\documentclass{report}

\usepackage[T2A]{fontenc}
\usepackage[utf8]{luainputenc}
\usepackage[english, russian]{babel}
\usepackage[pdftex]{hyperref}
\usepackage[14pt]{extsizes}
\usepackage{listings}
\usepackage{color}
\usepackage{geometry}
\usepackage{enumitem}
\usepackage{multirow}
\usepackage{graphicx}
\usepackage{indentfirst}
\usepackage{amsmath}


\geometry{a4paper,top=2cm,bottom=3cm,left=2cm,right=1.5cm}
\setlength{\parskip}{0.5cm}
\setlist{nolistsep, itemsep=0.3cm,parsep=0pt}

\lstset{language=C++,
		basicstyle=\footnotesize,
		keywordstyle=\color{blue}\ttfamily,
		stringstyle=\color{red}\ttfamily,
		commentstyle=\color{green}\ttfamily,
		morecomment=[l][\color{magenta}]{\#}, 
		tabsize=4,
		breaklines=true,
  		breakatwhitespace=true,
  		title=\lstname,       
}

\makeatletter
\renewcommand\@biblabel[1]{#1.\hfil}
\makeatother

\begin{document}

\begin{titlepage}

\begin{center}
Министерство науки и высшего образования Российской Федерации
\end{center}

\begin{center}
Федеральное государственное автономное образовательное учреждение высшего образования \\
Национальный исследовательский Нижегородский государственный университет им. Н.И. Лобачевского
\end{center}

\begin{center}
Институт информационных технологий, математики и механики
\end{center}

\vspace{4em}

\begin{center}
\textbf{\LargeОтчет по лабораторной работе} \\
\end{center}
\begin{center}
\textbf{\Large«Поразрядная сортировка для вещественных чисел с чётно-нечётным слиянием Бэтчера»} \\
\end{center}

\vspace{4em}

\newbox{\lbox}
\savebox{\lbox}{\hbox{text}}
\newlength{\maxl}
\setlength{\maxl}{\wd\lbox}
\hfill\parbox{7cm}{
\hspace*{5cm}\hspace*{-5cm}\textbf{Выполнил:} \\ студент группы 381806-3 \\ Зайцев А. Р.\\
\\
\hspace*{5cm}\hspace*{-5cm}\textbf{Проверил:}\\ доцент кафедры МОСТ, \\ кандидат технических наук \\ Сысоев А. В.\\
}
\vspace{\fill}

\begin{center}
Нижний Новгород \\ 2021
\end{center}

\end{titlepage}

\setcounter{page}{2}

% Содержание
\tableofcontents
\newpage

% Введение
\section*{Введение}
\addcontentsline{toc}{section}{Введение}
При работе с данными нам очень часто приходится их отсортировывать, и эту задачу решают методы сортировки. На сегодняшний день существует очень много разнообразных методов сортировки, от простых до сложных, от медленных до быстрых. Они уже просто необходимы при разработке программы, и трудно представить работу с данными без них.

\par Одной из таких сортировок является "поразрядная сортировка". Её реализация достаточно трудна, так как сортировка будет происходить по разрядам чисел, т.е. будут сравниваться поочерёдно каждый байт у чисел, и чем больше будет значение этого байта(всего 256 значений, т.к. байт содержит 8 бит, каждый из которых принимает 0 или 1), тем на данном этапе число будет больше. Я специально сказал 'на данном этапе', т.к. окончательно числа отсортируются после сравнений последнего разряда, а все промежуточные сравнения на текущем разряде нужны для правильной сортировки на следующем разряде. В частности числа, с которыми придётся работать в моей программе, будут вещественными(типа double). Этот тип хранится в памяти размером в 8 байт, и соответственно для сортировки чисел с таким типом данных нужно будет проделать 8 шагов, где на каждом шаге будут отсортировываться кокретные байты.

\par Помимо поразрядной сортировки в программе будет использоваться чётно-нечётное слияние Бэтчера. Суть этого слияния в том, что исходный массив данных разбивается на левую и правую половинку, каждая из которых отсортировывается, а после чего эти половинки сливаются в один массив таким образом, что на чётные индексы главного массива помещаются в отсортированном порядке элементы, стоящие на тех же чётных индексах в двух отсортированных половинках, и тоже самое происходит с нечётными индексами. После такого слияния дополнительно необходимо отсортировать элементы в результирующем массиве между собой(проделать swap при необходимости), это делается в один проход по массиву от начала до конца.

\par Таким образом в данной задаче будет рассмотрена совместная работа поразрядной сортировки и чётно-нечётного слияния Бэтчера над вектором вещественных чисел. 

\newpage

% Постановка задачи
\section*{Постановка задачи}
\addcontentsline{toc}{section}{Постановка задачи}
В рамках лабораторной работы нужно реализовать алгоритм чётно-нечётного слияния Бэтчера, внутри которого будет использоваться поразрядная сортировка, и всё это должно работать для вещественных чисел. Этот алгоритм должен уметь отсортировывать исходный массив чисел.

\par Требуется выполнить следующее:

\begin{enumerate}
\item Изучить и реализовать поразрядную сортировку и чётно-нечётное слияние Бэтчера.
\item Разработать и реализовать параллельный алгоритм данной задачи с использованием технологий OpenMP, TBB.
\item Проверить корректность работы алгоритмов с помощью набора автоматических тестов с использованием Google C++ Testing Framework для каждой из технологий.
\item Провести вычислительные эксперименты и сделать выводы.
\end{enumerate}
\newpage

% Метод решения
\section*{Метод решения}
\addcontentsline{toc}{section}{Метод решения}
Пусть нам дан вектор вещественных чисел, который нужно отсортировать. Для примера возьмём вектор из 16 чисел $N = \{n_1, n_2, ..., n_{16}\}$. Для начала разделим этот целый вектор на несколько маленьких одинаковой длины. Делать это будем, постепенно разбивая вектор на равные половинки. Также в моём алгоритме есть возможность задать границу, которая будет служить прекращением деления(если количество элементов в векторе равняется ей или становится меньше, то деление прекращается). Предположим, что в данном примере эта граница будет равняться 4.

\par Итак, начинаем алгоритм. Сначала делим исходный вектор из 16 чисел на 2 вектора по 8 чисел $N_1 = \{n_1, n_2, ..., n_8\}$, $N_2 = \{n_9, n_{10}, ..., n_{16}\}$. Теперь каждый из двух получившихся векторов снова делим на 2 равные половинки. Получится следующее: $N_{11} = \{n_1, n_2, n_3, n_4\}$, $N_{12} = \{n_5, n_6, n_7, n_8\}$, $N_{21} = \{n_9, n_{10}, n_{11}, n_{12}\}$, $N_{22} = \{n_{13}, n_{14}, n_{15}, n_{16}\}$. После этого деление уже не продолжается, так как мы задали границу в 4, а количество элементов в каждом подвекторе равняется 4.

\par Теперь каждый из этих подвекторов с 4-мя числами отсортировывается поразрядной сортировкой, и затем уже отсортированные подвектора подаются на чётно-нечётное слияние Бэтчера (подробный разбор поразрядной сортировки и чётно-нечётного слияния Бэтчера приведён в введении). Слияние будет происходит в обратном порядке делению на половинки. Т.е. отсортированные по отдельности $N_{11}$ и $N_{12}$ сливаются в $N_1$ по правилу чётных и нечётных индексов, аналогично $N_{21}$ и $N_{22}$ сливаются в $N_2$. $N_1$ и $N_2$ доотсортировываются одним проходом по ним(будет происходить swap соседних элементов по необходимости). И затем снова уже полностью отсортированные по отдельности $N_1$ и $N_2$ сливаются в вектор $N$, который также дополнительно доотсортировывается. В итоге мы получим отсортированный вектор из всех исходных чисел.

\par В общем случае, при большом количестве элементов, смысл алгоритма такой же. Мы просто делим пополам наши вектора, затем сортируем наименьшие из подвекторов и сливаем их в обратном порядке.

\newpage

% Схема распараллеливания
\section*{Схема распараллеливания}
\addcontentsline{toc}{section}{Схема распараллеливания}
Рассмотрим реализацию алгоритма поразрядной сортировки с чётно-нечётным слиянием Бэтчера с точки зрения задачи распараллеливания вычислений.

\par Распараллеливание я произвёл в следующих моментах:

\par 1. Был распараллелен цикл, передающий наименьшие подвектора чисел после деления на поразрядную сортировку. Т.е. за один момент времени одновременно стали отсортировываться несколько подвекторов, что сильно ускорило вычисления.
\par 2. Был распараллелен цикл, выполняющий слияние двух подвекторов в один.
\par 3. Также был распараллелен цикл, в котором доотсортировывались вектора, образованные из 2-ух подвекторов.

\newpage

% Описание программной реализации
\section*{Описание программной реализации}
\addcontentsline{toc}{section}{Описание программной реализации}
Вещественные числа в программе хранятся в виде вектора:
\begin{lstlisting}
std::vector<double> v;
\end{lstlisting}
Рассмотрим алгоритмы в программе.
\par Последовательный алгоритм поразрядной сортировки с чётно-нечётным слиянием Бэтчера для вещественных чисел реализован в следующей функции:
\begin{lstlisting}
void oddEvenMergeSortSeq(double* vec, int size, int border);
\end{lstlisting}
\par В качестве входных параметров передаётся адрес на изменяемый исходный вектор чисел, размер этого вектора и граница, влияющая на размер конечных подвекторов для поразрядной сортировки. Выходного параметра нет, т.к. изменения записываются в векторе, который передан по адресу.
\par Параллельные алгоритмы для реализаций с OpenMP и TBB представлены в функциях:
\begin{lstlisting}
void oddEvenMergeSortOmp(double* vec, int size, int border, int threads);
void oddEvenMergeSortTbb(double* vec, int size, int border, int threads);
\end{lstlisting}
\par В качестве входного параметра добавляется только целочисленное количество потоков, чтобы иметь возможность задавать исходное количество потоков по-разному. Во всём остальном эти функции аналогичны функции для последовательного варианта.

\newpage

% Подтверждение корректности
\section*{Подтверждение корректности}
\addcontentsline{toc}{section}{Подтверждение корректности}
Для подтверждения корректности в программе представлен набор тестов, разработанных с помощью Google Testing Framework.
\par Набор включает в себя тесты, проверяющие работоспособность всех функций. Они проверяют работоспособность создания вектора из случайных чисел, отдельно проверяют саму поразрядную сортировку, ну и корректность самих функций чётно-нечётного слияния Бэтчера с векторами вещественных чисел разного размера.

\par По окончании проверки успешное прохождение всех тестов подтверждает корректность работы написанной программы.
\newpage

% Результаты экспериментов
\section*{Результаты экспериментов}
\addcontentsline{toc}{section}{Результаты экспериментов}
Вычислительные эксперименты для оценки эффективности параллельного алгоритма рассматриваемой задачи проводились на оборудовании со следующей аппаратной конфигурацией:

\begin{itemize}
\item Процессор: AMD Ryzen 5 1600 Six-Core Processor 3.20 GHz, 6 ядер;
\item Оперативная память: 16 ГБ;
\item ОС: Windows 10 Pro.
\end{itemize}

\par В рамках эксперимента было вычислено время работы последовательного алгоритма и двух параллельных алгоритмов сортировки исходного вектора вещественных чисел с использованием чётно-нечётного слияния Бэтчера и поразрядной сортировки. В тестах отсортировывались вектора по 1000000, 10000000 и 50000000 элементов, а в качестве границы делимости была указана 1000, 10000 и 50000 соответственно. Тесты запускались на шести потоках.
\par Результаты экспериментов представлены ниже.

\begin{table}[!h]
\caption{Сравнение последовательного алгоритма с OpenMP.}
\centering
\begin{tabular}{lllll}
Размер вектора & Последовательно & Параллельно & Ускорение  \\
1000000        & 0.243         & 0.064     & 3.8       \\
10000000        & 2.545         & 0.629     & 4.05       \\
50000000       & 13.09         & 3.293     & 3.98       
\end{tabular}
\end{table}

\begin{table}[!h]
\caption{Сравнение последовательного алгоритма с TBB.}
\centering
\begin{tabular}{lllll}
Размер вектора & Последовательно & Параллельно & Ускорение  \\
1000000        & 0.243         & 0.055     & 4.42       \\
10000000        & 2.545         & 0.535     & 4.76       \\
50000000       & 13.09         & 2.78     & 4.71       
\end{tabular}
\end{table}

\par По данным экспериментов можно сделать вывод, что параллельный алгоритм работает быстрее последовательного. А между OpenMP и TBB с небольшим преимуществом в среднем в 1.16 раза выигрывает TBB. Но и при разработке реализация TBB вызывает больше трудностей, чем OpenMP.
\newpage

% Заключение
\section*{Заключение}
\addcontentsline{toc}{section}{Заключение}
В ходе выполнения лабораторной работы был реализован алгоритм чётно-нечётного слияния Бэтчера с использованием поразрядной сортировки. Этот алгоритм был сделан в последовательном варианте и в 2-ух параллельных: с использованием OpenMP и TBB.
\par По данным экспериментов удалось сравнить время работы последовательного алгоритма с каждым из 2-ух параллельных алгоритмов.
\par В итоге удалось достичь достаточно хорошего ускорения для обоих вариантов распараллеливания.
\newpage

% Список литературы
\begin{thebibliography}{1}
\addcontentsline{toc}{section}{Список литературы}
\bibitem{Sysoev} Сысоев А.В., Мееров И.Б., Свистунов А.Н., Курылев А.Л., Сенин А.В., Шишков А.В., Корняков К.В., Сиднев А.А. «Параллельное программирование в системах с общей памятью. Инструментальная поддержка». Учебно-методические материалы по программе повышения квалификации «Технологии высокопроизводительных вычислений для обеспечения учебного процесса и научных исследований». Нижний Новгород, 2007, 110 с.
\bibitem{Wiki1} Wikipedia: the free encyclopedia [Электронный ресурс] // URL: https://ru.wikipedia.org/wiki/Поразрядная сортировка (дата обращения: 22.03.2021)
\end{thebibliography}
\newpage

% Приложение
\section*{Приложение}
\addcontentsline{toc}{section}{Приложение}
В этом разделе находится листинг всего кода, написанного в рамках лабораторной работы.
\par 1. Библиотека с последовательной реализацией. Файл: RadixSortSeq.h
\begin{lstlisting}
// Copyright 2021 Zaitsev Andrey
#ifndef MODULES_TASK_1_ZAITSEV_A_RADIX_SORT_SEQ_RADIXSORTSEQ_H_
#define MODULES_TASK_1_ZAITSEV_A_RADIX_SORT_SEQ_RADIXSORTSEQ_H_

#include <vector>

void merge(double *vec, double *out, int center, int size, int odd_even);
void oddEvenMergeSort(double *vec, double *out, int size, int border);
void radixSort(double *vec, double *out, int size);
double* createVector(double* vec, int size, double a, double b);
bool check(double* vec, int size);

#endif  // MODULES_TASK_1_ZAITSEV_A_RADIX_SORT_SEQ_RADIXSORTSEQ_H_
\end{lstlisting}
\par 2. Библиотека с последовательной реализацией. Файл: RadixSortSeq.cpp
\begin{lstlisting}
// Copyright 2021 Zaitsev Andrey
#include <random>
#include <ctime>
#include <vector>
#include <algorithm>
#include "../../../modules/task_1/zaitsev_a_radix_sort_seq/RadixSortSeq.h"

double* createVector(double* vec, int size, double a, double b) {
  std::mt19937 gen;
  gen.seed(static_cast<unsigned int>(time(0)));
  std::uniform_real_distribution<double> distr(a, b);
  for (int i = 0; i < size; i++) {
    vec[i] = distr(gen);
  }
  return vec;
}

bool check(double* vec, int size) {
  for (int i = 0; i < size - 1; i++) {
    if (vec[i + 1] >= vec[i]) {
      i++;
    } else {
      return false;
    }
  }
  return true;
}

void radixSort(double *vec, double *out, int size) {
  double* swap = nullptr;
  for (int i = 0; i < 7; i++) {
    unsigned char* arr = (unsigned char*)vec;
    int cnt[256] = { 0 };
    int res = 0;
    for (int j = 0; j < size; j++)
      cnt[arr[8 * j + i]]++;
    for (int j = 0; j < 256; j++) {
      int tmp_ = cnt[j];
      cnt[j] = res;
      res += tmp_;
    }
    for (int j = 0; j < size; j++) {
      int ind = 8 * j + i;
      out[cnt[arr[ind]]] = vec[j];
      cnt[arr[ind]]++;
    }

    swap = vec;
    vec = out;
    out = swap;
  }

  unsigned char* arr2 = (unsigned char*)vec;
  int cnt[256] = { 0 };
  int shift[256] = { 0 };
  for (int i = 0; i < size; i++)
    cnt[arr2[8 * i + 7]]++;
  shift[255] = cnt[255] - 1;
  for (int i = 254; i >= 128; i--)
    shift[i] = shift[i + 1] + cnt[i];
  shift[0] = shift[128] + 1;
  for (int i = 1; i < 128; i++)
    shift[i] = shift[i - 1] + cnt[i - 1];
  for (int i = 0; i < size; i++) {
    if (arr2[8 * i + 7] < 128)
      out[shift[arr2[8 * i + 7]]++] = vec[i];
    else
      out[shift[arr2[8 * i + 7]]--] = vec[i];
  }
}

void merge(double *vec, double *out, int center, int size, int odd_even) {
  for (int i = odd_even; i < center; i += 2)
    out[i] = vec[i];
  double *inp2 = vec + center;
  int par1 = odd_even;
  int par2 = odd_even;
  int ind = odd_even;
  while ((par1 < center) && (par2 < size)) {
    if (out[par1] > inp2[par2]) {
      vec[ind] = inp2[par2];
      par2 += 2;
    } else {
      vec[ind] = out[par1];
      par1 += 2;
    }
    ind += 2;
  }
  if (par1 != center) {
    for (int j = par1; j < center; j += 2, ind += 2)
      vec[ind] = out[j];
  } else {
    for (int j = par2; j < size; j += 2, ind += 2)
      vec[ind] = inp2[j];
  }
}

void oddEvenMergeSort(double *vec, double *out, int size, int border) {
  if (size <= border) {
    radixSort(vec, out, size);
  } else {
    int center = size / 2 + (size / 2) % 2;
    oddEvenMergeSort(vec, out, center, border);
    oddEvenMergeSort(vec + center, out + center, size - center, border);
    merge(vec, out, center, size - center, 0);
    merge(vec, out, center, size - center, 1);
    for (int i = 1; i < (size + 1) / 2; i++) {
      if (vec[2 * i] < vec[2 * i - 1]) {
        double tmp = vec[2 * i - 1];
        vec[2 * i - 1] = vec[2 * i];
        vec[2 * i] = tmp;
      }
    }
  }
}
\end{lstlisting}

\par 3. Библиотека с последовательной реализацией. Файл: main.cpp
\begin{lstlisting}
// Copyright 2021 Zaitsev Andrey
#include <gtest/gtest.h>
#include <vector>
#include <algorithm>
#include "./RadixSortSeq.h"

TEST(Test_Seq, Size_5_Without_Random) {
  int size = 5;
  double* vec = new double[size] {4.33, -6.541,
    16.84, 2.662, -12.544};
  double* tmp = new double[size];
  bool result = false;
  radixSort(vec, tmp, size);
  result = check(vec, size);
  ASSERT_EQ(true, result);
  delete[] tmp;
  delete[] vec;
}

TEST(Test_Seq, Size_1000_Random_Positive) {
  int size = 1000;
  double* vec = new double[size];
  vec = createVector(vec, size, 100.0, 1000.0);
  double* tmp = new double[size];
  bool result = false;
  radixSort(vec, tmp, size);
  result = check(vec, size);
  ASSERT_EQ(true, result);
  delete[] tmp;
  delete[] vec;
}

TEST(Test_Seq, Size_1000_Random_Negative) {
  int size = 1000;
  double* vec = new double[size];
  vec = createVector(vec, size, -800.0, 0.0);
  double* tmp = new double[size];
  bool result = false;
  radixSort(vec, tmp, size);
  result = check(vec, size);
  ASSERT_EQ(true, result);
  delete[] tmp;
  delete[] vec;
}

TEST(Test_Seq, Size_1000_Random_Pos_And_Neg) {
  int size = 1000;
  double* vec = new double[size];
  vec = createVector(vec, size, -500.0, 500.0);
  double* tmp = new double[size];
  bool result = false;
  radixSort(vec, tmp, size);
  result = check(vec, size);
  ASSERT_EQ(true, result);
  delete[] tmp;
  delete[] vec;
}

TEST(Test_Seq, Exact_Answer) {
  int size = 7;
  double* vec = new double[size] {5.32, 8.25, 3.463, 43.733,
    120.632, 75.86, 53.41};
  double* answer = new double[size] {3.463, 5.32, 8.25,
    43.733, 53.41, 75.86, 120.632};
  double* tmp = new double[size];
  bool result = false;
  radixSort(vec, tmp, size);
  result = check(vec, size);
  ASSERT_EQ(true, result);
  delete[] tmp;
  delete[] vec;
  delete[] answer;
}

TEST(Test_Seq, Size_100_Batcher) {
  int size = 100;
  double* vec = new double[size];
  vec = createVector(vec, size, 0.0, 1000.0);
  double* tmp = new double[size];
  bool result = false;
  oddEvenMergeSort(vec, tmp, size, 6);
  result = check(vec, size);
  ASSERT_EQ(true, result);
  delete[] tmp;
  delete[] vec;
}

TEST(Test_Seq, Size_1000_Batcher_Border_20) {
  int size = 1000;
  double* vec = new double[size];
  vec = createVector(vec, size, -1000.0, 1000.0);
  double* tmp = new double[size];
  bool result = false;
  oddEvenMergeSort(vec, tmp, size, 20);
  result = check(vec, size);
  ASSERT_EQ(true, result);
  delete[] tmp;
  delete[] vec;
}

TEST(Test_Seq, Size_1000_Batcher_Border_1000) {
  int size = 1000;
  double* vec = new double[size];
  vec = createVector(vec, size, -1000.0, 1000.0);
  double* tmp = new double[size];
  bool result = false;
  oddEvenMergeSort(vec, tmp, size, 1000);
  result = check(vec, size);
  ASSERT_EQ(true, result);
  delete[] tmp;
  delete[] vec;
}

int main(int argc, char** argv) {
  ::testing::InitGoogleTest(&argc, argv);
  return RUN_ALL_TESTS();
}
\end{lstlisting}

\par 4. Библиотека с реализацией на OpenMP. Файл: RadixSortOmp.h
\begin{lstlisting}
// Copyright 2021 Zaitsev Andrey
#ifndef MODULES_TASK_2_ZAITSEV_A_RADIX_SORT_RADIXSORTOMP_H_
#define MODULES_TASK_2_ZAITSEV_A_RADIX_SORT_RADIXSORTOMP_H_

#include <omp.h>
#include <vector>

void merge(double *vec, double *out, int center, int size, int odd_even);
void oddEvenMergeSortSeq(double* vec, int size, int border);
void oddEvenMergeSortOmp(double* vec, int size, int border, int threads);
void radixSort(double *vec, double *out, int size);
std::vector<double> createVector(int size, double a, double b);
bool check(double* vec, int size);

#endif  // MODULES_TASK_2_ZAITSEV_A_RADIX_SORT_RADIXSORTOMP_H_
\end{lstlisting}

\par 5. Библиотека с реализацией на OpenMP. Файл: RadixSortOmp.cpp
\begin{lstlisting}
// Copyright 2021 Zaitsev Andrey
#include <omp.h>
#include <random>
#include <ctime>
#include <utility>
#include <vector>
#include <queue>
#include <stack>
#include "../../../modules/task_2/zaitsev_a_radix_sort/RadixSortOmp.h"

std::vector<double> createVector(int size, double a, double b) {
  std::mt19937 gen;
  gen.seed(static_cast<unsigned int>(time(0)));
  std::uniform_real_distribution<double> distr(a, b);
  std::vector<double> vec;
  for (int i = 0; i < size; i++) {
    vec.push_back(distr(gen));
  }
  return vec;
}

bool check(double* vec, int size) {
  for (int i = 0; i < size - 1; i++) {
    if (vec[i + 1] >= vec[i]) {
      i++;
    } else {
      return false;
    }
  }
  return true;
}

void radixSort(double *vec, double *out, int size) {
  double* swap = nullptr;
  for (int i = 0; i < 7; i++) {
    unsigned char* temp_vec_one_byte = (unsigned char*)vec;
    int cnt[256] = { 0 };
    int res = 0;
    for (int j = 0; j < size; j++)
      cnt[temp_vec_one_byte[8 * j + i]]++;
    for (int j = 0; j < 256; j++) {
      int tmp_ = cnt[j];
      cnt[j] = res;
      res += tmp_;
    }
    for (int j = 0; j < size; j++) {
      int ind = 8 * j + i;
      out[cnt[temp_vec_one_byte[ind]]] = vec[j];
      cnt[temp_vec_one_byte[ind]]++;
    }

    swap = vec;
    vec = out;
    out = swap;
  }

  unsigned char* temp_vec_one_byte = (unsigned char*)vec;
  int cnt[256] = { 0 };
  int shift[256] = { 0 };
  for (int i = 0; i < size; i++)
    cnt[temp_vec_one_byte[8 * i + 7]]++;
  shift[255] = cnt[255] - 1;
  for (int i = 254; i >= 128; i--)
    shift[i] = shift[i + 1] + cnt[i];
  shift[0] = shift[128] + 1;
  for (int i = 1; i < 128; i++)
    shift[i] = shift[i - 1] + cnt[i - 1];
  for (int i = 0; i < size; i++) {
    if (temp_vec_one_byte[8 * i + 7] < 128)
      out[shift[temp_vec_one_byte[8 * i + 7]]++] = vec[i];
    else
      out[shift[temp_vec_one_byte[8 * i + 7]]--] = vec[i];
  }
}

void merge(double *vec, double *out, int center, int size, int odd_even) {
  for (int i = odd_even; i < center; i += 2)
    out[i] = vec[i];
  double *vec_from_center = vec + center;
  int par1 = odd_even;
  int par2 = odd_even;
  int ind = odd_even;
  while ((par1 < center) && (par2 < size)) {
    if (out[par1] > vec_from_center[par2]) {
      vec[ind] = vec_from_center[par2];
      par2 += 2;
    } else {
      vec[ind] = out[par1];
      par1 += 2;
    }
    ind += 2;
  }
  if (par1 != center) {
    for (int j = par1; j < center; j += 2, ind += 2)
      vec[ind] = out[j];
  } else {
    for (int j = par2; j < size; j += 2, ind += 2)
      vec[ind] = vec_from_center[j];
  }
}

void oddEvenMergeSortSeq(double* vec, int size, int border) {
  double* temp_vec = new double[size];
  std::vector<std::pair<int, int>> piecesOfVector;
  std::stack<std::pair<int, std::pair<int, int>>> preparingForMerges;
  std::queue<std::pair<int, int>> curr_level;
  std::queue<std::pair<int, int>> next_level;

  preparingForMerges.push({ 0, {0, 0} });
  curr_level.push({ 0, size });

  std::pair<int, int> current;

  while (true) {
    current = curr_level.front();
    curr_level.pop();
    if (current.second > border) {
      int center = current.second / 2 + (current.second / 2) % 2;
      next_level.push({ current.first, center });
      next_level.push({ current.first + center, current.second - center });
      preparingForMerges.push({ current.first,
        {center, current.second - center} });
    } else {
      piecesOfVector.push_back(current);
    }
    if (curr_level.empty()) {
      if (!next_level.empty()) {
        curr_level = next_level;
        next_level = std::queue<std::pair<int, int>>();
        preparingForMerges.push({ 0, {0, 0} });
      } else {
        break;
      }
    }
  }
  preparingForMerges.pop();

  for (size_t i = 0; i < piecesOfVector.size(); i++) {
    radixSort(vec + piecesOfVector[i].first,
      temp_vec + piecesOfVector[i].first, piecesOfVector[i].second);
  }
  std::vector<std::pair<int, std::pair<int, int>>> merges_order;

  while (!preparingForMerges.empty()) {
    while (preparingForMerges.top().second.second != 0) {
      merges_order.push_back(preparingForMerges.top());
      preparingForMerges.pop();
    }
    preparingForMerges.pop();

    for (size_t i = 0; i < merges_order.size() * 2; i++) {
      int start = merges_order[i / 2].first;
      int center = merges_order[i / 2].second.first;
      int size_after_center = merges_order[i / 2].second.second;
      merge(vec + start, temp_vec + start, center, size_after_center, i % 2);
    }

    for (size_t i = 0; i < merges_order.size(); i++) {
      int start = merges_order[i].first;
      int center = merges_order[i].second.first;
      int size_after_center = merges_order[i].second.second;
      for (int j = 1; j < (center + size_after_center + 1) / 2; j++) {
        if (vec[start + 2 * j - 1] > vec[start + 2 * j]) {
          double tmp = vec[start + 2 * j];
          vec[start + 2 * j] = vec[start + 2 * j - 1];
          vec[start + 2 * j - 1] = tmp;
        }
      }
    }
    merges_order.clear();
  }
  delete[] temp_vec;
}

void oddEvenMergeSortOmp(double* vec, int size, int border, int threads) {
  double* temp_vec = new double[size];
  std::vector<std::pair<int, int>> piecesOfVector;
  std::stack<std::pair<int, std::pair<int, int>>> preparingForMerges;
  std::queue<std::pair<int, int>> curr_level;
  std::queue<std::pair<int, int>> next_level;

  preparingForMerges.push({ 0, {0, 0} });
  curr_level.push({ 0, size });

  std::pair<int, int> current;

  while (true) {
    current = curr_level.front();
    curr_level.pop();
    if (current.second > border) {
      int center = current.second / 2 + (current.second / 2) % 2;
      next_level.push({ current.first, center });
      next_level.push({ current.first + center, current.second - center });
      preparingForMerges.push({ current.first,
        {center, current.second - center} });
    } else {
      piecesOfVector.push_back(current);
    }
    if (curr_level.empty()) {
      if (!next_level.empty()) {
        curr_level = next_level;
        next_level = std::queue<std::pair<int, int>>();
        preparingForMerges.push({ 0, {0, 0} });
      } else {
        break;
      }
    }
  }
  preparingForMerges.pop();

#pragma omp parallel for num_threads(threads)
  for (int i = 0; i < static_cast<int>(piecesOfVector.size()); i++) {
    radixSort(vec + piecesOfVector[i].first,
      temp_vec + piecesOfVector[i].first, piecesOfVector[i].second);
  }
  std::vector<std::pair<int, std::pair<int, int>>> merges_order;

  while (!preparingForMerges.empty()) {
    while (preparingForMerges.top().second.second != 0) {
      merges_order.push_back(preparingForMerges.top());
      preparingForMerges.pop();
    }
    preparingForMerges.pop();

#pragma omp parallel for num_threads(threads)
    for (int i = 0; i < static_cast<int>(merges_order.size() * 2); i++) {
      int start = merges_order[i / 2].first;
      int center = merges_order[i / 2].second.first;
      int size_after_center = merges_order[i / 2].second.second;
      merge(vec + start, temp_vec + start, center, size_after_center, i % 2);
    }

#pragma omp parallel for num_threads(threads)
    for (int i = 0; i < static_cast<int>(merges_order.size()); i++) {
      int start = merges_order[i].first;
      int center = merges_order[i].second.first;
      int size_after_center = merges_order[i].second.second;
      for (int j = 1; j < (center + size_after_center + 1) / 2; j++) {
        if (vec[start + 2 * j - 1] > vec[start + 2 * j]) {
          double tmp = vec[start + 2 * j];
          vec[start + 2 * j] = vec[start + 2 * j - 1];
          vec[start + 2 * j - 1] = tmp;
        }
      }
    }
    merges_order.clear();
  }
  delete[] temp_vec;
}
\end{lstlisting}

\par 6. Библиотека с реализацией на OpenMP. Файл: main.cpp
\begin{lstlisting}
// Copyright 2021 Zaitsev Andrey
#include <gtest/gtest.h>
#include <vector>
#include <iostream>
#include "./RadixSortOmp.h"

TEST(Test_Omp, Size_5_Without_Random) {
  int size = 5;
  double* vec = new double[size] {5.21, -8.652,
  26.32, 3.289, -14.842};
  double* tmp = new double[size];
  bool result = false;
  radixSort(vec, tmp, size);
  result = check(vec, size);
  ASSERT_EQ(true, result);
  delete[] tmp;
  delete[] vec;
}

TEST(Test_Omp, Size_1000_Random_Pos_And_Neg) {
  int size = 1000;
  std::vector<double> vec = createVector(size, -1000.0, 1000.0);
  double* tmp = new double[size];
  bool result = false;
  radixSort(vec.data(), tmp, size);
  result = check(vec.data(), size);
  ASSERT_EQ(true, result);
  delete[] tmp;
}

TEST(Test_Omp, Size_1000_Batcher_Seq_Border_20) {
  int size = 1000;
  std::vector<double> vec = createVector(size, -1000.0, 1000.0);
  bool result = false;
  oddEvenMergeSortSeq(vec.data(), size, 20);
  result = check(vec.data(), size);
  ASSERT_EQ(true, result);
}

TEST(Test_Omp, Size_100_Batcher_Omp_1thr_Border_25) {
  int size = 100;
  int thr = 1;
  std::vector<double> vec = createVector(size, -100.0, 100.0);
  bool result = false;
  oddEvenMergeSortOmp(vec.data(), size, 25, thr);
  result = check(vec.data(), size);
  ASSERT_EQ(true, result);
}

TEST(Test_Omp, Size_100_Batcher_Omp_4thr_Border_25) {
  int size = 100;
  int thr = 4;
  std::vector<double> vec = createVector(size, -1000.0, 1000.0);
  bool result = false;
  oddEvenMergeSortOmp(vec.data(), size, 25, thr);
  result = check(vec.data(), size);
  ASSERT_EQ(true, result);
}

TEST(Test_Omp, Size_1000_Batcher_Omp_4thr_Border_60) {
  int size = 1000;
  int thr = 4;
  std::vector<double> vec = createVector(size, -1000.0, 1000.0);
  bool result = false;
  oddEvenMergeSortOmp(vec.data(), size, 60, thr);
  result = check(vec.data(), size);
  ASSERT_EQ(true, result);
}

TEST(Test_Omp, Test_Equality_Seq_Par_1000) {
  int size = 1000;
  int thr = 5;
  std::vector<double> v1 = createVector(size, -1000.0, 1000.0);
  std::vector<double> v2 = v1;
  double t1 = omp_get_wtime();
  oddEvenMergeSortSeq(v1.data(), size, 50);
  double t2 = omp_get_wtime();
  std::cout << "Seq: " << t2 - t1 << std::endl;
  t1 = omp_get_wtime();
  oddEvenMergeSortOmp(v2.data(), size, 50, thr);
  t2 = omp_get_wtime();
  std::cout << "Omp: " << t2 - t1 << std::endl;
  ASSERT_EQ(v1, v2);
}

TEST(Test_Omp, Test_Equality_Seq_Par_2000) {
  int size = 2000;
  int thr = 5;
  std::vector<double> v1 = createVector(size, -1000.0, 1000.0);
  std::vector<double> v2 = v1;
  double t1 = omp_get_wtime();
  oddEvenMergeSortSeq(v1.data(), size, 50);
  double t2 = omp_get_wtime();
  std::cout << "Seq: " << t2 - t1 << std::endl;
  t1 = omp_get_wtime();
  oddEvenMergeSortOmp(v2.data(), size, 50, thr);
  t2 = omp_get_wtime();
  std::cout << "Omp: " << t2 - t1 << std::endl;
  ASSERT_EQ(v1, v2);
}

TEST(Test_Omp, Test_Equality_Seq_Par_3000) {
  int size = 3000;
  int thr = 5;
  std::vector<double> v1 = createVector(size, -1000.0, 1000.0);
  std::vector<double> v2 = v1;
  double t1 = omp_get_wtime();
  oddEvenMergeSortSeq(v1.data(), size, 50);
  double t2 = omp_get_wtime();
  std::cout << "Seq: " << t2 - t1 << std::endl;
  t1 = omp_get_wtime();
  oddEvenMergeSortOmp(v2.data(), size, 50, thr);
  t2 = omp_get_wtime();
  std::cout << "Omp: " << t2 - t1 << std::endl;
  ASSERT_EQ(v1, v2);
}

TEST(Test_Omp, Test_Equality_Seq_Par_4000) {
  int size = 4000;
  int thr = 6;
  std::vector<double> v1 = createVector(size, -1000.0, 1000.0);
  std::vector<double> v2 = v1;
  double t1 = omp_get_wtime();
  oddEvenMergeSortSeq(v1.data(), size, 50);
  double t2 = omp_get_wtime();
  std::cout << "Seq: " << t2 - t1 << std::endl;
  t1 = omp_get_wtime();
  oddEvenMergeSortOmp(v2.data(), size, 50, thr);
  t2 = omp_get_wtime();
  std::cout << "Omp: " << t2 - t1 << std::endl;
  ASSERT_EQ(v1, v2);
}

TEST(Test_Omp, Test_Equality_Seq_Par_1000000) {
	int size = 1000000;
	int thr = 6;
	std::vector<double> v1 = createVector(size, -1000.0, 1000.0);
	std::vector<double> v2 = v1;
	double t1 = omp_get_wtime();
	oddEvenMergeSortSeq(v1.data(), size, 1000);
	double t2 = omp_get_wtime();
	std::cout << "Seq: " << t2 - t1 << std::endl;
	t1 = omp_get_wtime();
	oddEvenMergeSortOmp(v2.data(), size, 1000, thr);
	t2 = omp_get_wtime();
	std::cout << "Omp: " << t2 - t1 << std::endl;
	ASSERT_EQ(v1, v2);
}

int main(int argc, char** argv) {
  ::testing::InitGoogleTest(&argc, argv);
  return RUN_ALL_TESTS();
}
\end{lstlisting}

\par 7. Библиотека с реализацией на TBB. Файл: RadixSortTbb.h
\begin{lstlisting}
// Copyright 2021 Zaitsev Andrey
#ifndef MODULES_TASK_3_ZAITSEV_A_RADIX_SORT_RADIXSORTTBB_H_
#define MODULES_TASK_3_ZAITSEV_A_RADIX_SORT_RADIXSORTTBB_H_

#include <vector>

void merge(double *vec, double *out, int center, int size, int odd_even);
void oddEvenMergeSortSeq(double* vec, int size, int border);
void oddEvenMergeSortTbb(double* vec, int size, int border, int threads);
void radixSort(double *vec, double *out, int size);
std::vector<double> createVector(int size, double a, double b);
bool check(double* vec, int size);

#endif  // MODULES_TASK_3_ZAITSEV_A_RADIX_SORT_RADIXSORTTBB_H_
\end{lstlisting}

\par 8. Библиотека с реализацией на TBB. Файл: RadixSortTbb.cpp
\begin{lstlisting}
// Copyright 2021 Zaitsev Andrey
#include <tbb/tbb.h>
#include <random>
#include <ctime>
#include <utility>
#include <vector>
#include <queue>
#include <stack>
#include "../../../modules/task_3/zaitsev_a_radix_sort/RadixSortTbb.h"

std::vector<double> createVector(int size, double a, double b) {
  std::mt19937 gen;
  gen.seed(static_cast<unsigned int>(time(0)));
  std::uniform_real_distribution<double> distr(a, b);
  std::vector<double> vec;
  for (int i = 0; i < size; i++) {
    vec.push_back(distr(gen));
  }
  return vec;
}

bool check(double* vec, int size) {
  for (int i = 0; i < size - 1; i++) {
    if (vec[i + 1] >= vec[i]) {
      i++;
    } else {
      return false;
    }
  }
  return true;
}

void radixSort(double *vec, double *out, int size) {
  double* swap = nullptr;
  for (int i = 0; i < 7; i++) {
    unsigned char* temp_vec_one_byte = (unsigned char*)vec;
    int cnt[256] = { 0 };
    int res = 0;
    for (int j = 0; j < size; j++)
      cnt[temp_vec_one_byte[8 * j + i]]++;
    for (int j = 0; j < 256; j++) {
      int tmp_ = cnt[j];
      cnt[j] = res;
      res += tmp_;
    }
    for (int j = 0; j < size; j++) {
      int ind = 8 * j + i;
      out[cnt[temp_vec_one_byte[ind]]] = vec[j];
      cnt[temp_vec_one_byte[ind]]++;
    }

    swap = vec;
    vec = out;
    out = swap;
  }

  unsigned char* temp_vec_one_byte = (unsigned char*)vec;
  int cnt[256] = { 0 };
  int shift[256] = { 0 };
  for (int i = 0; i < size; i++)
    cnt[temp_vec_one_byte[8 * i + 7]]++;
  shift[255] = cnt[255] - 1;
  for (int i = 254; i >= 128; i--)
    shift[i] = shift[i + 1] + cnt[i];
  shift[0] = shift[128] + 1;
  for (int i = 1; i < 128; i++)
    shift[i] = shift[i - 1] + cnt[i - 1];
  for (int i = 0; i < size; i++) {
    if (temp_vec_one_byte[8 * i + 7] < 128)
      out[shift[temp_vec_one_byte[8 * i + 7]]++] = vec[i];
    else
      out[shift[temp_vec_one_byte[8 * i + 7]]--] = vec[i];
  }
}

void merge(double *vec, double *out, int center, int size, int odd_even) {
  for (int i = odd_even; i < center; i += 2)
    out[i] = vec[i];
  double *vec_from_center = vec + center;
  int par1 = odd_even;
  int par2 = odd_even;
  int ind = odd_even;
  while ((par1 < center) && (par2 < size)) {
    if (out[par1] > vec_from_center[par2]) {
      vec[ind] = vec_from_center[par2];
      par2 += 2;
    } else {
      vec[ind] = out[par1];
      par1 += 2;
    }
    ind += 2;
  }
  if (par1 != center) {
    for (int j = par1; j < center; j += 2, ind += 2)
      vec[ind] = out[j];
  } else {
    for (int j = par2; j < size; j += 2, ind += 2)
      vec[ind] = vec_from_center[j];
  }
}

void oddEvenMergeSortSeq(double* vec, int size, int border) {
  double* temp_vec = new double[size];
  std::vector<std::pair<int, int>> piecesOfVector;
  std::stack<std::pair<int, std::pair<int, int>>> preparingForMerges;
  std::queue<std::pair<int, int>> curr_level;
  std::queue<std::pair<int, int>> next_level;

  preparingForMerges.push({ 0, {0, 0} });
  curr_level.push({ 0, size });

  std::pair<int, int> current;

  while (true) {
    current = curr_level.front();
    curr_level.pop();
    if (current.second > border) {
      int center = current.second / 2 + (current.second / 2) % 2;
      next_level.push({ current.first, center });
      next_level.push({ current.first + center, current.second - center });
      preparingForMerges.push({ current.first,
        {center, current.second - center} });
    } else {
      piecesOfVector.push_back(current);
    }
    if (curr_level.empty()) {
      if (!next_level.empty()) {
        curr_level = next_level;
        next_level = std::queue<std::pair<int, int>>();
        preparingForMerges.push({ 0, {0, 0} });
      } else {
        break;
      }
    }
  }
  preparingForMerges.pop();

  for (size_t i = 0; i < piecesOfVector.size(); i++) {
    radixSort(vec + piecesOfVector[i].first,
      temp_vec + piecesOfVector[i].first, piecesOfVector[i].second);
  }
  std::vector<std::pair<int, std::pair<int, int>>> merges_order;

  while (!preparingForMerges.empty()) {
    while (preparingForMerges.top().second.second != 0) {
      merges_order.push_back(preparingForMerges.top());
      preparingForMerges.pop();
    }
    preparingForMerges.pop();

    for (size_t i = 0; i < merges_order.size() * 2; i++) {
      int start = merges_order[i / 2].first;
      int center = merges_order[i / 2].second.first;
      int size_after_center = merges_order[i / 2].second.second;
      merge(vec + start, temp_vec + start, center, size_after_center, i % 2);
    }

    for (size_t i = 0; i < merges_order.size(); i++) {
      int start = merges_order[i].first;
      int center = merges_order[i].second.first;
      int size_after_center = merges_order[i].second.second;
      for (int j = 1; j < (center + size_after_center + 1) / 2; j++) {
        if (vec[start + 2 * j - 1] > vec[start + 2 * j]) {
          double tmp = vec[start + 2 * j];
          vec[start + 2 * j] = vec[start + 2 * j - 1];
          vec[start + 2 * j - 1] = tmp;
        }
      }
    }
    merges_order.clear();
  }
  delete[] temp_vec;
}

void oddEvenMergeSortTbb(double* vec, int size, int border, int threads) {
  double* temp_vec = new double[size];
  std::vector<std::pair<int, int>> piecesOfVector;
  std::stack<std::pair<int, std::pair<int, int>>> preparingForMerges;
  std::queue<std::pair<int, int>> curr_level;
  std::queue<std::pair<int, int>> next_level;

  preparingForMerges.push({ 0, {0, 0} });
  curr_level.push({ 0, size });

  std::pair<int, int> current;

  while (true) {
    current = curr_level.front();
    curr_level.pop();
    if (current.second > border) {
      int center = current.second / 2 + (current.second / 2) % 2;
      next_level.push({ current.first, center });
      next_level.push({ current.first + center, current.second - center });
      preparingForMerges.push({ current.first,
        {center, current.second - center} });
    } else {
      piecesOfVector.push_back(current);
    }
    if (curr_level.empty()) {
      if (!next_level.empty()) {
        curr_level = next_level;
        next_level = std::queue<std::pair<int, int>>();
        preparingForMerges.push({ 0, {0, 0} });
      } else {
        break;
      }
    }
  }
  preparingForMerges.pop();

  int n = static_cast<int>(piecesOfVector.size());
  tbb::task_scheduler_init init(threads);
  tbb::parallel_for(tbb::blocked_range<int>(0, n, 1),
    [&](const tbb::blocked_range<int>& range) {
      int begin = range.begin();
      int end = range.end();
      for (int i = begin; i < end; i++) {
          radixSort(vec + piecesOfVector[i].first,
              temp_vec + piecesOfVector[i].first, piecesOfVector[i].second);
      }
    });
  init.terminate();

  std::vector<std::pair<int, std::pair<int, int>>> merges_order;

  while (!preparingForMerges.empty()) {
    while (preparingForMerges.top().second.second != 0) {
      merges_order.push_back(preparingForMerges.top());
      preparingForMerges.pop();
    }
    preparingForMerges.pop();

    n = static_cast<int>(merges_order.size() * 2);
    tbb::parallel_for(tbb::blocked_range<int>(0, n, 1),
      [&](const tbb::blocked_range<int>& range) {
        int begin = range.begin();
        int end = range.end();
        for (int i = begin; i < end; i++) {
          int start = merges_order[i / 2].first;
          int center = merges_order[i / 2].second.first;
          int size_after_center = merges_order[i / 2].second.second;
          merge(vec + start, temp_vec + start,
            center, size_after_center, i % 2);
        }
      });

    n = static_cast<int>(merges_order.size());
    tbb::parallel_for(tbb::blocked_range<int>(0, n, 1),
      [&](const tbb::blocked_range<int>& range) {
        int begin = range.begin();
        int end = range.end();
        for (int i = begin; i < end; i++) {
          int start = merges_order[i].first;
          int center = merges_order[i].second.first;
          int size_after_center = merges_order[i].second.second;
          for (int j = 1; j < (center + size_after_center + 1) / 2; j++) {
            if (vec[start + 2 * j - 1] > vec[start + 2 * j]) {
              double tmp = vec[start + 2 * j];
              vec[start + 2 * j] = vec[start + 2 * j - 1];
              vec[start + 2 * j - 1] = tmp;
            }
          }
        }
      });
    merges_order.clear();
  }
  delete[] temp_vec;
}
\end{lstlisting}

\par 9. Библиотека с реализацией на TBB. Файл: main.cpp
\begin{lstlisting}
// Copyright 2021 Zaitsev Andrey
#include <gtest/gtest.h>
#include <tbb/tick_count.h>
#include <iostream>
#include <vector>
#include "./RadixSortTbb.h"

TEST(Test_Tbb, Size_5_Without_Random) {
  int size = 5;
  double* vec = new double[size] {5.21, -8.652,
  26.32, 3.289, -14.842};
  double* tmp = new double[size];
  bool result = false;
  radixSort(vec, tmp, size);
  result = check(vec, size);
  ASSERT_EQ(true, result);
  delete[] tmp;
  delete[] vec;
}

TEST(Test_Tbb, Size_1000_Random_Pos_And_Neg) {
  int size = 1000;
  std::vector<double> vec = createVector(size, -1000.0, 1000.0);
  double* tmp = new double[size];
  bool result = false;
  radixSort(vec.data(), tmp, size);
  result = check(vec.data(), size);
  ASSERT_EQ(true, result);
  delete[] tmp;
}

TEST(Test_Tbb, Size_1000_Batcher_Seq_Border_20) {
  int size = 1000;
  std::vector<double> vec = createVector(size, -1000.0, 1000.0);
  bool result = false;
  oddEvenMergeSortSeq(vec.data(), size, 20);
  result = check(vec.data(), size);
  ASSERT_EQ(true, result);
}

TEST(Test_Tbb, Size_100_Batcher_Tbb_1thr_Border_25) {
  int size = 100;
  int thr = 1;
  std::vector<double> vec = createVector(size, -100.0, 100.0);
  bool result = false;
  oddEvenMergeSortTbb(vec.data(), size, 25, thr);
  result = check(vec.data(), size);
  ASSERT_EQ(true, result);
}

TEST(Test_Tbb, Size_100_Batcher_Tbb_4thr_Border_25) {
  int size = 100;
  int thr = 4;
  std::vector<double> vec = createVector(size, -1000.0, 1000.0);
  bool result = false;
  oddEvenMergeSortTbb(vec.data(), size, 25, thr);
  result = check(vec.data(), size);
  ASSERT_EQ(true, result);
}

TEST(Test_Tbb, Size_1000_Batcher_Tbb_4thr_Border_60) {
  int size = 1000;
  int thr = 4;
  std::vector<double> vec = createVector(size, -1000.0, 1000.0);
  bool result = false;
  oddEvenMergeSortTbb(vec.data(), size, 60, thr);
  result = check(vec.data(), size);
  ASSERT_EQ(true, result);
}

TEST(Test_Tbb, Test_Equality_Seq_Par_10000) {
  int size = 10000;
  int thr = 6;
  std::vector<double> v1 = createVector(size, -1000.0, 1000.0);
  std::vector<double> v2 = v1;
  tbb::tick_count t1 = tbb::tick_count::now();
  oddEvenMergeSortSeq(v1.data(), size, 100);
  tbb::tick_count t2 = tbb::tick_count::now();
  std::cout << "Seq: " <<
    static_cast<double>((t2 - t1).seconds()) << std::endl;
  t1 = tbb::tick_count::now();
  oddEvenMergeSortTbb(v2.data(), size, 100, thr);
  t2 = tbb::tick_count::now();
  std::cout << "Tbb: " <<
    static_cast<double>((t2 - t1).seconds()) << std::endl;
  ASSERT_EQ(v1, v2);
}

TEST(Test_Tbb, Test_Equality_Seq_Par_20000) {
  int size = 20000;
  int thr = 6;
  std::vector<double> v1 = createVector(size, -1000.0, 1000.0);
  std::vector<double> v2 = v1;
  tbb::tick_count t1 = tbb::tick_count::now();
  oddEvenMergeSortSeq(v1.data(), size, 100);
  tbb::tick_count t2 = tbb::tick_count::now();
  std::cout << "Seq: " <<
    static_cast<double>((t2 - t1).seconds()) << std::endl;
  t1 = tbb::tick_count::now();
  oddEvenMergeSortTbb(v2.data(), size, 100, thr);
  t2 = tbb::tick_count::now();
  std::cout << "Tbb: " <<
    static_cast<double>((t2 - t1).seconds()) << std::endl;
  ASSERT_EQ(v1, v2);
}

TEST(Test_Tbb, Test_Equality_Seq_Par_30000) {
  int size = 30000;
  int thr = 6;
  std::vector<double> v1 = createVector(size, -1000.0, 1000.0);
  std::vector<double> v2 = v1;
  tbb::tick_count t1 = tbb::tick_count::now();
  oddEvenMergeSortSeq(v1.data(), size, 100);
  tbb::tick_count t2 = tbb::tick_count::now();
  std::cout << "Seq: " <<
    static_cast<double>((t2 - t1).seconds()) << std::endl;
  t1 = tbb::tick_count::now();
  oddEvenMergeSortTbb(v2.data(), size, 100, thr);
  t2 = tbb::tick_count::now();
  std::cout << "Tbb: " <<
    static_cast<double>((t2 - t1).seconds()) << std::endl;
  ASSERT_EQ(v1, v2);
}

TEST(Test_Tbb, Test_Equality_Seq_Par_50000) {
  int size = 50000;
  int thr = 6;
  std::vector<double> v1 = createVector(size, -1000.0, 1000.0);
  std::vector<double> v2 = v1;
  tbb::tick_count t1 = tbb::tick_count::now();
  oddEvenMergeSortSeq(v1.data(), size, 100);
  tbb::tick_count t2 = tbb::tick_count::now();
  std::cout << "Seq: " <<
    static_cast<double>((t2 - t1).seconds()) << std::endl;
  t1 = tbb::tick_count::now();
  oddEvenMergeSortTbb(v2.data(), size, 100, thr);
  t2 = tbb::tick_count::now();
  std::cout << "Tbb: " <<
    static_cast<double>((t2 - t1).seconds()) << std::endl;
  ASSERT_EQ(v1, v2);
}

TEST(Test_Tbb, Test_Equality_Seq_Par_1000000) {
  int size = 1000000;
  int thr = 6;
  std::vector<double> v1 = createVector(size, -1000.0, 1000.0);
  std::vector<double> v2 = v1;
  tbb::tick_count t1 = tbb::tick_count::now();
  oddEvenMergeSortSeq(v1.data(), size, 1000);
  tbb::tick_count t2 = tbb::tick_count::now();
  std::cout << "Seq: " <<
    static_cast<double>((t2 - t1).seconds()) << std::endl;
  t1 = tbb::tick_count::now();
  oddEvenMergeSortTbb(v2.data(), size, 1000, thr);
  t2 = tbb::tick_count::now();
  std::cout << "Tbb: " <<
    static_cast<double>((t2 - t1).seconds()) << std::endl;
  ASSERT_EQ(v1, v2);
}

int main(int argc, char** argv) {
  ::testing::InitGoogleTest(&argc, argv);
  return RUN_ALL_TESTS();
}
\end{lstlisting}

\end{document}
