\documentclass{report}

\usepackage[T2A]{fontenc}
\usepackage[utf8]{luainputenc}
\usepackage[english, russian]{babel}
\usepackage[pdftex]{hyperref}
\usepackage[14pt]{extsizes}
\usepackage{listings}
\usepackage{color}
\usepackage{geometry}
\usepackage{enumitem}
\usepackage{multirow}
\usepackage{graphicx}
\usepackage{indentfirst}

\geometry{a4paper,top=2cm,bottom=3cm,left=2cm,right=1.5cm}
\setlength{\parskip}{0.5cm}
\setlist{nolistsep, itemsep=0.3cm,parsep=0pt}

\lstset{language=C++,
		basicstyle=\footnotesize,
		keywordstyle=\color{blue}\ttfamily,
		stringstyle=\color{red}\ttfamily,
		commentstyle=\color{green}\ttfamily,
		morecomment=[l][\color{magenta}]{\#}, 
		tabsize=4,
		breaklines=true,
  		breakatwhitespace=true,
  		title=\lstname,       
}

\makeatletter
\renewcommand\@biblabel[1]{#1.\hfil}
\makeatother

\begin{document}

\begin{titlepage}

\begin{center}
Министерство науки и высшего образования Российской Федерации
\end{center}

\begin{center}
Федеральное государственное автономное образовательное учреждение высшего образования \\
Национальный исследовательский Нижегородский государственный университет им. Н.И. Лобачевского
\end{center}

\begin{center}
Институт информационных технологий, математики и механики
\end{center}

\vspace{4em}

\begin{center}
\textbf{\LargeОтчет по лабораторной работе} \\
\end{center}
\begin{center}
\textbf{\Large«Линейная фильтрация изображений (горизонтальное разбиение). Ядро Гаусса 3x3.»} \\
\end{center}

\vspace{4em}

\newbox{\lbox}
\savebox{\lbox}{\hbox{text}}
\newlength{\maxl}
\setlength{\maxl}{\wd\lbox}
\hfill\parbox{7cm}{
\hspace*{5cm}\hspace*{-5cm}\textbf{Выполнил:} \\ студент группы 381808-2 \\ Евсеев А. Д.\\
\\
\hspace*{5cm}\hspace*{-5cm}\textbf{Проверил:}\\ доцент кафедры МОСТ, \\ кандидат технических наук \\ Сысоев А. В.\\
}
\vspace{\fill}

\begin{center} Нижний Новгород \\ 2021 \end{center}

\end{titlepage}

\setcounter{page}{2}

% Содержание
\tableofcontents
\newpage

% Введение
\section*{Введение}
\addcontentsline{toc}{section}{Введение}
\par Фильтр Гаусса - это метод фильтрации изображений, который использует нормальное распределение (также называемое Гауссовым распределением) для вычисления преобразования, применяемого к каждому пикселю изображения. Данный фильтр используется для создания эффекта размытия изображения, а также для того, чтобы уменьшить шумы на изображении.
\par Тот факт, что данный фильтр применяется к каждому пикселю в отдельности независимо от остальных, открывает возможности для существенного ускорения его работы путем распараллеливания вычислений между разными ядрами процессора.
\par В данной работе будет рассмотрена фильтрация изображений "фильтром Гаусса" с ядром Гаусса 3х3 и реализация последовательной и нескольких параллельных версий алгоритма с использованием разных технологий выполнения параллельных вычислений.
\newpage

% Постановка задачи
\section*{Постановка задачи}
\addcontentsline{toc}{section}{Постановка задачи}
\par В ходе данной лабораторной работы требуется:
    \begin{itemize}
        \item изучить фильтрацию методом Гаусса;
        \item написать последовательную реализацию алгоритма, выполняющего фильтрацию;
        \item написать параллельные реализации алгоритма, выполняющего фильтрацию, с помощью OpenMP, TBB;
        \item выполнить сравнение и сделать выводы об эффективности написанных реализаций;
        \item проверить корректность работы алгоритма.
    \end{itemize}
\newpage

% Описание алгоритма
\section*{Описание алгоритма}
\addcontentsline{toc}{section}{Описание алгоритма}
    \par Фильтр Гаусса - это широко используемый в обработке изображений фильтр. Главным образом он используется для устранения шумов на изображении и сделать эффект размытия. Гауссово размывание применяет функцию Гаусса (\ref{eq:normal_dist}), которая также выражает нормальное распределение в математической статистике, к каждому пикселю в изображении.
    \begin{equation}
        \label{eq:normal_dist}
        \frac{1}{\sigma\sqrt{2\pi}}\exp\left(-\frac{x^2 + y^2}{2\sigma^2}\right)
    \end{equation}
    \par Генерируется ядро размером 3x3 с использованием вышеприведенной формулы таким образом, чтобы сумма всех чисел в ядре была равна единице для того, чтобы не делать изображение ярче или темнее. Далее требуется применить ядро к каждому пикселю изображения.
    \par Новый цвет для пикселя с индексами [x][y] будет равен сумме произведений значения цветовой компоненты каждого из пикселей на соответствующее значение из ядра ($pixel[x][y] * kernel[x][y]$). Полученное значение нормализуется так, чтобы оно оказалось в диапазоне [0, 255].
\newpage

% Описание схемы распараллеливания
\section*{Схема распараллеливания}
\addcontentsline{toc}{section}{Схема распараллеливания}
\par Для эффективного распределения задач по потокам использовалась схема горизонтального разбиения. Она выглядит следующим образом: изображение разбивается на N горизонтальных полосок, где N - число потоков, на которые мы хотим распределить вычисления, с приблизительно равной высотой. Далее в каждом потоке отдельно обрабатывается каждая из N частей изображения.
\newpage

% Описание программной реализации
\section*{Описание программной реализации}
\addcontentsline{toc}{section}{Описание программной реализации}
Создание матрицы заданного размера, заполненной значениями от 0 до 255:
\begin{lstlisting}
Matrix RandMatrix(int height, int width);
\end{lstlisting}
\par В качестве входных параметров передаются:
\begin{itemize}
\item количество строк.
\item количество столбцов.
\end{itemize}
\par Функция, которая формирует ядро Гаусса.
\begin{lstlisting}
Matrix GaussKernel(int R, double sigma)
\end{lstlisting}
\par В качестве входных данных передаются радиус и сигма.
\par Алгоритм параллельной линейной фильтрации изображения фильтром Гаусса вызывается с помощью функции:
\begin{lstlisting}
Matrix OmpGauss(const Matrix& mat, int height, int width, int R, double sigma)
\end{lstlisting}
\par В качестве входных параметров передаются те же параметры, что и в последовательной схеме.
\par Последовательная схема:
\begin{lstlisting}
Matrix SeqGauss(const Matrix& mat, int height, int width, int R, double sigma)
\end{lstlisting}
\par В качестве входных параметров передаются:
\begin{itemize}
\item константная ссылка на матрицу, в которой находятся значения пикселей исходного изображения.
\item число строк.
\item число столбцов.
\item радиус.
\item сигма.
\end{itemize}
\newpage

% Подтверждение корректности
\section*{Подтверждение корректности}
\addcontentsline{toc}{section}{Подтверждение корректности}
\par Для подтверждения корректности в программе содержится набор тестов(5 штук), разработанных с помощью использования Google C++ Testing Framework.
\par Данные тесты проверяют сценарии исключений при неккоректном задании матрицы(с отрицательным количеством столбцов), корректность вычислений(то есть сравнение результата последовательного метода и результата параллельного метода, а также сравнение с заранее посчитанной матрицей вручную) и также эффективность выполнения последовательного и параллельного метода.
\par Успешное прохождение всех тестов доказывает корректность работы всей программы.
\newpage

% Результаты экспериментов
\section*{Результаты экспериментов}
\addcontentsline{toc}{section}{Результаты экспериментов}
Вычислительные эксперименты для оценки эффективности параллельной и последовательной линейной фильтрации изображения фильтром Гаусса производились на оборудовании со следующей аппаратной конфигурацией:

\begin{itemize}
\item Процессор:Intel(R) Core(TM) i5-8300H CPU, 4000 GHz, ядер: 4;
\item Оперативная память: 2x4,0 GB(DDR4, 2666 MHz);
\item ОС: Microsoft Windows 10 Professional.
\end{itemize}

\par Для сравнения скорости выполнения алгоритма была выбрана матрица размером 1500*1000 . 
\par Результаты экспериментов представлены в следующей таблице.

 \begin{table}[htbp]
        \centering
        \begin{tabular}{|c|c|c|c|c|c|c|}
            \hline
                             & 1       & 2       & 3       & 4        & 6        & 8        \\ \hline
            Последовательная & \multicolumn{6}{c|}{3.00508}                                 \\ \hline
            OpenMP           & 2.98192 & 1.54635 & 1.00003 & 0.757113 & 0.598993 & 0.470119 \\ \hline
            TBB              & 2.97817 & 1.51188 & 1.0098  & 0.763663 & 0.576086 & 0.47195  \\ \hline
        \end{tabular}
        \caption{Результаты экспериментов.}
        \label{tab:results}
        По вертикали - число потоков. \\
        По горизонтали - тестируемая реализация. \\
        Время указано в секундах.
    \end{table}

\par Параллельная реализация окаазлась быстрее, по сравнению с последовательной реализацией, особенно, это становится виднее с ростом данных. 
\newpage

% Заключение
\section*{Заключение}
\addcontentsline{toc}{section}{Заключение}
В результате лабораторной работы были разработаны последовательная и несколько параллельных реализаций линейной фильтрации изображения фильтром Гаусса с ядром Гаусса 3х3.
\par Основной целью данной лабораторной работы была реализация эффективной параллельной версии. Эта задача была успешно достигнута, что подтверждается результатами экспериментов, проведенных в ходе работы.
\par Кроме того, были разработаныя тесты, созданные для данного программного проекта с использованием Google C++ Testing Framework и необходимые для подтверждения корректности работы программы и для демонстрации функционала.
\newpage

% Список литературы
\section*{Список литературы}
    \addcontentsline{toc}{section}{Список литературы}
    \begin{enumerate}
        \item Сысоев А.В., Мееров И.Б., Сиднев А.А. Средства разработки параллельных программ для систем с общей памятью. Библиотека Intel Threading Building Blocks. Учебно-методические материалы по программе повышения квалификации «Технологии высокопроизводительных вычислений для обеспечения учебного процесса и научных исследований». Нижний Новгород, 2007, 86 с.
        \item Guide into OpenMP: Easy multithreading programming for C++. URL: \url{https://bisqwit.iki.fi/story/howto/openmp/}
        \item Intel Threading Building Blocks User Guide. URL: \url{https://www.threadingbuildingblocks.org/docs/help/tbb_userguide/parallel_for.html}
        \item Турлапов В.Е., Гетманская А.А., Васильев Е.П., Дубровская М.В. Методические указания для проведения лабораторных работ по курсу "КОМПЬЮТЕРНАЯ ГРАФИКА": Учебное пособие. – Нижний Новгород, Национальный Исследовательский Нижегородский государственный университет им. Н.И. Лобачевского, 2018 – 96 с.
    \end{enumerate}
\newpage

% Приложение
\section*{Приложение}
\addcontentsline{toc}{section}{Приложение}
Реализация OpenMP.
\begin{lstlisting}
// evseev_a_gauss_omp.h

// Copyright 2021 Evseev Alexander
#ifndef MODULES_TASK_2_EVSEEV_A_GAUSS_OMP_EVSEEV_A_GAUSS_OMP_H_
#define MODULES_TASK_2_EVSEEV_A_GAUSS_OMP_EVSEEV_A_GAUSS_OMP_H_

#include <algorithm>
#include <ctime>
#include <iostream>
#include <random>
#include <vector>

using Matrix = std::vector<std::vector<double>>;
Matrix RandMatrix(int height, int width);
int clamp(int value, int max, int min);
Matrix GaussKernel(int R, double sigma);
Matrix SeqGauss(const Matrix& mat, int height, int width, int R, double sigma);
Matrix OmpGauss(const Matrix& mat, int height, int width, int R, double sigma);
void printMatrix(Matrix mat, int height, int width);

#endif  // MODULES_TASK_2_EVSEEV_A_GAUSS_OMP_EVSEEV_A_GAUSS_OMP_H_


\end{lstlisting}
\begin{lstlisting}
// evseev_a_gauss_omp.cpp

// Copyright 2021 Evseev Alexander
#include <algorithm>
#include <cmath>
#include <ctime>
#include <random>
#include <stdexcept>
#include <vector>

#include "../../../modules/task_2/evseev_a_gauss_omp/evseev_a_gauss_omp.h"

int clamp(int value, int max, int min) {
  if (value < min) {
    return min;
  }
  if (value > max) {
    return max;
  }
  return value;
}
Matrix RandMatrix(int height, int width) {
  if ((height <= 0) || (width <= 0)) throw std::invalid_argument("Invalid size");
  std::mt19937 gen;
  gen.seed(static_cast<int>(time(0)));
  Matrix matrix(height, std::vector<double>(width));
  for (int i = 0; i < height; i++) {
    for (int j = 0; j < width; j++) {
      matrix[i][j] = gen() % 100;
    }
  }
  return matrix;
}

Matrix GaussKernel(int R, double sigma) {
  const int size = 2 * R + 1;
  double norm = 0;
  Matrix kernel(size, std::vector<double>(size));
  for (int64_t i = -R; i <= R; i++) {
    for (int64_t j = -R; j <= R; j++) {
      kernel[i + R][j + R] =
          static_cast<double>(exp(-(i * i + j * j) / (sigma * sigma)));
      norm = norm + kernel[i + R][j + R];
    }
  }
  for (int64_t i = 0; i < size; i++) {
    for (int64_t j = 0; j < size; j++) {
      kernel[i][j] = kernel[i][j] / norm;
    }
  }
  return kernel;
}

Matrix SeqGauss(const Matrix& mat, int height, int width, int R, double sigma) {
  Matrix finMat(height, std::vector<double>(width));
  Matrix kernel = GaussKernel(R, sigma);
  for (int64_t x = 0; x < height; x++) {
    for (int64_t y = 0; y < width; y++) {
      int finValue = 0;
      for (int64_t i = -R; i <= R; i++) {
        for (int64_t j = -R; j <= R; j++) {
          double value = mat[x][y];
          finValue += value * kernel[i + R][j + R];
        }
      }
      finValue = clamp(finValue, 255, 0);
      finMat[x][y] = finValue;
    }
  }
  return finMat;
}

Matrix OmpGauss(const Matrix& mat, int height, int width, int R, double sigma) {
  Matrix finMat(height, std::vector<double>(width));
  Matrix kernel = GaussKernel(R, sigma);
#pragma omp parallel
  {
#pragma omp for collapse(2) schedule(static)
    for (int64_t x = 0; x < height; x++) {
      for (int64_t y = 0; y < width; y++) {
        int finValue = 0;
        for (int64_t i = -R; i <= R; i++) {
          for (int64_t j = -R; j <= R; j++) {
            double value = mat[x][y];
            finValue += value * kernel[i + R][j + R];
          }
        }
        finValue = clamp(finValue, 255, 0);
        finMat[x][y] = finValue;
      }
    }
  }
  return finMat;
}



\end{lstlisting}
\begin{lstlisting}
// Copyright 2021 Evseev Alexander
#include <gtest/gtest.h>
#include <omp.h>
#include <vector>
#include<iostream>
#include "./evseev_a_gauss_omp.h"

TEST(GaussFilter_Test, invalid_matrix) {
  int hight = 0;
  int widht = -1;
  ASSERT_ANY_THROW(RandMatrix(hight, widht));
}
TEST(OMP_GaussFilter_Test, Test_1500x1000x60thr) {
  int rows = 1500;
  int cols = 1000;
  double sigma = 7;
  int radius = 2;
  Matrix SEQ(rows, std::vector<double>(cols));
  Matrix OMP(rows, std::vector<double>(cols));
  Matrix mat(rows, std::vector<double>(cols));
  Matrix kernel = GaussKernel(radius, sigma);
  mat = RandMatrix(rows, cols);
  double start = omp_get_wtime();
  SEQ = SeqGauss(mat, rows, cols, radius, sigma);
  double end = omp_get_wtime();
  std::cout << "Sequential time: " << end - start << std::endl;
  start = omp_get_wtime();
  omp_set_num_threads(60);
  OMP = OmpGauss(mat, rows, cols, radius, sigma);
  end = omp_get_wtime();
  std::cout << "Parallel time: " << end - start << std::endl;
  ASSERT_EQ(SEQ, OMP);
}

TEST(OMP_GaussFilter_Test, Test_400x400_20thr) {
  int rows = 400;
  int cols = 400;
  double sigma = 7;
  int radius = 2;
  Matrix SEQ(rows, std::vector<double>(cols));
  Matrix OMP(rows, std::vector<double>(cols));
  Matrix mat(rows, std::vector<double>(cols));
  Matrix kernel = GaussKernel(radius, sigma);
  mat = RandMatrix(rows, cols);
  double start = omp_get_wtime();
  SEQ = SeqGauss(mat, rows, cols, radius, sigma);
  double end = omp_get_wtime();
  std::cout << "Sequential time: " << end - start << std::endl;
  start = omp_get_wtime();
  omp_set_num_threads(20);
  OMP = OmpGauss(mat, rows, cols, radius, sigma);
  end = omp_get_wtime();
  std::cout << "Parallel time: " << end - start << std::endl;
  ASSERT_EQ(SEQ, OMP);
}

TEST(GaussFilter_Test, Test_4x3) {
  int height = 4;
  int width = 3;
  double sigma = 1;
  int radius = 1;
  Matrix mat(height, std::vector<double>(width));
  Matrix gaussSeq(height, std::vector<double>(width));
  Matrix gaussOmp(height, std::vector<double>(width));
  Matrix resMat(height, std::vector<double>(width));
  Matrix KernelMat = GaussKernel(radius, sigma);
  mat[0][0] = 105;
  mat[0][1] = 177;
  mat[0][2] = 111;
  mat[1][0] = 200;
  mat[1][1] = 28;
  mat[1][2] = 151;
  mat[2][0] = 230;
  mat[2][1] = 44;
  mat[2][2] = 88;
  mat[3][0] = 112;
  mat[3][1] = 128;
  mat[3][2] = 155;

  resMat[0][0] = 98;
  resMat[0][1] = 170;
  resMat[0][2] = 104;
  resMat[1][0] = 194;
  resMat[1][1] = 25;
  resMat[1][2] = 146;
  resMat[2][0] = 228;
  resMat[2][1] = 38;
  resMat[2][2] = 81;
  resMat[3][0] = 109;
  resMat[3][1] = 122;
  resMat[3][2] = 147;

  gaussSeq = SeqGauss(mat, height, width, radius, sigma);
  gaussOmp = OmpGauss(mat, height, width, radius, sigma);
  ASSERT_EQ(resMat, gaussSeq);
  ASSERT_EQ(resMat, gaussOmp);
}

TEST(GaussFilter_Test, Test_8x1) {
  int height = 8;
  int width = 1;
  double sigma = 1;
  int radius = 1;
  Matrix mat(height, std::vector<double>(width));
  Matrix gaussSeq(height, std::vector<double>(width));
  Matrix gaussOmp(height, std::vector<double>(width));
  Matrix resMat(height, std::vector<double>(width));
  Matrix KernelMat = GaussKernel(radius, sigma);
  mat[0][0] = 21;
  mat[1][0] = 67;
  mat[2][0] = 2;
  mat[3][0] = 2;
  mat[4][0] = 77;
  mat[5][0] = 55;
  mat[6][0] = 12;
  mat[7][0] = 56;

  resMat[0][0] = 14;
  resMat[1][0] = 66;
  resMat[2][0] = 0;
  resMat[3][0] = 0;
  resMat[4][0] = 73;
  resMat[5][0] = 50;
  resMat[6][0] = 7;
  resMat[7][0] = 50;

  gaussSeq = SeqGauss(mat, height, width, radius, sigma);
  gaussOmp = OmpGauss(mat, height, width, radius, sigma);
  ASSERT_EQ(resMat, gaussSeq);
  ASSERT_EQ(resMat, gaussOmp);
}


\end{lstlisting}
Реализация TBB
\begin{lstlisting}
// Copyright 2021 Evseev Alexander
#ifndef MODULES_TASK_3_EVSEEV_A_GAUSS_TBB_EVSEEV_A_GAUSS_TBB_H_
#define MODULES_TASK_3_EVSEEV_A_GAUSS_TBB_EVSEEV_A_GAUSS_TBB_H_


#include <tbb/parallel_for.h>
#include <tbb/tick_count.h>
#include <algorithm>
#include <ctime>
#include <iostream>
#include <random>
#include <vector>

using Matrix = std::vector<std::vector<double>>;
Matrix RandMatrix(int height, int width);
int clamp(int value, int max, int min);
Matrix GaussKernel(int R, double sigma);
Matrix SeqGauss(const Matrix& mat, int height, int width, int R, double sigma);
Matrix TbbGauss(const Matrix& mat, int height, int width, int R, double sigma);
void printMatrix(Matrix mat, int height, int width);

#endif  // MODULES_TASK_3_EVSEEV_A_GAUSS_TBB_EVSEEV_A_GAUSS_TBB_H_

\end{lstlisting}

\begin{lstlisting}
// Copyright 2021 Evseev Alexander
#include <tbb/parallel_for.h>
#include <algorithm>
#include <cmath>
#include <ctime>
#include <random>
#include <stdexcept>
#include <vector>

#include "../../../modules/task_3/evseev_a_gauss_tbb/evseev_a_gauss_tbb.h"

int clamp(int value, int max, int min) {
  if (value < min) {
    return min;
  }
  if (value > max) {
    return max;
  }
  return value;
}
Matrix RandMatrix(int height, int width) {
  if ((height <= 0) || (width <= 0))
    throw std::invalid_argument("Invalid size");
  std::mt19937 gen;
  gen.seed(static_cast<int>(time(0)));
  Matrix matrix(height, std::vector<double>(width));
  for (int i = 0; i < height; i++) {
    for (int j = 0; j < width; j++) {
      matrix[i][j] = gen() % 100;
    }
  }
  return matrix;
}

Matrix GaussKernel(int R, double sigma) {
  const int size = 2 * R + 1;
  double norm = 0;
  Matrix kernel(size, std::vector<double>(size));
  for (int64_t i = -R; i <= R; i++) {
    for (int64_t j = -R; j <= R; j++) {
      kernel[i + R][j + R] =
          static_cast<double>(exp(-(i * i + j * j) / (sigma * sigma)));
      norm = norm + kernel[i + R][j + R];
    }
  }
  for (int64_t i = 0; i < size; i++) {
    for (int64_t j = 0; j < size; j++) {
      kernel[i][j] = kernel[i][j] / norm;
    }
  }
  return kernel;
}

Matrix SeqGauss(const Matrix& mat, int height, int width, int R, double sigma) {
  Matrix finMat(height, std::vector<double>(width));
  Matrix kernel = GaussKernel(R, sigma);
  for (int64_t x = 0; x < height; x++) {
    for (int64_t y = 0; y < width; y++) {
      int finValue = 0;
      for (int64_t i = -R; i <= R; i++) {
        for (int64_t j = -R; j <= R; j++) {
          double value = mat[x][y];
          finValue += value * kernel[i + R][j + R];
        }
      }
      finValue = clamp(finValue, 255, 0);
      finMat[x][y] = finValue;
    }
  }
  return finMat;
}

Matrix TbbGauss(const Matrix& mat, int height, int width, int R, double sigma) {
  Matrix finMat(height, std::vector<double>(width));
  Matrix kernel = GaussKernel(R, sigma);

  tbb::parallel_for(0, height, [&](int x) {
    tbb::parallel_for(0, width, [&](int y) {
      int finValue = 0;
      for (int64_t i = -R; i <= R; i++) {
        for (int64_t j = -R; j <= R; j++) {
          double value = mat[x][y];
          finValue += value * kernel[i + R][j + R];
        }
      }
      finValue = clamp(finValue, 255, 0);
      finMat[x][y] = finValue;
    });
  });
  return finMat;
}



\end{lstlisting}

\begin{lstlisting}
// Copyright 2021 Evseev Alexander
#include <gtest/gtest.h>
#include <tbb/tick_count.h>
#include <iostream>
#include <vector>
#include "./evseev_a_gauss_tbb.h"

TEST(GaussFilter_Test, invalid_matrix) {
  int hight = 0;
  int widht = -1;
  ASSERT_ANY_THROW(RandMatrix(hight, widht));
}
TEST(TBB_GaussFilter_Test, Test_1500x1000) {
  int rows = 1500;
  int cols = 1000;
  double sigma = 7;
  int radius = 2;
  Matrix SEQ(rows, std::vector<double>(cols));
  Matrix TBB(rows, std::vector<double>(cols));
  Matrix mat(rows, std::vector<double>(cols));
  Matrix kernel = GaussKernel(radius, sigma);
  mat = RandMatrix(rows, cols);
  tbb::tick_count start = tbb::tick_count::now();
  SEQ = SeqGauss(mat, rows, cols, radius, sigma);
  tbb::tick_count end = tbb::tick_count::now();
  std::cout << "Sequential time: " << (end - start).seconds() << std::endl;
  start = tbb::tick_count::now();
  TBB = TbbGauss(mat, rows, cols, radius, sigma);
  end = tbb::tick_count::now();
  std::cout << "Parallel time: " << (end - start).seconds() << std::endl;
  ASSERT_EQ(SEQ, TBB);
}

TEST(TBB_GaussFilter_Test, Test_400x400) {
  int rows = 400;
  int cols = 400;
  double sigma = 7;
  int radius = 2;
  Matrix SEQ(rows, std::vector<double>(cols));
  Matrix TBB(rows, std::vector<double>(cols));
  Matrix mat(rows, std::vector<double>(cols));
  Matrix kernel = GaussKernel(radius, sigma);
  mat = RandMatrix(rows, cols);
  tbb::tick_count start = tbb::tick_count::now();
  SEQ = SeqGauss(mat, rows, cols, radius, sigma);
  tbb::tick_count end = tbb::tick_count::now();
  std::cout << "Sequential time: " << (end - start).seconds() << std::endl;
  start = tbb::tick_count::now();
  TBB = TbbGauss(mat, rows, cols, radius, sigma);
  end = tbb::tick_count::now();
  std::cout << "Parallel time: " << (end - start).seconds() << std::endl;
  ASSERT_EQ(SEQ, TBB);
}

TEST(GaussFilter_Test, Test_4x3) {
  int height = 4;
  int width = 3;
  double sigma = 1;
  int radius = 1;
  Matrix mat(height, std::vector<double>(width));
  Matrix gaussSeq(height, std::vector<double>(width));
  Matrix gaussTbb(height, std::vector<double>(width));
  Matrix resMat(height, std::vector<double>(width));
  Matrix KernelMat = GaussKernel(radius, sigma);
  mat[0][0] = 105;
  mat[0][1] = 177;
  mat[0][2] = 111;
  mat[1][0] = 200;
  mat[1][1] = 28;
  mat[1][2] = 151;
  mat[2][0] = 230;
  mat[2][1] = 44;
  mat[2][2] = 88;
  mat[3][0] = 112;
  mat[3][1] = 128;
  mat[3][2] = 155;

  resMat[0][0] = 98;
  resMat[0][1] = 170;
  resMat[0][2] = 104;
  resMat[1][0] = 194;
  resMat[1][1] = 25;
  resMat[1][2] = 146;
  resMat[2][0] = 228;
  resMat[2][1] = 38;
  resMat[2][2] = 81;
  resMat[3][0] = 109;
  resMat[3][1] = 122;
  resMat[3][2] = 147;

  gaussSeq = SeqGauss(mat, height, width, radius, sigma);
  gaussTbb = TbbGauss(mat, height, width, radius, sigma);
  ASSERT_EQ(resMat, gaussSeq);
  ASSERT_EQ(resMat, gaussTbb);
}

TEST(GaussFilter_Test, Test_8x1) {
  int height = 8;
  int width = 1;
  double sigma = 1;
  int radius = 1;
  Matrix mat(height, std::vector<double>(width));
  Matrix gaussSeq(height, std::vector<double>(width));
  Matrix gaussTbb(height, std::vector<double>(width));
  Matrix resMat(height, std::vector<double>(width));
  Matrix KernelMat = GaussKernel(radius, sigma);
  mat[0][0] = 21;
  mat[1][0] = 67;
  mat[2][0] = 2;
  mat[3][0] = 2;
  mat[4][0] = 77;
  mat[5][0] = 55;
  mat[6][0] = 12;
  mat[7][0] = 56;

  resMat[0][0] = 14;
  resMat[1][0] = 66;
  resMat[2][0] = 0;
  resMat[3][0] = 0;
  resMat[4][0] = 73;
  resMat[5][0] = 50;
  resMat[6][0] = 7;
  resMat[7][0] = 50;

  gaussSeq = SeqGauss(mat, height, width, radius, sigma);
  gaussTbb = TbbGauss(mat, height, width, radius, sigma);
  ASSERT_EQ(resMat, gaussSeq);
  ASSERT_EQ(resMat, gaussTbb);
}



\end{lstlisting}
\end{document}
