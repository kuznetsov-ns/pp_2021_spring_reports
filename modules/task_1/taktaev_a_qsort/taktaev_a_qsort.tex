\documentclass{report}

\usepackage[T2A]{fontenc}
\usepackage[utf8]{luainputenc}
\usepackage[english, russian]{babel}
\usepackage[pdftex]{hyperref}
\usepackage[14pt]{extsizes}
\usepackage{listings}
\usepackage{color}
\usepackage{geometry}
\usepackage{enumitem}
\usepackage{multirow}
\usepackage{graphicx}
\usepackage{indentfirst}
\usepackage{amsmath}


\geometry{a4paper,top=2cm,bottom=3cm,left=2cm,right=1.5cm}
\setlength{\parskip}{0.5cm}
\setlist{nolistsep, itemsep=0.3cm,parsep=0pt}

\lstset{language=C++,
		basicstyle=\footnotesize,
		keywordstyle=\color{blue}\ttfamily,
		stringstyle=\color{red}\ttfamily,
		commentstyle=\color{green}\ttfamily,
		morecomment=[l][\color{magenta}]{\#}, 
		tabsize=4,
		breaklines=true,
  		breakatwhitespace=true,
  		title=\lstname,       
}

\makeatletter
\renewcommand\@biblabel[1]{#1.\hfil}
\makeatother

\begin{document}

\begin{titlepage}

\begin{center}
Министерство науки и высшего образования Российской Федерации
\end{center}

\begin{center}
Федеральное государственное автономное образовательное учреждение высшего образования \\
Национальный исследовательский Нижегородский государственный университет им. Н.И. Лобачевского
\end{center}

\begin{center}
Институт информационных технологий, математики и механики
\end{center}

\vspace{4em}

\begin{center}
\textbf{\LargeОтчет по лабораторной работе} \\
\end{center}
\begin{center}
\textbf{\Large«Сортировка Хоара с простым слиянием»} \\
\end{center}

\vspace{4em}

\newbox{\lbox}
\savebox{\lbox}{\hbox{text}}
\newlength{\maxl}
\setlength{\maxl}{\wd\lbox}
\hfill\parbox{7cm}{
\hspace*{5cm}\hspace*{-5cm}\textbf{Выполнил:} \\ студент группы 381806-1 \\ Тактаев А. А.\\
\\
\hspace*{5cm}\hspace*{-5cm}\textbf{Проверил:}\\ доцент кафедры МОСТ, \\ кандидат технических наук \\ Сысоев А. В.\\
}
\vspace{\fill}

\begin{center} Нижний Новгород \\ 2021 \end{center}

\end{titlepage}

\setcounter{page}{2}

% Содержание
\tableofcontents
\newpage

% Введение
\section*{Введение}
\addcontentsline{toc}{section}{Введение}
Как правило, как людям, так и компьютерам удобнее работать с такими наборами или списками данных, которые упорядочены по определенному свойству. Одним из способов упорядочения множества данных является сортировка.
\par Но обычно сортировки представляют собой алгоритмы, требующие многократных обходов сортируемого диапазона. Отсюда следует, что чем больше количество объектов, которые необходимо отсортировать, тем больше времени требуется для сортировки. Уменьшить затраты времени на сортировку может помочь распараллеливание реализации алгоритма с помощью его запуска на нескольких процессорах вычислительного оборудования.
\par Таким образом, мы можем разделить исходный набор данных на части, для каждой вычислительной единицы поровну, отсортировать их параллельно, а затем объединить, чтобы получить набор с начальным размером.
\par Итак, в данной лабораторной работе будет реализован один из алгоритмов сортировок - сортировка Хоара, а также алгоритм простого слияния.

\newpage

% Постановка задачи
\section*{Постановка задачи}
\addcontentsline{toc}{section}{Постановка задачи}
В рамках лабораторной работы необходимо реализовать последовательную и параллельную реализации алгоритма сортировки Хоара с простым слиянием, проверить корректность работы алгоритмов, сравнить эффективность последовательной и параллельной реализаций.
\par Для реализации параллельной версии необходимо использовать библиотеки OpenMP и TBB. Для проверки корректности работы алгоритмов используется Google Testing Framework.
\newpage

% Метод решения
\section*{Описание алгоритмов}
\addcontentsline{toc}{section}{Метод решения}
Алгоритм Хоара позволяет отсортировать исходный массив данных, для которых определена операция сравнения. Для наглядного примера была произведена реализация на массиве чисел типа double по возрастанию значений.
\par Опишем шаги алгоритма:
\begin{enumerate}
\item В начале алгоритма выберем опорный элемент массива (в реализации --- элемент в середине массива).
\item Все элементы, меньшие опорного, переместим в массиве влево относительно него, а все остальные --- вправо.
\item Рекурсивно применим шаги 1 и 2 к левому и правому относительно опорного элемента подмассивам (включая сам элемент). Выход из рекурсии осуществляется за счет проверки размера подмассива, который должен быть строго больше 1.
\end{enumerate}
\par В зависимости от выбора опорного элемента и самого массива сложность алгоритма варьируется от $ O(n*log_2{n})$ до $O(n^2)$, где $n$ --- количество элементов в массиве.

\newpage

% Схема распараллеливания
\section*{Схема распараллеливания}
\addcontentsline{toc}{section}{Схема распараллеливания}
Для распараллеливания алгоритма сортировки Хоара необходимо разделить массив на равные по размеру фрагменты, количество которых равно количеству потоков. Если размер исходного массива не делится нацело на количество потоков, необходимо работу с остаточными элементами производить на последнем по счету потоке. Затем эти фрагменты сортируются параллельно на разных потоках. Далее выполняется простое слияние отсортированных фрагментов массива на одном потоке.

\newpage

% Описание программной реализации
\section*{Описание программной реализации}
\addcontentsline{toc}{section}{Описание программной реализации}
Алгоритм сортировки Хоара вызывается с помощью функции:
\begin{lstlisting}
void qSortSeq(std::vector<double> *arr, int left, int right);
\end{lstlisting}
\par Входными параметрами являются: исходный массив, а также левый и правый индекс сортируемой части этого массива. Выходные данные отсутствуют, массив сортируется на месте.
\par Реализация распараллеливания метода Симпсона представлена в функциях:
\begin{lstlisting}
void qSortOmp(std::vector<double> *arr);
\end{lstlisting}
\begin{lstlisting}
void qSortTbb(std::vector<double> *arr);
\end{lstlisting}
\par На вход они принимают только исходный массив.
\par Слияние осуществляется за счет функции:
\begin{lstlisting}
void merge(double* a, int size_a, double* b, int size_b);
\end{lstlisting}
\par На вход она принимает два массива и их размеры. Массивы "сливаются" на месте.
\newpage

% Подтверждение корректности
\section*{Подтверждение корректности}
\addcontentsline{toc}{section}{Подтверждение корректности}
Для подтверждения корректности в программе представлен набор тестов, разработанных с Google Testing Framework.
\par Набор представляет из себя тесты, которые проверяют корректность вычислений (сравниваются массивы, полученные из последовательного и параллельного алгоритма), а также эффективность (вычисление занимаемого последовательной и параллельной сортировок времени и сравнение полученных данных).
\par Успешное прохождение всех тестов подтверждает корректность работы написанной программы.
\newpage

% Результаты экспериментов
\section*{Результаты экспериментов}
\addcontentsline{toc}{section}{Результаты экспериментов}
Вычислительные эксперименты для оценки эффективности параллельного алгоритма сортировки Хоара проводились на оборудовании со следующей аппаратной конфигурацией:

\begin{itemize}
\item Процессор: Intel Core i5-9400f, 2.90 GHz, 6 ядер;
\item Оперативная память: 16 ГБ;
\item ОС: Microsoft Windows 10 Home 64-bit.
\end{itemize}

\par В рамках эксперимента было вычислено время работы последовательного и параллельного алгоритма сортировки Хоара.
\par В качестве тестируемых данных генерируется массив случайных чисел типа double от -1000 до 1000. Размер массива --- 1300000 элементов.
\par Результаты экспериментов представлены ниже.

\begin{table}[!h]
\caption{Результаты экспериментов версии с OpenMP}
\centering
\begin{tabular}{lllll}
Число потоков & Последовательно (OpenMP) & Параллельно & Ускорение  \\
4        & 0.706244         & 0.181495     & 3.89126      \\
6        & 0.702349         & 0.153063     & 4.58864       \\
10       & 0.726828        & 0.169834     & 4.27964      
\end{tabular}
\end{table}

\begin{table}[!h]
\caption{Результаты экспериментов версии с TBB}
\centering
\begin{tabular}{lllll}
Число потоков & Последовательно (TBB) & Параллельно & Ускорение  \\
4        & 0.707749         & 0.213238     & 3.31905       \\
6        & 0.720922        & 0.20903     & 3.44889       \\
10       & 0.707474         & 0.210033     & 3.36839       
\end{tabular}
\end{table}

\par По данным экспериментов видно, что параллельный алгоритм сортировки Хоара работает значительно быстрее, но ускорение не идеально за счет слияния упорядоченных подмассивов на одном потоке.
\newpage

% Заключение
\section*{Заключение}
\addcontentsline{toc}{section}{Заключение}
В ходе выполнения лабораторной работы был разработан последовательный и два параллельных алгоритма сортировки Хоара с простым слиянием.
\par По данным экспериментов удалось сравнить время работы последовательной и параллельных реализаций. Выявлено, что параллельные алгоритмы работают несколько быстрее, что показывает преимущество параллельных вычислений.
\par Для подтверждения корректности работы программы написаны тесты с использованием Google Testing Framework.
\newpage

% Список литературы
\begin{thebibliography}{1}
\addcontentsline{toc}{section}{Список литературы}
\bibitem{wiki} Википедия: Быстрая сортировка // URL \url {https://en.wikipedia.org/wiki/Quicksort} (дата обращения: 01.03.2021)
\bibitem{wiki1} Википедия: Сортировка слиянием // URL \url {https://en.wikipedia.org/wiki/Merge_sort} (дата обращения: 01.03.2021)
\end{thebibliography}
\newpage

% Приложение
\section*{Приложение}
\addcontentsline{toc}{section}{Приложение}
В этом разделе находится листинг всего кода, написанного в рамках лабораторной работы.
\par Реализация с использованием технологии OpenMP:
\begin{lstlisting}
// qsort.h

// Copyright 2021 Taktaev Artem
#ifndef MODULES_TASK_2_TAKTAEV_A_QSORT_QSORT_H_
#define MODULES_TASK_2_TAKTAEV_A_QSORT_QSORT_H_

#include <vector>

std::vector<double> createRandomVector(int vec_size);
void qSortSeq(std::vector<double> *arr, int left, int right);
void merge(double* a, int size_a, double* b, int size_b);
void qSortOmp(std::vector<double> *arr);

#endif  // MODULES_TASK_2_TAKTAEV_A_QSORT_QSORT_H_

\end{lstlisting}
\begin{lstlisting}
// qsort.cpp

// Copyright 2021 Taktaev Artem
#include "../../../modules/task_2/taktaev_a_qsort/qsort.h"

#include <omp.h>

#include <random>
#include <utility>
#include <deque>

std::vector<double> createRandomVector(int vec_size) {
    if (vec_size <= 0) throw "Vector's size must be > 0.";

    std::random_device rand_d;
    std::mt19937 gen(rand_d());
    std::vector<double> res_vec(vec_size);

    for (int i = 0; i < vec_size; i++)
        res_vec[i] = static_cast<double>(gen() % 2001) - 1000.0;

    return res_vec;
}

void qSortSeq(std::vector<double> *arr, int left, int right) {
    if (left >= right) throw "Left idx must be >= that right one.";

    int pidx = (left + right) / 2;
    double p = arr->at(pidx);
    int i = left, j = right;

    do {
        while (arr->at(i) < p) i++;
        while (arr->at(j) > p) j--;
        if (i <= j) {
            if (i < j) std::swap(arr->at(i), arr->at(j));
            i++;
            j--;
        }
    } while (i <= j);

    if (j > left) qSortSeq(arr, left, j);
    if (i < right) qSortSeq(arr, i, right);
}

void merge(double* a, int size_a, double* b, int size_b) {
    if (size_a < 1) throw "Size of a must be > 0";
    if (size_b < 1) throw "Size of b must be > 0";
    int i = 0, j = 0, k = 0;
    int size_c = size_a + size_b;
    double* c = new double[size_c];
    while ((i < size_a) && (j < size_b)) {
        if (a[i] <= b[j]) {
            c[k++] = a[i++];
        } else {
            c[k++] = b[j++];
        }
    }

    if (i == size_a) {
        while (j < size_b) {
            c[k++] = b[j++];
        }
    } else {
        while (i < size_a) {
            c[k++] = a[i++];
        }
    }

    for (int i = 0; i < size_c; i++) {
        a[i] = c[i];
    }
}

void qSortOmp(std::vector<double> *arr) {
    #pragma omp parallel
    {
        int n_threads = omp_get_num_threads();
        int part_vec_size = static_cast<int>(arr->size()) / n_threads;
            int thread_id = omp_get_thread_num();
            if (thread_id != n_threads - 1) {
                qSortSeq(arr, thread_id * part_vec_size, (thread_id + 1) * part_vec_size - 1);
            } else {
                qSortSeq(arr, thread_id * part_vec_size, static_cast<int>(arr->size()) - 1);
            }
    }

    #pragma omp parallel
    {
        int n_threads = omp_get_num_threads();
        #pragma omp master
        {
            int part_vec_size = static_cast<int>(arr->size()) / n_threads;
            std::deque<std::pair<int, int>> idx_size_deque;
            for (int i = 0; i < n_threads - 1; i++) {
                idx_size_deque.push_back({i, part_vec_size});
            }
            int n_last = static_cast<int>(arr->size()) - (n_threads - 1) * part_vec_size;
            idx_size_deque.push_back({n_threads - 1, n_last});
            std::pair<int, int> arr_to_merge1;
            std::pair<int, int> arr_to_merge2;
            while (idx_size_deque.size() > 1) {
                arr_to_merge1 = idx_size_deque.front();
                idx_size_deque.pop_front();
                arr_to_merge2 = idx_size_deque.front();
                idx_size_deque.pop_front();
                merge(arr->data() + arr_to_merge1.first * part_vec_size, arr_to_merge1.second,
                 arr->data() + arr_to_merge2.first * part_vec_size, arr_to_merge2.second);
                idx_size_deque.push_front({arr_to_merge1.first, arr_to_merge1.second + arr_to_merge2.second});
            }
        }
    }
}

\end{lstlisting}
\begin{lstlisting}
// main.cpp

// Copyright 2021 Taktaev Artem
#include <gtest/gtest.h>
#include <omp.h>

#include <vector>
#include <iostream>

#include "../../../modules/task_2/taktaev_a_qsort/qsort.h"

TEST(QSort_OMP, Test_Correct_Creating_Vector) {
    int n = 1000;
    std::vector<double> vec(n);
    ASSERT_NO_THROW(vec = createRandomVector(n));
}

TEST(QSort_OMP, Test_Sequential_Sorting_Rand_Odd_Vec) {
    int n = 11;
    std::vector<double> vec = createRandomVector(n);
    std::vector<double> vec_copy = vec;
    qSortSeq(&vec, 0, n - 1);
    std::sort(vec_copy.begin(), vec_copy.end());
    ASSERT_EQ(vec, vec_copy);
}

TEST(QSort_OMP, Test_Sequential_Sorting_Sorted_Vec) {
    int n = 15;
    std::vector<double> vec(n);
    for (int i = 0; i < n; i++) {
        vec[i] = i;
    }
    std::vector<double> vec_copy = vec;
    qSortSeq(&vec, 0, n - 1);
    std::sort(vec_copy.begin(), vec_copy.end());
    ASSERT_EQ(vec, vec_copy);
}

TEST(QSort_OMP, Test_Sequential_Sorting_Reverse_Sorted_Vec) {
    int n = 12;
    std::vector<double> vec(n);
    for (int i = 0; i < n; i++) {
        vec[i] = n - i;
    }
    std::vector<double> vec_copy = vec;
    qSortSeq(&vec, 0, n - 1);
    std::sort(vec_copy.begin(), vec_copy.end());
    ASSERT_EQ(vec, vec_copy);
}

TEST(QSort_OMP, Test_Sequential_Sorting_Eq_Elements_Vec) {
    int n = 18;
    std::vector<double> vec(n);
    for (int i = 0; i < n; i++) {
        vec[i] = 1;
    }
    std::vector<double> vec_copy = vec;
    qSortSeq(&vec, 0, n - 1);
    std::sort(vec_copy.begin(), vec_copy.end());
    ASSERT_EQ(vec, vec_copy);
}

TEST(QSort_OMP, Test_Correct_Simple_Merge) {
    std::vector<double> vec = {1, 3, 5, 7, 9, 2, 4, 6, 8, 10};
    std::vector<double> vec_copy = vec;
    std::sort(vec_copy.begin(), vec_copy.end());
    merge(&vec[0], 5, &vec[5], 5);
    ASSERT_EQ(vec_copy, vec);
}

TEST(QSort_OMP, Test_Parallel_Sorting_Rand_Even_Vec) {
    int n = 1300;
    std::vector<double> vec = createRandomVector(n);
    std::vector<double> vec_copy = vec;
    qSortOmp(&vec);
    std::sort(vec_copy.begin(), vec_copy.end());

    ASSERT_EQ(vec, vec_copy);
}

TEST(QSort_OMP, Test_Parallel_Sorting_Rand_Odd_Vec) {
    int n = 1301;
    std::vector<double> vec = createRandomVector(n);
    std::vector<double> vec_copy = vec;
    qSortOmp(&vec);
    std::sort(vec_copy.begin(), vec_copy.end());

    ASSERT_EQ(vec, vec_copy);
}

TEST(QSort_OMP, Test_Parallel_And_Seq_Do_The_Same) {
    int n = 1000;
    std::vector<double> vec = createRandomVector(n);
    std::vector<double> vec_copy = vec;
    qSortOmp(&vec);
    qSortSeq(&vec_copy, 0, n - 1);

    ASSERT_EQ(vec, vec_copy);
}

TEST(QSort_OMP, Test_Parallel_Sorting_Sorted_Vec) {
    int n = 150;
    std::vector<double> vec(n);
    for (int i = 0; i < n; i++) {
        vec[i] = i;
    }
    std::vector<double> vec_copy = vec;
    qSortOmp(&vec);
    std::sort(vec_copy.begin(), vec_copy.end());
    ASSERT_EQ(vec, vec_copy);
}

TEST(QSort_OMP, Test_Parallel_Sorting_Reverse_Sorted_Vec) {
    int n = 120;
    std::vector<double> vec(n);
    for (int i = 0; i < n; i++) {
        vec[i] = n - i;
    }
    std::vector<double> vec_copy = vec;
    qSortOmp(&vec);
    std::sort(vec_copy.begin(), vec_copy.end());
    ASSERT_EQ(vec, vec_copy);
}

TEST(QSort_OMP, Test_Parallel_Sorting_Eq_Elements_Vec) {
    int n = 180;
    std::vector<double> vec(n);
    for (int i = 0; i < n; i++) {
        vec[i] = 1;
    }
    std::vector<double> vec_copy = vec;
    qSortOmp(&vec);
    std::sort(vec_copy.begin(), vec_copy.end());
    ASSERT_EQ(vec, vec_copy);
}

TEST(QSort_OMP, Test_Creating_Vector_Exception) {
    int n = -1;
    ASSERT_ANY_THROW(std::vector<double> vec = createRandomVector(n));
}

TEST(QSort_OMP, Test_Out_Of_Bounds_Exception) {
    int n = 11;
    std::vector<double> vec = createRandomVector(n);
    ASSERT_ANY_THROW(qSortSeq(&vec, -1, n));
}

TEST(QSort_OMP, Test_Index_Overlap_Exception) {
    int n = 10;
    std::vector<double> vec = createRandomVector(n);
    ASSERT_ANY_THROW(qSortSeq(&vec, n - 1, 0));
}

/*TEST(QSort_OMP, Test_Time) {
    int n = 1300000;
    double start = 0, finish = 0, time_seq = 0, time_par = 0;
    int n_thr = 6;  // thread num
    std::vector<double> vec = createRandomVector(n);
    std::vector<double> vec_copy = vec;
    omp_set_num_threads(n_thr);
    start = omp_get_wtime();
    qSortSeq(&vec, 0, n - 1);
    finish = omp_get_wtime();
    time_seq = finish - start;
    std::cout << "Time sequential: " << time_seq << std::endl;
    start = omp_get_wtime();
    qSortOmp(&vec_copy);
    finish = omp_get_wtime();
    time_par = finish - start;
    std::cout << "Time parallel: " << time_par << std::endl;
    std::cout << "Acceleration " << time_seq / time_par << std::endl;
    ASSERT_EQ(vec, vec_copy);
}*/

int main(int argc, char** argv) {
    ::testing::InitGoogleTest(&argc, argv);
    return RUN_ALL_TESTS();
}

\end{lstlisting}

\par Реализация с использованием технологии TBB:

\begin{lstlisting}
// qsort.h

// Copyright 2021 Taktaev Artem
#ifndef MODULES_TASK_3_TAKTAEV_A_QSORT_QSORT_H_
#define MODULES_TASK_3_TAKTAEV_A_QSORT_QSORT_H_

#include <vector>

std::vector<double> createRandomVector(int vec_size);
void qSortSeq(std::vector<double> *arr, int left, int right);
void merge(double* a, int size_a, double* b, int size_b);
void qSortTbb(std::vector<double> *arr);

#endif  // MODULES_TASK_3_TAKTAEV_A_QSORT_QSORT_H_

\end{lstlisting}
\begin{lstlisting}
// qsort.cpp

// Copyright 2021 Taktaev Artem
#include "../../../modules/task_3/taktaev_a_qsort/qsort.h"

#include <tbb/blocked_range.h>
#include <tbb/parallel_for.h>

#include <map>
#include <random>
#include <utility>

std::vector<double> createRandomVector(int vec_size) {
    if (vec_size <= 0) throw "Vector's size must be > 0.";

    std::random_device rand_d;
    std::mt19937 gen(rand_d());
    std::vector<double> res_vec(vec_size);

    for (int i = 0; i < vec_size; i++)
        res_vec[i] = static_cast<double>(gen() % 2001) - 1000.0;

    return res_vec;
}

void qSortSeq(std::vector<double> *arr, int left, int right) {
    if (left >= right) throw "Left idx must be < that right one.";

    int pidx = (left + right) / 2;
    double p = arr->at(pidx);
    int i = left, j = right;

    do {
        while (arr->at(i) < p) i++;
        while (arr->at(j) > p) j--;
        if (i <= j) {
            if (i < j) std::swap(arr->at(i), arr->at(j));
            i++;
            j--;
        }
    } while (i <= j);

    if (j > left) qSortSeq(arr, left, j);
    if (i < right) qSortSeq(arr, i, right);
}

void merge(double* a, int size_a, double* b, int size_b) {
    if (size_a < 1 || size_b < 1) return;
    int i = 0, j = 0, k = 0;
    int size_c = size_a + size_b;
    double* c = new double[size_c];
    while ((i < size_a) && (j < size_b)) {
        if (a[i] <= b[j]) {
            c[k++] = a[i++];
        } else {
            c[k++] = b[j++];
        }
    }

    if (i == size_a) {
        while (j < size_b) {
            c[k++] = b[j++];
        }
    } else {
        while (i < size_a) {
            c[k++] = a[i++];
        }
    }

    for (int i = 0; i < size_c; i++) {
        a[i] = c[i];
    }
}

void qSortTbb(std::vector<double> *arr) {
    int size = static_cast<int>(arr->size());
    int min_grain_size = 10;  // the less possible grain size
    int divider = 3;  // the part of arr that can be a grain size
    int part_arr_size = size / divider;
    int grain_size = part_arr_size > min_grain_size ? part_arr_size : min_grain_size;
    tbb::parallel_for(tbb::blocked_range<int>(0, size, grain_size), [&arr](const tbb::blocked_range<int>& r) {
        qSortSeq(arr, r.begin(), r.end() - 1);
    });
    int i_prev = 0;
    int k = 0;
    for (int i = 1; i < size; i++) {
        if (arr->at(i) < arr->at(i - 1)) {
            if (k++ > 0)
                merge(arr->data(), i_prev, arr->data() + i_prev, i - i_prev);
            i_prev = i;
        }
    }
    merge(arr->data(), i_prev, arr->data() + i_prev, size - i_prev);
}

\end{lstlisting}
\begin{lstlisting}
// main.cpp

// Copyright 2021 Taktaev Artem
#include <gtest/gtest.h>
#include <tbb/tick_count.h>

#include <vector>
#include <iostream>

#include "../../../modules/task_3/taktaev_a_qsort/qsort.h"

TEST(QSort_TBB, Test_Correct_Creating_Vector) {
    int n = 1000;
    std::vector<double> vec(n);
    ASSERT_NO_THROW(vec = createRandomVector(n));
}

TEST(QSort_TBB, Test_Sequential_Sorting_Rand_Odd_Vec) {
    int n = 11;
    std::vector<double> vec = createRandomVector(n);
    std::vector<double> vec_copy = vec;
    qSortSeq(&vec, 0, n - 1);
    std::sort(vec_copy.begin(), vec_copy.end());
    ASSERT_EQ(vec, vec_copy);
}

TEST(QSort_TBB, Test_Sequential_Sorting_Sorted_Vec) {
    int n = 15;
    std::vector<double> vec(n);
    for (int i = 0; i < n; i++) {
        vec[i] = i;
    }
    std::vector<double> vec_copy = vec;
    qSortSeq(&vec, 0, n - 1);
    std::sort(vec_copy.begin(), vec_copy.end());
    ASSERT_EQ(vec, vec_copy);
}

TEST(QSort_TBB, Test_Sequential_Sorting_Reverse_Sorted_Vec) {
    int n = 12;
    std::vector<double> vec(n);
    for (int i = 0; i < n; i++) {
        vec[i] = n - i;
    }
    std::vector<double> vec_copy = vec;
    qSortSeq(&vec, 0, n - 1);
    std::sort(vec_copy.begin(), vec_copy.end());
    ASSERT_EQ(vec, vec_copy);
}

TEST(QSort_TBB, Test_Sequential_Sorting_Eq_Elements_Vec) {
    int n = 18;
    std::vector<double> vec(n);
    for (int i = 0; i < n; i++) {
        vec[i] = 1;
    }
    std::vector<double> vec_copy = vec;
    qSortSeq(&vec, 0, n - 1);
    std::sort(vec_copy.begin(), vec_copy.end());
    ASSERT_EQ(vec, vec_copy);
}

TEST(QSort_TBB, Test_Correct_Simple_Merge) {
    std::vector<double> vec = {1, 3, 5, 7, 9, 2, 4, 6, 8, 10};
    std::vector<double> vec_copy = vec;
    std::sort(vec_copy.begin(), vec_copy.end());
    merge(&vec[0], 5, &vec[5], 5);
    ASSERT_EQ(vec_copy, vec);
}

TEST(QSort_TBB, Test_Parallel_Sorting_Rand_Even_Vec) {
    int n = 1300;
    std::vector<double> vec = createRandomVector(n);
    std::vector<double> vec_copy = vec;
    qSortTbb(&vec);
    std::sort(vec_copy.begin(), vec_copy.end());

    ASSERT_EQ(vec, vec_copy);
}

TEST(QSort_TBB, Test_Parallel_Sorting_Rand_Odd_Vec) {
    int n = 1301;
    std::vector<double> vec = createRandomVector(n);
    std::vector<double> vec_copy = vec;
    qSortTbb(&vec);
    std::sort(vec_copy.begin(), vec_copy.end());

    ASSERT_EQ(vec, vec_copy);
}

TEST(QSort_TBB, Test_Parallel_And_Seq_Do_The_Same) {
    int n = 1000;
    std::vector<double> vec = createRandomVector(n);
    std::vector<double> vec_copy = vec;
    qSortTbb(&vec);
    qSortSeq(&vec_copy, 0, n - 1);

    ASSERT_EQ(vec, vec_copy);
}

TEST(QSort_TBB, Test_Parallel_Sorting_Sorted_Vec) {
    int n = 150;
    std::vector<double> vec(n);
    for (int i = 0; i < n; i++) {
        vec[i] = i;
    }
    std::vector<double> vec_copy = vec;
    qSortTbb(&vec);
    std::sort(vec_copy.begin(), vec_copy.end());
    ASSERT_EQ(vec, vec_copy);
}

TEST(QSort_TBB, Test_Parallel_Sorting_Reverse_Sorted_Vec) {
    int n = 120;
    std::vector<double> vec(n);
    for (int i = 0; i < n; i++) {
        vec[i] = n - i;
    }
    std::vector<double> vec_copy = vec;
    qSortTbb(&vec);
    std::sort(vec_copy.begin(), vec_copy.end());
    ASSERT_EQ(vec, vec_copy);
}

TEST(QSort_TBB, Test_Parallel_Sorting_Eq_Elements_Vec) {
    int n = 180;
    std::vector<double> vec(n);
    for (int i = 0; i < n; i++) {
        vec[i] = 1;
    }
    std::vector<double> vec_copy = vec;
    qSortTbb(&vec);
    std::sort(vec_copy.begin(), vec_copy.end());
    ASSERT_EQ(vec, vec_copy);
}

TEST(QSort_TBB, Test_Creating_Vector_Exception) {
    int n = -1;
    ASSERT_ANY_THROW(std::vector<double> vec = createRandomVector(n));
}

TEST(QSort_TBB, Test_Out_Of_Bounds_Exception) {
    int n = 11;
    std::vector<double> vec = createRandomVector(n);
    ASSERT_ANY_THROW(qSortSeq(&vec, -1, n));
}

TEST(QSort_TBB, Test_Index_Overlap_Exception) {
    int n = 10;
    std::vector<double> vec = createRandomVector(n);
    ASSERT_ANY_THROW(qSortSeq(&vec, n - 1, 0));
}

/*TEST(QSort_TBB, Test_Time) {
    int n = 3000000;
    double start = 0, finish = 0, time_seq = 0, time_par = 0;
    std::vector<double> vec = createRandomVector(n);
    std::vector<double> vec_copy = vec;
    tbb::tick_count t1_seq = tbb::tick_count::now();
    qSortSeq(&vec, 0, n - 1);
    tbb::tick_count t2_seq = tbb::tick_count::now();
    time_seq = (t2_seq - t1_seq).seconds();
    std::cout << "Time sequential: " << time_seq << std::endl;
    tbb::tick_count t1_par = tbb::tick_count::now();
    qSortTbb(&vec_copy);
    tbb::tick_count t2_par = tbb::tick_count::now();
    time_par = (t2_par - t1_par).seconds();
    std::cout << "Time parallel: " << time_par << std::endl;
    std::cout << "Acceleration " << time_seq / time_par << std::endl;
    ASSERT_EQ(vec, vec_copy);
}*/

int main(int argc, char** argv) {
    ::testing::InitGoogleTest(&argc, argv);
    return RUN_ALL_TESTS();
}

\end{lstlisting}

\end{document}
