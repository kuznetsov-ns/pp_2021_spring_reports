\documentclass{report}

\usepackage[T2A]{fontenc}
\usepackage[utf8]{luainputenc}
\usepackage[english, russian]{babel}
\usepackage[pdftex]{hyperref}
\usepackage[14pt]{extsizes}
\usepackage{listings}
\usepackage{color}
\usepackage{geometry}
\usepackage{enumitem}
\usepackage{multirow}
\usepackage{graphicx}
\usepackage{indentfirst}
\usepackage{amsmath}

\geometry{a4paper,top=2cm,bottom=3cm,left=2cm,right=1.5cm}
\setlength{\parskip}{0.5cm}
\setlist{nolistsep, itemsep=0.3cm,parsep=0pt}

\lstset{language=C++,
		basicstyle=\footnotesize,
		keywordstyle=\color{blue}\ttfamily,
		stringstyle=\color{red}\ttfamily,
		commentstyle=\color{green}\ttfamily,
		morecomment=[l][\color{magenta}]{\#}, 
		tabsize=4,
		breaklines=true,
  		breakatwhitespace=true,
  		title=\lstname,       
}

\makeatletter
\renewcommand\@biblabel[1]{#1.\hfil}
\makeatother

\begin{document}

\begin{titlepage}

\begin{center}
Министерство науки и высшего образования Российской Федерации
\end{center}

\begin{center}
Федеральное государственное автономное образовательное учреждение высшего образования \\
Национальный исследовательский Нижегородский государственный университет им. Н.И. Лобачевского
\end{center}

\begin{center}
Институт информационных технологий, математики и механики
\end{center}

\vspace{4em}

\begin{center}
\textbf{\LargeОтчет по лабораторной работе} \\
\end{center}
\begin{center}
\textbf{\Large«Вычисление многомерных интегралов с использованием многошаговой схемы (метод Симпсона).»} \\
\end{center}

\vspace{4em}

\newbox{\lbox}
\savebox{\lbox}{\hbox{text}}
\newlength{\maxl}
\setlength{\maxl}{\wd\lbox}
\hfill\parbox{7cm}{
\hspace*{5cm}\hspace*{-5cm}\textbf{Выполнил:} \\ студент группы 381808-2 \\ Алехин Д.А. \\
\\
\hspace*{5cm}\hspace*{-5cm}\textbf{Проверил:}\\ доцент кафедры МОСТ, \\ кандидат технических наук \\ Сысоев А. В. \\
}
\vspace{\fill}

\begin{center} Нижний Новгород \\ 2021 \end{center}

\end{titlepage}

\setcounter{page}{2}

% Содержание
\tableofcontents
\newpage

% Введение
\section*{Введение}
\addcontentsline{toc}{section}{Введение}
Задача нахождения значение определенного интеграла не всегда имеет точное решения. Действительно, первообразную подынтегральной функции во многих случаях не удается представить в виде элементарной функции. В этом случае мы не можем точно вычислить определенный интеграл по формуле Ньютона-Лейбница.  Поэтому были разработаны методы, которые находят значение определенного интеграла с высокой точностью. Одним из них является Метод Симпсона, который позволяет вычислить значение интеграла с малой погрешностью.
\newpage

% Постановка задачи
\section*{Постановка задачи}
\addcontentsline{toc}{section}{Постановка задачи}
Целью данной лабораторной работы является разработка последовательной и параллельной версий метода Симпсона, для решения кратного интеграла. 
\par Решением задачи является численные ответ - действительное число. Исходными данными является решаемый интеграл и пределы интегрирования для каждого интеграла.
\par Необходимо реализовать последовательную и параллельную версию программы. Для реализации параллельной версии необходимо использовать средства OMP, TBB и std::thread. Для проверки корректности работы программы необходимо использовать GTest.
\newpage

% Описание алгоритма
\section*{Описание алгоритма}
\addcontentsline{toc}{section}{Описание алгоритма}
На вход алгоритма поступают:
\begin{itemize}
\item Функция f, от которой необходино найти интеграл
\item Пределы интегрирования $[a_1,b_1],...,[a_k, b_k]$
\item Точность n
\end{itemize}
\par Описание алгоритма:
\begin{itemize}
\item Каждый кратный интеграл разбиваем на $N=2n$ отрезков и для каждого интеграла определям шаг интегрирования $h_k=(b_k-a_k)/N$.
\item Пронумеруем границы отрезков интегрирования от $0$ до $N$.
\item Найдем сумму значений функции во всех точках с четным номером. Обозначим это значение за $evenSum$.
\item Найдем сумму значений функции во всех точках с нечетным номером. Обозначим это значение за $oddSum$.
\item Примем за $result=f(a)+f(b)+evenSum*2+4*oddSum$
\item Поделим result на $(b_k-a_b)$ для каждого предела интегрирования.
\item Численным ответом задачи будет $result=result/(3*N)$

\end{itemize}

\newpage

% Описание схемы распараллеливания
\section*{Описание схемы распараллеливания}
\addcontentsline{toc}{section}{Описание схемы распараллеливания}

Алгоритм является простым для распараллеливания. Так как интегралы разбиты на $N$ отрезков, и вычисление значения интеграла не зависит от других отрезков, то можно распределить данные отрезки по вычислительным узлам. Численным ответом решения задачи будет являться сумма результатов полученных на каждом вычислительном узле. 
\par Для достижения данного результата в OpenMP существует деректива препроцессора $pragma omp parallel for reduction$, а в TBB $tbb::parallel_reduce$.
\newpage

% Описание программной реализации
\section*{Описание программной реализации}
\addcontentsline{toc}{section}{Описание программной реализации}\
Входными параметрами всех реализаций алгоритма являются:
\begin{itemize}
\item $std::vector<pair::<double, double>> scope$ - пределы интегрирования
\item Функция $f$
\item Точность интегрирования $n$
\end{itemize}

\par Для вызова последовательной реализации используется метод 
\begin{lstlisting}
simpsonMethod(scope, f, n)
\end{lstlisting}
\par Для вызова OMP реализации используется метод:
\begin{lstlisting}
simpsonMethodOMP(scope, f, n)
\end{lstlisting}
\par Для вызова TBB реализации используется метод:
\begin{lstlisting}
simpsonMethodTBB(scope, f, n)
\end{lstlisting}
\par Для вызова STD реализации используется метод:
\begin{lstlisting}
simpsonMethodSTD(scope, f, n)
\end{lstlisting}

\newpage

% Результаты экспериментов
\section*{Результаты экспериментов}
\addcontentsline{toc}{section}{Результаты экспериментов}
Вычислительные эксперименты проводились на оборудовании со следующей аппаратной конфигурацией:

\begin{itemize}
\item Процессор: Intel(R) Core(TM) i5-8400 CPU @ 3.80GHz, ядер: 6
\item Оперативная память: 16 ГБ DDR4;
\item ОС: Microsoft Windows 10 LTSC
\end{itemize}

\par Тестирование производилось при $N=100$ млн., $f=sin(x_1)+sin(x_2)+sin(x_3)$
\\

\begin{table}[!h]
\begin{center}
\begin{tabular}{lllllll}
Реализация & Время & Ускорение   \\
SEQ        & 3.93 & 1           \\
OMP        & 1.34 & 2.93       \\
TBB        & 1.25 & 3.14       \\
STD        & 1.3 & 3.02        \\

\end{tabular}
\end{center}
\caption{Результаты вычислительных экспериментов}
\centering
\end{table}

\par Из полученных результатов видно, что удалось получить значительное ускорение в параллельных реализациях метода Стронгина. Исходя из этого можно сделать вывод, что данный алгоритм хорошо поддается распараллеливанию.
\newpage

% Заключение
\section*{Заключение}
\addcontentsline{toc}{section}{Заключение}
В результате лабораторной работы были разработаны последовательная и параллельная реализации метода Стронгина.
\par Были разработанна последовательная версия алгоритма,  и на ее основе были полученны параллельные реализации с использованием OpenMP, TBB и std::thread. 
\par Из результатов вычислительных экспериментов видно, что параллельные реализации алгоритма являются эффективными и, в среднем, обеспечивают ускорение в 3 раза.
\newpage

% Литература
\begin{thebibliography}{1}
\addcontentsline{toc}{section}{Литература}
\bibitem{} «Параллельное программирование на OpenMP»
\\URL:\url {http://ccfit.nsu.ru/arom/data/openmp.pdf}
\bibitem{Sidnev} Сиднев А.А., Мееров И.Б., Сысоев А.В. «Разработка параллельных программ в системах с общей памятью с использованием библиотеки Intel Threading Building Blocks (TBB)».
\\URL:\url {http://hpc-education.ru/files/lectures/meerov/ppt06.pdf}
\bibitem{} «Параллельное программирование. С++. Thread Support Library. Atomic Operations Library.»
\\URL:\url {https://ppt-online.org/184002}
\end{thebibliography}
\newpage

% Приложение
\section*{Приложение}
\addcontentsline{toc}{section}{Приложение}
В данном разделе находится реализация методов последовательной и параллельной реализции алгоритма
\begin{lstlisting}
// simpsont.cpp

// Copyright 2021 Alekhin Denis

#include <omp.h>
#include <iostream>
#include "../../modules/task_2/alekhin_d_simpson_omp/simpson.h"

double simpsonMethod(std::vector<std::pair<double, double>> scope,
  std::function<double(std::vector<double>)> func,
  int precision) {
  if (scope.size() == 0) {
    throw "Error: scope can't be empty!";
  }
  for (size_t i = 0; i < scope.size(); i++) {
    if (scope[i].first >= scope[i].second) {
      throw "Error: wrong scope interval!";
    }
  }
  if (precision <= 0) {
    throw "Error: precision can't be less than 1";
  }

  int dimention = scope.size();
  std::vector<double>
    h(dimention),
    a(dimention),
    b(dimention);
  for (int i = 0; i < dimention; i++) {
    h[i] = (scope[i].second - scope[i].first) / precision;
    a[i] = scope[i].first;
    b[i] = scope[i].second;
  }

  double evenSum = 0, oddSum = 0;
  std::vector<double> point(a);
  for (int i = 1; i < precision; i++) {
    for (int j = 0; j < dimention; j++) {
      point[j] += h[j];
    }
    if (i % 2 == 0) {
      evenSum += func(point);
    } else {
      oddSum += func(point);
    }
  }

  double result = func(a) + func(b) + 2.0 * evenSum + 4.0 * oddSum;

  for (int i = 0; i < dimention; i++) {
    result *= b[i] - a[i];
  }

  result /= 3.0 * precision;
  return result;
}

double simpsonMethodOMP(
  std::vector<std::pair<double, double>> scope,
  std::function<double(std::vector<double>)> func,
  int precision) {
  if (scope.size() == 0) {
    throw "Error: scope can't be empty!";
  }

  for (size_t i = 0; i < scope.size(); i++) {
    if (scope[i].first >= scope[i].second) {
      throw "Error: wrong scope interval!";
    }
  }
  if (precision <= 0) {
    throw "Error: precision can't be less than 1";
  }

  int dimention = scope.size();
  std::vector<double>
    h(dimention),
    a(dimention),
    b(dimention);
  for (int i = 0; i < dimention; i++) {
    h[i] = (scope[i].second - scope[i].first) / precision;
    a[i] = scope[i].first;
    b[i] = scope[i].second;
  }

  double result = func(a) + func(b);
  std::vector<double> oddPoint(dimention), evenPoint(dimention);
#pragma omp parallel for reduction(+:result) firstprivate(oddPoint, evenPoint)
  for (int i = 1; i < precision; i += 2) {
    for (int j = 0; j < dimention; j++) {
      oddPoint[j] = a[j] + h[j] * i;
      evenPoint[j] = (i + 1 < precision) ? oddPoint[j] + h[j] : 0;
    }
    result +=  2.0 * func(evenPoint) + 4.0 * func(oddPoint);
  }

  for (int i = 0; i < dimention; i++) {
    result *= b[i] - a[i];
  }

  result /= 3.0 * precision;

  return result;
}

double simpsonMethodTBB(
  std::vector<std::pair<double, double>> scope,
  std::function<double(std::vector<double>)> func,
  int precision) {
  if (scope.size() == 0) {
    throw "Error: scope can't be empty!";
  }

  for (size_t i = 0; i < scope.size(); i++) {
    if (scope[i].first >= scope[i].second) {
      throw "Error: wrong scope interval!";
    }
  }
  if (precision <= 0) {
    throw "Error: precision can't be less than 1";
  }

  int dimention = scope.size();
  std::vector<double>
    h(dimention),
    a(dimention),
    b(dimention);
  for (int i = 0; i < dimention; i++) {
    h[i] = (scope[i].second - scope[i].first) / precision;
    a[i] = scope[i].first;
    b[i] = scope[i].second;
  }

  double result =
    tbb::parallel_reduce(tbb::blocked_range<int>(1, precision), 0.0,
    [&](const tbb::blocked_range<int>& range, double result) {
      std::vector<double> oddPoint(dimention), evenPoint(dimention);
      for (int i = range.begin(); i < range.end(); i++) {
        if (i % 2 == 0) {
          for (int j = 0; j < dimention; j++) {
            evenPoint[j] = a[j] + h[j] * i;
          }
          result += 2.0 * func(evenPoint);
        } else {
          for (int j = 0; j < dimention; j++) {
            oddPoint[j] = a[j] + h[j] * i;
          }
          result += 4.0 * func(oddPoint);
        }
      }
      return result;
    }, std::plus<double>());

  result += func(a) + func(b);

  for (int i = 0; i < dimention; i++) {
    result *= b[i] - a[i];
  }

  result /= 3.0 * precision;

  return result;
}

\end{lstlisting}
\end{document}