\documentclass{report}

\usepackage[T2A]{fontenc}
\usepackage[utf8]{luainputenc}
\usepackage[english, russian]{babel}
\usepackage[pdftex]{hyperref}
\usepackage[14pt]{extsizes}
\usepackage{listings}
\usepackage{color}
\usepackage{geometry}
\usepackage{enumitem}
\usepackage{multirow}
\usepackage{graphicx}
\usepackage{indentfirst}
\usepackage{amsmath}


\geometry{a4paper,top=2cm,bottom=3cm,left=2cm,right=1.5cm}
\setlength{\parskip}{0.5cm}
\setlist{nolistsep, itemsep=0.3cm,parsep=0pt}

\lstset{language=C++,
		basicstyle=\footnotesize,
		keywordstyle=\color{blue}\ttfamily,
		stringstyle=\color{red}\ttfamily,
		commentstyle=\color{green}\ttfamily,
		morecomment=[l][\color{magenta}]{\#},
		tabsize=4,
		breaklines=true,
  		breakatwhitespace=true,
  		title=\lstname,
}

\makeatletter
\renewcommand\@biblabel[1]{#1.\hfil}
\makeatother

\begin{document}

\begin{titlepage}

\begin{center}
Министерство науки и высшего образования Российской Федерации
\end{center}

\begin{center}
Федеральное государственное автономное образовательное учреждение высшего образования \\
Национальный исследовательский Нижегородский государственный университет им. Н.И. Лобачевского
\end{center}

\begin{center}
Институт информационных технологий, математики и механики
\end{center}

\vspace{4em}

\begin{center}
\textbf{\LargeОтчет по лабораторной работе} \\
\end{center}
\begin{center}
\textbf{\Large«Сортировка Хоара с простым слиянием»} \\
\end{center}

\vspace{4em}

\newbox{\lbox}
\savebox{\lbox}{\hbox{text}}
\newlength{\maxl}
\setlength{\maxl}{\wd\lbox}
\hfill\parbox{7cm}{
\hspace*{5cm}\hspace*{-5cm}\textbf{Выполнил:} \\ студент группы 381808-1 \\ Визгалов А. И.\\
\\
\hspace*{5cm}\hspace*{-5cm}\textbf{Проверил:}\\ доцент кафедры МОСТ, \\ кандидат технических наук \\ Сысоев А. В.\\
}
\vspace{\fill}

\begin{center} Нижний Новгород \\ 2021 \end{center}

\end{titlepage}

\setcounter{page}{2}

% Содержание
\tableofcontents
\newpage

% Введение
\section*{Введение}
\addcontentsline{toc}{section}{Введение}
Сортировка Хоара (англ. quicksort) — алгоритм сортировки, разработанный английским информатиком Тони Хоаром во время его работы в МГУ в 1960 году. Является одним из самых быстрых известных универсальных алгоритмов сортировки массивов: в среднем $O(n*log{n})$ обменов при упорядочении $n$ элементов; из-за наличия ряда недостатков на практике обычно используется с некоторыми доработками.

\newpage

% Постановка задачи
\section*{Постановка задачи}
\addcontentsline{toc}{section}{Постановка задачи}
Требуется разработать программу, позволяющую осуществлять сортировку массива целых чисел методом Хоара с простым слиянием.
\par Программа должна содержать три версии:
\begin{enumerate}
\item Последовательная реализация.
\item Параллельная реализация, использующая стандарт OpenMP.
\item Параллельная реализация, использующая библиотеку Intel Threading Building Blocks (TBB).
\end{enumerate}
\par С помощью разработанной программы, требуется провести вычислительные эксперименты и по их результатам сравнить время работы последовательной и параллельных версий, а затем сделать вывод об эффективности параллельных реализаций.
\par Также необходимо экспериментально убедиться в корректности реализованных алгоритмов с помощью набора тестов и фреймворка GoogleTest.
\newpage

% Описание алгоритма
\section*{Описание алгоритма}
\addcontentsline{toc}{section}{Описание алгоритма}
Алгоритм Хоара является существенно улучшенным вариантом алгоритма сортировки с помощью прямого обмена («Пузырьковая сортировка», «Шейкерная сортировка»), обладающего низкой эффективностью. Принципиальное отличие состоит в том, что в первую очередь производятся перестановки на наибольшем возможном расстоянии и после каждого прохода элементы делятся на две независимые группы.
\par Общая идея алгоритма состоит в следующем:
\begin{enumerate}
\item Выбрать из массива элемент, называемый опорным. Это может быть любой из элементов массива (в нашей реализации --- последний). От выбора опорного элемента не зависит корректность алгоритма, но в отдельных случаях может сильно зависеть его эффективность.
\item Сравнить все остальные элементы с опорным и переставить их в массиве так, чтобы разбить массив на три непрерывных отрезка, следующих друг за другом: «элементы меньшие опорного», «равные» и «большие».
\item Для отрезков «меньших» и «больших» значений выполнить рекурсивно ту же последовательность операций, если длина отрезка больше единицы.
\end{enumerate}

\newpage

% Схема распараллеливания
\section*{Схема распараллеливания}
\addcontentsline{toc}{section}{Схема распараллеливания}
Для выполнения сортировки Хоара параллельно, необходимо предварительно разделить массив на равные подмассивы, количество которых равно количеству потоков. Если массив не делится нацело на число потоков, остаточные элементы добавляются к последнему по счету подмассиву.
\par Далее в каждом потоке сортировка выполняется на соответствующем потоку подмассиве.
\par В конце отсортированные подмассивы объединяются в один массив методом простого слияния. На каждом шаге выбирается меньший из двух первых элементов подмассивов и записывается в результирующий массив. Счетчики номеров элементов результирующего массива и подмассива, из которого был взят элемент, увеличиваются на 1. Когда один из подмассивов заканчивается, все оставшиеся элементы второго подмассива добавляются в результирующий массив.
\newpage

% Описание программной реализации
\section*{Описание программной реализации}
\addcontentsline{toc}{section}{Описание программной реализации}

Основная функция программы - runHoareSort. Она принимает на вход указатель на массив целых чисел типа int и производит сортировку методом Хоара с простым слиянием. В параллельных версиях программы сортировка выполняется параллельно.
\begin{lstlisting}
void runHoareSort(std::vector<int>* vec);
\end{lstlisting}

\par Основная функция использует несколько служебных функций:
\begin{enumerate}
\item Для сортировки подмассива в одном потоке:
\begin{lstlisting}
void hoareSort(std::vector<int>* vec, int low, int high);
\end{lstlisting}
\item Для выбора опорного элемента и перемещения меньших значений влево:
\begin{lstlisting}
int partition(std::vector<int>* vec, int low, int high);
\end{lstlisting}
\item Для слияния подмассивов:
\begin{lstlisting}
void merge(std::vector<int>* vec, int left, int mid, int right);
\end{lstlisting}
\end{enumerate}

\par Параллельная версия на OpenMP использует следующую директиву перед циклом для запуска сортировки Хоара на подмассивах:
\begin{lstlisting}
#pragma omp parallel for
\end{lstlisting}

\par Параллельная версия на TBB использует следующую конструкцию:
\begin{lstlisting}
tbb::task_scheduler_init init(numThreads);

tbb::parallel_for(tbb::blocked_range<size_t>(0, segments.size(), 1),
    [&segments](const tbb::blocked_range<size_t>& range) {
        for (size_t i = range.begin(); i != range.end(); ++i) {
            hoareSort(&segments[i], 0, segments[i].size() - 1);
        }
    }, tbb::simple_partitioner());
\end{lstlisting}

\par В целях тестирования программы была также разработана следующая функция генерации массива случайных чисел:
\begin{lstlisting}
std::vector<int> getRandomVector(int size, int type = 0);
\end{lstlisting}

\newpage

% Результаты экспериментов
\section*{Результаты экспериментов}
\addcontentsline{toc}{section}{Результаты экспериментов}
Вычислительные эксперименты для оценки эффективности программы проводились на компьютере следующей аппаратной конфигурации:

\begin{itemize}
\item Процессор: Intel Core i7-4790K, 4.00 GHz, 4 физических ядра, 8 потоков (с учетом Hyper-Threading);
\item Оперативная память: 8 ГБ;
\item ОС: Linux Ubuntu 20.04 64-bit.
\end{itemize}

\par Исходными данными для эксперимента являлся массив из 10000000 случайных целых чисел со значениями из полного диапазона типа данных int.
\par В ходе эксперимента было установлено время работы последовательной и параллельных версий программы при разном количестве потоков. В таблице приведены усредненные результаты пяти запусков программы для каждого случая. Значения приведены в миллисекундах.

\begin{table}[!h]
\caption{Результаты экспериментов}
\centering
\begin{tabular}{llll}
Число потоков & OpenMP & TBB & Последовательная версия \\
2 & 4188 & 4149 & 7566 \\
4 & 2704 & 2503 & 7464 \\
8 & 2879 & 2301 & 7538 \\
\end{tabular}
\end{table}

\par Результаты экспериментов позволяют установить, что параллельные версии алгоритма работают значительно быстрее последовательной, однако добиться ускорения в $N$ раз, когда параллельная версия на $N$ потоках работала бы в $N$ раз быстрее последовательной, не удается.
\par Кроме того, эксперименты показывают более высокую эффективность TBB версии, по сравнению с OpenMP, а также что она более оптимально использует логические ядра процессора. В OpenMP версии при числе потоков, превышающем число физических ядер процессора, наблюдается небольшое, но уменьшение производительности, а в TBB версии - наоборот.

Для подтверждения корректности реализованных алгоритмов, программа проходит подготовленный набор тестов. Они проверяют как решение задачи целиком (после сортировки массив должен быть отсортирован), так и отдельных ее частей (слияние массивов разного размера, сортировка разных типов массивов и т.п.).
\par Успешное прохождение всех тестов подтверждает корректность работы программы.
\newpage

% Заключение
\section*{Заключение}
\addcontentsline{toc}{section}{Заключение}
В ходе выполнения лабораторной работы были реализованы три версии программы (последовательная и две параллельных), позволяющей осуществлять сортировку массива целых чисел методом Хоара с простым слиянием.
\par Проведенные эксперименты позволили сравнить время работы последовательной и параллельных версий и оценить их эффективность.
\par Разработанный набор тестов позволяет убедиться в корректности работы программы.
\newpage

% Список литературы
\begin{thebibliography}{1}
\addcontentsline{toc}{section}{Список литературы}
\bibitem{source1} Гергель В. П., Стронгин Р. Г. Основы параллельных вычислений для многопроцессорных вычислительных систем. – 2003.
\bibitem{source2} Cormen, Thomas H., et al. Introduction to algorithms. MIT press, 2009.
\bibitem{source3} Tanenbaum, Andrew S. Structured computer organization. Pearson Education India, 2016.
\bibitem{source4} Stroustrup, Bjarne. The C++ programming language. Pearson Education India, 2000.
\bibitem{source5} Chandra, Rohit, et al. Parallel programming in OpenMP. Morgan kaufmann, 2001.
\end{thebibliography}
\newpage

% Приложение
\section*{Приложение}
\addcontentsline{toc}{section}{Приложение}
Исходный код программ, разработанных в ходе лабораторной работы.

\par Последовательная реализация:
\begin{lstlisting}
// hoare_sort_simple_merge.h

// Copyright 2021 Vizgalov Anton

#ifndef MODULES_TASK_1_VIZGALOV_A_HOARE_SORT_SIMPLE_MERGE_HOARE_SORT_SIMPLE_MERGE_H_
#define MODULES_TASK_1_VIZGALOV_A_HOARE_SORT_SIMPLE_MERGE_HOARE_SORT_SIMPLE_MERGE_H_

#include <vector>

int partition(std::vector<int>* vec, int low, int high);
void hoareSort(std::vector<int>* vec, int low, int high);
void merge(std::vector<int>* vec, int left, int mid, int right);

void runHoareSort(std::vector<int>* vec);

std::vector<int> getRandomVector(int size, int type = 0);

#endif  // MODULES_TASK_1_VIZGALOV_A_HOARE_SORT_SIMPLE_MERGE_HOARE_SORT_SIMPLE_MERGE_H_



// hoare_sort_simple_merge.cpp

// Copyright 2021 Vizgalov Anton

#include <random>
#include <limits>
#include <functional>
#include <algorithm>
#include <stdexcept>

#include "../../../modules/task_1/vizgalov_a_hoare_sort_simple_merge/hoare_sort_simple_merge.h"

int partition(std::vector<int>* vec, int low, int high) {
    int pivot = vec->at(high);
    int smallerElementIdx = low - 1;

    for (int i = low; i < high; ++i) {
        if (vec->at(i) < pivot) {
            std::swap(vec->at(++smallerElementIdx), vec->at(i));
        }
    }

    std::swap(vec->at(smallerElementIdx + 1), vec->at(high));

    return smallerElementIdx + 1;
}

void hoareSort(std::vector<int>* vec, int low, int high) {
    if (low < high) {
        int partitionIdx = partition(vec, low, high);

        hoareSort(vec, low, partitionIdx - 1);
        hoareSort(vec, partitionIdx + 1, high);
    }
}

void merge(std::vector<int>* vec, int left, int mid, int right) {
    int leftIdx = 0;
    int rightIdx = 0;
    int mergedIdx = left;
    int sizeLeft = mid - left + 1;
    int sizeRight = right - mid;

    while (leftIdx < sizeLeft && rightIdx < sizeRight) {
        if (vec->at(left + leftIdx) <= vec->at(mid + 1 + rightIdx)) {
            vec->at(mergedIdx) = vec->at(left + leftIdx);
            leftIdx++;
        } else {
            vec->at(mergedIdx) = vec->at(mid + 1 + rightIdx);
            rightIdx++;
        }
        mergedIdx++;
    }

    while (leftIdx < sizeLeft) {
        vec->at(mergedIdx) = vec->at(left + leftIdx);
        leftIdx++;
        mergedIdx++;
    }

    while (rightIdx < sizeRight) {
        vec->at(mergedIdx) = vec->at(mid + 1 + rightIdx);
        rightIdx++;
        mergedIdx++;
    }
}

void runHoareSort(std::vector<int>* vec) {
    int numSegments = 4;  // "nthreads" in parallel version
    int elementsPerSegment = vec->size() / numSegments + ((vec->size() % numSegments == 0) ? 0 : 1);

    for (int i = 0; i < numSegments; ++i) {
        hoareSort(vec,
                  i * elementsPerSegment,
                  std::min((i + 1) * elementsPerSegment - 1, static_cast<int>(vec->size() - 1)));
    }

    for (int i = 0; i < numSegments - 1; ++i) {
        merge(vec,
              0,
              (i + 1) * elementsPerSegment - 1,
              std::min((i + 2) * elementsPerSegment - 1, static_cast<int>(vec->size() - 1)));
    }
}

// Types:
// 0 - any integers
// 1 - only positive integers
// 2 - any integers in non-decreasing order (sorted)
// 3 - any integers in non-increasing order (reverse-sorted)
std::vector<int> getRandomVector(int size, int type) {
    std::vector<int> newVec(size);

    std::random_device dev;
    std::mt19937 randomGenerator(dev());

    if (type == 1) {
        std::uniform_int_distribution<int> randomValue(1, std::numeric_limits<int>::max());

        for (int& item : newVec) {
            item = randomValue(randomGenerator);
        }
    } else if (type >= 0 && type <= 3) {
        std::uniform_int_distribution<int> randomValue(std::numeric_limits<int>::min(),
                                                       std::numeric_limits<int>::max());

        for (int& item : newVec) {
            item = randomValue(randomGenerator);
        }

        if (type == 2) {
            sort(newVec.begin(), newVec.end());
        } else if (type == 3) {
            sort(newVec.begin(), newVec.end(), std::greater<int>());
        }
    } else {
        throw std::invalid_argument("Invalid type of vector chosen");
    }

    return newVec;
}



// main.cpp

// Copyright 2021 Vizgalov Anton

#include <gtest/gtest.h>
#include <algorithm>

#include "./hoare_sort_simple_merge.h"

TEST(Hoare_Sort_Simple_Merge, Hoare_Sort_Empty_Vector) {
    std::vector<int> vec = getRandomVector(0);
    int lastIdx = static_cast<int>(vec.size() - 1);

    ASSERT_NO_THROW(hoareSort(&vec, 0, lastIdx));

    EXPECT_TRUE(std::is_sorted(vec.begin(), vec.end()));
}

TEST(Hoare_Sort_Simple_Merge, Hoare_Sort_Only_Positive_Numbers) {
    std::vector<int> vec = getRandomVector(100, 1);
    int lastIdx = static_cast<int>(vec.size() - 1);

    ASSERT_NO_THROW(hoareSort(&vec, 0, lastIdx));

    EXPECT_TRUE(std::is_sorted(vec.begin(), vec.end()));
}

TEST(Hoare_Sort_Simple_Merge, Hoare_Sort_All_Numbers_Small_Size) {
    std::vector<int> vec = getRandomVector(100);
    int lastIdx = static_cast<int>(vec.size() - 1);

    ASSERT_NO_THROW(hoareSort(&vec, 0, lastIdx));

    EXPECT_TRUE(std::is_sorted(vec.begin(), vec.end()));
}

TEST(Hoare_Sort_Simple_Merge, Hoare_Sort_All_Numbers_Big_Size) {
    std::vector<int> vec = getRandomVector(10000);
    int lastIdx = static_cast<int>(vec.size() - 1);

    ASSERT_NO_THROW(hoareSort(&vec, 0, lastIdx));

    EXPECT_TRUE(std::is_sorted(vec.begin(), vec.end()));
}

TEST(Hoare_Sort_Simple_Merge, Hoare_Sort_Already_Sorted) {
    std::vector<int> vec = getRandomVector(100, 2);
    int lastIdx = static_cast<int>(vec.size() - 1);

    ASSERT_NO_THROW(hoareSort(&vec, 0, lastIdx));

    EXPECT_TRUE(std::is_sorted(vec.begin(), vec.end()));
}

TEST(Hoare_Sort_Simple_Merge, Hoare_Sort_Worst_Case_Unsorted) {
    std::vector<int> vec = getRandomVector(100, 3);
    int lastIdx = static_cast<int>(vec.size() - 1);

    ASSERT_NO_THROW(hoareSort(&vec, 0, lastIdx));

    EXPECT_TRUE(std::is_sorted(vec.begin(), vec.end()));
}

TEST(Hoare_Sort_Simple_Merge, Merge_Left_Empty) {
    std::vector<int> vec = getRandomVector(100);
    int lastIdx = static_cast<int>(vec.size() - 1);
    hoareSort(&vec, 0, lastIdx);

    ASSERT_NO_THROW(merge(&vec, 0, 0, vec.size() - 1));

    EXPECT_TRUE(std::is_sorted(vec.begin(), vec.end()));
}

TEST(Hoare_Sort_Simple_Merge, Merge_Right_Empty) {
    std::vector<int> vec = getRandomVector(100);
    int lastIdx = static_cast<int>(vec.size() - 1);
    hoareSort(&vec, 0, lastIdx);

    ASSERT_NO_THROW(merge(&vec, 0, vec.size() - 1, vec.size() - 1));

    EXPECT_TRUE(std::is_sorted(vec.begin(), vec.end()));
}

TEST(Hoare_Sort_Simple_Merge, Merge_Same_Size) {
    std::vector<int> vec = getRandomVector(100);
    int lastIdx = static_cast<int>(vec.size() - 1);
    int midIdx = vec.size() / 2;
    hoareSort(&vec, 0, midIdx);
    hoareSort(&vec, midIdx + 1, lastIdx);

    ASSERT_NO_THROW(merge(&vec, 0, midIdx, vec.size() - 1));

    EXPECT_TRUE(std::is_sorted(vec.begin(), vec.end()));
}

TEST(Hoare_Sort_Simple_Merge, Merge_Different_Size) {
    std::vector<int> vec = getRandomVector(100);
    int lastIdx = static_cast<int>(vec.size() - 1);
    int divisionIdx = vec.size() / 8;
    hoareSort(&vec, 0, divisionIdx);
    hoareSort(&vec, divisionIdx + 1, lastIdx);

    ASSERT_NO_THROW(merge(&vec, 0, divisionIdx, vec.size() - 1));

    EXPECT_TRUE(std::is_sorted(vec.begin(), vec.end()));
}

TEST(Hoare_Sort_Simple_Merge, Merge_Large_Vectors) {
    std::vector<int> vec = getRandomVector(10000);
    int lastIdx = static_cast<int>(vec.size() - 1);
    int midIdx = vec.size() / 2;
    hoareSort(&vec, 0, midIdx);
    hoareSort(&vec, midIdx + 1, lastIdx);

    ASSERT_NO_THROW(merge(&vec, 0, midIdx, vec.size() - 1));

    EXPECT_TRUE(std::is_sorted(vec.begin(), vec.end()));
}

TEST(Hoare_Sort_Simple_Merge, Run_Hoare_Sort_Even_Size) {
    std::vector<int> vec = getRandomVector(100);

    ASSERT_NO_THROW(runHoareSort(&vec));

    EXPECT_TRUE(std::is_sorted(vec.begin(), vec.end()));
}

TEST(Hoare_Sort_Simple_Merge, Run_Hoare_Sort_Odd_Size) {
    std::vector<int> vec = getRandomVector(111);

    ASSERT_NO_THROW(runHoareSort(&vec));

    EXPECT_TRUE(std::is_sorted(vec.begin(), vec.end()));
}

int main(int argc, char **argv) {
    ::testing::InitGoogleTest(&argc, argv);
    return RUN_ALL_TESTS();
}

\end{lstlisting}

\par Параллельная реализация (OpenMP):
\begin{lstlisting}
// hoare_sort_simple_merge.h

// Copyright 2021 Vizgalov Anton

#ifndef MODULES_TASK_2_VIZGALOV_A_HOARE_SORT_SIMPLE_MERGE_HOARE_SORT_SIMPLE_MERGE_H_
#define MODULES_TASK_2_VIZGALOV_A_HOARE_SORT_SIMPLE_MERGE_HOARE_SORT_SIMPLE_MERGE_H_

#include <vector>

int partition(std::vector<int>* vec, int low, int high);
void hoareSort(std::vector<int>* vec, int low, int high);
void merge(std::vector<int>* vec, int left, int mid, int right);

void runHoareSort(std::vector<int>* vec);

std::vector<int> getRandomVector(int size, int type = 0);

#endif  // MODULES_TASK_2_VIZGALOV_A_HOARE_SORT_SIMPLE_MERGE_HOARE_SORT_SIMPLE_MERGE_H_



// hoare_sort_simple_merge.cpp

// Copyright 2021 Vizgalov Anton

#include <omp.h>
#include <random>
#include <limits>
#include <functional>
#include <algorithm>
#include <stdexcept>

#include "../../../modules/task_2/vizgalov_a_hoare_sort_simple_merge/hoare_sort_simple_merge.h"

int partition(std::vector<int>* vec, int low, int high) {
    int pivot = vec->at(high);
    int smallerElementIdx = low - 1;

    for (int i = low; i < high; ++i) {
        if (vec->at(i) < pivot) {
            std::swap(vec->at(++smallerElementIdx), vec->at(i));
        }
    }

    std::swap(vec->at(smallerElementIdx + 1), vec->at(high));

    return smallerElementIdx + 1;
}

void hoareSort(std::vector<int>* vec, int low, int high) {
    if (low < high) {
        int partitionIdx = partition(vec, low, high);

        hoareSort(vec, low, partitionIdx - 1);
        hoareSort(vec, partitionIdx + 1, high);
    }
}

void merge(std::vector<int>* vec, int left, int mid, int right) {
    int leftIdx = 0;
    int rightIdx = 0;
    int mergedIdx = left;
    int sizeLeft = mid - left + 1;
    int sizeRight = right - mid;

    while (leftIdx < sizeLeft && rightIdx < sizeRight) {
        if (vec->at(left + leftIdx) <= vec->at(mid + 1 + rightIdx)) {
            vec->at(mergedIdx) = vec->at(left + leftIdx);
            leftIdx++;
        } else {
            vec->at(mergedIdx) = vec->at(mid + 1 + rightIdx);
            rightIdx++;
        }
        mergedIdx++;
    }

    while (leftIdx < sizeLeft) {
        vec->at(mergedIdx) = vec->at(left + leftIdx);
        leftIdx++;
        mergedIdx++;
    }

    while (rightIdx < sizeRight) {
        vec->at(mergedIdx) = vec->at(mid + 1 + rightIdx);
        rightIdx++;
        mergedIdx++;
    }
}

void runHoareSort(std::vector<int>* vec) {
    int numSegments = omp_get_max_threads();
    int elementsPerSegment = vec->size() / numSegments + ((vec->size() % numSegments == 0) ? 0 : 1);

    #pragma omp parallel for
    for (int i = 0; i < numSegments; ++i) {
        hoareSort(vec,
                  i * elementsPerSegment,
                  std::min((i + 1) * elementsPerSegment - 1, static_cast<int>(vec->size() - 1)));
    }

    for (int i = 0; i < numSegments - 1; ++i) {
        merge(vec,
              0,
              (i + 1) * elementsPerSegment - 1,
              std::min((i + 2) * elementsPerSegment - 1, static_cast<int>(vec->size() - 1)));
    }
}

// Types:
// 0 - any integers
// 1 - only positive integers
// 2 - any integers in non-decreasing order (sorted)
// 3 - any integers in non-increasing order (reverse-sorted)
std::vector<int> getRandomVector(int size, int type) {
    std::vector<int> newVec(size);

    std::random_device dev;
    std::mt19937 randomGenerator(dev());

    if (type == 1) {
        std::uniform_int_distribution<int> randomValue(1, std::numeric_limits<int>::max());

        for (int& item : newVec) {
            item = randomValue(randomGenerator);
        }
    } else if (type >= 0 && type <= 3) {
        std::uniform_int_distribution<int> randomValue(std::numeric_limits<int>::min(),
                                                       std::numeric_limits<int>::max());

        for (int& item : newVec) {
            item = randomValue(randomGenerator);
        }

        if (type == 2) {
            sort(newVec.begin(), newVec.end());
        } else if (type == 3) {
            sort(newVec.begin(), newVec.end(), std::greater<int>());
        }
    } else {
        throw std::invalid_argument("Invalid type of vector chosen");
    }

    return newVec;
}



// main.cpp

// Copyright 2021 Vizgalov Anton

#include <gtest/gtest.h>
#include <algorithm>

#include "./hoare_sort_simple_merge.h"

TEST(Hoare_Sort_Simple_Merge_OpenMP, Hoare_Sort_Empty_Vector) {
    std::vector<int> vec = getRandomVector(0);
    int lastIdx = static_cast<int>(vec.size() - 1);

    ASSERT_NO_THROW(hoareSort(&vec, 0, lastIdx));

    EXPECT_TRUE(std::is_sorted(vec.begin(), vec.end()));
}

TEST(Hoare_Sort_Simple_Merge_OpenMP, Hoare_Sort_Only_Positive_Numbers) {
    std::vector<int> vec = getRandomVector(100, 1);
    int lastIdx = static_cast<int>(vec.size() - 1);

    ASSERT_NO_THROW(hoareSort(&vec, 0, lastIdx));

    EXPECT_TRUE(std::is_sorted(vec.begin(), vec.end()));
}

TEST(Hoare_Sort_Simple_Merge_OpenMP, Hoare_Sort_All_Numbers_Small_Size) {
    std::vector<int> vec = getRandomVector(100);
    int lastIdx = static_cast<int>(vec.size() - 1);

    ASSERT_NO_THROW(hoareSort(&vec, 0, lastIdx));

    EXPECT_TRUE(std::is_sorted(vec.begin(), vec.end()));
}

TEST(Hoare_Sort_Simple_Merge_OpenMP, Hoare_Sort_All_Numbers_Big_Size) {
    std::vector<int> vec = getRandomVector(10000);
    int lastIdx = static_cast<int>(vec.size() - 1);

    ASSERT_NO_THROW(hoareSort(&vec, 0, lastIdx));

    EXPECT_TRUE(std::is_sorted(vec.begin(), vec.end()));
}

TEST(Hoare_Sort_Simple_Merge_OpenMP, Hoare_Sort_Already_Sorted) {
    std::vector<int> vec = getRandomVector(100, 2);
    int lastIdx = static_cast<int>(vec.size() - 1);

    ASSERT_NO_THROW(hoareSort(&vec, 0, lastIdx));

    EXPECT_TRUE(std::is_sorted(vec.begin(), vec.end()));
}

TEST(Hoare_Sort_Simple_Merge_OpenMP, Hoare_Sort_Worst_Case_Unsorted) {
    std::vector<int> vec = getRandomVector(100, 3);
    int lastIdx = static_cast<int>(vec.size() - 1);

    ASSERT_NO_THROW(hoareSort(&vec, 0, lastIdx));

    EXPECT_TRUE(std::is_sorted(vec.begin(), vec.end()));
}

TEST(Hoare_Sort_Simple_Merge_OpenMP, Merge_Left_Empty) {
    std::vector<int> vec = getRandomVector(100);
    int lastIdx = static_cast<int>(vec.size() - 1);
    hoareSort(&vec, 0, lastIdx);

    ASSERT_NO_THROW(merge(&vec, 0, 0, vec.size() - 1));

    EXPECT_TRUE(std::is_sorted(vec.begin(), vec.end()));
}

TEST(Hoare_Sort_Simple_Merge_OpenMP, Merge_Right_Empty) {
    std::vector<int> vec = getRandomVector(100);
    int lastIdx = static_cast<int>(vec.size() - 1);
    hoareSort(&vec, 0, lastIdx);

    ASSERT_NO_THROW(merge(&vec, 0, vec.size() - 1, vec.size() - 1));

    EXPECT_TRUE(std::is_sorted(vec.begin(), vec.end()));
}

TEST(Hoare_Sort_Simple_Merge_OpenMP, Merge_Same_Size) {
    std::vector<int> vec = getRandomVector(100);
    int lastIdx = static_cast<int>(vec.size() - 1);
    int midIdx = vec.size() / 2;
    hoareSort(&vec, 0, midIdx);
    hoareSort(&vec, midIdx + 1, lastIdx);

    ASSERT_NO_THROW(merge(&vec, 0, midIdx, vec.size() - 1));

    EXPECT_TRUE(std::is_sorted(vec.begin(), vec.end()));
}

TEST(Hoare_Sort_Simple_Merge_OpenMP, Merge_Different_Size) {
    std::vector<int> vec = getRandomVector(100);
    int lastIdx = static_cast<int>(vec.size() - 1);
    int divisionIdx = vec.size() / 8;
    hoareSort(&vec, 0, divisionIdx);
    hoareSort(&vec, divisionIdx + 1, lastIdx);

    ASSERT_NO_THROW(merge(&vec, 0, divisionIdx, vec.size() - 1));

    EXPECT_TRUE(std::is_sorted(vec.begin(), vec.end()));
}

TEST(Hoare_Sort_Simple_Merge_OpenMP, Merge_Large_Vectors) {
    std::vector<int> vec = getRandomVector(10000);
    int lastIdx = static_cast<int>(vec.size() - 1);
    int midIdx = vec.size() / 2;
    hoareSort(&vec, 0, midIdx);
    hoareSort(&vec, midIdx + 1, lastIdx);

    ASSERT_NO_THROW(merge(&vec, 0, midIdx, vec.size() - 1));

    EXPECT_TRUE(std::is_sorted(vec.begin(), vec.end()));
}

TEST(Hoare_Sort_Simple_Merge_OpenMP, Run_Hoare_Sort_Even_Size) {
    std::vector<int> vec = getRandomVector(1000000);

    ASSERT_NO_THROW(runHoareSort(&vec));

    EXPECT_TRUE(std::is_sorted(vec.begin(), vec.end()));
}

TEST(Hoare_Sort_Simple_Merge_OpenMP, Run_Hoare_Sort_Odd_Size) {
    std::vector<int> vec = getRandomVector(1111111);

    ASSERT_NO_THROW(runHoareSort(&vec));

    EXPECT_TRUE(std::is_sorted(vec.begin(), vec.end()));
}

int main(int argc, char **argv) {
    ::testing::InitGoogleTest(&argc, argv);
    return RUN_ALL_TESTS();
}

\end{lstlisting}

\par Параллельная реализация (TBB):
\begin{lstlisting}
// hoare_sort_simple_merge.h

// Copyright 2021 Vizgalov Anton

#ifndef MODULES_TASK_3_VIZGALOV_A_HOARE_SORT_SIMPLE_MERGE_HOARE_SORT_SIMPLE_MERGE_H_
#define MODULES_TASK_3_VIZGALOV_A_HOARE_SORT_SIMPLE_MERGE_HOARE_SORT_SIMPLE_MERGE_H_

#include <vector>

int partition(std::vector<int>* vec, int low, int high);
void hoareSort(std::vector<int>* vec, int low, int high);
std::vector<int> merge(const std::vector<int>& vec1,
                       const std::vector<int>& vec2);

void runHoareSort(std::vector<int>* vec, int numThreads = 4);

std::vector<int> getRandomVector(int size, int type = 0);

#endif  // MODULES_TASK_3_VIZGALOV_A_HOARE_SORT_SIMPLE_MERGE_HOARE_SORT_SIMPLE_MERGE_H_



// hoare_sort_simple_merge.cpp

// Copyright 2021 Vizgalov Anton

#include <tbb/tbb.h>
#include <random>
#include <climits>
#include <functional>
#include <algorithm>
#include <stdexcept>

#include "../../../modules/task_3/vizgalov_a_hoare_sort_simple_merge/hoare_sort_simple_merge.h"

int partition(std::vector<int>* vec, int low, int high) {
    int pivot = vec->at(high);
    int smallerElementIdx = low - 1;

    for (int i = low; i < high; ++i) {
        if (vec->at(i) < pivot) {
            std::swap(vec->at(++smallerElementIdx), vec->at(i));
        }
    }

    std::swap(vec->at(smallerElementIdx + 1), vec->at(high));

    return smallerElementIdx + 1;
}

void hoareSort(std::vector<int>* vec, int low, int high) {
    if (low < high) {
        int partitionIdx = partition(vec, low, high);

        hoareSort(vec, low, partitionIdx - 1);
        hoareSort(vec, partitionIdx + 1, high);
    }
}

std::vector<int> merge(const std::vector<int>& vec1,
                       const std::vector<int>& vec2) {
    std::vector<int> merged(vec1.size() + vec2.size());

    int i = 0;
    int j = 0;
    int k = 0;
    while (i < static_cast<int>(vec1.size())
           && j < static_cast<int>(vec2.size())) {
        if (vec1[i] < vec2[j]) {
            merged[k] = vec1[i];
            i++;
        } else {
            merged[k] = vec2[j];
            j++;
        }
        k++;
    }
    while (i < static_cast<int>(vec1.size())) {
        merged[k] = vec1[i];
        k++;
        i++;
    }
    while (j < static_cast<int>(vec2.size())) {
        merged[k] = vec2[j];
        k++;
        j++;
    }

    return merged;
}

void runHoareSort(std::vector<int>* vec, int numThreads) {
    if (static_cast<int>(vec->size()) < numThreads) {
        hoareSort(vec, 0, vec->size() - 1);
        return;
    }
    if (vec->size() == 1) {
        return;
    }
    tbb::task_scheduler_init init(numThreads);

    int elementsPerSegment = vec->size() / numThreads;

    std::vector<std::vector<int>> segments;
    std::vector<int> segment;

    for (int i = 0; i < numThreads; ++i) {
        if (i == numThreads - 1) {
            segment.insert(segment.end(),
                           vec->begin() + i * elementsPerSegment,
                           vec->end());
        } else {
            segment.insert(segment.end(),
                           vec->begin() + i * elementsPerSegment,
                           vec->begin() + (i + 1) * elementsPerSegment);
        }

        segments.push_back(segment);
        segment.clear();
    }

    tbb::parallel_for(tbb::blocked_range<size_t>(0, segments.size(), 1),
    [&segments](const tbb::blocked_range<size_t>& range) {
        for (size_t i = range.begin(); i != range.end(); ++i) {
            hoareSort(&segments[i], 0, segments[i].size() - 1);
        }
    }, tbb::simple_partitioner());

    *vec = segments[0];
    for (size_t i = 1; i < segments.size(); ++i) {
        *vec = merge(*vec, segments[i]);
    }
}

// Types:
// 0 - any integers
// 1 - only positive integers
// 2 - any integers in non-decreasing order (sorted)
// 3 - any integers in non-increasing order (reverse-sorted)
std::vector<int> getRandomVector(int size, int type) {
    std::vector<int> newVec(size);

    std::random_device dev;
    std::mt19937 randomGenerator(dev());

    if (type == 1) {
        std::uniform_int_distribution<int> randomValue(1, INT_MAX);

        for (int& item : newVec) {
            item = randomValue(randomGenerator);
        }
    } else if (type >= 0 && type <= 3) {
        std::uniform_int_distribution<int> randomValue(INT_MIN, INT_MAX);

        for (int& item : newVec) {
            item = randomValue(randomGenerator);
        }

        if (type == 2) {
            sort(newVec.begin(), newVec.end());
        } else if (type == 3) {
            sort(newVec.begin(), newVec.end(), std::greater<int>());
        }
    } else {
        throw std::invalid_argument("Invalid type of vector chosen");
    }

    return newVec;
}



// main.cpp

// Copyright 2021 Vizgalov Anton

#include <tbb/tbb.h>
#include <gtest/gtest.h>
#include <algorithm>

#include "./hoare_sort_simple_merge.h"

TEST(Hoare_Sort_Simple_Merge_TBB, Hoare_Sort_Empty_Vector) {
    std::vector<int> vec = getRandomVector(0);
    int lastIdx = static_cast<int>(vec.size() - 1);

    ASSERT_NO_THROW(hoareSort(&vec, 0, lastIdx));

    EXPECT_TRUE(std::is_sorted(vec.begin(), vec.end()));
}

TEST(Hoare_Sort_Simple_Merge_TBB, Hoare_Sort_Only_Positive_Numbers) {
    std::vector<int> vec = getRandomVector(100, 1);
    int lastIdx = static_cast<int>(vec.size() - 1);

    ASSERT_NO_THROW(hoareSort(&vec, 0, lastIdx));

    EXPECT_TRUE(std::is_sorted(vec.begin(), vec.end()));
}

TEST(Hoare_Sort_Simple_Merge_TBB, Hoare_Sort_All_Numbers_Small_Size) {
    std::vector<int> vec = getRandomVector(100);
    int lastIdx = static_cast<int>(vec.size() - 1);

    ASSERT_NO_THROW(hoareSort(&vec, 0, lastIdx));

    EXPECT_TRUE(std::is_sorted(vec.begin(), vec.end()));
}

TEST(Hoare_Sort_Simple_Merge_TBB, Hoare_Sort_All_Numbers_Big_Size) {
    std::vector<int> vec = getRandomVector(10000);
    int lastIdx = static_cast<int>(vec.size() - 1);

    ASSERT_NO_THROW(hoareSort(&vec, 0, lastIdx));

    EXPECT_TRUE(std::is_sorted(vec.begin(), vec.end()));
}

TEST(Hoare_Sort_Simple_Merge_TBB, Hoare_Sort_Already_Sorted) {
    std::vector<int> vec = getRandomVector(100, 2);
    int lastIdx = static_cast<int>(vec.size() - 1);

    ASSERT_NO_THROW(hoareSort(&vec, 0, lastIdx));

    EXPECT_TRUE(std::is_sorted(vec.begin(), vec.end()));
}

TEST(Hoare_Sort_Simple_Merge_TBB, Hoare_Sort_Worst_Case_Unsorted) {
    std::vector<int> vec = getRandomVector(100, 3);
    int lastIdx = static_cast<int>(vec.size() - 1);

    ASSERT_NO_THROW(hoareSort(&vec, 0, lastIdx));

    EXPECT_TRUE(std::is_sorted(vec.begin(), vec.end()));
}

TEST(Hoare_Sort_Simple_Merge_TBB, Merge_Left_Empty) {
    std::vector<int> vec1 = getRandomVector(100);
    std::vector<int> vec2 = getRandomVector(0);
    hoareSort(&vec1, 0, vec1.size() - 1);

    std::vector<int> vec = merge(vec1, vec2);

    EXPECT_TRUE(std::is_sorted(vec.begin(), vec.end()));
}

TEST(Hoare_Sort_Simple_Merge_TBB, Merge_Right_Empty) {
    std::vector<int> vec1 = getRandomVector(0);
    std::vector<int> vec2 = getRandomVector(100);
    hoareSort(&vec2, 0, vec2.size() - 1);

    std::vector<int> vec = merge(vec1, vec2);

    EXPECT_TRUE(std::is_sorted(vec.begin(), vec.end()));
}

TEST(Hoare_Sort_Simple_Merge_TBB, Merge_Same_Size) {
    std::vector<int> vec1 = getRandomVector(100);
    std::vector<int> vec2 = getRandomVector(100);
    hoareSort(&vec1, 0, vec1.size() - 1);
    hoareSort(&vec2, 0, vec2.size() - 1);

    std::vector<int> vec = merge(vec1, vec2);

    EXPECT_TRUE(std::is_sorted(vec.begin(), vec.end()));
}

TEST(Hoare_Sort_Simple_Merge_TBB, Merge_Different_Size) {
    std::vector<int> vec1 = getRandomVector(75);
    std::vector<int> vec2 = getRandomVector(25);
    hoareSort(&vec1, 0, vec1.size() - 1);
    hoareSort(&vec2, 0, vec2.size() - 1);

    std::vector<int> vec = merge(vec1, vec2);

    EXPECT_TRUE(std::is_sorted(vec.begin(), vec.end()));
}

TEST(Hoare_Sort_Simple_Merge_TBB, Merge_Large_Vectors) {
    std::vector<int> vec1 = getRandomVector(5000);
    std::vector<int> vec2 = getRandomVector(5000);
    hoareSort(&vec1, 0, vec1.size() - 1);
    hoareSort(&vec2, 0, vec2.size() - 1);

    std::vector<int> vec = merge(vec1, vec2);

    EXPECT_TRUE(std::is_sorted(vec.begin(), vec.end()));
}

TEST(Hoare_Sort_Simple_Merge_TBB, Run_Hoare_Sort_Even_Size) {
    std::vector<int> vec = getRandomVector(100000);

    ASSERT_NO_THROW(runHoareSort(&vec));

    EXPECT_TRUE(std::is_sorted(vec.begin(), vec.end()));
}

TEST(Hoare_Sort_Simple_Merge_TBB, Run_Hoare_Sort_Odd_Size) {
    std::vector<int> vec = getRandomVector(111111);

    ASSERT_NO_THROW(runHoareSort(&vec));

    EXPECT_TRUE(std::is_sorted(vec.begin(), vec.end()));
}

TEST(Hoare_Sort_Simple_Merge_TBB, Compare_Time) {
    std::vector<int> vec1 = getRandomVector(1000000);

    tbb::tick_count t0 = tbb::tick_count::now();
    ASSERT_NO_THROW(runHoareSort(&vec1, 4));
    tbb::tick_count t1 = tbb::tick_count::now();
    std::cout << "Time (parallel): " << (t1 - t0).seconds() << std::endl;

    std::vector<int> vec2 = getRandomVector(1000000);

    tbb::tick_count t3 = tbb::tick_count::now();
    ASSERT_NO_THROW(hoareSort(&vec2, 0, vec2.size() - 1));
    tbb::tick_count t4 = tbb::tick_count::now();
    std::cout << "Time (non-parallel): " << (t4 - t3).seconds() << std::endl;
}

int main(int argc, char **argv) {
    ::testing::InitGoogleTest(&argc, argv);
    return RUN_ALL_TESTS();
}

\end{lstlisting}

\end{document}
