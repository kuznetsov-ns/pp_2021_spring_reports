\documentclass{report}

\usepackage[T2A]{fontenc}
\usepackage[utf8]{luainputenc}
\usepackage[english, russian]{babel}
\usepackage[pdftex]{hyperref}
\usepackage[14pt]{extsizes}
\usepackage{listings}
\usepackage{color}
\usepackage{geometry}
\usepackage{enumitem}
\usepackage{multirow}
\usepackage{graphicx}
\usepackage{indentfirst}
\usepackage{amsmath}

\geometry{a4paper,top=2cm,bottom=3cm,left=2cm,right=1.5cm}
\setlength{\parskip}{0.5cm}
\setlist{nolistsep, itemsep=0.3cm,parsep=0pt}

\lstset{language=C++,
		basicstyle=\footnotesize,
		keywordstyle=\color{blue}\ttfamily,
		stringstyle=\color{red}\ttfamily,
		commentstyle=\color{green}\ttfamily,
		morecomment=[l][\color{magenta}]{\#}, 
		tabsize=4,
		breaklines=true,
  		breakatwhitespace=true,
  		title=\lstname,       
}

\addto\captionsrussian{\renewcommand \contentsname {\LARGE Содержание}}

\makeatletter
\renewcommand\@biblabel[1]{#1.\hfil}
\makeatother

\begin{document}

\begin{titlepage}

\begin{center}
МИНИСТЕРСТВО НАУКИ И ВЫСШЕГО ОБРАЗОВАНИЯ РОССИЙСКОЙ ФЕДЕРАЦИИ\\
Федеральное государственное автономное образовательное учреждение высшего образования \\
\textbf{"Национальный исследовательский Нижегородский государственный университет им. Н.И. Лобачевского" (ННГУ)}
\end{center}

\begin{center}
\textbf{Институт информационных технологий, математики и механики}
\end{center}

\vspace{4em}

\begin{center}
Отчет по лабораторной работе \\
\end{center}
\begin{center}
\textbf{\Large«Умножение разреженных матриц. Элементы комплексного типа. Формат хранения матрицы - столбцовый (CCS).»} \\
\end{center}

\vspace{4em}

\newbox{\lbox}
\savebox{\lbox}{\hbox{text}}
\newlength{\maxl}
\setlength{\maxl}{\wd\lbox}
\hfill\parbox{7cm}{
\hspace*{5cm}\hspace*{-5cm}\textbf{Выполнил:} \\ студент группы 381806-3 \\ Лудина Д.А.\\
\\
\hspace*{5cm}\hspace*{-5cm}\textbf{Проверил:}\\ доцент кафедры МОСТ, \\ кандидат технических наук \\ Сысоев А. В.\\
}
\vspace{\fill}

\begin{center} Нижний Новгород \\ 2021 \end{center}

\end{titlepage}

\setcounter{page}{2}
% содержание
\tableofcontents{}
\clearpage

\begin{center}
\section*{Введение}
\end{center}
\addcontentsline{toc}{section}{Введение}
\parРазреженная матрица - это матрица, состоящая приемущественно из нулевых элементов. Строгого определения того, сколько элементов должно быть нулевыми, чтобы матрица считалась разреженной не существует, но общий критерий состоит в том, что количество ненулевых элементов приблизительно равно числу строк или столбцов.
\par Так как в разреженных матрицах большинство значений - нули, то любая арифметическая операция с этими матрицами увеличивает вычислительные затраты. Для того чтобы их снизить
были придуманы разные способы хранений разреженных матриц. Самые распространенные из них это координатный, строковый(CRS) и столбцовый(CCS) форматы. В данной работе будет рассмотрен только последний из перечисленных.\\
Столбцовый формат хранения представляет собой структуру данных, позволяющую хранить матрицу в виде трех массивов. Первый хранит значениия ненулевых элементов исходной матрицы. Второй массив хранит индексы строк для каждого ненулевого элемента. Третий хранит индекс начала каждого столбца.
\par Умножение является одной из основных операций с матрицами, в том числе и с рязреженными. Поэтому, в данной работе рассматривается алгоритм умножения разреженных матриц, хранящихся в столбцовом формате, а также примененение к нему разных методов распараллеливания.
\newpage

\begin{center}
\section*{Постановка задач}
\end{center}
\addcontentsline{toc}{section}{Постановка задач}
В данной лабораторной работе необходимо:
\begin{enumerate}
\item Реализовать последовательный алгоритм умножения разреженных матриц в столбцовом формате(CCS).
\item Реализовать параллельный алгоритм умножения разреженных матриц в столбцовом формате(CCS) с использованием технологии OpenMP.
\item Реализовать параллельный алгоритм умножения разреженных матриц в столбцовом формате(CCS) с использованием технологии TBB.
\item Провести вычислительные эксперементы и сравнить время работы трех алгоритмов.
\item Сделать вывод на основе полученных результатов.
\end{enumerate}
\newpage

\begin{center}
\section*{Описание алгоритма}
\end{center}
\addcontentsline{toc}{section}{Описание алгоритма}
Алгоритм преобразования матрицы в столбцовый формат хранения:
\par Проходим по столбцу матрицы, в нем по каждой строке. Если значение не нулевое, то увеличиваем переменную, хранящую количество ненулевых элементов. В вектор значений помещаем данный элемент и записываем индекс строки в соответсвующий вектор.
По окончании внутреннего цикла заполняем вектор начала нового столбца в векторе значений.\\
\\
Алгоритм транспонирования:
\begin{enumerate}
\item Создаем три вектора, которые будут хранить транспонированную матрицу в CCS формате.
\item Просматриваем масив индексов по строкам исходной матрицы и считаем количество в каждой строке(столбце транспонированной матрицы). Записываем результат в третий вектор.
\item Подсчитываем индексы начала каждого столбца в транспонированной матрицы.
\item Правильно распологаем в структуре результирующей матрицы значения из исходной.
\item Возвращаем транспонированную матрицу.
\end{enumerate}
Алгоритм умножения двух разреженных матриц:
\begin{enumerate}
\item Создаем результирующую матрицу.
\item Транспонируем матрицу A. Это необходимо сделать, так как умножение производится строка на столбец, а формат хранения матриц столбцовый, поэтому рациональнее сначала транспонировать первую матрицу.
\item Проходим по столбцам матрицы B и для каждого перебираем столбцы транспонированной матрицы A.
\item Выполняем скалярное умножение. Если оно не ранво нулю, то заполняем результирующие векторы для столбцового формата хранения.
\item После того, как мы перебрали все столбцы матрицы B, записываем в вектор, хранящий начало каждого столбца, число ненулевых элемнтов.
\item Возвращаем результирующую матрицу.
\end{enumerate}
\newpage

\begin{center}
\section*{Схема распараллеливания}
\end{center}
\addcontentsline{toc}{section}{Схема распараллеливания}
\par Главной идеей распараллеливания алгоритма умножения разреженных матриц в столбцовом формате хранения является перебор столбцов матрицы B, то есть распараллеливание происходит по внешнему циклу. Мы распределяем столбцы между потоками и работаем с ними независимо.
\par Для более эффективного результата необходимо оптимизировать сам алгоримт умножения. Добавляем вектор, хранящий индексы расположения ненулевых элементов в столбце и инициализируем его -1.
Проходим по каждому столбцу матрицы В и заполняем новый вектор. Затем рассматриваем все ненулевые элементы столбца матрицы А и проверяем значение. Если оно равно -1(в первом векторе нет соответсвующего элемента), то выполянть умножение не надо. В противном случае, выполняем умножение.
\newpage

\begin{center}
\section*{Описание программной реализации}
\end{center}
\addcontentsline{toc}{section}{Описание программной реализации}
\par Разреженная матрица в столбцовом формате хранения представлена следующей структурой:
\begin{lstlisting}
struct SparseMatrix {
  int rows, cols;
  std::vector<int> col_idx;
  std::vector<int> row;
  std::vector<std::complex<int>> value;

  SparseMatrix() : rows(0), cols(0) {}
  SparseMatrix(int _rows, int _cols) : rows(_rows), cols(_cols) {}
  SparseMatrix(int _rows, int _cols,
    const std::vector<int>& _col_idx, const std::vector<int>& _row,
    const std::vector<std::complex<int>>& _value) : rows(_rows), cols(_cols), col_idx(_col_idx),
    row(_row), value(_value) {}
  SparseMatrix(int _rows, int _cols, std::vector<std::complex<int>> matrix);
  SparseMatrix transpose();
};
\end{lstlisting}
\par Реализация алгоритма перевода матрицы в столбцовый формат хранения представлена функцией:
\begin{lstlisting}
  SparseMatrix(int _rows, int _cols, std::vector<std::complex<int>> matrix);
\end{lstlisting}
В качетве входных парамтров передаются число строк, число столбцов и вектор значений. На выходе результирующая матрица в столбцовом формате.
\par Реализация алгоритма транпонирования представлена функцией:
\begin{lstlisting}
  SparseMatrix transpose();
\end{lstlisting}
Выходным параметром является траспонированная разреженная матрица.
\par Функция умножения двух разреженных матриц:
\begin{lstlisting}
  SparseMatrix Multiplication(SparseMatrix A, SparseMatrix B);
\end{lstlisting}
На вход подаются две разреженные матрицы, на выходе результирующая матрица.
\par Реализация параллельного алгоритма умножения представлена в следующих функциях:
\begin{lstlisting}
  SparseMatrix MultiplicationOpenMP(SparseMatrix A, SparseMatrix B);
  SparseMatrix MultiplicationTBB(SparseMatrix A, SparseMatrix B);
\end{lstlisting}
На вход подаются две разреженные матрицы, на выходе результирующая матрица.
\par Также в программе реализована вспомогательная функция, генерирующая разреженную матрицу
\begin{lstlisting}
  SparseMatrix getRandomSparseMatrix(int rows, int cols, int notZero);
\end{lstlisting}
В качетве входных парамтров передаются число строк, число столбцов и число ненулевых значений в каждом столбце. На выходе результирующая матрица в столбцовом формате хранения.
\newpage

\begin{center}
\section*{Подтверждение корректности}
\end{center}
\addcontentsline{toc}{section}{Подтверждение корректности}
\par Для подтверждения корректности в программе реализована система тестов, разработанная с помощью использования Google C++ Testing Framework.
\par Эта система включает в себя тесты, проверяющие все алгоритмы написанной программы. При проверке умножения в последовательной реализации сравнивались вектора столбцового формата хранения результирующей матрицы с ожидаемыми. А в паралльленой версии сравниваются значение последовательного алгоритма умножения разреженных матриц с параллельным.
\par Успешное прохождение данных тестов доказывает корректность работы всей программы.
\newpage

\begin{center}
\section*{Результаты эксперементов}
\end{center}
\addcontentsline{toc}{section}{Результаты эксперементов}
\par Вычислительные эксперименты для оценки эффективности алгоритма умножения двух разреженных матриц в столбцовом формате хранения проводились на оборудовании со следующей аппаратной конфигурацией:
\begin{enumerate}
\item Процессор: Intel(R) Core(TM) i5-8250U CPU @ 1.60GHz 1.80GHz, ядер: 4, логических процессоров: 8;
\item Оперативная память: 6 Gb, 2133 MHz;
\item Операционная система: Windows 10 Домашняя;
\item Среда разработки: Visual Studio 2019.
\end{enumerate}
В тестах умножались матрицы размером 1000$\times$1000, 2000$\times$2000, 3000$\times$3000, 4000$\times$4000, 5000$\times$5000, 10000$\times$10000. Количество используемых потоков определялось автоматически.
\begin{table}[!h]
\caption{Результаты вычислительных экспериментов. Сравнение последовательного алгоритма с OpenMP}
\centering
\begin{tabular}{|p{4cm}|p{4cm}|p{4cm}|p{3cm}|}
\hline
Размер матрицы & Последовательный алгоритм,сек & Параллельный алгоритм,сек & Ускорение  \\\hline
1000 & 0.0036702 & 0.0033403 & 1.0988  \\\hline
2000  & 0.0425559 & 0.0080879 & 5.262  \\\hline
3000  & 0.108678 & 0.0276799 & 3.926  \\\hline
4000  & 0.169397 & 0.0418563 & 4.047  \\\hline
5000  & 0.256634 & 0.0678537 & 3.782  \\\hline
10000 & 0.962818 & 0.242603 & 3.989  \\
\hline
\end{tabular}
\end{table}

\begin{table}[!h]
\caption{Результаты вычислительных экспериментов. Сравнение последовательного алгоритма с TBB}
\centering
\begin{tabular}{|p{4cm}|p{4cm}|p{4cm}|p{3cm}|}
\hline
Размер матрицы & Последовательный алгоритм,сек & Параллельный алгоритм,сек & Ускорение  \\\hline
1000  & 0.0128584 & 0.0113118 & 1.138  \\\hline
2000  & 0.0275358 & 0.0091291 & 3.016  \\\hline
3000  & 0.0806581 & 0.0203535 & 3.982  \\\hline
4000  & 0.137639 & 0.0343931 & 4.002  \\\hline
5000  & 0.23257 & 0.0641605 & 3.625  \\\hline
10000 & 1.144314 & 0.325898 & 3.511 \\
\hline
\end{tabular}
\end{table}

\par На основе полученных данных, можно сделать вывод, что в большинстве случаев параллельный алгоритм с использованием OpenMP более эффективен. Максимальное ускорение в 4 раза.
\newpage

\begin{center}
\section*{Заключение}
\end{center}
\addcontentsline{toc}{section}{Заключение}
\par В ходе данной лабораторной работы мною были реализованы три алгоритма умножения двух разреженных матриц в столбцовом формате хранения: последовательное умножение, параллельнное умножение с импользованием технологии OpenMP и параллельнное умножение с импользованием технологии TBB. Кроме того, был реализован алгоритм транспонирования матрицы и преобразование матрицы в столбцовый формат хранения. А также были разработаны и доведены до успешного выполнения тесты, созданные для данного программного проекта с использованием Google C++ Testing Framework и необходимые для подтверждения корректности работы программы.
\par Основной целью работы была реализация параллельной версии алгоритма умножения, которая должна работать быстрее по времени, чем последовательная.
Вычислительные эксперименты нам это показали.
\newpage

\addcontentsline{toc}{section}{Литература}
\begin{thebibliography}{}
\bibitem{1} Гергель В. П. Теория и практика параллельных вычислений. – 2007.
\bibitem{2} Гергель В. П., Стронгин Р. Г. Основы параллельных вычислений для многопроцессорных вычислительных систем. – 2003.
\bibitem{3}Wikipedia: the free encyclopedia [Электронный ресурс] // URL: https://ru-wiki.ru/wiki/\verb|Разреженная_матрица| (дата обращения: 15.03.2021)
\end{thebibliography}{}
\newpage

\begin{center}
\section*{Приложение}
\end{center}
\addcontentsline{toc}{section}{Приложение}
В этом разделе находится листинг всего кода, написанного в рамках лабораторной работы.
\begin{lstlisting}
// sparse_matrix_complex_ccs.h

// Copyright 2021 Ludina Daria
#ifndef MODULES_TASK_3_LUDINA_D_SPARSE_MATRIX_COMPLEX_CCS_SPARSE_MATRIX_COMPLEX_CCS_H_
#define MODULES_TASK_3_LUDINA_D_SPARSE_MATRIX_COMPLEX_CCS_SPARSE_MATRIX_COMPLEX_CCS_H_

#include <omp.h>
#include <tbb/tbb.h>
#include <complex>
#include <vector>
#include <random>
#include <ctime>
#include <iostream>

struct SparseMatrix {
  int rows, cols;
  std::vector<int> col_idx;
  std::vector<int> row;
  std::vector<std::complex<int>> value;

  SparseMatrix() : rows(0), cols(0) {}
  SparseMatrix(int _rows, int _cols) : rows(_rows), cols(_cols) {}
  SparseMatrix(int _rows, int _cols,
    const std::vector<int>& _col_idx, const std::vector<int>& _row,
    const std::vector<std::complex<int>>& _value) : rows(_rows), cols(_cols), col_idx(_col_idx),
    row(_row), value(_value) {}
  SparseMatrix(int _rows, int _cols, std::vector<std::complex<int>> matrix);
  SparseMatrix transpose();
};

SparseMatrix Multiplication(SparseMatrix A, SparseMatrix B);
SparseMatrix getRandomSparseMatrix(int rows, int cols, int notZero);
SparseMatrix MultiplicationOpenMP(SparseMatrix A, SparseMatrix B);
SparseMatrix MultiplicationTBB(SparseMatrix A, SparseMatrix B);

#endif  // MODULES_TASK_3_LUDINA_D_SPARSE_MATRIX_COMPLEX_CCS_SPARSE_MATRIX_COMPLEX_CCS_H_


//sparse_matrix_complex_ccs.cpp

// Copyright 2021 Ludina Daria
#include "../../../modules/task_3/ludina_d_sparse_matrix_complex_ccs/sparse_matrix_complex_ccs.h"

SparseMatrix::SparseMatrix(int _rows, int _cols, std::vector<std::complex<int>> matrix) {
  rows = _rows;
  cols = _cols;
  int not_zero_value = 0;
  col_idx.push_back(0);

  for (int j = 0; j < _cols; j++) {
    for (int i = 0; i < _rows; i++) {
      if (matrix[cols * i + j] != std::complex<int>(0, 0)) {
        not_zero_value++;
        value.push_back(matrix[cols * i + j]);
        row.push_back(i);
      }
    }
    col_idx.push_back(not_zero_value);
  }
}

SparseMatrix SparseMatrix::transpose() {
  std::vector<std::complex<int>> valueT(value.size());
  std::vector<int> col_idxT(col_idx.size());
  std::vector<int> rowT(row.size());
  int size = row.size();
  for (int i = 0; i < size; i++) {
    col_idxT[row[i] + 1]++;
  }
  int tmp, index, r;
  int s = 0;
  for (int i = 1; i <= rows; i++) {
    tmp = col_idxT[i];
    col_idxT[i] = s;
    s += tmp;
  }
  for (int i = 0; i < cols; i++) {
    for (int j = col_idx[i]; j < col_idx[i + 1]; j++) {
      r = row[j];
      index = col_idxT[r + 1];
      rowT[index] = i;
      valueT[index] = value[j];
      col_idxT[r + 1]++;
    }
  }
  SparseMatrix res = SparseMatrix(rows, cols, col_idxT, rowT, valueT);
  return res;
}

SparseMatrix Multiplication(SparseMatrix A, SparseMatrix B) {
  if (A.cols != B.rows)
    throw "Size col A not equal size row B";
  SparseMatrix result(A.rows, B.cols);
  A = A.transpose();
  result.col_idx.push_back(0);
  for (int i = 0; i < B.cols; i++) {
    for (int j = 0; j < A.cols; j++) {
      std::complex<int> sum = { 0, 0 };
      for (int k = B.col_idx[i]; k < B.col_idx[i + 1]; k++) {
        for (int n = A.col_idx[j]; n < A.col_idx[j + 1]; n++) {
          if (B.row[k] == A.row[n]) {
            sum += B.value[k] * A.value[n];
            break;
          }
        }
      }
      if (sum != std::complex<int>(0, 0)) {
        result.value.push_back(sum);
        result.row.push_back(j);
      }
    }
    result.col_idx.push_back(static_cast<int>(result.value.size()));
  }
  return result;
}

SparseMatrix getRandomSparseMatrix(int rows, int cols, int notZero) {
  SparseMatrix result;
  std::vector<std::complex<int>> val;
  std::vector<int> col_i, r;
  col_i.push_back(0);
  std::mt19937 gen;
  gen.seed(static_cast<unsigned int>(time(0)));
  for (int j = 0; j < cols; j++) {
    for (int i = 0; i < notZero; i++) {
      val.push_back(std::complex<int>(
        static_cast<int>(gen() % 100),
        static_cast<int>(gen() % 100)));
      r.push_back(gen() % cols);
    }
    col_i.push_back(static_cast<int>(r.size()));
  }
  result.value = val;
  result.row = r;
  result.col_idx = col_i;
  return result;
}

SparseMatrix MultiplicationOpenMP(SparseMatrix A, SparseMatrix B) {
  if (A.cols != B.rows)
    throw "Size col A not equal size row B";
  SparseMatrix res(A.rows, B.cols);
  std::vector<std::vector<std::complex<int>>> val(res.cols);
  std::vector<std::vector<int>> r(res.cols);
  A = A.transpose();
  res.col_idx.push_back(0);
  int i, j, k;

#pragma omp parallel
  {
    std::vector<int> tmp(res.cols);
#pragma omp for private(j, k) schedule(static)
    for (i = 0; i < res.cols; i++) {
      tmp.assign(res.cols, -1);
      for (j = B.col_idx[i]; j < B.col_idx[i + 1]; j++) {
        tmp[B.row[j]] = j;
      }
      for (j = 0; j < res.cols; j++) {
        std::complex<int> sum = { 0, 0 };
        for (k = A.col_idx[j]; k < A.col_idx[j + 1]; k++) {
          if (tmp[A.row[k]] != -1) {
            sum += B.value[tmp[A.row[k]]] * A.value[k];
          }
        }
        if (sum != std::complex<int>(0, 0)) {
          val[i].emplace_back(sum);
          r[i].emplace_back(j);
        }
      }
    }
  }
  for (int i = 0; i < res.cols; i++) {
    res.value.insert(res.value.end(), val[i].begin(), val[i].end());
    res.row.insert(res.row.end(), r[i].begin(), r[i].end());
    res.col_idx.insert(res.col_idx.end(), r[i].begin(), r[i].end());
  }
  return res;
}

SparseMatrix MultiplicationTBB(SparseMatrix A, SparseMatrix B) {
  if (A.cols != B.rows)
    throw "Size col A not equal size row B";
  SparseMatrix res(A.rows, B.cols);
  std::vector<std::vector<std::complex<int>>> val(res.cols);
  std::vector<std::vector<int>> res_row(res.cols);
  A = A.transpose();
  res.col_idx.push_back(0);
  tbb::task_scheduler_init init;
  int my_grainsize = 10;
  tbb::parallel_for(tbb::blocked_range<int>(0, B.cols, my_grainsize),
    [&](const tbb::blocked_range<int>& r) {
    std::vector<int> tmp(res.cols);
    for (int i = r.begin(); i < r.end(); i++) {
      tmp.assign(res.cols, -1);
      for (int j = B.col_idx[i]; j < B.col_idx[i + 1]; j++) {
        tmp[B.row[j]] = j;
      }
      for (int j = 0; j < res.cols; j++) {
        std::complex<int> sum = { 0, 0 };
        for (int k = A.col_idx[j]; k < A.col_idx[j + 1]; k++) {
          if (tmp[A.row[k]] != -1) {
            sum += B.value[tmp[A.row[k]]] * A.value[k];
          }
        }
        if (sum != std::complex<int>(0, 0)) {
          val[i].emplace_back(sum);
          res_row[i].emplace_back(j);
        }
      }
    }
  });
  init.terminate();
  for (int i = 0; i < res.cols; i++) {
    res.value.insert(res.value.end(), val[i].begin(), val[i].end());
    res.row.insert(res.row.end(), res_row[i].begin(), res_row[i].end());
    res.col_idx.insert(res.col_idx.end(), res_row[i].begin(), res_row[i].end());
  }
  return res;
}


//main.cpp

// Copyright 2021 Ludina Daria
#include <gtest/gtest.h>
#include "../../../modules/task_3/ludina_d_sparse_matrix_complex_ccs/sparse_matrix_complex_ccs.h"

TEST(Sparse_Matrix_CCS_3, CreateCCS) {
  std::vector<std::complex<int>> a = {
    {0, 0},  {3, 0}, {0, 0}, {7, 0},
    {0, 0}, {0, 0},  {5, 0}, {0, 0},
    {8, 0},  {0, 0},  {0, 0}, {0, 0},
    {0, 0},  {0, 0},  {1, 0}, {11, 0}
  };
  SparseMatrix SMatrix(4, 4, a);
  std::vector<int> row1 = { 2, 0, 1, 3, 0, 3 };
  std::vector<int> colidx = { 0, 1, 2, 4, 6 };
  std::vector<std::complex<int>> val = { {8, 0} , {3, 0}, {5, 0}, {1, 0}, {7, 0}, {11, 0} };
  ASSERT_EQ(val, SMatrix.value);
  ASSERT_EQ(row1, SMatrix.row);
  ASSERT_EQ(colidx, SMatrix.col_idx);
}

TEST(Sparse_Matrix_CCS_3, Transpose) {
  std::vector<std::complex<int>> a = {
    {0, 0},  {3, 0}, {0, 0}, {7, 0},
    {0, 0}, {0, 0},  {8, 0}, {0, 0},
    {0, 0},  {0, 0},  {0, 0}, {0, 0},
    {9, 0},  {0, 0},  {15, 0}, {16, 0}
  };
  SparseMatrix SMatrix(4, 4, a);
  SparseMatrix SMatrixT = SMatrix.transpose();
  std::vector<int> row1 = { 1, 3, 2, 0, 2, 3 };
  std::vector<int> colidx = { 0, 2, 3, 3, 6 };
  std::vector<std::complex<int>> val = { {3, 0} , {7, 0}, {8, 0}, {9, 0}, {15, 0}, {16, 0} };
  ASSERT_EQ(val, SMatrixT.value);
  ASSERT_EQ(row1, SMatrixT.row);
  ASSERT_EQ(colidx, SMatrixT.col_idx);
}

TEST(Sparse_Matrix_CCS_3, Multiplication_1) {
  std::vector<std::complex<int>> a = {
    {1, 0}, {0, 0},
    {0, 0}, {2, 0} };
  SparseMatrix A(2, 2, a);
  std::vector<std::complex<int>> b = {
    {0, 0}, {0, 0},
    {0, 0}, {0, 0} };
  SparseMatrix B(2, 2, b);
  SparseMatrix result = Multiplication(A, B);
  std::vector<std::complex<int>> res = {};
  ASSERT_EQ(res, result.value);
}

TEST(Sparse_Matrix_CCS_3, Multiplication_2) {
  std::vector<std::complex<int>> a = {
    {0, 0}, {5, 0}, {0, 0},
    {3, 4}, {0, 0}, {1, -2},
    {1, 0}, {0, 0}, {0, 0} };
  SparseMatrix A(3, 3, a);
  std::vector<std::complex<int>> b = {
    {0, 0}, {8, 0}, {0, 0},
    {1, 0}, {2, -4}, {0, 0},
    {0, 0}, {0, 0}, {2, -1} };
  SparseMatrix B(3, 3, b);
  SparseMatrix result = Multiplication(A, B);
  std::vector<std::complex<int>> res = { {5, 0}, {10, -20}, {24, 32}, {8, 0}, {0, -5} };
  ASSERT_EQ(res, result.value);
}

TEST(Sparse_Matrix_CCS_3, Multiplication_TBB) {
  std::vector<std::complex<int>> a = {
    {0, 0}, {5, 0}, {0, 0},
    {3, 4}, {0, 0}, {1, -2},
    {1, 0}, {0, 0}, {0, 0} };
  SparseMatrix A(3, 3, a);
  std::vector<std::complex<int>> b = {
    {0, 0}, {8, 0}, {0, 0},
    {1, 0}, {2, -4}, {0, 0},
    {0, 0}, {0, 0}, {2, -1} };
  SparseMatrix B(3, 3, b);
  SparseMatrix result = MultiplicationTBB(A, B);
  std::vector<std::complex<int>> res = { {5, 0}, {10, -20}, {24, 32}, {8, 0}, {0, -5} };
  ASSERT_EQ(res, result.value);
}

TEST(Sparse_Matrix_CCS_3, Multiplication_TBB_2) {
  SparseMatrix A = getRandomSparseMatrix(100, 100, 20);
  SparseMatrix B = getRandomSparseMatrix(100, 100, 40);
  SparseMatrix result_1 = Multiplication(A, B);
  SparseMatrix result_2 = MultiplicationTBB(A, B);
  ASSERT_EQ(result_1.value, result_2.value);
  ASSERT_EQ(result_1.col_idx, result_2.col_idx);
  ASSERT_EQ(result_1.row, result_2.row);
}

int main(int argc, char** argv) {
  ::testing::InitGoogleTest(&argc, argv);
  return RUN_ALL_TESTS();
}
\end{lstlisting}
\end{document}
