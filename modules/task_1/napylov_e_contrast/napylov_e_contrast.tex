\documentclass{report}

\usepackage[T2A]{fontenc}
\usepackage[utf8]{luainputenc}
\usepackage[english, russian]{babel}
\usepackage[pdftex]{hyperref}
\usepackage[14pt]{extsizes}
\usepackage{listings}
\usepackage{color}
\usepackage{geometry}
\usepackage{enumitem}
\usepackage{multirow}
\usepackage{graphicx}
\usepackage{indentfirst}

\geometry{a4paper,top=2cm,bottom=3cm,left=2cm,right=1.5cm}
\setlength{\parskip}{0.5cm}
\setlist{nolistsep, itemsep=0.3cm,parsep=0pt}

\addto\captionsrussian{\renewcommand \contentsname {\LARGE Содержание}}

\lstset{
    language=C++,
	basicstyle=\footnotesize,
	keywordstyle=\color{blue}\ttfamily,
	stringstyle=\color{red}\ttfamily,
	commentstyle=\color{green}\ttfamily,
	morecomment=[l][\color{magenta}]{\#}, 
	tabsize=2,
	title=\lstname,       
}

\begin{document}

	\begin{titlepage}

		\begin{center}
			Министерство науки и высшего образования Российской Федерации
		\end{center}

		\begin{center}
			Федеральное государственное автономное образовательное учреждение высшего образования \\
			Национальный исследовательский Нижегородский государственный университет им. Н.И. Лобачевского
		\end{center}

		\begin{center}
			Институт информационных технологий, математики и механики
		\end{center}

		\vspace{4em}

		\begin{center}
			\textbf{\LargeОтчет по лабораторной работе} \\
		\end{center}
		\begin{center}
			\textbf{\Large«Повышение контраста полутонового изображения посредством линейной растяжки гистограммы»} \\
		\end{center}

		\vspace{4em}

		\newbox{\lbox}
		\savebox{\lbox}{\hbox{text}}
		\newlength{\maxl}
		\setlength{\maxl}{\wd\lbox}
		\hfill\parbox{7cm}{
			\hspace*{5cm}\hspace*{-5cm}\textbf{Выполнил:} \\ студент группы 381806-2 \\ Напылов Е. И.\\
			\\
			\hspace*{5cm}\hspace*{-5cm}\textbf{Проверил:}\\ доцент кафедры МОСТ, \\ кандидат технических наук \\ Сысоев А. В.\\
		}
		\vspace{\fill}

		\begin{center} Нижний Новгород \\ 2021 \end{center}

	\end{titlepage}


	\tableofcontents

	\newpage
	

	\section*{Введение}
	\addcontentsline{toc}{section}{Введение}
	\par Контрастность изображения - это разница интенсивностей пикселей самой светлой и самой тёмной частей изображения. Для анализа контраста часто используют гистограмму. Гистограмма - это график, показывающий распределение количества пикселей каждой интенсивности. Распределение для изображения с хорошим контрастом приближается к равномерному. Если интенсивности сосредоточены в одном месте, то такое изображение обладает низким контрастом.
	
	\par Достаточно часто фотографии приходится делать при плохих световых условиях. Это относится и к изображениям, полученным с помощью автоматических систем. Фотографии, полученные при таких услових, обладают низкой детализацией. Анализ и распознавание таких изображений без предварительной обработки приводит к плохим результатам. Если перед анализом провести даже самую простую обработку - перевести в оттенки серого и выполнить коррекцию гистограммы, то результаты станут в разы лучше.
	
	\par Самым простым способом повышения контраста является алгоритм линейного растяжения гистограммы. Этот способ хорошо работает для изображений с узким диапазоном яркостей. Идея заключается в том, чтобы растянуть узкий интервал яркостей исходного изображения в более широкий. Перед растяжением можно применить фильтр, который уберет не информативные края гистограммы. Этот прием может усилить эффект от линейной коррекции.
	
	\newpage
    \section*{Постановка задачи}
    \addcontentsline{toc}{section}{Постановка задачи}
    \par Целью работы является изучение и реализация алгоритма улучшения контраста изображения с помощью алгоритма линейного растяжения гистограммы и его ускорение с помощью использования параллельных вычислений. Требуется реализовать последовательный алгоритм и убедится в его корректности. Затем на основе последовательного алгоритма необходимо разработать параллельные версии с использованием библиотек OpenMP, TBB, Thread. После тестирования параллельных версий программы необходимо сравнить производительность с последовательной версией. Для проверки корректности работы алгоритмов так же требуется разработать программу, которая будет применять их к реальным изображениям.
    
    \newpage
    \section*{Описание алгоритма}
    \addcontentsline{toc}{section}{Описание алгоритма}
    \par Алгоритм линейного растяжения состоит из двух частей. 
    \par Сначала производится поиск максимального и минимального значений пикселей.
    \par На втором шаге производится замена пикселя по формуле: \\
    $g(x, y) = \frac{f(x,y)-min}{max - min} * 255$ (1), где \\$g(x, y)$ - новое значение пикселя с координатами $x, y$, \\$f(x, y)$ - старое значение, \\$max, min$ - значения, полученные на предыдущем шаге.
    
    \par Данная формула применяется к каждому пикселю изображения.
    \par Изображение хранится в виде линейного вектора, элементами которого являются значения пикселей в виде unsigned char. Данная схема хранения является оптимальной, т.к. в кэш процессора попадает больше необходимых данных, а использование типа unsigned char позволяет использовать минимальное количество памяти для хранения 8-битных полутоновых изображений. 
    
    \newpage
    \section*{Схема распараллеливания}
    \addcontentsline{toc}{section}{Схема распараллеливания}
    
    \par Алгоритм повышения контраста состоит только из циклов. Следовательно, он хорошо поддается распараллеливанию.
    \par Сначала производится параллельный поиск минимального и максимального пикселей. Для того, чтобы не замедлять программу критическими секциями используются два вектора, размер которых равен числу потоков. Каждый поток ищет минимум и максимум в своем блоке данных. Локальные значения помещаются в ячейку вектора, соответствующую номеру данного потока. После выполнения параллельного поиска этих значений происходит редукция - из локальных значений выбирается максимум и минимум. Эта операция выполняется последовательно.
    \par После поиска минимального и максимального значения происходит растяжение гистограммы с помощью формулы (1). Как и в предыдущем шаге изображение делится на блоки, которые обрабатываются потоками.
    \par Стоит отметить, что в реализации на OpenMP и TBB используется автоматическое распределение данных. В случае с std Thread это происходит вручную: всем потокам достается одинаковое количество данных, за исключением последнего - ему достается еще и остаток от деления.
    \par Во всех параллельных версиях использовалась только автоматическая синхронизация при выходе из параллельной области.
    
    \newpage
    \section*{Описание программной реализации}
    \addcontentsline{toc}{section}{Описание программной реализации}
    \par Изображение хранится в векторе с типом unsigned char. Для удобства использован синоним:
    \begin{lstlisting}
    typedef std::vector<unsigned char> VecImage;
    \end{lstlisting}
    \textbf{Open MP версия}
    \begin{lstlisting}
    VecImage add_contrast_omp(VecImage image)
    \end{lstlisting}
    \par Поиск минимума и максимума находится в отдельной функции.
    \begin{lstlisting}
    std::pair<unsigned char, unsigned char> minmax_omp(const VecImage& image);
    \end{lstlisting}
    \par Для поиска локальных минимумов и максимумов создается параллельная область, а цикл распараллелен с параметрами по умолчанию (статическое распределение). После выхода из параллельной области происходит редукция локальных значений.
    \begin{lstlisting}
    #pragma omp parallel
    {
    int thread_id = omp_get_thread_num();
    #pragma omp for
    for (int i = 0; i < static_cast<int>(image.size()); i++) {/*local min max*/}
    }
    \end{lstlisting}
    \par Цикл, преобразующий пиксели, распараллелен как поиск минимума и максимума, но без редукции.
    \begin{lstlisting}
    #pragma omp parallel for
    for (int i = 0; i < static_cast<int>(image.size()); i++) {/*change pixels(1)*/}
    \end{lstlisting}
    % -------------------------------------------------------------------------------------
    \newpage
    \textbf{TBB версия}
    \begin{lstlisting}
    VecImage add_contrast_tbb(VecImage image)
    \end{lstlisting}
    \par Поиск минимума и максимума находится в отдельной функции.
    \begin{lstlisting}
    std::pair<unsigned char, unsigned char> minmax_tbb(const VecImage& image);
    \end{lstlisting}
    \par Для поиска минимума и максимума создается одномерное итерационное пространство с параметром grainsize по умолчанию. Чтобы не создавать отдельный класс для функтора используется лямбда-выражение.
    \begin{lstlisting}
      tbb::parallel_for(tbb::blocked_range<int>(0, sz),
        [&](const tbb::blocked_range<int>& range) {
            for (int i = range.begin(); i < range.end(); i++) {/*local min max*/}
    \end{lstlisting}
    После выхода из параллельной области происходит редукция локальных значений.
    \par Цикл, преобразующий пиксели, распараллелен как поиск минимума и максимума, но без редукции.
    \begin{lstlisting}
    tbb::parallel_for(tbb::blocked_range<int>(0, sz),
        [&](const tbb::blocked_range<int>& range) {
            for (int i = range.begin(); i < range.end(); i++) {/*change pixels(1)*/}
    \end{lstlisting}    
    % -------------------------------------------------------------------------------------

    \newpage
    \textbf{STD версия}
    \begin{lstlisting}
    VecImage add_contrast_std(VecImage image)
    \end{lstlisting}
    Распределение данных происходит по следующей схеме:
    \begin{lstlisting}
    const int count = image.size() / num_threads;
    const int rem = image.size() % num_threads;
    \end{lstlisting}
    
    Для поиска минимума и максимума создается команда потоков, в которой каждый поток выполняет функцию minmax\_tf. Она принимает на вход стартовый индекс и количество элементов.
    \begin{lstlisting}
    for (int i = 0; i < num_threads; i++) {
        threads_vec.push_back(std::thread(minmax_tf, image, start, size,
                                                &min_vec, &max_vec, i));
    }
    for (int i = 0; i < num_threads; i++) threads_vec[i].join();
    \end{lstlisting}
    \par После этого происходит редукция локальных значений.
    \par Для выполнения растяжения используется та же команда потоков, выполняющая функцию contrast\_tf. Она принимает на вход стартовый индекс и количество элементов.
    \begin{lstlisting}
    for (int i = 0; i < num_threads; i++) {
        threads_vec.push_back(std::thread(contrast_tf, &image,
                                    starts[i], sizes[i], min_c, max_c));
    }

    for (int i = 0; i < num_threads; i++) threads_vec[i].join();
    \end{lstlisting}
    % -------------------------------------------------------
    \textbf{Вспомогательные алгоритмы}
    \par Для тестирования программы на стабильность при большом размере входных данных использована функция, которая генерирует случайное изображение заданного размера.
    \begin{lstlisting}
    VecImage RandomVector(int size);
    \end{lstlisting}
    
    \newpage
    \section*{Тестирование}
    \addcontentsline{toc}{section}{Тестирование}
    \par Для того, чтобы убедиться в корректности работы алгоритма, был использован Google testing framework (gtest).
    \par В рамках данного фреймворка были созданы тесты, которые проверяют соответствие результата алгоритма и значений, вычисленных вручную. Для тестирования параллельных версий так же производится сравнение с последовательной на случайном изображении большого размера.
    \par Кроме использования модульных тестов было проведено тестирование на реальных изображениях. Для визуализации использовалась библиотека OpenCV. Все алгоритмы показали одинаковый правильный результат.
    \par Тесты были успешно пройдены, следовательно, программа работает корректно.
    
    \newpage
    \section*{Результаты экспериментов}
    \addcontentsline{toc}{section}{Результаты экспериментов}
    
    \parЭксперимент проводился на компьютере, обладающим следующими характеристиками:
    \begin{enumerate}
        \item Процессор: Intel Core i5-8265U, 1.6 GHz*, 4 cores, 8 threads
        \item Оперативная память 8 GB, DDR4, 2400 MHz
        \item Операционная система Microsoft Windows 10 Pro x64
    \end{enumerate}
    * Частота была зафиксирована на 1.6 GHz для более точных результатов. \\
    Экспериментальные данные: изображение в в виде вектора с типом unsiged char и размером 5000*5000 пикселей.\\
    
    \par Тестирование на 2-х потоках.
    \begin{table}[!h]
    \centering
    \begin{tabular}{| r | r | r |}
    \hline
    Версия & Время (с) & Ускорение \\[5pt]
    \hline
    SEQ & 4.479 & -     \\
    OMP & 2.312 & 1.931 \\
    TBB & 2.357 & 1.900 \\
    STD & 2.795 & 1.602 \\
    \hline
    \end{tabular}
    \end{table}
    
    \par Тестирование на 4-х потоках.
    \begin{table}[!h]
    \centering
    \begin{tabular}{| r | r | r |}
    \hline
    Версия & Время (с) & Ускорение \\[5pt]
    \hline
    SEQ & 4.531 & -     \\
    OMP & 1.193 & 3.798 \\
    TBB & 1.227 & 3.692 \\
    STD & 1.443 & 3.139 \\
    \hline
    \end{tabular}
    \end{table}

    \par Версии с использованием OpenMP и TBB на четырех потоках дают приблизительно одинаковое ускорение в 3.7-3.8 раз. Версия с использовании std Thread отстает от них на 15\%. На двух потоках ускорение OpenMP и TBB близко к 2, а std Thread отстает на те же 15\%. Вероятно это связано с тем, что в OpenMP и TBB автоматическое распределение работы происходит оптимальнее. Из-за того, что создание потоков и распределение работы приводит к накладным расходам, ускорение не может быть точно равно числу потоков.
    
    \newpage
	\section*{Заключение}
	\addcontentsline{toc}{section}{Заключение}
	
	\par В ходе работы был изучен алгоритм повышения контраста полутонового изображения с помощью линейного растяжения гистограммы. Был разработан последовательный алгоритм для решения задачи. После успешного тестирования с помощью фреймворка Google Test и тестирования с реальными изображениями были разработаны три параллельные версии с использованием библиотек OpenMP, TBB и стандартной библиотеки Thread. Данные версии так же были протестированы.
	
	В среднем все параллельные алгоритмы на четырех-ядерном процессоре показали прирост производительности  в 3.1-3.8 раз по сравнению с последовательной версией. Так же стоит отметить, что использование OpenMP и TBB облегчает задачу для программиста по сравнению с std Thread, т.к. в этих библиотеках предусмотрено автоматическое распределение данных.
	
	\newpage
	\begin{thebibliography}{1}
		\addcontentsline{toc}{section}{Список литературы}
		\bibitem{1} Н. Н. Красильников - Цифровая обработка 2D и 3D-изображений
		\bibitem{2} Гонсалес Р., Вудс Р - Цифровая обработка изображений
		\bibitem{3} Гергель В. П. - Теория и практика параллельных вычислений
		\bibitem{4} Даутов Руслан - Открытая книга по технологии OpenMP
		\bibitem{5} Intel(R) Threading Building Blocks Reference Manual
		\bibitem{6} Мюссер Д., Дердж Ж., Сейни А. - C++ и STL. Справочное руководство
	\end{thebibliography}
	
	\newpage
	\section*{Приложение}
	\addcontentsline{toc}{section}{Приложение}
	
	\textbf{Последовательная версия}
	\begin{lstlisting}
	
// contrast.h
// Copyright 2021 Napylov Evgenii
#ifndef MODULES_TASK_1_NAPYLOV_E_CONTRAST_CONTRAST_H_
#define MODULES_TASK_1_NAPYLOV_E_CONTRAST_CONTRAST_H_

#include <iostream>
#include <vector>
#include <cassert>
#include <random>
#include <ctime>
#include <algorithm>

typedef std::vector<unsigned char> VecImage;
typedef std::vector<std::vector<unsigned char>> Image;

void print_vec(const VecImage& vec);

VecImage image_to_vec(const Image& image, int w, int h);

Image vec_to_image(const VecImage& vec, int w, int h);

VecImage RandomVector(int size);

VecImage add_contrast(VecImage image, unsigned char down = 0,
                                      unsigned char up = 255);

#endif  // MODULES_TASK_1_NAPYLOV_E_CONTRAST_CONTRAST_H_


// contrast.cpp
// Copyright 2021 Napylov Evgenii
#include "../../../modules/task_1/napylov_e_contrast/contrast.h"

void print_vec(const VecImage& vec) {
    for (auto val : vec) {
        std::cout << static_cast<int>(val) << ' ';
    }
    std::cout << std::endl;
}

VecImage image_to_vec(const Image& image, int w, int h) {
    VecImage res(w * h);
    int k = 0;
    for (int i = 0; i < w; i++) {
        for (int j = 0; j < h; j++) {
            res[k++] = image[i][j];
        }
    }
    return res;
}

Image vec_to_image(const VecImage& vec, int w, int h) {
    Image res(w);
    for (int i = 0; i < w; i++) {
        res[i].resize(h);
        for (int j = 0; j < h; j++) {
            res[i][j] = vec[h * i + j];
        }
    }
    return res;
}

VecImage RandomVector(int size) {
    static std::mt19937 gen(time(0));
    VecImage result(size);
    std::uniform_int_distribution<unsigned int> distr(0, 255);
    for (int i = 0; i < size; i++) {
        result[i] = static_cast<unsigned char>(distr(gen));
    }
    return result;
}

VecImage add_contrast(VecImage image, unsigned char down, unsigned char up) {
    assert(up > down);
    unsigned char min = *std::min_element(image.begin(), image.end());
    unsigned char max = *std::max_element(image.begin(), image.end());
    if (max == min) {
        return image;
    } else {
        for (size_t i = 0; i < image.size(); i++) {
            image[i] = round((static_cast<double>((image[i] - min))
                / static_cast<double>((max - min)))* (up - down));
        }
        return image;
    }
}


// main.cpp
// Copyright 2021 Napylov Evgenii
#include <gtest/gtest.h>
#include "./contrast.h"

TEST(Linear_stretch_contrast, Manual_Calc_1) {
    int w = 3; int h = 2;
    Image image = { {0, 1}, {1, 1}, {0, 1} };
    Image correct = { {0, 255}, {255, 255}, {0, 255} };
    Image res = vec_to_image(add_contrast(image_to_vec(image, w, h)), w, h);
    for (int i = 0; i < w; i++) {
        for (int j = 0; j < h; j++) {
            ASSERT_EQ(correct[i][j], res[i][j]);
        }
    }
}

TEST(Linear_stretch_contrast, Manual_Calc_2) {
    int w = 2; int h = 2;
    Image image = { {5, 123}, {28, 240} };
    Image correct = { {0, 128}, {25, 255} };
    Image res = vec_to_image(add_contrast(image_to_vec(image, w, h)), w, h);
    for (int i = 0; i < w; i++) {
        for (int j = 0; j < h; j++) {
            ASSERT_EQ(correct[i][j], res[i][j]);
        }
    }
}

TEST(Linear_stretch_contrast, Manual_Calc_3) {
    int w = 3; int h = 3;
    Image image = { {57, 120, 99}, {32, 17, 64}, {1, 55, 200} };
    Image correct = { {72, 152, 126}, {40, 21, 81}, {0, 69, 255} };
    Image res = vec_to_image(add_contrast(image_to_vec(image, w, h)), w, h);
    for (int i = 0; i < w; i++) {
        for (int j = 0; j < h; j++) {
            ASSERT_EQ(correct[i][j], res[i][j]);
        }
    }
}

TEST(Linear_stretch_contrast, Manual_Calc_4) {
    int w = 3; int h = 2;
    Image image = { {1, 2}, {3, 4}, {5, 1} };
    Image correct = { {0, 64}, {128, 191}, {255, 0} };
    Image res = vec_to_image(add_contrast(image_to_vec(image, w, h)), w, h);
    for (int i = 0; i < w; i++) {
        for (int j = 0; j < h; j++) {
            ASSERT_EQ(correct[i][j], res[i][j]);
        }
    }
}

TEST(Linear_stretch_contrast, Random_No_Throw) {
    VecImage image = RandomVector(1920 * 1080);
    VecImage res;
    ASSERT_NO_THROW(res = add_contrast(image));
}


	\end{lstlisting}
	
	\textbf{OMP версия}
    \begin{lstlisting}

// contrast.h
// Copyright 2021 Napylov Evgenii
#ifndef MODULES_TASK_2_NAPYLOV_E_CONTRAST_CONTRAST_H_
#define MODULES_TASK_2_NAPYLOV_E_CONTRAST_CONTRAST_H_

#include <omp.h>
#include <vector>


typedef std::vector<unsigned char> VecImage;

void print_vec(const VecImage& vec);

VecImage RandomVector(int size);

VecImage add_contrast(VecImage image);

VecImage add_contrast_omp(VecImage image);

#endif  // MODULES_TASK_2_NAPYLOV_E_CONTRAST_CONTRAST_H_


// contrast.cpp
// Copyright 2021 Napylov Evgenii

#include <iostream>
#include <cassert>
#include <random>
#include <ctime>
#include <algorithm>

#include "../../../modules/task_2/napylov_e_contrast/contrast.h"

void print_vec(const VecImage& vec) {
    for (auto val : vec) {
        std::cout << static_cast<int>(val) << ' ';
    }
    std::cout << std::endl;
}

VecImage RandomVector(int size) {
    static std::mt19937 gen(time(0));
    VecImage result(size);
    std::uniform_int_distribution<unsigned int> distr(0, 255);
    for (int i = 0; i < size; i++) {
        result[i] = static_cast<unsigned char>(distr(gen));
    }
    return result;
}

VecImage add_contrast(VecImage image) {
    unsigned char min = *std::min_element(image.begin(), image.end());
    unsigned char max = *std::max_element(image.begin(), image.end());
    if (max == min) {
        return image;
    } else {
        for (size_t i = 0; i < image.size(); i++) {
            image[i] = round((static_cast<double>((image[i] - min))
                / static_cast<double>((max - min)))* 255);
        }
        return image;
    }
}

std::pair<unsigned char, unsigned char> minmax_omp(const VecImage& image) {
    unsigned char min_col = 255;
    unsigned char max_col = 0;

    int max_threads = omp_get_max_threads();
    std::vector<unsigned char> min_vec(max_threads);
    std::fill(min_vec.begin(), min_vec.end(), 255);
    std::vector<unsigned char> max_vec(max_threads);
    std::fill(max_vec.begin(), max_vec.end(), 0);

    #pragma omp parallel
    {
        int thread_id = omp_get_thread_num();
        #pragma omp for
        for (int i = 0; i < static_cast<int>(image.size()); i++) {
            if (image[i] > max_vec[thread_id]) {
                max_vec[thread_id] = image[i];
            }
            if (image[i] < min_vec[thread_id]) {
                min_vec[thread_id] = image[i];
            }
        }
    }
    // Reduction
    max_col = *std::max_element(max_vec.begin(), max_vec.end());
    min_col = *std::min_element(min_vec.begin(), min_vec.end());
    return std::pair<unsigned char, unsigned char>(min_col, max_col);
}

VecImage add_contrast_omp(VecImage image) {
    std::pair<unsigned char, unsigned char> minmax = minmax_omp(image);
    unsigned char min_col = minmax.first;
    unsigned char max_col = minmax.second;

    if (max_col == min_col) {
        return image;
    } else {
        #pragma omp parallel for
        for (int i = 0; i < static_cast<int>(image.size()); i++) {
            image[i] = round((static_cast<double>((image[i] - min_col))
                / static_cast<double>((max_col - min_col))) * 255);
        }
        return image;
    }
}


// main.cpp
// Copyright 2021 Napylov Evgenii
#include <gtest/gtest.h>
#include <iostream>

#include "./contrast.h"

TEST(Linear_stretch_contrast_OMP, Manual_Calc_1) {
    VecImage image = {0, 1, 1, 1, 0, 1 };
    VecImage correct = { 0, 255, 255, 255, 0, 255 };
    VecImage res = add_contrast_omp(image);
    for (size_t i = 0; i < image.size(); i++) {
        ASSERT_EQ(correct[i], res[i]);
    }
}

TEST(Linear_stretch_contrast_OMP, Manual_Calc_2) {
    VecImage image = { 5, 123, 28, 240 };
    VecImage correct = { 0, 128, 25, 255 };
    VecImage res = add_contrast_omp(image);
    for (size_t i = 0; i < image.size(); i++) {
        ASSERT_EQ(correct[i], res[i]);
    }
}

TEST(Linear_stretch_contrast_OMP, Manual_Calc_3) {
    VecImage image = { 57, 120, 99, 32, 17, 64, 1, 55, 200 };
    VecImage correct = { 72, 152, 126, 40, 21, 81, 0, 69, 255 };
    VecImage res = add_contrast_omp(image);
    for (size_t i = 0; i < image.size(); i++) {
        ASSERT_EQ(correct[i], res[i]);
    }
}

TEST(Linear_stretch_contrast_OMP, Manual_Calc_4) {
    VecImage image = { 1, 2, 3, 4, 5, 1 };
    VecImage correct = { 0, 64, 128, 191, 255, 0 };
    VecImage res = add_contrast_omp(image);
    for (size_t i = 0; i < image.size(); i++) {
        ASSERT_EQ(correct[i], res[i]);
    }
}

TEST(Linear_stretch_contrast_OMP, Seq_vs_Omp) {
    double t1, t2, dt_seq, dt_omp;
    VecImage image = RandomVector(200 * 400);

    t1 = omp_get_wtime();
    VecImage res_seq = add_contrast(image);
    t2 = omp_get_wtime();
    dt_seq = t2 - t1;

    t1 = omp_get_wtime();
    VecImage res_omp = add_contrast_omp(image);
    t2 = omp_get_wtime();
    dt_omp = t2 - t1;

    for (size_t i = 0; i < image.size(); i++) {
        ASSERT_EQ(res_seq[i], res_omp[i]);
    }

    std::cout << "seq t: " << dt_seq << "\nomp t: " << dt_omp << std::endl;
}

int main(int argc, char** argv) {
    ::testing::InitGoogleTest(&argc, argv);
    return RUN_ALL_TESTS();
}
    \end{lstlisting}
    
    \textbf{TBB версия}
    \begin{lstlisting}
// contrast.h
// Copyright 2021 Napylov Evgenii
#ifndef MODULES_TASK_3_NAPYLOV_E_CONTRAST_CONTRAST_H_
#define MODULES_TASK_3_NAPYLOV_E_CONTRAST_CONTRAST_H_

#include <vector>


typedef std::vector<unsigned char> VecImage;

void print_vec(const VecImage& vec);

VecImage RandomVector(int size);

VecImage add_contrast(VecImage image);

VecImage add_contrast_tbb(VecImage image);

#endif  // MODULES_TASK_3_NAPYLOV_E_CONTRAST_CONTRAST_H_

// contrast.cpp
// Copyright 2021 Napylov Evgenii
#include <tbb/tbb.h>
#include <iostream>
#include <random>
#include <ctime>
#include <algorithm>

#include "../../../modules/task_3/napylov_e_contrast/contrast.h"

const int num_threads = 8;

void print_vec(const VecImage& vec) {
    for (auto val : vec) {
        std::cout << static_cast<int>(val) << ' ';
    }
    std::cout << std::endl;
}

VecImage RandomVector(int size) {
    static std::mt19937 gen(time(0));
    VecImage result(size);
    std::uniform_int_distribution<unsigned int> distr(0, 255);
    for (int i = 0; i < size; i++) {
        result[i] = static_cast<unsigned char>(distr(gen));
    }
    return result;
}

VecImage add_contrast(VecImage image) {
    unsigned char min = *std::min_element(image.begin(), image.end());
    unsigned char max = *std::max_element(image.begin(), image.end());
    if (max == min) {
        return image;
    } else {
        for (size_t i = 0; i < image.size(); i++) {
            image[i] = round((static_cast<double>((image[i] - min))
                / static_cast<double>((max - min))) * 255);
        }
        return image;
    }
}

std::pair<unsigned char, unsigned char> minmax_tbb(const VecImage& image) {
    std::vector<unsigned char> min_vec(num_threads);
    std::fill(min_vec.begin(), min_vec.end(), 255);
    std::vector<unsigned char> max_vec(num_threads);
    std::fill(max_vec.begin(), max_vec.end(), 0);

    int sz = image.size();

    tbb::task_scheduler_init init(num_threads);

    tbb::parallel_for(tbb::blocked_range<int>(0, sz),
        [&](const tbb::blocked_range<int>& range) {
            for (int i = range.begin(); i < range.end(); i++) {
                int thr_id = tbb::task_arena::current_thread_index();
                if (image[i] > max_vec[thr_id]) {
                    max_vec[thr_id] = image[i];
                }
                if (image[i] < min_vec[thr_id]) {
                    min_vec[thr_id] = image[i];
                }
            }
        });

    unsigned char max_col = *std::max_element(max_vec.begin(), max_vec.end());
    unsigned char min_col = *std::min_element(min_vec.begin(), min_vec.end());
    return std::pair<unsigned char, unsigned char>(min_col, max_col);
}

VecImage add_contrast_tbb(VecImage image) {
    auto minmax = minmax_tbb(image);
    unsigned char min_col = minmax.first;
    unsigned char max_col = minmax.second;

    int sz = image.size();

    tbb::task_scheduler_init init(num_threads);

    tbb::parallel_for(tbb::blocked_range<int>(0, sz),
        [&](const tbb::blocked_range<int>& range) {
            for (int i = range.begin(); i < range.end(); i++) {
                image[i] = round((static_cast<double>((image[i] - min_col))
                    / static_cast<double>((max_col - min_col))) * 255);
            }
        });

    return image;
}

// main.cpp
// Copyright 2021 Napylov Evgenii
#include <tbb/tick_count.h>
#include <gtest/gtest.h>
#include <iostream>

#include "./contrast.h"


TEST(Linear_stretch_contrast_TBB, Manual_Calc_1) {
    VecImage image = {0, 1, 1, 1, 0, 1 };
    image = add_contrast_tbb(image);
    VecImage correct = { 0, 255, 255, 255, 0, 255 };
    VecImage res = add_contrast_tbb(image);
    for (size_t i = 0; i < image.size(); i++) {
        ASSERT_EQ(correct[i], res[i]);
    }
}

TEST(Linear_stretch_contrast_TBB, Manual_Calc_2) {
    VecImage image = { 5, 123, 28, 240 };
    VecImage correct = { 0, 128, 25, 255 };
    VecImage res = add_contrast_tbb(image);
    for (size_t i = 0; i < image.size(); i++) {
        ASSERT_EQ(correct[i], res[i]);
    }
}

TEST(Linear_stretch_contrast_TBB, Manual_Calc_3) {
    VecImage image = { 57, 120, 99, 32, 17, 64, 1, 55, 200 };
    VecImage correct = { 72, 152, 126, 40, 21, 81, 0, 69, 255 };
    VecImage res = add_contrast_tbb(image);
    for (size_t i = 0; i < image.size(); i++) {
        ASSERT_EQ(correct[i], res[i]);
    }
}

TEST(Linear_stretch_contrast_TBB, Manual_Calc_4) {
    VecImage image = { 1, 2, 3, 4, 5, 1 };
    VecImage correct = { 0, 64, 128, 191, 255, 0 };
    VecImage res = add_contrast_tbb(image);
    for (size_t i = 0; i < image.size(); i++) {
        ASSERT_EQ(correct[i], res[i]);
    }
}

TEST(Linear_stretch_contrast_TBB, Seq_vs_Tbb) {
    VecImage image = RandomVector(1920 * 1080);
    tbb::tick_count t0, t1;
    double dt_seq, dt_tbb;

    t0 = tbb::tick_count::now();
    VecImage res_seq = add_contrast(image);
    t1 = tbb::tick_count::now();
    dt_seq = (t1 - t0).seconds();

    t0 = tbb::tick_count::now();
    VecImage res_tbb = add_contrast_tbb(image);
    t1 = tbb::tick_count::now();
    dt_tbb = (t1 - t0).seconds();

    for (size_t i = 0; i < image.size(); i++) {
        ASSERT_EQ(res_seq[i], res_tbb[i]);
    }

    std::cout << "seq_t: " << dt_seq << std::endl;
    std::cout << "tbb_t: " << dt_tbb << std::endl;
}

int main(int argc, char** argv) {
    ::testing::InitGoogleTest(&argc, argv);
    return RUN_ALL_TESTS();
}
    \end{lstlisting}
    
    \textbf{STD версия}
    \begin{lstlisting}
// contrast.h
// Copyright 2021 Napylov Evgenii
#ifndef MODULES_TASK_4_NAPYLOV_E_CONTRAST_CONTRAST_H_
#define MODULES_TASK_4_NAPYLOV_E_CONTRAST_CONTRAST_H_


#include <vector>


typedef std::vector<unsigned char> VecImage;

void print_vec(const VecImage& vec);

VecImage RandomVector(int size);

VecImage add_contrast(VecImage image);

VecImage add_contrast_std(VecImage image);

#endif  // MODULES_TASK_4_NAPYLOV_E_CONTRAST_CONTRAST_H_

// contrast.cpp
// Copyright 2021 Napylov Evgenii

#include <iostream>
#include <cassert>
#include <random>
#include <ctime>
#include <algorithm>
#include "../../../3rdparty/unapproved/unapproved.h"  // thread

#include "../../../modules/task_4/napylov_e_contrast/contrast.h"

void print_vec(const VecImage& vec) {
    for (auto val : vec) {
        std::cout << static_cast<int>(val) << ' ';
    }
    std::cout << std::endl;
}

VecImage RandomVector(int size) {
    static std::mt19937 gen(time(0));
    VecImage result(size);
    std::uniform_int_distribution<unsigned int> distr(0, 255);
    for (int i = 0; i < size; i++) {
        result[i] = static_cast<unsigned char>(distr(gen));
    }
    return result;
}

VecImage add_contrast(VecImage image) {
    unsigned char min = *std::min_element(image.begin(), image.end());
    unsigned char max = *std::max_element(image.begin(), image.end());
    if (max == min) {
        return image;
    } else {
        for (size_t i = 0; i < image.size(); i++) {
            image[i] = round((static_cast<double>((image[i] - min))
                / static_cast<double>((max - min))) * 255);
        }
        return image;
    }
}

void minmax_tf(const VecImage& image, int start, int size,
    std::vector<unsigned char>* min_vec,
    std::vector<unsigned char>* max_vec, int num) {
    for (int i = start; i < start+size; i++) {
        if (image[i] > max_vec->at(num)) {
            max_vec->at(num) = image[i];
        }
        if (image[i] < min_vec->at(num)) {
            min_vec->at(num) = image[i];
        }
    }
}

void contrast_tf(VecImage* image, int start, int size, int min_c, int max_c) {
    for (int i = start; i < start+size; i++) {
        image->at(i) = round((static_cast<double>((image->at(i) - min_c))
                    / static_cast<double>((max_c - min_c))) * 255);
    }
}

VecImage add_contrast_std(VecImage image) {
    int num_threads = 8;
    std::vector<std::thread> threads_vec;

    // remove unnecessary threads if there are more than the number of pixels
    if (num_threads > static_cast<int>(image.size())) {
        num_threads = image.size();
    }

    const int count = image.size() / num_threads;
    const int rem = image.size() % num_threads;

    std::vector<unsigned char> min_vec(num_threads);
    std::fill(min_vec.begin(), min_vec.end(), 255);
    std::vector<unsigned char> max_vec(num_threads);
    std::fill(max_vec.begin(), max_vec.end(), 0);

    std::vector<int> starts;
    std::vector<int> sizes;

    int start = 0;
    // data distr. and min/max team
    for (int i = 0; i < num_threads; i++) {
        int size;
        if (i != num_threads - 1) {
            size = count;
        } else {
            size = count + rem;
        }
        sizes.push_back(size);
        if (i != 0) {
            start += sizes[i - 1];
        }
        starts.push_back(start);
        threads_vec.push_back(std::thread(minmax_tf, image, start, size,
                                &min_vec, &max_vec, i));
    }

    // blocks the main thread -> synchronized
    for (int i = 0; i < num_threads; i++) threads_vec[i].join();

    unsigned char min_c = *std::min_element(min_vec.begin(), min_vec.end());
    unsigned char max_c = *std::max_element(max_vec.begin(), max_vec.end());

    threads_vec.clear();

    // stretching team
    for (int i = 0; i < num_threads; i++) {
        threads_vec.push_back(std::thread(contrast_tf, &image,
                                    starts[i], sizes[i], min_c, max_c));
    }

    for (int i = 0; i < num_threads; i++) threads_vec[i].join();

    return image;
}

// main.cpp
// Copyright 2021 Napylov Evgenii
#include <gtest/gtest.h>
#include <time.h>
#include <iostream>
#include "./contrast.h"

TEST(Linear_stretch_contrast_STD, Manual_Calc_1) {
     VecImage image = {0, 1, 1, 1, 0, 1 };
     VecImage correct = { 0, 255, 255, 255, 0, 255 };
     VecImage res = add_contrast_std(image);
     for (size_t i = 0; i < image.size(); i++) {
         ASSERT_EQ(correct[i], res[i]);
     }
}

TEST(Linear_stretch_contrast_STD, Manual_Calc_2) {
     VecImage image = { 5, 123, 28, 240 };
     VecImage correct = { 0, 128, 25, 255 };
     VecImage res = add_contrast_std(image);
     for (size_t i = 0; i < image.size(); i++) {
         ASSERT_EQ(correct[i], res[i]);
     }
}

TEST(Linear_stretch_contrast_STD, Manual_Calc_3) {
     VecImage image = { 57, 120, 99, 32, 17, 64, 1, 55, 200 };
     VecImage correct = { 72, 152, 126, 40, 21, 81, 0, 69, 255 };
     VecImage res = add_contrast_std(image);
     for (size_t i = 0; i < image.size(); i++) {
         ASSERT_EQ(correct[i], res[i]);
     }
}

TEST(Linear_stretch_contrast_STD, Manual_Calc_4) {
     VecImage image = { 1, 2, 3, 4, 5, 1 };
     VecImage correct = { 0, 64, 128, 191, 255, 0 };
     VecImage res = add_contrast_std(image);
     for (size_t i = 0; i < image.size(); i++) {
         ASSERT_EQ(correct[i], res[i]);
     }
}

TEST(Linear_stretch_contrast_STD, Seq_vs_Std) {
    VecImage image = RandomVector(1920 * 1080);

    clock_t t1 = clock();
    VecImage res_seq = add_contrast(image);
    clock_t t2 = clock();
    double t_seq = static_cast<double>(t2 - t1) / CLOCKS_PER_SEC;

    t1 = clock();
    VecImage res_std = add_contrast_std(image);
    t2 = clock();
    double t_std = static_cast<double>(t2 - t1) / CLOCKS_PER_SEC;

    for (size_t i = 0; i < image.size(); i++) {
        ASSERT_EQ(res_seq[i], res_std[i]);
    }

    std::cout << "seq_t: " << t_seq << "\nstd_t: " << t_std << std::endl;
}

int main(int argc, char** argv) {
    ::testing::InitGoogleTest(&argc, argv);
    return RUN_ALL_TESTS();
}
    \end{lstlisting}
\end{document}