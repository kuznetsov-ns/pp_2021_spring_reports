\documentclass{report}

\usepackage[T2A]{fontenc}
\usepackage[utf8]{luainputenc}
\usepackage[english, russian]{babel}
\usepackage[pdftex]{hyperref}
\usepackage[14pt]{extsizes}
\usepackage{listings}
\usepackage{color}
\usepackage{geometry}
\usepackage{enumitem}
\usepackage{multirow}
\usepackage{graphicx}
\usepackage{indentfirst}
\usepackage{amsmath}


\geometry{a4paper,top=2cm,bottom=3cm,left=2cm,right=1.5cm}
\setlength{\parskip}{0.5cm}
\setlist{nolistsep, itemsep=0.3cm,parsep=0pt}

\lstset{language=C++,
		basicstyle=\footnotesize,
		keywordstyle=\color{blue}\ttfamily,
		stringstyle=\color{red}\ttfamily,
		commentstyle=\color{green}\ttfamily,
		morecomment=[l][\color{magenta}]{\#}, 
		tabsize=4,
		breaklines=true,
  		breakatwhitespace=true,
  		title=\lstname,       
}

\makeatletter
\renewcommand\@biblabel[1]{#1.\hfil}
\makeatother

\begin{document}

\begin{titlepage}

\begin{center}
Министерство науки и высшего образования Российской Федерации
\end{center}

\begin{center}
Федеральное государственное автономное образовательное учреждение высшего образования \\
Национальный исследовательский Нижегородский государственный университет им. Н.И. Лобачевского
\end{center}

\begin{center}
Институт информационных технологий, математики и механики
\end{center}

\vspace{4em}

\begin{center}
\textbf{\LargeОтчет по лабораторной работе} \\
\end{center}
\begin{center}
\textbf{\Large«Поразрядная сортировка для целых чисел с простым слиянием.»} \\
\end{center}

\vspace{4em}

\newbox{\lbox}
\savebox{\lbox}{\hbox{text}}
\newlength{\maxl}
\setlength{\maxl}{\wd\lbox}
\hfill\parbox{7cm}{
\hspace*{5cm}\hspace*{-5cm}\textbf{Выполнил:} \\ студент группы 381806-3 \\ Макарычев С. Д.\\
\\
\hspace*{5cm}\hspace*{-5cm}\textbf{Проверил:}\\ доцент кафедры МОСТ, \\ кандидат технических наук \\ Сысоев А. В.\\
}
\vspace{\fill}

\begin{center} Нижний Новгород \\ 2021 \end{center}

\end{titlepage}

\setcounter{page}{2}

% Содержание
\tableofcontents
\newpage

% Введение
\section*{Введение}
\addcontentsline{toc}{section}{Введение}
К списку самых популярных задач в программировании относится сортировка данных в порядке возрастания или убывания по какому-либо ключу. Одной из таких задач является поразрядная сортировка для целых чисел.
\par Главным преимуществом данного алгоритма является работа за линейное время. Также его можно применить ко всем объектам, которые можно поделить на условные ``разряды''. В частности целые числа типа \verb|int|.
\par Реализация поразрядной сортировки существует в двух вариантах: LSD(сортировка производится по разрядам справа налево) и MSD(сортировка производится по разрядам слева направо). В данной работе реализуется поразрядная сортировка первого типа.
\par Таким образом, нами будет рассмотрены последовательная реализация сортировки и ее распараллеливание с использованием разных технологий.
\newpage
% Конец введения

% Постановка задачи
\section*{Постановка задачи}
\addcontentsline{toc}{section}{Постановка задачи}

В рамках лабораторной работы необходимо выполнить несколько задач:

\begin{enumerate} 

\item Изучить алгоритм поразрядной сортировки для целых чисел
\item Реализовать последовательную версию алгоритма и создать автоматические тесты для проверки корректности его работы
\item Реализовать параллельную версию алгоритма, использовав технологию OpenMP. Создать автоматические тесты для проверки корректности его работы
\item Реализовать параллельную версию алгоритма, использовав технологию TBB. Создать автоматические тесты для проверки корректности его работы
\item Провести вычислительные эксперименты и сделать выводы

\end{enumerate} 

\newpage
% Конец постановки задачи

% Метод решения
\section*{Метод решения}
\addcontentsline{toc}{section}{Метод решения}
В нашем случае поразрядная сортировка сортирует элементы по разрядам чисел, начиная с крайнего правого разряда и заканчивая крайним левым. За разряд считаем байт числа.
\par Сортировка протекает следующим образом: начинаем идти по байтам числа с младшего байта, создаем массив счетчиков размером 256, так как каждый байт равен числу от 0 до 255. Смотрим значение текущего байта числа и и увеличиваем элемент в массиве счетчиков с индексом, равным этому значению, на единицу. Далее присваиваем элементу массива счетчиков значение, равное сумме элементов до данного. Производим окончательную расстановку: для каждого числа исходного массива мы знаем, сколько чисел меньше него - это значение хранится в массиве счетчиков. Таким образом, нам известно окончательное место числа в упорядоченном массиве: если есть K чисел меньше данного, то оно должно стоять на позиции K+1. Осуществляем проход по входному массиву слева направо, одновременно заполняя выходной массив.
\par Так как у целых чисел со знаком левый бит старшего байта выделен для определения знака числа, то для корректной работы алгоритма мы из исходного массива создадим два вспомогательных массива(для положительных и отрицательных чисел), произведем их сортировку по отдельности и сольем их в исходный массив(снчала отсортированный массив отрицательных чисел, потом отсортированный массив положительных чисел).
\newpage
% Конец метод решения
% Схема распараллеливания
\begin{center}\section*{Схема распараллеливания}\end{center}
\addcontentsline{toc}{section}{Схема распараллеливания}
\subsection*{Технология OpenMP}
\addcontentsline{toc}{subsection}{Технология OpenMP}
Для организации параллельного блока исполнения использовалась директива препроцессора \verb|#pragma omp parallel|.
\par Исходный массив делится на равные части и распределяется по потокам, в случае если число элементов не делится нацело на число потоков, оставшиеся числа при делении числа элементов на число потоков, добавляются в нулевой поток.
\par Затем каждый поток вызывает функцию поразрядной сортировки для своего выделенного массива.
\par В конце необходимо сделать слияние всех отсортированных массивов в один. Слияние происходит в нулевом потоке, который поочередно вызвыет функцию слияния, в которую передает участки отсортированного массива.
\subsection*{Технология TBB}
\addcontentsline{toc}{subsection}{Технология TBB}
Для решения задачи будем использовать функцию \verb|tbb::parallel_for|. Первым параметром которой является иттерационное пространство - в нашем случае \verb|tbb::blocked_range<int>(0, sizeVec, grainsize)|, где первый и второй параметр - левая и правая граница полуинтервала, третий параметр - размер порции вычислений. Вторым парметром является функтор, в котором вызывается функция сортировки для каждой части массива.
\par В конце происходит слияние такое же, как в версии OpenMP.
\newpage

% Описание программной реализации
\section*{Описание программной реализации}
\addcontentsline{toc}{section}{Описание программной реализации}
В программе используются функции, представленные ниже.
Последовательный алгоритм поразрядной сортировки вызывается с помощью функции:
\begin{lstlisting}
void signedRadixSort(std::vector<int>* sortVec);
\end{lstlisting}
\par В качестве входного параметра передается указатель на сортируемый вектор чисел.
\par Реализация параллельного алгоритма представлена в функциях:
\begin{lstlisting}
void signedRadixSortOmp(std::vector<int>* sortVec);
void signedRadixSortTbb(std::vector<int>* sortVec);
\end{lstlisting}
\par В качестве входного параметра передается указатель на сортируемый вектор чисел.

\newpage

% Подтверждение корректности
\section*{Подтверждение корректности}
\addcontentsline{toc}{section}{Подтверждение корректности}
Для подтверждения корректности в программе представлен набор тестов, разработанных с помощью Google Testing Framework.
\par Набор представляет из себя тесты, которые проверяют корректность вычислений: сравниваются отсортированный с помощью функции стандартной библиотеки вектор и вектор, отсортированный с помощью написанной нами функции.

\par Успешное прохождение всех тестов подтверждает корректность работы написанной программы.
\newpage

% Результаты экспериментов
\section*{Результаты экспериментов}
\addcontentsline{toc}{section}{Результаты экспериментов}
Вычислительные эксперименты для оценки эффективности параллельного алгоритма Фокса были проведены на оборудовании со следующей аппаратной конфигурацией:

\begin{itemize}
\item Процессор: Intel Core i3-9100F, 3.60 GHz, 4 ядра;
\item Оперативная память: 16,0 ГБ (DDR4);
\item ОС: Майкрософт Windows 10 Pro 64-bit.
\end{itemize}

\par В рамках эксперимента было вычислено время работы параллельного
и последовательного алгоритмов поразрядной сортировки. Размеры вектора в эксперименте: \verb|1500000| , \verb|15000000| и \verb|150000000|.
\par  Тесты запускались на четырех потоках. Результаты экспериментов представлены ниже.

\begin{table}[!h]
\caption{Результаты экспериментов. Сравнение с OpenMP.}
\centering
\begin{tabular}{lllll}
Размер вектора & Последовательно & Параллельно & Ускорение  \\
1500000        & 0.427         & 0.132     & 3.3       \\
15000000        & 4.2         & 1.4     & 2.9       \\
150000000       & 41.7179         & 12.198     & 3.42       
\end{tabular}
\end{table}

\begin{table}[!h]
\caption{Результаты экспериментов. Сравнение с TBB.}
\centering
\begin{tabular}{lllll}
Размер вектора & Последовательно & Параллельно & Ускорение  \\
1500000        & 	0.4231         & 0.199     & 	2.122       \\
15000000        & 4.235         & 1.5459     & 2.74       \\
150000000       & 41.8891         & 15.5868     & 2.68748       
\end{tabular}
\end{table}
\par Из полученных результатов можно сделать несколько выводов:
\begin{enumerate} 
\item Благодаря параллельному исполнению время работы сортировки сократилось.
\item Сложность и затратность в выполнении такого этапа сортировки, как слияние отсортированных массивов, сильно увеличивает время выполнения сортировки.

\end{enumerate} 

\newpage
% Конец результатов экспериментов

% Заключение
\begin{center}\section*{Заключение}\end{center}
\addcontentsline{toc}{section}{Заключение}

В результате выполнения данной работы был изучен алгоритм поразрядной сортировки и возможность его распараллеливания с помощью разбиения исходного массива на части с последующим простым слиянием. Были реализованы: последовательная версия, OpenMP и TBB реализации. Выполнены замеры времени выполнения, на основании которых можно сделать вывод, что распараллеливание данного алгоритма даёт ускорение, равное количеству физических ядер на компьютере.
\newpage
% Конец заключения

% Список литературы
\begin{thebibliography}{1}
\addcontentsline{toc}{section}{Список литературы}
\bibitem{algolist} Поразрядная сортировка. [Электронный ресурс] // URL: \url {https://http://algolist.ru/sort/radix_sort.php} (дата обращения: 25.03.2021)
\end{thebibliography}
\newpage

% Приложение
\section*{Приложение}
\addcontentsline{toc}{section}{Приложение}
В этом разделе находится весь код, написанный в рамках лабораторной работы.
\begin{lstlisting}[language=C++]
//Seq bitwise_sort.h
// Copyright 2021 Makarychev Sergey
#ifndef  MODULES_TASK_1_MAKARYCHEV_S_BITWISE_SORT_BITWISE_SORT_H_
#define MODULES_TASK_1_MAKARYCHEV_S_BITWISE_SORT_BITWISE_SORT_H_
#include <iostream>
#include <vector>

std::vector<int> getRandomVector(int sizeVec);
void createCounters(int* sortVec, int* counters, int sizeVec);
void signedRadix(int byteNumber, int sizeVec, int* sourceVec, int* destVec, int* count);
void signedRadixSort(int* sortVec, int sizeVec);

#endif  // MODULES_TASK_1_MAKARYCHEV_S_BITWISE_SORT_BITWISE_SORT_H_

//Seq bitwise_sort.cpp

// Copyright 2021 Makarychev Sergey
#include "../../../modules/task_1/makarychev_s_bitwise_sort/bitwise_sort.h"
#include <vector>
#include <random>

std::vector<int> getRandomVector(int sizeVec) {
    if (sizeVec < 0)
        throw "Wrong vector size";
    std::vector<int> rVec(sizeVec);
    std::random_device rd;
    std::mt19937 mersenne(rd());
    std::uniform_int_distribution<> uid(-1000, 2000);
    for (int i = 0; i < sizeVec; i++)
        rVec[i] = uid(mersenne);
    return rVec;
}

void createCounters(int* sortVec, int* counters, int sizeVec) {
  unsigned char* bytePointer = (unsigned char*)sortVec;
  unsigned char* dataEnd = (unsigned char*)(sortVec + sizeVec);
  int s = static_cast<int>(sizeof(int));
  while (bytePointer != dataEnd) {
    for (int i = 0; i < s; i++)
      counters[256 * i + *bytePointer++]++;
  }
}
void signedRadix(int byteNumber, int sizeVec, int* sourceVec, int* destVec, int* count) {
  int sum = 0;
  int* countPointer;
  if (byteNumber == sizeof(int) - 1) {
    int numNegative = 0;
    for (int i = 128; i < 256; i++)
      numNegative += count[i];

    sum = numNegative;
    countPointer = count;
    int tmp;
    for (int i = 0; i < 128; ++i, ++countPointer) {
      tmp = *countPointer;
      *countPointer = sum;
      sum += tmp;
    }
    sum = 0;
    countPointer = count + 128;
    for (int i = 0; i < 128; ++i, ++countPointer) {
      tmp = *countPointer;
      *countPointer = sum;
      sum += tmp;
    }
  } else {
    countPointer = count;
    int tmp;
    for (int i = 256; i > 0; --i, ++countPointer) {
      tmp = *countPointer;
      *countPointer = sum;
      sum += tmp;
    }
  }

  unsigned char* bytePointer = (unsigned char*)sourceVec + byteNumber;
  int* sourceVecPointer = sourceVec;
  for (int i = sizeVec; i > 0; --i, bytePointer += sizeof(int), ++sourceVecPointer) {
    countPointer = count + *bytePointer;
    destVec[*countPointer] = *sourceVecPointer;
    ++(*countPointer);
  }
}

void signedRadixSort(int* sortVec, int sizeVec) {
  int* out = new int[sizeVec];
  std::vector<int> counters(sizeof(int) * 256);
  int* count;
  createCounters(sortVec, counters.data(), sizeVec);
  int s = static_cast<int>(sizeof(int));
  for (int i = 0; i < s; i++) {
    count = counters.data() + 256 * i;
    signedRadix(i, sizeVec, sortVec, out, count);
    std::swap(sortVec, out);
  }
  delete[] out;
}

//Seq main.cpp

// Copyright 2021 Makarychev Sergey
#include <gtest/gtest.h>
#include <algorithm>
#include <vector>
#include "./bitwise_sort.h"

TEST(Sequential_Radix_Sort, sort_vector_of_random_value_1) {
    std::vector<int> vec1 = getRandomVector(50);
    std::vector<int> vec2 = vec1;
    std::sort(vec2.begin(), vec2.begin() + vec2.size());
    signedRadixSort(vec1.data(), vec1.size());
    ASSERT_EQ(vec1, vec2);
}

TEST(Sequential_Radix_Sort, sort_vector_of_random_value_2) {
    std::vector<int> vec1 = getRandomVector(78);
    std::vector<int> vec2 = vec1;
    std::sort(vec2.begin(), vec2.begin() + vec2.size());
    signedRadixSort(vec1.data(), vec1.size());
    ASSERT_EQ(vec1, vec2);
}

TEST(Sequential_Radix_Sort, sort_vector_of_litle_positive_value) {
    std::vector<int> vec1 = {1, 3, 7, 4, 0, 7, 0, 9};
    std::vector<int> vec2 = vec1;
    std::sort(vec2.begin(), vec2.begin() + vec2.size());
    signedRadixSort(vec1.data(), vec1.size());
    ASSERT_EQ(vec1, vec2);
}

TEST(Sequential_Radix_Sort, sort_vector_of_litle_negative_value) {
    std::vector<int> vec1 = { -1, -8, -4, -10, -1, -6, -6, -6 };
    std::vector<int> vec2 = vec1;
    signedRadixSort(vec1.data(), vec1.size());
    std::sort(vec2.begin(), vec2.begin() + vec2.size());
    ASSERT_EQ(vec1, vec2);
}

TEST(Sequential_Radix_Sort, sort_vector_of_zero_value) {
    std::vector<int> vec1 = { 0, 0, 0, 0, 0, 0, 0, 0};
    std::vector<int> vec2 = vec1;
    signedRadixSort(vec1.data(), vec1.size());
    std::sort(vec2.begin(), vec2.begin() + vec2.size());
    ASSERT_EQ(vec1, vec2);
}

TEST(Sequential_Radix_Sort, sort_vector_of_mixed_value) {
    std::vector<int> vec1 = { 2, -99947, -9, 87456, -987456, 0 };
    std::vector<int> vec2 = vec1;
    signedRadixSort(vec1.data(), vec1.size());
    std::sort(vec2.begin(), vec2.begin() + vec2.size());
    ASSERT_EQ(vec1, vec2);
}

TEST(Sequential_Radix_Sort, sort_just_sorted_vector) {
    std::vector<int> vec1 = { -99999, -1, 0, 8, 9, 15, 10265, 235458 };
    std::vector<int> vec2 = vec1;
    signedRadixSort(vec1.data(), vec1.size());
    std::sort(vec2.begin(), vec2.begin() + vec2.size());
    ASSERT_EQ(vec1, vec2);
}
//OpenMP bitwise_sort.h
// Copyright 2021 Makarychev Sergey
#ifndef  MODULES_TASK_2_MAKARYCHEV_S_BITWISE_SORT_BITWISE_SORT_H_
#define MODULES_TASK_2_MAKARYCHEV_S_BITWISE_SORT_BITWISE_SORT_H_
#include <omp.h>
#include <iostream>
#include <vector>

std::vector<int> getRandomVector(int sizeVec);
void createCounters(int* sortVec, int* counters, int sizeVec);
void signedRadixSort(std::vector<int>* sortVec);
void countersSort(int numByte, std::vector<int>* sortVec,
  std::vector<int>* output);
void unsignedRadixSort(std::vector<int>* sortVec);
void signedRadixSortParallel(std::vector<int>* sortVec,
  int leftIndex, int rightIndex, int sizeVec);
void mergeOrderVec(int* vec1, int size1, int* vec2, int size2);
void signedRadixSortOmp(std::vector<int>* sortVec);

#endif  // MODULES_TASK_2_MAKARYCHEV_S_BITWISE_SORT_BITWISE_SORT_H_

//OpenMP bitwise_sort.cpp
// Copyright 2021 Makarychev Sergey
#include <omp.h>
#include <random>
#include <algorithm>
#include <vector>
#include "../../../modules/task_2/makarychev_s_bitwise_sort/bitwise_sort.h"

std::vector<int> getRandomVector(int sizeVec) {
    if (sizeVec < 0)
        throw "Wrong vector size";
    std::vector<int> rVec(sizeVec);
    std::random_device rd;
    std::mt19937 mersenne(rd());
    std::uniform_int_distribution<> uid(-1000, 2000);
    for (int i = 0; i < sizeVec; i++)
        rVec[i] = uid(mersenne);
    return rVec;
}

void mergeOrderVec(int* vec1, int size1,  int* vec2, int size2) {
    int* resVec = new int[size1 + size2];
    int i = 0, s = 0, j = 0;
    while (i < size1 && j < size2) {
        if (vec1[i] < vec2[j])
            resVec[s++] = vec1[i++];
        else
            resVec[s++] = vec2[j++];
    }
    while (i < size1)
        resVec[s++] = vec1[i++];
    while (j < size2)
        resVec[s++] = vec2[j++];
    i = s = 0;
    while (i < size1 + size2)
        vec1[i++] = resVec[s++];
    delete[] resVec;
}

void createCounters(std::vector<int>* sortVec, int* counters, int numByte) {
    unsigned char* bytePointer = (unsigned char*)sortVec->data();
    int s = static_cast<int>(sizeof(int));
    for (size_t i = 0; i < sortVec->size(); i++) {
        counters[bytePointer[s * i + numByte]]++;
    }
}

void countersSort(int numByte, std::vector<int>* sortVec,
    std::vector<int>* output) {
    unsigned char* bytePointer = (unsigned char*)sortVec->data();
    const int sizeCounter = 256;
    int counters[sizeCounter] = { 0 };
    int typeSize = static_cast<int>(sizeof(int));
    createCounters(sortVec, counters, numByte);
    int value = 0;
    for (int i = 0; i < sizeCounter; i++) {
        int tmp = counters[i];
        counters[i] = value;
        value += tmp;
    }
    int index = 0;
    for (size_t i = 0; i < sortVec->size(); i++) {
        index = bytePointer[typeSize * i + numByte];
        output->at(counters[index]++) = sortVec->at(i);
    }
}

void unsignedRadixSort(std::vector<int>* sortVec) {
    std::vector<int>* outbuf = new std::vector<int>(sortVec->size());
    for (int i = 0; i < 4; i++) {
        countersSort(i, sortVec, outbuf);
        std::swap(sortVec, outbuf);
    }
}

void signedRadixSort(std::vector<int>* sortVec) {
    int positiveNum = 0;
    int negativeNum = 0;
    for (size_t i = 0; i < sortVec->size(); i++) {
        if (sortVec->at(i) < 0)
            negativeNum++;
        else
            positiveNum++;
    }
     std::vector<int> positiveNumVec(positiveNum);
    std::vector<int> negativeNumVec(negativeNum);

    int it1 = 0, it2 = 0;
    for (size_t i = 0; i < sortVec->size(); i++) {
        if (sortVec->at(i) >= 0)
            positiveNumVec[it1++] = sortVec->at(i);
        else
            negativeNumVec[it2++] = sortVec->at(i);
    }

    unsignedRadixSort(&positiveNumVec);
    unsignedRadixSort(&negativeNumVec);

    int current_buff = 0;
    for (size_t i = 0; i < negativeNumVec.size() ; i++) {
        sortVec->at(i) = negativeNumVec.at(current_buff);
        current_buff++;
    }
    current_buff = 0;
    for (size_t i = negativeNumVec.size();
        i < positiveNumVec.size() + negativeNumVec.size(); i++) {
        sortVec->at(i) = positiveNumVec.at(current_buff);
        current_buff++;
    }
}



void signedRadixSortParallel(std::vector<int>* sortVec,
    int leftIndex, int rightIndex, int sizeVec) {

    int positiveNum = 0;
    int negativeNum = 0;
    for (int i = leftIndex; i <= rightIndex; i++) {
        if (sortVec->at(i) < 0)
            negativeNum++;
        else
            positiveNum++;
    }
    std::vector<int> positiveNumVec(positiveNum);
    std::vector<int> negativeNumVec(negativeNum);

    int it1 = 0, it2 = 0;
    for (int i = leftIndex; i <= rightIndex; i++) {
        if (sortVec->at(i) >= 0)
            positiveNumVec[it1++] = sortVec->at(i);
        else
            negativeNumVec[it2++] = sortVec->at(i);
    }

    unsignedRadixSort(&positiveNumVec);
    unsignedRadixSort(&negativeNumVec);

    int currentId = 0;
    for (size_t i = leftIndex; i < negativeNumVec.size() + leftIndex; i++) {
        sortVec->at(i) = negativeNumVec.at(currentId);
        currentId++;
    }
    currentId = 0;
    for (size_t i = leftIndex + negativeNumVec.size();
        i < positiveNumVec.size() + leftIndex + negativeNumVec.size(); i++) {
        sortVec->at(i) = positiveNumVec.at(currentId);
        currentId++;
    }
}

void signedRadixSortOmp(std::vector<int>* sortVec) {
    int numberThreads = 0;
    int sizeVec = sortVec->size();
#pragma omp parallel
    {
        int sizePartVec = 0;
        numberThreads = omp_get_num_threads();
        sizePartVec = sizeVec / numberThreads;
        int remainder = sizeVec % numberThreads;
        int threadId = omp_get_thread_num();
#pragma omp master
        {
            sizePartVec += remainder;
            remainder = 0;
        }
        signedRadixSortParallel(sortVec,  threadId * sizePartVec + remainder,
            (threadId + 1)* sizePartVec + remainder - 1, sizePartVec);
    }

    for (int i = 1; i < numberThreads; i++) {
        int remainder = sizeVec % numberThreads;
        int current_size = sortVec->size() / numberThreads;
        if (i == 0) {
            current_size = sortVec->size() /
                numberThreads + sortVec->size() % numberThreads;
        } else {
            current_size = sortVec->size() / numberThreads;
        }
        mergeOrderVec(sortVec->data(), current_size * i + remainder,
            sortVec->data() + current_size * i  + remainder, current_size);
    }
}

//OpenMP main.cpp
// Copyright 2021 Makarychev Sergey
#include <gtest/gtest.h>
#include <algorithm>
#include <vector>
#include "./bitwise_sort.h"

TEST(Sequential_Radix_Sort, sort_vector_of_random_value) {
    std::vector<int> vec1 = getRandomVector(78);
    std::vector<int> vec2 = vec1;
    std::sort(vec2.begin(), vec2.begin() + vec2.size());
    signedRadixSort(&vec1);
    ASSERT_EQ(vec1, vec2);
}

TEST(Omp_Radix_Sort, sort_vector_of_litle_positive_value) {
    std::vector<int> vec1 = {1, 3, 7, 4, 0, 7, 0, 9};
    std::vector<int> vec2 = vec1;
    std::sort(vec2.begin(), vec2.begin() + vec2.size());
    signedRadixSortOmp(&vec1);
    ASSERT_EQ(vec1, vec2);
}

TEST(Omp_Radix_Sort, sort_vector_of_seq_omp) {
    std::vector<int> vec1 = { 1, 3, 7, 4, 0, 7, 0, 9 };
    std::vector<int> vec2 = vec1;
    signedRadixSortOmp(&vec1);
    signedRadixSort(&vec2);
    ASSERT_EQ(vec1, vec2);
}

TEST(Omp_Radix_Sort, sort_vector_of_litle_negative_value) {
    std::vector<int> vec1 = { -1, -8, -4, -10, -1, -6, -6, -6 };
    std::vector<int> vec2 = vec1;
    signedRadixSortOmp(&vec1);
    std::sort(vec2.begin(), vec2.begin() + vec2.size());
    ASSERT_EQ(vec1, vec2);
}

TEST(Omp_Radix_Sort, sort_vector_of_zero_value) {
    std::vector<int> vec1 = { 0, 0, 0, 0, 0, 0, 0, 0};
    std::vector<int> vec2 = vec1;
    signedRadixSortOmp(&vec1);
    std::sort(vec2.begin(), vec2.begin() + vec2.size());
    ASSERT_EQ(vec1, vec2);
}

TEST(Omp_Radix_Sort, sort_vector_of_mixed_value) {
    std::vector<int> vec1 = { 2, -99947, -9, 87456, -987456, 0 };
    std::vector<int> vec2 = vec1;
    signedRadixSortOmp(&vec1);
    std::sort(vec2.begin(), vec2.begin() + vec2.size());
    ASSERT_EQ(vec1, vec2);
}

TEST(Omp_Radix_Sort, sort_just_sorted_vector) {
    std::vector<int> vec1 = { -99999, -1, 0, 8, 9, 15, 10265, 235458 };
    std::vector<int> vec2 = vec1;
    signedRadixSortOmp(&vec1);
    std::sort(vec2.begin(), vec2.begin() + vec2.size());
    ASSERT_EQ(vec1, vec2);
}

/*TEST(Omp_Radix_Sort, sort_vector_of_random_value_with_time) {
    std::vector<int> vec1 = getRandomVector(150000000);
    std::vector<int> vec2 = vec1;
    std::cout << std::endl;
    double t1 = omp_get_wtime();
    signedRadixSortOmp(&vec1);
    double t2 = omp_get_wtime();
    double ompTime = t2 - t1;
    std::cout << "Par: " << ompTime << std::endl;
    t1 = omp_get_wtime();
    signedRadixSort(&vec2);
    t2 = omp_get_wtime();
    double seqTime = t2 - t1;
    std::cout << "Seq: " << seqTime << std::endl;
    std::cout << "Acceleration: " << seqTime / ompTime << std::endl;
    ASSERT_EQ(vec1, vec2);
}*/

//TBB bitwise_sort.h
// Copyright 2021 Makarychev Sergey
#ifndef  MODULES_TASK_3_MAKARYCHEV_S_BITWISE_SORT_BITWISE_SORT_H_
#define MODULES_TASK_3_MAKARYCHEV_S_BITWISE_SORT_BITWISE_SORT_H_
#include <iostream>
#include <vector>

std::vector<int> getRandomVector(int sizeVec);
void createCounters(int* sortVec, int* counters, int sizeVec);
void signedRadixSort(std::vector<int>* sortVec);
void countersSort(int numByte, std::vector<int>* sortVec,
  std::vector<int>* output);
void unsignedRadixSort(std::vector<int>* sortVec);
void signedRadixSortParallel(std::vector<int>* sortVec,
  int leftIndex, int rightIndex);
void mergeOrderVec(int* vec1, int size1, int* vec2, int size2);
void signedRadixSortTbb(std::vector<int>* sortVec);

#endif  // MODULES_TASK_3_MAKARYCHEV_S_BITWISE_SORT_BITWISE_SORT_H_

//TBB bitwise_sort.cpp
// Copyright 2021 Makarychev Sergey
#include <tbb/blocked_range.h>
#include <tbb/parallel_for.h>
#include <random>
#include <algorithm>
#include <vector>
#include "../../../modules/task_3/makarychev_s_bitwise_sort/bitwise_sort.h"

std::vector<int> getRandomVector(int sizeVec) {
    if (sizeVec < 0)
        throw "Wrong vector size";
    std::vector<int> rVec(sizeVec);
    std::random_device rd;
    std::mt19937 mersenne(rd());
    std::uniform_int_distribution<> uid(-1000, 2000);
    for (int i = 0; i < sizeVec; i++)
        rVec[i] = uid(mersenne);
    return rVec;
}

void mergeOrderVec(int* vec1, int size1, int* vec2, int size2) {
    int* resVec = new int[size1 + size2];
    int i = 0, s = 0, j = 0;
    while (i < size1 && j < size2) {
        if (vec1[i] < vec2[j])
            resVec[s++] = vec1[i++];
        else
            resVec[s++] = vec2[j++];
    }
    while (i < size1)
        resVec[s++] = vec1[i++];
    while (j < size2)
        resVec[s++] = vec2[j++];
    i = s = 0;
    while (i < size1 + size2)
        vec1[i++] = resVec[s++];
    delete[] resVec;
}

void createCounters(std::vector<int>* sortVec, int* counters, int numByte) {
    unsigned char* bytePointer = (unsigned char*)sortVec->data();
    int s = static_cast<int>(sizeof(int));
    for (size_t i = 0; i < sortVec->size(); i++) {
        counters[bytePointer[s * i + numByte]]++;
    }
}

void countersSort(int numByte, std::vector<int>* sortVec,
    std::vector<int>* output) {
    unsigned char* bytePointer = (unsigned char*)sortVec->data();
    const int sizeCounter = 256;
    int counters[sizeCounter] = { 0 };
    int typeSize = static_cast<int>(sizeof(int));
    createCounters(sortVec, counters, numByte);
    int value = 0;
    for (int i = 0; i < sizeCounter; i++) {
        int tmp = counters[i];
        counters[i] = value;
        value += tmp;
    }
    int index = 0;
    for (size_t i = 0; i < sortVec->size(); i++) {
        index = bytePointer[typeSize * i + numByte];
        output->at(counters[index]++) = sortVec->at(i);
    }
}

void unsignedRadixSort(std::vector<int>* sortVec) {
    std::vector<int>* outbuf = new std::vector<int>(sortVec->size());
    for (int i = 0; i < 4; i++) {
        countersSort(i, sortVec, outbuf);
        std::swap(sortVec, outbuf);
    }
}

void signedRadixSort(std::vector<int>* sortVec) {
    int positiveNum = 0;
    int negativeNum = 0;
    for (size_t i = 0; i < sortVec->size(); i++) {
        if (sortVec->at(i) < 0)
            negativeNum++;
        else
            positiveNum++;
    }
    std::vector<int> positiveNumVec(positiveNum);
    std::vector<int> negativeNumVec(negativeNum);

    int it1 = 0, it2 = 0;
    for (size_t i = 0; i < sortVec->size(); i++) {
        if (sortVec->at(i) >= 0)
            positiveNumVec[it1++] = sortVec->at(i);
        else
            negativeNumVec[it2++] = sortVec->at(i);
    }

    unsignedRadixSort(&positiveNumVec);
    unsignedRadixSort(&negativeNumVec);

    int current_buff = 0;
    for (size_t i = 0; i < negativeNumVec.size(); i++) {
        sortVec->at(i) = negativeNumVec.at(current_buff);
        current_buff++;
    }
    current_buff = 0;
    for (size_t i = negativeNumVec.size();
        i < positiveNumVec.size() + negativeNumVec.size(); i++) {
        sortVec->at(i) = positiveNumVec.at(current_buff);
        current_buff++;
    }
}



void signedRadixSortParallel(std::vector<int>* sortVec,
    int leftIndex, int rightIndex) {

    int positiveNum = 0;
    int negativeNum = 0;
    for (int i = leftIndex; i <= rightIndex; i++) {
        if (sortVec->at(i) < 0)
            negativeNum++;
        else
            positiveNum++;
    }
    std::vector<int> positiveNumVec(positiveNum);
    std::vector<int> negativeNumVec(negativeNum);

    int it1 = 0, it2 = 0;
    for (int i = leftIndex; i <= rightIndex; i++) {
        if (sortVec->at(i) >= 0)
            positiveNumVec[it1++] = sortVec->at(i);
        else
            negativeNumVec[it2++] = sortVec->at(i);
    }

    unsignedRadixSort(&positiveNumVec);
    unsignedRadixSort(&negativeNumVec);

    int currentId = 0;
    for (size_t i = leftIndex; i < negativeNumVec.size() + leftIndex; i++) {
        sortVec->at(i) = negativeNumVec.at(currentId);
        currentId++;
    }
    currentId = 0;
    for (size_t i = leftIndex + negativeNumVec.size();
        i < positiveNumVec.size() + leftIndex + negativeNumVec.size(); i++) {
        sortVec->at(i) = positiveNumVec.at(currentId);
        currentId++;
    }
}

void signedRadixSortTbb(std::vector<int>* sortVec) {
    int previous = 0;
    int numberThreads = 4;
    int sizeVec = sortVec->size();
    int sizePartVec = sizeVec / numberThreads;
    int grainsize = sizePartVec;
    tbb::parallel_for(tbb::blocked_range<int>(0, sizeVec, grainsize),
        [&sortVec](const tbb::blocked_range<int>& r) {
        signedRadixSortParallel(sortVec, r.begin(), r.end() - 1);
        });
    for (int i = 1; i < sizeVec; i++) {
        if (sortVec->at(i) < sortVec->at(i - 1)) {
                mergeOrderVec(sortVec->data(), previous,
                    sortVec->data() + previous, i - previous);
                previous = i;
        }
    }
    mergeOrderVec(sortVec->data(), previous, sortVec->data() + previous,
        sizeVec - previous);
}

//TBB main.cpp
// Copyright 2021 Makarychev Sergey
#include <gtest/gtest.h>
#include <tbb/tick_count.h>
#include <algorithm>
#include <vector>
#include "./bitwise_sort.h"

TEST(Sequential_Radix_Sort, sort_vector_of_random_value) {
    std::vector<int> vec1 = getRandomVector(78);
    std::vector<int> vec2 = vec1;
    std::sort(vec2.begin(), vec2.begin() + vec2.size());
    signedRadixSort(&vec1);
    ASSERT_EQ(vec1, vec2);
}

TEST(Omp_Radix_Sort, sort_vector_of_litle_positive_value) {
    std::vector<int> vec1 = { 1, 3, 7, 4, 0, 7, 0, 9 };
    std::vector<int> vec2 = vec1;
    std::sort(vec2.begin(), vec2.begin() + vec2.size());
    signedRadixSortTbb(&vec1);
    ASSERT_EQ(vec1, vec2);
}

TEST(Omp_Radix_Sort, sort_vector_of_seq_omp) {
    std::vector<int> vec1 = { 1, 3, 7, 4, 0, 7, 0, 9 };
    std::vector<int> vec2 = vec1;
    signedRadixSortTbb(&vec1);
    signedRadixSort(&vec2);
    ASSERT_EQ(vec1, vec2);
}

TEST(Omp_Radix_Sort, sort_vector_of_litle_negative_value) {
    std::vector<int> vec1 = { -1, -8, -4, -10, -1, -6, -6, -6 };
    std::vector<int> vec2 = vec1;
    signedRadixSortTbb(&vec1);
    std::sort(vec2.begin(), vec2.begin() + vec2.size());
    ASSERT_EQ(vec1, vec2);
}

TEST(Omp_Radix_Sort, sort_vector_of_zero_value) {
    std::vector<int> vec1 = { 0, 0, 0, 0, 0, 0, 0, 0 };
    std::vector<int> vec2 = vec1;
    signedRadixSortTbb(&vec1);
    std::sort(vec2.begin(), vec2.begin() + vec2.size());
    ASSERT_EQ(vec1, vec2);
}

TEST(Omp_Radix_Sort, sort_vector_of_mixed_value) {
    std::vector<int> vec1 = { 2, -99947, -9, 87456, -987456, 0 };
    std::vector<int> vec2 = vec1;
    signedRadixSortTbb(&vec1);
    std::sort(vec2.begin(), vec2.begin() + vec2.size());
    ASSERT_EQ(vec1, vec2);
}

TEST(Omp_Radix_Sort, sort_just_sorted_vector) {
    std::vector<int> vec1 = { -99999, -1, 0, 8, 9, 15, 10265, 235458 };
    std::vector<int> vec2 = vec1;
    signedRadixSortTbb(&vec1);
    std::sort(vec2.begin(), vec2.begin() + vec2.size());
    ASSERT_EQ(vec1, vec2);
}

TEST(Seqvssspar, Test_Only_Positive1) {
    std::vector<int> vec1 = getRandomVector(1500000);
    std::vector<int> vec2 = vec1;
    tbb::tick_count t1Seq = tbb::tick_count::now();
    signedRadixSort(&vec1);
    tbb::tick_count t2Seq = tbb::tick_count::now();
    double time_seq = (t2Seq - t1Seq).seconds();
    std::cout << "Seq: " << time_seq << std::endl;
    tbb::tick_count t1_par = tbb::tick_count::now();
    signedRadixSortTbb(&vec2);
    tbb::tick_count t2_par = tbb::tick_count::now();
    double time_par = (t2_par - t1_par).seconds();
    std::cout << "Par: " << time_par << std::endl;
    std::cout << "Acceleration " << time_seq / time_par << std::endl;
    ASSERT_EQ(vec2, vec1);
}

TEST(Seqvssspar, Test_Only_Positive2) {
    std::vector<int> vec1 = getRandomVector(15000000);
    std::vector<int> vec2 = vec1;
    tbb::tick_count t1Seq = tbb::tick_count::now();
    signedRadixSort(&vec1);
    tbb::tick_count t2Seq = tbb::tick_count::now();
    double time_seq = (t2Seq - t1Seq).seconds();
    std::cout << "Seq: " << time_seq << std::endl;
    tbb::tick_count t1_par = tbb::tick_count::now();
    signedRadixSortTbb(&vec2);
    tbb::tick_count t2_par = tbb::tick_count::now();
    double time_par = (t2_par - t1_par).seconds();
    std::cout << "Par: " << time_par << std::endl;
    std::cout << "Acceleration " << time_seq / time_par << std::endl;
    ASSERT_EQ(vec2, vec1);
}

 TEST(Seqvssspar, Test_Only_Positive) {
     std::vector<int> vec1 = getRandomVector(150000000);
     std::vector<int> vec2 = vec1;
     tbb::tick_count t1Seq = tbb::tick_count::now();
     signedRadixSort(&vec1);
     tbb::tick_count t2Seq = tbb::tick_count::now();
     double time_seq = (t2Seq - t1Seq).seconds();
     std::cout << "Seq: " << time_seq << std::endl;
     tbb::tick_count t1_par = tbb::tick_count::now();
     signedRadixSortTbb(&vec2);
     tbb::tick_count t2_par = tbb::tick_count::now();
     double time_par = (t2_par - t1_par).seconds();
     std::cout << "Par: " << time_par << std::endl;
     std::cout << "Acceleration " << time_seq / time_par << std::endl;
     ASSERT_EQ(vec2, vec1);
 }

\end{lstlisting}
\end{document}