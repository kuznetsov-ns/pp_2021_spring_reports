\documentclass{report}

\usepackage[T2A]{fontenc}
\usepackage[utf8]{luainputenc}
\usepackage[english, russian]{babel}
\usepackage[pdftex]{hyperref}
\usepackage[14pt]{extsizes}
\usepackage{listings}
\usepackage{color}
\usepackage{geometry}
\usepackage{enumitem}
\usepackage{multirow}
\usepackage{graphicx}
\usepackage{indentfirst}
\usepackage[ruled,vlined,linesnumbered]{algorithm2e}

\geometry{a4paper,top=2cm,bottom=3cm,left=2cm,right=1.5cm}
\setlength{\parskip}{0.5cm}
\setlist{nolistsep, itemsep=0.3cm,parsep=0pt}

\lstset{
    backgroundcolor=color{gray!10},  
    language=C++,
    basicstyle=\footnotesize,
    keywordstyle=\color{blue}\ttfamily,
    stringstyle=\color{red}\ttfamily,
    commentstyle=\color{green}\ttfamily,
    morecomment=[l][\color{magenta}]{\#},
    tabsize=4,
    breaklines=true,
    breakatwhitespace=true,
    captionpos=b, 
    extendedchars=true,
    showspaces=false,               
    showtabs=false,
    title=\lstname,
}

\makeatletter
\renewcommand\@biblabel[1]{#1.\hfil}
\makeatother

\begin{document}

\begin{titlepage}

\begin{center}
Министерство науки и высшего образования Российской Федерации
\end{center}

\begin{center}
Федеральное государственное автономное образовательное учреждение высшего образования \\
Национальный исследовательский Нижегородский государственный университет им. Н.И. Лобачевского
\end{center}

\begin{center}
Институт информационных технологий, математики и механики
\end{center}

\vspace{4em}

\begin{center}
\textbf{\LargeОтчет по лабораторной работе} \\
\end{center}
\begin{center}
\textbf{\Large«Умножение плотных матриц. Элементы типа double. Алгоритм Штрассена.»} \\
\end{center}

\vspace{4em}

\newbox{\lbox}
\savebox{\lbox}{\hbox{text}}
\newlength{\maxl}
\setlength{\maxl}{\wd\lbox}
\hfill\parbox{7cm}{
\hspace*{5cm}\hspace*{-5cm}\textbf{Выполнил:} \\ студент группы 381808-2 \\ Крень Полина\\
\\
\hspace*{5cm}\hspace*{-5cm}\textbf{Проверил:}\\ доцент кафедры МОСТ, \\ кандидат технических наук \\ Сысоев А. В.\\
}
\vspace{\fill}

\begin{center} 
Нижний Новгород \\ 2021 
\end{center}
\end{titlepage}

\setcounter{page}{2}

\newpage

% Введение
\section*{Введение}
\addcontentsline{toc}{section}{Введение}
Умножение матриц это одна из основных операций над матрицами. Сложность вычисления произведения матриц по определению составляет  , однако существуют более эффективные алгоритмы, применяющиеся для больших матриц. Вопрос о предельной скорости умножения больших матриц, также как и вопрос о построении наиболее быстрых и устойчивых практических алгоритмов умножения больших матриц остаётся одной из нерешённых проблем линейной алгебры. Примерами более эффективных алгоритмов умножения матриц являются: алгоритм Штрассена, алгоритм Пана, алгоритм Бини. В рамках данных лабораторных работ будет рассмотрен алгоритм Штрассена, разработанный в 1969 году Фолькером Штрассеном.
\par
В отличие от традиционного алгоритма умножения матриц, что даёт выигрыш на больших плотных матрицах начиная от 64×64. Несмотря на то, что алгоритм Штрассена является асимптотически не самым быстрым из существующих алгоритмов быстрого умножения матриц, он проще программируется и эффективнее при умножении матриц относительно малого размера, поэтому именно он чаще используется на практике.
\par
Целью данной работы является реализация параллельноого умножения плотных матриц с использованием алгоритма Штрассена и сравнение ее производительности с последовательной реализацией.
\newpage

% Постановка задачи
\section*{Постановка задачи}
\addcontentsline{toc}{section}{Постановка задачи}
В рамках лабораторной работы нужно реализовать параллельный алгоритм Штрассена для умножения плотных матриц.
\par
Требуется выполнить следующее: последовательный алгоритм Штрассена, параллельный алгоритм Штрассена  с использованием технологий OpenMP и TBB, провести вычислительные эксперименты, сравнить время работы параллельной и последовательной реализаций.
\newpage

% Метод решения
\section*{Метод решения}
\addcontentsline{toc}{section}{Метод решения}
Рассматривается умножение квадратных матриц размера N × N, где N = p· $2^k$ и p < M. Умножение двух матриц записывается как C = A · B, где A, B и C матрицы размера N × N. Метод Штрассена для умножения матриц основан на рекурсивном делении каждой перемножаемой матрицы на 4 подматрицы и выполнения операций над ними. Требование к размеру матриц (N = p· $2^k$ и p < M) нужно, чтобы было возможным разбить каждую матрицу на 4 ${{N}\over{2}}$ × ${{N}\over{2}}$ подматрицы и выполнить умножение по формуле 1. Под M здесь понимается пороговое значение размера матриц, после которого вызывается стандартный алгоритм.
\par
$C = \left[\begin{array}{ccccc} 
A_{11} & A_{12} \\ 
A_{21} & A_{22} \\ 
\end{array}\right]$ × 
$\left[\begin{array}{ccccc} 
B_{11} & B_{12} \\ 
B_{21} & B_{22} \\  
\end{array}\right]$ =
$\left[\begin{array}{ccccc} 
(S_{1} + S_{4} - S_{5} + S_{7} & S_{3} + S_{5} \\ 
S_{2} + S_{4} & S_{1} - S_{2} + S_{3} + S_{6} \\  
\end{array}\right]$ (1), 
где
\begin{center} 
$S_1=(A_{11}+A_{22})*(B_{11}+B_{22})$

$S_2=(A_{21}+A_{22})*B_{11}$

$S_3=A_{11}*(B_{12}-B_{22})$

$S_4=A_{22}*(B_{21}-B_{11})$

$S_5=(A_{11}+A_{12})*B_{22}$

$S_6=(A_{21}-A_{11})*(B_{11}+B_{12})$

$S_7=(A_{12}-A_{22})*(B_{21}+B_{22})$
\end{center}
\par
Таким образом одна процедура умножения матриц размера N × N уменьшается до 7 умножений матриц размера ${{N}\over{2}}$ × ${{N}\over{2}}$ (при стандартном подходе требуется 8 умножений). Далее происходит разбиение каждого перемножения рекурсивно, до тех пор пока размер матриц не достигнет порогового значения размера матрицы.

\newpage

% Схема распараллеливания
\section*{Схема распараллеливания}
\addcontentsline{toc}{section}{Схема распараллеливания}
Была реализована самая простая схема распараллеливания - все 7 умножений подматриц выполняются в отдельных потоках (потоки порождаются на всех уровнях рекурсии).
\par
\begin{center} 
$S_1=(A_{11}+A_{22})*(B_{11}+B_{22})$

$S_2=(A_{21}+A_{22})*B_{11}$

$S_3=A_{11}*(B_{12}-B_{22})$

$S_4=A_{22}*(B_{21}-B_{11})$

$S_5=(A_{11}+A_{12})*B_{22}$

$S_6=(A_{21}-A_{11})*(B_{11}+B_{12})$

$S_7=(A_{12}-A_{22})*(B_{21}+B_{22})$
\end{center}
Деление матрицы происходит следующим образом: делим количество столбцов и количество строк на корень из количества процессов, остатки от деления распределяем между процессами.
\par
В реализации с использования технологии OpenMP также был распараллелен процесс разбиения матрицы на части и слияние частей в единую матрицу.
\newpage

% Описание программной реализации
\section*{Описание программной реализации}
\addcontentsline{toc}{section}{Описание программной реализации}
\par Для параллельной реализации алгоритма Штрассена были использованы технологии: OpenMP и TBB.
\par Для реализации алгоритмы были созданы методы:
\begin{itemize}
\item Деление матрицы на части: 
\end{itemize}
\begin{lstlisting}
	void SplitMatrixIntoPieces(const std::vector<double>& matrix, std::vector<double>* A11, std::vector<double>* A22, std::vector<double>* A12, std::vector<double>* A21, const int MatrixSize);
\end{lstlisting}
\begin{itemize}
\item Слияние частей матрицы в единую матрицу:
\end{itemize}
\begin{lstlisting}
	void RebuildMatrix(std::vector<double>* MatrixResult,
    const std::vector<double>& A11, const std::vector<double>& A22, const std::vector<double>& A12, const std::vector<double>& A21, const int MatrixSize);
\end{lstlisting}
\par Для последовательной реализации алгоритма Штрассена был создан следующий метод: 
\begin{lstlisting}
    void StrassenAlgorithm(const std::vector<double>& matrix1, const std::vector<double>& matrix2, std::vector<double>* result);
\end{lstlisting}
\par Для параллельной реализации с использованием технологии OpenMP используется директива: 
\verb|#pragma omp parallel sections|. 
\par Для реализация алгоритма Штрассена с помощью технологии OpenMP создан следующий метод:
\begin{lstlisting}
    void StrassenAlgorithm_omp(const std::vector<double>& matrix1,const std::vector<double>& matrix2, std::vector<double>* MatrixResult);
\end{lstlisting}

\par Для параллельной реализации с использованием технологии TBB используется: 
\verb|tbb::task_group|.
\par Для реализация алгоритма Штрассена с помощью технологии TBB создан следующий метод:
\begin{lstlisting}
    void StrassenAlgorithm_tbb(const std::vector<double>& matrix1, const std::vector<double>& matrix2, std::vector<double>* matrix3);
\end{lstlisting}

\newpage
% Результаты экспериментов
\section*{Результаты экспериментов}
\addcontentsline{toc}{section}{Результаты экспериментов}
Эксперименты проводились на 4-х ядерном компьютере с операционной системой – Windows 10. Результаты экспериментов приведены в таблице:
\begin{table}[!h]
\caption{Результаты вычислительных экспериментов}
\centering
\begin{tabular}{lllll}
Количество потоков & OpenMp & TBB & Разница\\
1 & 1.002 & 1.027 & 0.025\\
2 & 1.856 & 1.535 & 0.321\\
4 & 2.624 & 2.376 & 0.248\\
\end{tabular}
\end{table}
\par Можно сделать вывод о том, что для распараллеливания алгоритма Штрассена лучшей является технология OpenMP.

\par Оценить корректность работы программы могут результаты гугл-тестов.
\newpage

% Заключение
\section*{Заключение}
\addcontentsline{toc}{section}{Заключение}
Результатом лабораторной работы является параллельная реализация алгоритма Штрассена.

\par Параллельная реализация алгоритма с использованием технологий OpenMP и TBB выполнена хорошо. Данный факт следует из результатов экспериментов, проведенных в ходе работы: лучшая технология для реализации параллельного алгоритма Штрассена является OpenMP.

\newpage

% Список литературы
\begin{thebibliography}{1}
\addcontentsline{toc}{section}{Список литературы}
\bibitem{OMP} Гергель В.П. Учебный курс «Введение в методы параллельного программирования», раздел «Параллельное программирование с использованием OpenMP». Нижний Новгород, ННГУ, 2007.
\bibitem{TBB} А.А. Сиднев, А.В. Сысоев, И.Б. Мееров. Учебный курс «Технологии разработки параллельных программ», раздел «Создание параллельной программы», «Библиотека Intel Threading Building Blocks~--- краткое описание». Нижний Новгород, ННГУ, 2007. 

\end{thebibliography}
\newpage

% Приложение
\section*{Приложение}
\addcontentsline{toc}{section}{Приложение}
\begin{lstlisting}

#include "../../modules/task_2/kren_p_matrix_omp/matrix.h"
#include <math.h>
#include <omp.h>
#include <algorithm>
#include <vector>

void AdditionMatrix(const std::vector<double>& matrix1,
                    const std::vector<double>& matrix2,
                    std::vector<double>* matrix3) {
  for (unsigned int i = 0; i < matrix3->size(); i++) {
    matrix3->at(i) += matrix1[i];
  }
  for (unsigned int i = 0; i < matrix3->size(); i++) {
    matrix3->at(i) += matrix2[i];
  }
}

void SubtractionMatrix(const std::vector<double>& matrix1,
                       const std::vector<double>& matrix2,
                       std::vector<double>* matrix3) {
  for (unsigned int i = 0; i < matrix3->size(); i++) {
    matrix3->at(i) += matrix1[i];
  }
  for (unsigned int i = 0; i < matrix3->size(); i++) {
    matrix3->at(i) -= matrix2[i];
  }
}

void MultiplicationMatrix(const std::vector<double>& matrix1,
                          const std::vector<double>& matrix2,
                          std::vector<double>* matrix3, const int MatrixSize) {
  for (int i = 0; i < MatrixSize; i++) {
    for (int j = 0; j < MatrixSize; j++) {
      for (int k = 0; k < MatrixSize; k++) {
        matrix3->at(i * MatrixSize + j) +=
            matrix1[i * MatrixSize + k] * matrix2[k * MatrixSize + j];
      }
    }
  }
}

int CheckMatrixSize(const int MatrixSize) {
  int MatrixResultSize = 2;
  while (MatrixSize > MatrixResultSize) {
    MatrixResultSize = MatrixResultSize * 2;
  }
  return MatrixResultSize;
}

std::vector<double> ResizeMatrix(const std::vector<double>& matrix,
                                 const int MatrixSize) {
  const int MatrixResultSize = CheckMatrixSize(MatrixSize);
  std::vector<double> MatrixResult(MatrixResultSize * MatrixResultSize);
  for (int i = 0; i < MatrixResultSize * MatrixResultSize; i++) {
    MatrixResult[i] = 0;
  }
  for (int i = 0; i < MatrixSize; i++) {
    for (int j = 0; j < MatrixSize; j++) {
      MatrixResult[i * MatrixResultSize + j] = matrix[i * MatrixSize + j];
    }
  }
  return MatrixResult;
}

void SplitMatrixIntoPiecesOmp(const std::vector<double>& matrix,
                              std::vector<double>* A11,
                              std::vector<double>* A22,
                              std::vector<double>* A12,
                              std::vector<double>* A21, const int MatrixSize) {
  const int SizeOfPiece = MatrixSize / 2;
#pragma omp parallel for default(shared)
  for (int i = 0; i < SizeOfPiece; i++) {
    for (int j = 0; j < SizeOfPiece; j++) {
      A11->at(i * SizeOfPiece + j) = matrix[i * MatrixSize + j];
      A12->at(i * SizeOfPiece + j) = matrix[i * MatrixSize + j + SizeOfPiece];
      A21->at(i * SizeOfPiece + j) = matrix[(i + SizeOfPiece) * MatrixSize + j];
      A22->at(i * SizeOfPiece + j) =
          matrix[(i + SizeOfPiece) * MatrixSize + j + SizeOfPiece];
    }
  }
}

std::vector<double> RebuildMatrix(const std::vector<double>& A11,
                                  const std::vector<double>& A22,
                                  const std::vector<double>& A12,
                                  const std::vector<double>& A21,
                                  const int MatrixSize) {
  const int MatrixResultSize = MatrixSize * 2;
  std::vector<double> MatrixResult(MatrixResultSize * MatrixResultSize);
  for (int i = 0; i < MatrixSize; i++) {
    for (int j = 0; j < MatrixSize; j++) {
      MatrixResult[i * MatrixResultSize + j] = A11[i * MatrixSize + j];
      MatrixResult[i * MatrixResultSize + j + MatrixSize] =
          A12[i * MatrixSize + j];
      MatrixResult[(i + MatrixSize) * MatrixResultSize + j] =
          A21[i * MatrixSize + j];
      MatrixResult[(i + MatrixSize) * MatrixResultSize + j + MatrixSize] =
          A22[i * MatrixSize + j];
    }
  }
  return MatrixResult;
}

void RebuildMatrixOmp(std::vector<double>* MatrixResult,
                      const std::vector<double>& A11,
                      const std::vector<double>& A22,
                      const std::vector<double>& A12,
                      const std::vector<double>& A21, const int MatrixSize) {
  const int MatrixResultSize = MatrixSize * 2;
#pragma omp parallel for default(shared)
  for (int i = 0; i < MatrixSize; i++) {
    for (int j = 0; j < MatrixSize; j++) {
      MatrixResult->at(i * MatrixResultSize + j) = A11[i * MatrixSize + j];
      MatrixResult->at(i * MatrixResultSize + j + MatrixSize) =
          A12[i * MatrixSize + j];
      MatrixResult->at((i + MatrixSize) * MatrixResultSize + j) =
          A21[i * MatrixSize + j];
      MatrixResult->at((i + MatrixSize) * MatrixResultSize + j + MatrixSize) =
          A22[i * MatrixSize + j];
    }
  }
}

void StrassenAlgorithm_omp(const std::vector<double>& matrix1,
                           const std::vector<double>& matrix2,
                           std::vector<double>* MatrixResult) {
  int MatrixSize = sqrt(MatrixResult->size());

  if (MatrixSize <= 2) {
    return MultiplicationMatrix(matrix1, matrix2, MatrixResult, MatrixSize);
  }

  MatrixSize = MatrixSize / 2;
  int MatrixResultSize = MatrixSize * MatrixSize;

  std::vector<double> a11(MatrixResultSize);
  std::vector<double> a21(MatrixResultSize);
  std::vector<double> a12(MatrixResultSize);
  std::vector<double> a22(MatrixResultSize);
  std::vector<double> b11(MatrixResultSize);
  std::vector<double> b12(MatrixResultSize);
  std::vector<double> b21(MatrixResultSize);
  std::vector<double> b22(MatrixResultSize);

  SplitMatrixIntoPiecesOmp(matrix1, &a11, &a22, &a12, &a21, MatrixSize * 2);
  SplitMatrixIntoPiecesOmp(matrix2, &b11, &b22, &b12, &b21, MatrixSize * 2);

  std::vector<double> p1(MatrixResultSize);
  std::vector<double> p2(MatrixResultSize);
  std::vector<double> p3(MatrixResultSize);
  std::vector<double> p4(MatrixResultSize);
  std::vector<double> p5(MatrixResultSize);
  std::vector<double> p6(MatrixResultSize);
  std::vector<double> p7(MatrixResultSize);

#pragma omp parallel sections default(shared)
  {
#pragma omp section
    {
      std::vector<double> MatrixTemp1(MatrixResultSize);
      std::vector<double> MatrixTemp2(MatrixResultSize);
      AdditionMatrix(a11, a22, &MatrixTemp1);
      AdditionMatrix(b11, b22, &MatrixTemp2);
      StrassenAlgorithm_omp(MatrixTemp1, MatrixTemp2, &p1);
    }
#pragma omp section
    {
      std::vector<double> MatrixTemp1(MatrixResultSize);
      AdditionMatrix(a21, a22, &MatrixTemp1);
      StrassenAlgorithm_omp(MatrixTemp1, b11, &p2);
    }
#pragma omp section
    {
      std::vector<double> MatrixTemp1(MatrixResultSize);
      SubtractionMatrix(b12, b22, &MatrixTemp1);
      StrassenAlgorithm_omp(a11, MatrixTemp1, &p3);
    }
#pragma omp section
    {
      std::vector<double> MatrixTemp1(MatrixResultSize);
      SubtractionMatrix(b21, b11, &MatrixTemp1);
      StrassenAlgorithm_omp(a22, MatrixTemp1, &p4);
    }
#pragma omp section
    {
      std::vector<double> MatrixTemp1(MatrixResultSize);
      AdditionMatrix(a11, a12, &MatrixTemp1);
      StrassenAlgorithm_omp(MatrixTemp1, b22, &p5);
    }
#pragma omp section
    {
      std::vector<double> MatrixTemp1(MatrixResultSize);
      std::vector<double> MatrixTemp2(MatrixResultSize);
      SubtractionMatrix(a21, a11, &MatrixTemp1);
      AdditionMatrix(b11, b12, &MatrixTemp2);
      StrassenAlgorithm_omp(MatrixTemp1, MatrixTemp2, &p6);
    }
#pragma omp section
    {
      std::vector<double> MatrixTemp1(MatrixResultSize);
      std::vector<double> MatrixTemp2(MatrixResultSize);
      SubtractionMatrix(a12, a22, &MatrixTemp1);
      AdditionMatrix(b21, b22, &MatrixTemp2);
      StrassenAlgorithm_omp(MatrixTemp1, MatrixTemp2, &p7);
    }
  }

  std::vector<double> c11(MatrixResultSize);
  std::vector<double> c12(MatrixResultSize);
  std::vector<double> c21(MatrixResultSize);
  std::vector<double> c22(MatrixResultSize);

  AdditionMatrix(p3, p5, &c12);
  AdditionMatrix(p2, p4, &c21);
  std::vector<double> MatrixTemp1(MatrixResultSize);
  std::vector<double> MatrixTemp2(MatrixResultSize);
  AdditionMatrix(p1, p4, &MatrixTemp1);
  AdditionMatrix(MatrixTemp1, p7, &MatrixTemp2);
  SubtractionMatrix(MatrixTemp2, p5, &c11);

  std::vector<double> MatrixTemp3(MatrixResultSize);
  std::vector<double> MatrixTemp4(MatrixResultSize);
  AdditionMatrix(p1, p3, &MatrixTemp3);
  AdditionMatrix(MatrixTemp3, p6, &MatrixTemp4);
  SubtractionMatrix(MatrixTemp4, p2, &c22);
  RebuildMatrixOmp(MatrixResult, c11, c22, c12, c21, MatrixSize);
}


#include "../../modules/task_3/kren_p_matrix_tbb/matrix.h"
#include <math.h>
#include <tbb/task_group.h>
#include <tbb/tbb.h>
#include <algorithm>
#include <vector>

void AdditionMatrix(const std::vector<double>& matrix1,
                    const std::vector<double>& matrix2,
                    std::vector<double>* matrix3) {
  for (unsigned int i = 0; i < matrix3->size(); i++) {
    matrix3->at(i) += matrix1[i];
  }
  for (unsigned int i = 0; i < matrix3->size(); i++) {
    matrix3->at(i) += matrix2[i];
  }
}

void SubtractionMatrix(const std::vector<double>& matrix1,
                       const std::vector<double>& matrix2,
                       std::vector<double>* matrix3) {
  for (unsigned int i = 0; i < matrix3->size(); i++) {
    matrix3->at(i) += matrix1[i];
  }
  for (unsigned int i = 0; i < matrix3->size(); i++) {
    matrix3->at(i) -= matrix2[i];
  }
}

void MultiplicationMatrix(const std::vector<double>& matrix1,
                          const std::vector<double>& matrix2,
                          std::vector<double>* matrix3, const int MatrixSize) {
  for (int i = 0; i < MatrixSize; i++) {
    for (int j = 0; j < MatrixSize; j++) {
      for (int k = 0; k < MatrixSize; k++) {
        matrix3->at(i * MatrixSize + j) +=
            matrix1[i * MatrixSize + k] * matrix2[k * MatrixSize + j];
      }
    }
  }
}

int CheckMatrixSize(const int MatrixSize) {
  int MatrixResultSize = 2;
  while (MatrixSize > MatrixResultSize) {
    MatrixResultSize = MatrixResultSize * 2;
  }
  return MatrixResultSize;
}

std::vector<double> ResizeMatrix(const std::vector<double>& matrix,
                                 const int MatrixSize) {
  const int MatrixResultSize = CheckMatrixSize(MatrixSize);
  std::vector<double> MatrixResult(MatrixResultSize * MatrixResultSize);
  for (int i = 0; i < MatrixResultSize * MatrixResultSize; i++) {
    MatrixResult[i] = 0;
  }
  for (int i = 0; i < MatrixSize; i++) {
    for (int j = 0; j < MatrixSize; j++) {
      MatrixResult[i * MatrixResultSize + j] = matrix[i * MatrixSize + j];
    }
  }
  return MatrixResult;
}

void SplitMatrixIntoPieces(const std::vector<double>& matrix,
                           std::vector<double>* A11, std::vector<double>* A22,
                           std::vector<double>* A12, std::vector<double>* A21,
                           const int MatrixSize) {
  const int SizeOfPiece = MatrixSize / 2;
  for (int i = 0; i < SizeOfPiece; i++) {
    for (int j = 0; j < SizeOfPiece; j++) {
      A11->at(i * SizeOfPiece + j) = matrix[i * MatrixSize + j];
      A12->at(i * SizeOfPiece + j) = matrix[i * MatrixSize + j + SizeOfPiece];
      A21->at(i * SizeOfPiece + j) = matrix[(i + SizeOfPiece) * MatrixSize + j];
      A22->at(i * SizeOfPiece + j) =
          matrix[(i + SizeOfPiece) * MatrixSize + j + SizeOfPiece];
    }
  }
}

void RebuildMatrix(std::vector<double>* MatrixResult,
                   const std::vector<double>& A11,
                   const std::vector<double>& A22,
                   const std::vector<double>& A12,
                   const std::vector<double>& A21, const int MatrixSize) {
  const int MatrixResultSize = MatrixSize * 2;
  for (int i = 0; i < MatrixSize; i++) {
    for (int j = 0; j < MatrixSize; j++) {
      MatrixResult->at(i * MatrixResultSize + j) = A11[i * MatrixSize + j];
      MatrixResult->at(i * MatrixResultSize + j + MatrixSize) =
          A12[i * MatrixSize + j];
      MatrixResult->at((i + MatrixSize) * MatrixResultSize + j) =
          A21[i * MatrixSize + j];
      MatrixResult->at((i + MatrixSize) * MatrixResultSize + j + MatrixSize) =
          A22[i * MatrixSize + j];
    }
  }
}

void StrassenAlgorithm_tbb(const std::vector<double>& matrix1,
                           const std::vector<double>& matrix2,
                           std::vector<double>* matrix3) {
  int MatrixSize = sqrt(matrix1.size());
  if (MatrixSize <= 2) {
    MultiplicationMatrix(matrix1, matrix2, matrix3, MatrixSize);
    return;
  }
  MatrixSize = MatrixSize / 2;
  int MatrixResultSize = MatrixSize * MatrixSize;

  std::vector<double> A11(MatrixResultSize);
  std::vector<double> A12(MatrixResultSize);
  std::vector<double> A21(MatrixResultSize);
  std::vector<double> A22(MatrixResultSize);
  std::vector<double> B11(MatrixResultSize);
  std::vector<double> B12(MatrixResultSize);
  std::vector<double> B21(MatrixResultSize);
  std::vector<double> B22(MatrixResultSize);

  SplitMatrixIntoPieces(matrix1, &A11, &A22, &A12, &A21, MatrixSize * 2);
  SplitMatrixIntoPieces(matrix2, &B11, &B22, &B12, &B21, MatrixSize * 2);

  std::vector<double> MainP1(MatrixResultSize);
  std::vector<double>* p1 = &MainP1;
  std::vector<double> MainP2(MatrixResultSize);
  std::vector<double>* p2 = &MainP2;
  std::vector<double> MainP3(MatrixResultSize);
  std::vector<double>* p3 = &MainP3;
  std::vector<double> MainP4(MatrixResultSize);
  std::vector<double>* p4 = &MainP4;
  std::vector<double> MainP5(MatrixResultSize);
  std::vector<double>* p5 = &MainP5;
  std::vector<double> MainP6(MatrixResultSize);
  std::vector<double>* p6 = &MainP6;
  std::vector<double> MainP7(MatrixResultSize);
  std::vector<double>* p7 = &MainP7;

  tbb::task_scheduler_init init(8);
  tbb::task_group tasks;
  tasks.run([=] {
    StrassenAlgorithm(AdditionMatrix(A11, A22, MatrixSize),
                      AdditionMatrix(B11, B22, MatrixSize), p1);
  });

  tasks.run([=] {
    StrassenAlgorithm(AdditionMatrix(A21, A22, MatrixSize), B11, p2);
  });

  tasks.run([=] {
    StrassenAlgorithm(A11, SubtractionMatrix(B12, B22, MatrixSize), p3);
  });

  tasks.run([=] {
    StrassenAlgorithm(A22, SubtractionMatrix(B21, B11, MatrixSize), p4);
  });

  tasks.run([=] {
    StrassenAlgorithm(AdditionMatrix(A11, A12, MatrixSize), B22, p5);
  });

  tasks.run([=] {
    StrassenAlgorithm(SubtractionMatrix(A21, A11, MatrixSize),
                      AdditionMatrix(B11, B12, MatrixSize), p6);
  });

  tasks.run([=] {
    StrassenAlgorithm(SubtractionMatrix(A12, A22, MatrixSize),
                      AdditionMatrix(B21, B22, MatrixSize), p7);
  });
  tasks.wait();

  std::vector<double> c11(MatrixResultSize);
  std::vector<double> c12(MatrixResultSize);
  std::vector<double> c21(MatrixResultSize);
  std::vector<double> c22(MatrixResultSize);

  AdditionMatrix(MainP3, MainP5, &c12);
  AdditionMatrix(MainP2, MainP4, &c21);
  AdditionMatrix(AdditionMatrix(MainP1, MainP4, MatrixSize),
                 SubtractionMatrix(MainP7, MainP5, MatrixSize), &c11);
  AdditionMatrix(SubtractionMatrix(MainP1, MainP2, MatrixSize),
                 AdditionMatrix(MainP3, MainP6, MatrixSize), &c22);

  RebuildMatrix(matrix3, c11, c22, c12, c21, MatrixSize);
}

\end{lstlisting}
    \end{document}