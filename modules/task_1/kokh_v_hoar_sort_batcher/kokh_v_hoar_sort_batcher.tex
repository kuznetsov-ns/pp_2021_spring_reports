\documentclass{report}

\usepackage[english, russian]{babel}
\usepackage{geometry}
\usepackage{listings}
\usepackage[T2A]{fontenc}
\usepackage[14pt]{extsizes}
\usepackage{verbatim}
\usepackage{multirow}
\usepackage{color}
\usepackage{graphicx}
\usepackage[titles]{tocloft}
\usepackage[hyphens]{url}


\geometry{a4paper,top=2cm,bottom=2cm,left=2cm,right=1.5cm}
\setlength{\parskip}{0.5cm}
\setcounter{tocdepth}{4}
\setcounter{secnumdepth}{4}
\sloppy

\definecolor{darkgreen}{rgb}{0.0, 0.5, 0.0}
\lstset{
language=C++,
basicstyle=\footnotesize,
numbers=left,
numberstyle=\tiny,
numberfirstline=true,
numbersep=5pt,
keywordstyle=\color{blue}\bfseries,
commentstyle=\color{darkgreen},
stringstyle=\color{red},
showspaces=false,
showstringspaces=false,
captionpos=t,
breaklines=true,
breakatwhitespace=false,
extendedchars=true,
frame=tb,
title=\lstname,
}

\begin{document}
\begin{titlepage}

\begin{center}
   
Министерство науки и высшего образования Российской Федерации
{Федеральное государственное автономное образовательное учреждение высшего образования} \\
\textbf{"Национальный исследовательский Нижегородский государственный университет им. Н.И. Лобачевского" (ННГУ)} \\
\textbf{Институт информационных технологий, математики и механики}

\vspace{\fill}

\textbf{\LargeОтчет по лабораторной работе \\}
\textbf{\large\\ «Сортировка Хоара с четно-нечетным слиянием Бэтчера»}

\vspace{\fill}

\hfill\parbox{8cm}{
\hspace*{5cm}\hspace*{-5cm}\textbf{Выполнил:} \\ Студент группы 381808-2 \\ Кох Владислав Альбертович \\ \\ \\
\hspace*{5cm}\hspace*{-5cm}\textbf{Проверил:}\\ Доцент кафедры МОСТ, \\ кандидат технических наук \\ Сысоев А. В.
\hbadness=20000
}

\vspace{\fill}

Нижний Новгород \\ 2021 г.
\end{center}

\end{titlepage}

\setcounter{page}{2}
\setlength{\cftsecindent}{0em}
\setlength{\cftsubsecindent}{1.25em}
\setlength{\cftsubsubsecindent}{2.5em}
\setlength{\cftsubsubsecnumwidth}{1.25em}
\tableofcontents


\newpage
\section*{Введение и постановка задачи}
\addcontentsline{toc}{section}{Введение и постановка задачи}
\par Быстрая сортировка, сортировка Хоара (англ. quicksort), часто называемая qsort это алгоритм сортировки, который был разработан английским информатиком Тони Хоаром. В ходе данной работы мы напишем последовательную реализацию этого алгоритма, а также распараллелим алгоритм с помощью Openmp и tbb на языке С++, а также сравним выйгрыш в скорости и сделаем соответствующие выводы.



\newpage
\section*{Описание алгоритма}
\addcontentsline{toc}{section}{Описание алгоритма}

\par Сортировка Хоара:
\par Выбираем опорный элемент в массиве, далее мы распрределяем элементы массива так, чтобы элементы больше опорного или равные ему становиилсь после него, а те, что меньше до него. Дальше мы применяем рекурсию, опять выбираем опорный элемент, но уже в подмассивах: слева и справа от прошлого опорного элемента и т.д. Условие выхода из рекурсии - в массиве остается только 1 элемент или их не остается вовсе.


\newpage
\section*{Распараллеливание}
\addcontentsline{toc}{section}{Распараллеливание}
\par Распараллеливание сортировки заключается в разделении массива на несколько частей. На каждую часть будет приходится по одному потоку. Чтобы завершить сортировку, нужно обратно соединить части массива, которые сортировались по потокам. Через четно-нечетное слияние Батчера мы разделяем элементы на четные и нечетные для пары массивов. Это тоже выполняется через потоки: один для нечетных элементов, другой для четных. 


\newpage
\section*{Программная реализация}
\addcontentsline{toc}{section}{Программная реализация}
\par Openmp:
\par Для реализации распараллеливания с помощью Openmp библиотеки мы использовали различные дериктивы. 
\par \verb|#pragma omp parallel for| - через нее мы разделяем массив на потоки и проводим сортировку поотдельности. Через \verb|omp_set_num_threads(num)| задаем кол-во потоков для данной операции. 
Далее мы аналогично проводим слияение через потоки путем дележки на четные и нечетные элементы. Каждый раз мы берем пару подмассивов, поэтому с каждым разом кол-во массивов и потоков уменьшается в 2 раза.
\par Tbb:
\par Для реализации распараллеливания с помощью tbb библиотеки мы  использовали tbb задачи или task. (tbb::task и task::execute)
 Будем делить массив рекурсивно по кол-ву потоков. 
\par Мы установим одну задачу для слияния массивов, будем лишь менять параметры, которые влияют на четность или нечетность(splitter).Далее мы выполняем параллельное слияние массивов через \verb|tbb::parallel_for| и опять рекурсивно повторяем алгоритм.


\newpage
\section*{Результаты и выводы}
\addcontentsline{toc}{section}{Результаты и выводы}
\par Число ядер: 8 ядер
\par Тесты проводились на массивах размерности 100 000 000

\begin{table}[!h]
\caption{Замеры времен}
\centering
\begin{tabular}{|c|c|c|c|c|c|c|c|}
\hline
\multirow{2}{*}
	{\begin{tabular}[c]{@{}c@{}}Кол-во\\ потоков\end{tabular}} & 
\multirow{2}{*}
	{\begin{tabular}[c]{@{}c@{}}Последовательный\\ алгоритм\end{tabular}} & 
\multirow{2}{*}
	{\begin{tabular}[c]{@{}c@{}}OpenMP\end{tabular}} & 
\multirow{2}{*}
	{\begin{tabular}[c]{@{}c@{}}TBB\end{tabular}} & 	

	
	\\ \hline
2   & 5.24    & 3,012        	& 3,134         	
\\ \hline
4   & 5.24    & 2.103        	& 2.096          	
\\ \hline
8   & 5.24    & 2.208           & 2.254         
\\ \hline
\end{tabular}
\end{table}
\par В результате множества тестов можно сформировать вывод о том, что параллельная реализация с помощью tbb и openmp дает большое ускорение во времени. Особенно это показательно на больших размерностях массива. Видно, что при увеличении кол-ва потоков ускорение растет медленнее и медленнее. Это связано прежде всего с созданием большого кол-во потоков для слияния.



\newpage
\section*{Заключение}
\addcontentsline{toc}{section}{Заключение}

\par В ходе выполнения лабораторной работы попараллельного программирования была запрограммирована последовательная версия сортировки Хоара, с четно-нечетным слияниям Батчера. Мы написали последовательную реализацию и распараллелили задачу с помощью tbb и openmp. Кроме этого мы произвели оценку производительности параллельных алгоритмов и выяснили. что параллельная реализация в обоих случаях дает существенных выйгрыш во времени.

\newpage
\section*{Список литературы}
\addcontentsline{toc}{section}{Список литературы}
\begin{enumerate}
\item Guide into OpenMP: Easy multithreading programming for C++. URL: 
\end{enumerate}


\end{document}