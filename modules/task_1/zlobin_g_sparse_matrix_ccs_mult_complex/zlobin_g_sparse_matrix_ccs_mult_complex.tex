\documentclass{report}

\usepackage[utf8]{inputenc}
\usepackage[english, russian]{babel}
\usepackage{amssymb, amsfonts, amsmath, color, enumerate, indentfirst, geometry, listings}
\usepackage[pdftex]{graphicx}

\geometry{a4paper, left=2cm, right=2cm, top=2cm, bottom=2cm}
\definecolor{darkgreen}{rgb}{0.0, 0.5, 0.0}
\lstset{
	language=C++,
	basicstyle=\footnotesize,
	numbers=left,
	numberstyle=\tiny,
	numberfirstline=true,
	numbersep=5pt,
	keywordstyle=\color{blue}\bfseries,
	commentstyle=\color{darkgreen},
	stringstyle=\color{red},
	showspaces=false,
	showstringspaces=false,
	captionpos=t,
	breaklines=true,
	breakatwhitespace=false,
	extendedchars=true,
	frame=tb,
	title=\lstname,
}

\begin{document}

\begin{titlepage}
	\begin{center}
		\textbf{Министерство науки и высшего образования Российской Федерации}
	\end{center}

	\begin{center}
		\textbf{Федеральное государственное автономное образовательное учреждение высшего образования \\
		Национальный исследовательский Нижегородский государственный университет им. Н.И. Лобачевского}
	\end{center}

	\begin{center}
	\LargeИнститут информационных технологий, математики и механики
	\end{center}

	\vspace{4em}

	\begin{center}
		\textbf{\LARGEОтчет по лабораторной работе} \\
	\end{center}

	\begin{center}
		\textbf{\LARGE«Умножение разреженных матриц. Элементы комплексного типа. Формат хранения матрицы - столбцовый (CCS).»} \\
	\end{center}

	\vspace{10em}

	\hfill\parbox{6cm}{
		\hspace*{5cm}\hspace*{-5cm}\textbf{Выполнил:} \\ студент группы 381808-1 \\ Злобин Г.М. \\
		\\
		\hspace*{5cm}\hspace*{-5cm}\textbf{Проверил:}\\ доцент кафедры МОСТ, к.т.н., \\ Сысоев А. В.
	}

	\vspace{\fill}

	\begin{center}
		Нижний Новгород \\ 2021 г.
	\end{center}

\end{titlepage}

\setcounter{page}{2}

\tableofcontents
\newpage

\section*{Введение}
\addcontentsline{toc}{section}{Введение}
\par Разреженные матрицы возникают естественным образом при постановке и решении оптимизационных задач, при численном решении дифференциальных уравнений, в теории графов и в других случаях. Разреженной называют матрицу, имеющую малый процент ненулевых элементов. Что позволяет использовать другие, зачастую более сложные, структуры хранения и методы обработки. Важно определить какая структура хранения будет более эффективна в рамках используемых алгоритмов и конкретной архитектуры. Эффективные методы хранения и обработки разреженных матриц вызывают интерес у широкого круга исследователей.
\newpage

\section*{Постановка задачи}
\addcontentsline{toc}{section}{Постановка задачи}
В ходе данной лабораторной работы требуется:
\begin{enumerate}
\item Реализовать последовательное умножение разреженных матриц. Элементы комплексного типа. Формат хранения матрицы – столбцовый (CCS).
\item Реализовать параллельное умножение разреженных матриц с испоьзованием технологии OpenMP. Элементы комплексного типа. Формат хранения матрицы – столбцовый (CCS).
\item Реализовать параллельное умножение разреженных матриц с испоьзованием технологии TBB. Элементы комплексного типа. Формат хранения матрицы – столбцовый (CCS).
\item Провести сравнение времени работы алгоритмов.
\item Проанализировать полученные результаты и сделать вывод.
\end{enumerate}
\newpage

\section*{Описание алгоритма}
\addcontentsline{toc}{section}{Описание алгоритма}
\par В лабораторной работе будет использоваться формат хранения - столбцовый (CCS). общем случае матрица называется разреженной, если  процент ее ненулевых элементов мал. Введём обозначения: m - число строк, n - число столбцов матрицы, nz - количество ненулевых элементов.
\par Краткое описание формата хранения:
\begin{itemize}
	\item В хранится в виде трех массивов:
	\begin{itemize}
		\item массив values хранит ненулевые значения элементов по столбцам, с первого по последний;
		\item массив rows номеров строк для каждого элемента массива values;
		\item массив collumnsIndexes индексов начала каждого столбца из массива rows;
	\end{itemize}
	\item Количество элементов массива collumnsIndexes = n + 1.
	\item i-ый элемент массива collumnsIndexes указывает на начало i-го столбца.
	\item Элементы столбца i в массиве values находятся по индексам от collumnsIndexes[i] до collumnsIndexes[i + 1] – 1 включительно. Т.о. обрабатывается случай пустых столбцов, а также добавляется «лишний» элемент в массив collumnsIndexes – устраняется особенность при доступе к элементам последнего столбца. collumnsIndexes[n] = nz.
\end{itemize}
\par Для выполнения умножени матриц требуется вычислять скалярные произведения строки одной матрицы и столбца другой матрицы. При вычислении скалярных произведений в столбцовом формате неудобно выделять строку, поэтому удобно будет провести транспонирование матрицы. Тогда для матриц в столбцовом формате A и B их умножение выполняется следующим образом:
\begin{itemize}
	\item Применить транспонирование к матрице A и получить $A^{T}$;
	\item Последовательное перемножить столбцы матриц $A^{T}$ и B;
\end{itemize}
\par Алгоритм транспонирования:
\begin{itemize}
	\item Сформировать N одномерных векторов для хранения целых чисел, а также N векторов для хранения комплексных чисел;
	\item В цикле просмотреть все столбцы исходной матрицы, для каждого столбца – все элементы. Пусть текущий элемент находится в строке i, столбце j, его значение равно v. Тогда добавим числа j и v в i-ые вектора для хранения целых и комплексных чисел (соответственно). Тем самым в векторах мы сформируем строки транспонированной матрицы;
	\item Последовательно скопируем данные из векторов в CCS-структуру транспонированной матрицы (rows и values), попутно формируя массив collumnsIndexes;
\end{itemize}
\newpage

\section*{Описание схемы распараллеливания}
\addcontentsline{toc}{section}{Описание схемы распараллеливания}
\par Идея распараллеливания алгоритма умножения разреженных матриц в столбцовом формате достаточно очевидна. В алгоритме умножения во внешнем цикле мы перебираем столбцы матрицы $A^{T}$ и далее работаем с ними независимо. Поэтому столбцы из внешнего цикла можно распределить между потоками.
\par Но необходимо записовать результаты с каждого потока в итоговую матрицу последовательно. Чтобы решить эту проблему нужно продублировать для каждого потока rows и values, а в конце объединить, массив collumnsIndexes обрабатывается бесконфликтно.
\newpage

\section*{Описание программной реализации}
\addcontentsline{toc}{section}{Описание программной реализации}
\par Матрица хранимая в формате CCS представлена с помощью класса MatrixCCS.
\par Поля класса MatrixCCS:
\begin{itemize}
	\item int \_nCollumns — число столбцов в матрице;
	\item int \_nRows — число строк в матрице;
	\item int \_nNotZero — число ненулевых элементов в матрице;
	\item std::vector<int> \_collumnsIndexes — массив индексов начала столбцов в массиве rows;
	\item std::vector<int> \_rows — массив индексов строк для элементов из values;
	\item std::vector<std::complex<int>> \_values — массив значений ненулевых элементов в матрице.
\end{itemize}
\par Методы класса MatrixCCS в каждой реализации:
\begin{itemize}
	\item GetNumCollumns — без аргументов, возвращает значение \_nCollumns;
	\item GetNumRows — без аргументов, возвращает значение \_nRows;
	\item GetNumNotZero — без аргументов, возвращает значение \_nNotZero;
	\item FillRandom — принимает аргументы: seed - параметер для генератора случайных чисел, min - минимальное генерируемое значение, max - максимальное генерируемое значение, не возвращает значение, заполняет матрицу рандомными значениями с помощью генератора случайных чисел;
	\item Transpose — без аргументов, возвращает транспонированную матрицу MatrixCCS;
\end{itemize}

\subsection*{Последовательная реализация}
\addcontentsline{toc}{subsection}{Последовательная реализация}
\par В последовательной реализации добавлена функция SeqMultiplication. Аргументы: A - матрица MatrixCCS, B - матрица MatrixCCS. Возвращает матрицу MatrixCCS полученную в результате умножения матрицы A на B. В данной версии параллелизм отсутствует.

\subsection*{OpenMP реализация}
\addcontentsline{toc}{subsection}{OpenMP реализация}
\par В OpenMP реализации добавлена функция OMPMultiplication. Аргументы: A - матрица MatrixCCS, B - матрица MatrixCCS. Возвращает матрицу MatrixCCS полученную в результате умножения матрицы A на B. В данной версии распараллеливание происходит с помощью директивы \#pragma omp parallel. До параллельной секции для каждого столбца создаётся вектор rows и values для записи результатов работы потока работающего над данным столбцом. В начале параллельной секции создаётся локальная переменная tempRows для алгоритма умножения. С помощью директивы \#pragma omp for распараллеливается внешний цикл в алгоритме умножения по столбцам матрицы $A^{T}$. После параллельной секции основной поток формирует из всех векторов rows и values итоговую матрицу и заполняет collunmsIndexes этой матрицы.

\subsection*{TBB реализация}
\addcontentsline{toc}{subsection}{TBB реализация}
\par В TBB реализации добавлена функция TBBMultiplication. Аргументы: A - матрица MatrixCCS, B - матрица MatrixCCS. Возвращает матрицу MatrixCCS полученную в результате умножения матрицы A на B. В данной версии распараллеливание происходит с помощью функции tbb\:\:parallel\_for и функтора который выполняет вычисления для соответсвующего blocked\_range. До функции tbb\:\:parallel\_for для каждого столбца создаётся вектор rows и values для записи результатов работы над соответсвующим столбцом и передаються в конструктор функтора. В начале параллельной секции создаётся локальная переменная tempRows для алгоритма умножения. После функции tbb\:\:parallel\_for основной поток формирует из всех векторов rows и values итоговую матрицу и заполняет collunmsIndexes этой матрицы.
\newpage

\section*{Результаты экспериментов}
\addcontentsline{toc}{section}{Результаты экспериментов}
\par Эксперименты состоят в замере времени выполнения умножения двух разреженных матриц. Замер производиться для матриц размером 4000x4000, которые случайно заполняются ненулевыми элементами. Результаты экспериментов:
\begin{table}[htbp]
	\centering
	\begin{tabular}{|c|c|c|c|} \hline
		Реализация & Последовательная & OpenMP & TBB\\ \hline
		Число потоков & \multicolumn{3}{|c|}{время [мс]} \\ \hline
		1 & 7786 & 7777 & 7766 \\ \hline
		2 & - & 3149 & 3279 \\ \hline
		4 & - & 1769 & 1736 \\ \hline
		6 & - & 1389 & 1343 \\ \hline
		8 & - & 1362 & 1316 \\ \hline
		12 & - & 1109 & 1109 \\ \hline
	\end{tabular}
	\caption{Время выполнения от числа потоков (nz = 1000000)}
\end{table}

\begin{table}[htbp]
	\centering
	\begin{tabular}{|c|c|c|c|} \hline
		Реализация & Последовательная & OpenMP & TBB\\ \hline
		Число ненулевых элементов & время [мс] & \multicolumn{2}{|c|}{ускорение} \\ \hline
		1000 & 34 & 3,778 & 3,778 \\ \hline
		10000 & 145 & 2,843 & 6,042 \\ \hline
		100000 & 483 & 3,412 & 3,828 \\ \hline
		400000 & 2171 & 2,901 & 3,31 \\ \hline
		800000 & 5052 & 4,395 & 4,517 \\ \hline
		1000000 & 6858 & 4,897 & 4,946 \\ \hline
	\end{tabular}
	\caption{Коэффициент ускорения относительно последовательной версии}
\end{table}
\newpage

\section*{Описание подтверждения корректности}
\addcontentsline{toc}{section}{Описание подтверждения корректности}
\par Для подтверждения корректности используются unit тесты написанные на основе GoogleTest. Тестируемые ситуации:
\begin{itemize}
	\item создание матрицы;
	\item транспанирование матрицы;
	\item транспанирование вектора;
	\item умножение элемента на элемент;
	\item умножение матрицы на вектор;
	\item умножение вектор на матрицы;
	\item умножение матрицы на нулевую матрицу;
	\item умножение матрицы на единичную матрицу;
	\item умножение не соответсвующих матриц;
	\item умножение матриц;
\end{itemize}
\par Тесты успешно прошли.
\newpage

\section*{Выводы из результатов}
\addcontentsline{toc}{section}{Выводы из результатов}
\par Из результатов экспериментов сделаны следующие выводы:
\begin{itemize}
	\item Реализованные алгоритмы умножения матриц в столбцовом формате корректны. Т.к. тесты успешно пройдены;
	\item TBB реализация эффективнее OMP реализации. В OMP реализации использовалось распределение работы между потоками по умолчанию т.е. статическое, а в TBB используется динамическое. Т.к. матрица разреженная, то работа не равномерно распределена между потоками, поэтому статическое расписание менее эффективно по сравнению с динамическим;
	\item При увеличении плотности матриц ускорение, по сравнению с последовательной версией, увеличивается. Т.к. увеличивается доля вычислений на один поток и влияние накладных расходов на итоговое ускорение уменьшается.
	\item Даже при малой плотности больших матриц TBB и OMP реализации эффективней последовательной. Т.к. алгоритм распаралелен относительно числа столбцов в матрице, поэтому даже при малой плотности матрицы алгоритм проходит по всем столбцам.
\end{itemize}
\newpage

\section*{Заключение}
\addcontentsline{toc}{section}{Заключение}
\par В рамках данной работы были реализованы последовательная, OMP и TBB версии алгоритма умножения разреженных матриц в столбцовом формате. Были проведеные и проанализированые эксперименты для оценки эффективности реализаций. Экспериметы показали что TBB реализация наиболее эффективная среди рассмотренных.
\newpage

\section*{Литература}
\addcontentsline{toc}{section}{Литература}
\begin{enumerate}
	\item Корняков К.В., Мееров И.Б., Сиднев А.А., Сысоев А.В., Шишков А.В.
	Инструменты параллельного программирования в системах с общей
	памятью. – Учебное пособие / Под ред. проф. В.П. Гергеля. – Н. Новгород: Изд-во Нижегородского госуниверситета, 2010. – 201 с.
	\item Джордж А., Лю Дж. Численное решение больших разреженных систем уравнений. – М.: Мир, 1984.
\end{enumerate}
\newpage

\section*{Приложение}
\addcontentsline{toc}{section}{Приложение}
В этом разделе приведён код, используемый при проведение экспериментов. \\
Лисининг OMP реализация matrix\_ccs\_mult.h
\begin{lstlisting}
// Copyright 2021 Zlobin George
#ifndef MODULES_TASK_2_ZLOBIN_G_SPARSE_MATRIX_CCS_MULT_COMPLEX_OMP_MATRIX_CCS_MULT_H_
#define MODULES_TASK_2_ZLOBIN_G_SPARSE_MATRIX_CCS_MULT_COMPLEX_OMP_MATRIX_CCS_MULT_H_

#include <vector>
#include <complex>

class MatrixCCS {
 public:
    MatrixCCS(int nCollumns, int nRows, int nNotZero);
    MatrixCCS(int nCollumns, int nRows, std::vector<std::complex<int>> matrix);
    MatrixCCS(int nCollumns, int nRows,
              std::vector<int> collumnsIndexes,
              std::vector<int> rows,
              std::vector<std::complex<int>> values);

    void FillRandom(unsigned seed = 132, int min = -1000, int max = 1000);
    MatrixCCS Transpose() const;

    int GetNumCollumns() { return _nCollumns; }
    int GetNumRows() { return _nRows; }
    int GetNumNotZero() { return _nNotZero; }

    //  void Print();
 private:
    int _nCollumns;
    int _nRows;
    int _nNotZero;

    std::vector<int> _collumnsIndexes;
    std::vector<int> _rows;
    std::vector<std::complex<int>> _values;

    friend MatrixCCS SeqMultiplication(const MatrixCCS& A, const MatrixCCS& B);
    friend MatrixCCS OMPMultiplication(const MatrixCCS& A, const MatrixCCS& B);
    friend bool operator==(const MatrixCCS& A, const MatrixCCS& B);
};

MatrixCCS SeqMultiplication(const MatrixCCS& A, const MatrixCCS& B);
MatrixCCS OMPMultiplication(const MatrixCCS& A, const MatrixCCS& B);
bool operator==(const MatrixCCS& A, const MatrixCCS& B);

#endif  // MODULES_TASK_2_ZLOBIN_G_SPARSE_MATRIX_CCS_MULT_COMPLEX_OMP_MATRIX_CCS_MULT_H_

\end{lstlisting}

Лисининг OMP реализация matrix\_ccs\_mult.cpp
\begin{lstlisting}
// Copyright 2021 Zlobin George
#include "../../../modules/task_2/zlobin_g_sparse_matrix_ccs_mult_complex_omp/matrix_ccs_mult.h"
#include <omp.h>
#include <random>
#include <algorithm>

#include <iostream>

MatrixCCS::MatrixCCS(int nCollumns, int nRows, int nNotZero) {
    if (nCollumns <= 0 || nRows <= 0) {
        throw "Incorrect matrix size";
    } else if (nNotZero < 0 || nNotZero > nCollumns * nRows) {
        throw "Incorrect number of not zero elements";
    }

    _nCollumns = nCollumns;
    _nRows = nRows;
    _nNotZero = nNotZero;

    _collumnsIndexes.reserve(_nCollumns + 1);
    _rows.reserve(_nRows);
    _values.reserve(_nNotZero);
}

MatrixCCS::MatrixCCS(int nCollumns, int nRows,
                     std::vector<std::complex<int>> matrix) {
    if (nCollumns <= 0 || nRows <= 0) {
        throw "Incorrect matrix size";
    } else if (static_cast<int>(matrix.size()) != nCollumns * nRows) {
        throw "Size of matrix doesn't match given parameters";
    }

    _nCollumns = nCollumns;
    _nRows = nRows;

    _collumnsIndexes.reserve(_nCollumns + 1);
    _rows.reserve(_nRows);

    _collumnsIndexes.push_back(0);
    for (int j = 0; j < _nCollumns; j++) {
        for (int i = 0; i < _nRows; i++) {
            if (matrix[i * _nCollumns + j] != std::complex<int>(0, 0)) {
                _values.push_back(matrix[i * _nCollumns + j]);
                _rows.push_back(i);
            }
        }
        _collumnsIndexes.push_back(static_cast<int>(_values.size()));
    }

    _nNotZero = _collumnsIndexes[_nCollumns];
}

MatrixCCS::MatrixCCS(int nCollumns, int nRows,
                     std::vector<int> collumnsIndexes,
                     std::vector<int> rows,
                     std::vector<std::complex<int>> values) {
    if (nCollumns <= 0 || nRows <= 0) {
        throw "Incorrect matrix size";
    } else if (static_cast<int>(values.size()) > nCollumns * nRows) {
        throw "More elements when possible";
    } else if (static_cast<int>(collumnsIndexes.size()) != nCollumns + 1) {
        throw "Incorrect collumns indexes";
    } else if (std::find_if(rows.begin(), rows.end(),
               [&nRows](int i) { return i >= nRows; }) != rows.end()) {
        throw "Incorrect rows";
    }

    _nCollumns = nCollumns;
    _nRows = nRows;
    _nNotZero = static_cast<int>(values.size());
    _collumnsIndexes = collumnsIndexes;
    _rows = rows;
    _values = values;
}

MatrixCCS MatrixCCS::Transpose() const {
    MatrixCCS transposed{_nRows, _nCollumns, _nNotZero};

    transposed._collumnsIndexes.resize(_nRows + 1);
    for (int i = 0; i < _nNotZero; i++) {
        transposed._collumnsIndexes[_rows[i] + 1]++;
    }

    int tmp, sum = 0;
    for (int i = 1; i <= _nRows; i++) {
        tmp = transposed._collumnsIndexes[i];
        transposed._collumnsIndexes[i] = sum;
        sum += tmp;
    }

    transposed._rows.resize(_nNotZero);
    transposed._values.resize(_nNotZero);

    int j1, j2;
    int row, column, place;
    std::complex<int> value;
    for (int i = 0; i < _nCollumns; i++) {
        j1 = _collumnsIndexes[i];
        j2 = _collumnsIndexes[i + 1];
        row = i;
        for (int j = j1; j < j2; j++) {
            value = _values[j];
            column = _rows[j];
            place = transposed._collumnsIndexes[column + 1];
            transposed._rows[place] = row;
            transposed._values[place] = value;
            transposed._collumnsIndexes[column + 1]++;
        }
    }

    return transposed;
}

MatrixCCS SeqMultiplication(const MatrixCCS& A, const MatrixCCS& B) {
    if (A._nCollumns != B._nRows) {
        throw "Can't multiply matrix which not corresponding";
    }

    MatrixCCS AT = A.Transpose();

    std::vector<int>* rows = new std::vector<int>[B._nCollumns];
    std::vector<std::complex<int>>* values = new std::vector<std::complex<int>>[B._nCollumns];

    std::vector<int> collumnsIndexes(B._nCollumns + 1, 0);

    std::vector<int> tempRows(B._nRows);
    for (int i = 0; i < B._nCollumns; i++) {
        // Clear tempRows
        tempRows = std::vector<int>(B._nRows, -1);

        // Fill tempRows with rows indexes
        int ks = B._collumnsIndexes[i];
        int kf = B._collumnsIndexes[i + 1];
        for (int k = ks; k < kf; k++) {
            int row = B._rows[k];
            tempRows[row] = k;
        }

        // Multiply on every collumn of AT
        for (int j = 0; j < AT._nCollumns; j++) {
            std::complex<int> value{0, 0};

            int ls = AT._collumnsIndexes[j];
            int lf = AT._collumnsIndexes[j + 1];
            for (int l = ls; l < lf; l++) {
                int rowAT = AT._rows[l];
                int rowB = tempRows[rowAT];
                if (rowB != -1) {
                    value += B._values[rowB] * AT._values[l];
                }
            }

            if (value != std::complex<int>(0, 0)) {
                values[i].push_back(value);
                rows[i].push_back(j);
                collumnsIndexes[i]++;
            }
        }
    }

    int nNotZero = 0;
    for (int i = 0; i < B._nCollumns; i++) {
        int tmp = collumnsIndexes[i];
        collumnsIndexes[i] = nNotZero;
        nNotZero += tmp;
    }
    collumnsIndexes[B._nCollumns] = nNotZero;

    MatrixCCS product{B._nCollumns, A._nRows, nNotZero};
    product._collumnsIndexes = collumnsIndexes;
    for (int i = 0; i < B._nCollumns; i++) {
        product._rows.insert(product._rows.end(), rows[i].begin(), rows[i].end());
        product._values.insert(product._values.end(), values[i].begin(), values[i].end());
    }

    delete[] rows;
    delete[] values;

    return product;
}

MatrixCCS OMPMultiplication(const MatrixCCS& A, const MatrixCCS& B) {
    if (A._nCollumns != B._nRows) {
        throw "Can't multiply matrix which not corresponding";
    }

    MatrixCCS AT = A.Transpose();

    std::vector<int>* rows = new std::vector<int>[B._nCollumns];
    std::vector<std::complex<int>>* values = new std::vector<std::complex<int>>[B._nCollumns];

    std::vector<int> collumnsIndexes(B._nCollumns + 1, 0);

#pragma omp parallel
    {
        std::vector<int> tempRows(B._nRows);
    #pragma omp for
        for (int i = 0; i < B._nCollumns; i++) {
            // Clear tempRows
            tempRows = std::vector<int>(B._nRows, -1);

            // Fill tempRows with rows indexes
            int ks = B._collumnsIndexes[i];
            int kf = B._collumnsIndexes[i + 1];
            for (int k = ks; k < kf; k++) {
                int row = B._rows[k];
                tempRows[row] = k;
            }

            // Multiply on every collumn of AT
            for (int j = 0; j < AT._nCollumns; j++) {
                std::complex<int> value{0, 0};

                int ls = AT._collumnsIndexes[j];
                int lf = AT._collumnsIndexes[j + 1];
                for (int l = ls; l < lf; l++) {
                    int rowAT = AT._rows[l];
                    int rowB = tempRows[rowAT];
                    if (rowB != -1) {
                        value += B._values[rowB] * AT._values[l];
                    }
                }

                if (value != std::complex<int>(0, 0)) {
                    values[i].push_back(value);
                    rows[i].push_back(j);
                    collumnsIndexes[i]++;
                }
            }
        }
    }

    int nNotZero = 0;
    for (int i = 0; i < B._nCollumns; i++) {
        int tmp = collumnsIndexes[i];
        collumnsIndexes[i] = nNotZero;
        nNotZero += tmp;
    }
    collumnsIndexes[B._nCollumns] = nNotZero;

    MatrixCCS product{B._nCollumns, A._nRows, nNotZero};
    product._collumnsIndexes = collumnsIndexes;
    for (int i = 0; i < B._nCollumns; i++) {
        product._rows.insert(product._rows.end(), rows[i].begin(), rows[i].end());
        product._values.insert(product._values.end(), values[i].begin(), values[i].end());
    }

    delete[] rows;
    delete[] values;

    return product;
}

void MatrixCCS::FillRandom(unsigned seed, int min, int max) {
    _collumnsIndexes.clear();
    _rows.clear();
    _values.clear();

    std::mt19937 random;
    random.seed(seed);

    std::complex<int> value;
    const double probBarrier = 1.0 * _nNotZero / (_nCollumns * _nRows);

    std::uniform_real_distribution<> probDistrib(0.0, 1.0);
    std::uniform_int_distribution<int> valueDistrib(min, max);

    _collumnsIndexes.push_back(0);
    for (int collumn = 0; collumn < _nCollumns; collumn++) {
        for (int row = 0; row < _nRows; row++) {
            if (static_cast<int>(_values.size()) == _nNotZero) {
                break;
            } else if (_nNotZero - static_cast<int>(_values.size()) + collumn * _nRows + row ==
                       _nCollumns * _nRows ||
                       (static_cast<int>(_values.size()) < _nNotZero &&
                        probDistrib(random) < probBarrier)) {
                do {
                    value = std::complex<int>(valueDistrib(random), valueDistrib(random));
                } while (value == std::complex<int>(0, 0));
                _values.push_back(value);
                _rows.push_back(row);
            }
        }
        _collumnsIndexes.push_back(static_cast<int>(_values.size()));
        if (static_cast<int>(_values.size()) == _nNotZero) {
            while (static_cast<int>(_collumnsIndexes.size()) < _nCollumns + 1) {
                _collumnsIndexes.push_back(0);
            }
            break;
        }
    }
}

// void MatrixCCS::Print() {
//     for (int i = 0; i < _nCollumns; i++) {
//         for (int k = _collumnsIndexes[i];
//                 k < _collumnsIndexes[i + 1]; k++) {
//             std::cout << "Column " << i << "\tRow " << _rows[k] <<
//                 "\tValueR" << _values[k].real() << "\tValueI" << _values[k].imag() << "\n";
//         }
//     }
// }

bool operator==(const MatrixCCS& A, const MatrixCCS& B) {
    return A._nCollumns == B._nCollumns &&
           A._nRows == B._nRows &&
           A._nNotZero == B._nNotZero &&
           A._collumnsIndexes == B._collumnsIndexes &&
           A._rows == B._rows &&
           A._values == B._values;
}

\end{lstlisting}

Лисининг TBB реализация matrix\_ccs\_mult.h
\begin{lstlisting}
// Copyright 2021 Zlobin George
#ifndef MODULES_TASK_3_ZLOBIN_G_SPARSE_MATRIX_CCS_MULT_COMPLEX_TBB_MATRIX_CCS_MULT_H_
#define MODULES_TASK_3_ZLOBIN_G_SPARSE_MATRIX_CCS_MULT_COMPLEX_TBB_MATRIX_CCS_MULT_H_

#include <tbb/tbb.h>
#include <vector>
#include <complex>

class MatrixCCS {
 public:
    MatrixCCS(int nCollumns, int nRows, int nNotZero);
    MatrixCCS(int nCollumns, int nRows, std::vector<std::complex<int>> matrix);
    MatrixCCS(int nCollumns, int nRows,
              std::vector<int> collumnsIndexes,
              std::vector<int> rows,
              std::vector<std::complex<int>> values);

    void FillRandom(unsigned seed = 132, int min = -1000, int max = 1000);
    MatrixCCS Transpose() const;

    int GetNumCollumns() { return _nCollumns; }
    int GetNumRows() { return _nRows; }
    int GetNumNotZero() { return _nNotZero; }

    //  void Print();
 private:
    int _nCollumns;
    int _nRows;
    int _nNotZero;

    std::vector<int> _collumnsIndexes;
    std::vector<int> _rows;
    std::vector<std::complex<int>> _values;

    friend MatrixCCS SeqMultiplication(const MatrixCCS& A, const MatrixCCS& B);
    friend MatrixCCS TBBMultiplication(const MatrixCCS& A, const MatrixCCS& B);
    friend bool operator==(const MatrixCCS& A, const MatrixCCS& B);

    friend class TBBMatrixMultiplicator;
};

class TBBMatrixMultiplicator {
 public:
    const MatrixCCS& AT;
    const MatrixCCS& B;

    std::vector<int>* const rows;
    std::vector<std::complex<int>>* const values;
    std::vector<int>* const collumnsIndexes;

    void operator()(const tbb::blocked_range<int> &range) const;

    TBBMatrixMultiplicator(const MatrixCCS& _AT,
                       const MatrixCCS& _B,
                       std::vector<int>* _rows,
                       std::vector<std::complex<int>>* _values,
                       std::vector<int>* _collumnsIndexes) :
    AT(_AT),
    B(_B),
    rows(_rows),
    values(_values),
    collumnsIndexes(_collumnsIndexes) {}
};

MatrixCCS SeqMultiplication(const MatrixCCS& A, const MatrixCCS& B);
MatrixCCS TBBMultiplication(const MatrixCCS& A, const MatrixCCS& B);
bool operator==(const MatrixCCS& A, const MatrixCCS& B);

#endif  // MODULES_TASK_3_ZLOBIN_G_SPARSE_MATRIX_CCS_MULT_COMPLEX_TBB_MATRIX_CCS_MULT_H_

\end{lstlisting}

Лисининг TBB реализация matrix\_ccs\_mult.cpp
\begin{lstlisting}
// Copyright 2021 Zlobin George
#include "../../../modules/task_3/zlobin_g_sparse_matrix_ccs_mult_complex_tbb/matrix_ccs_mult.h"
#include <random>
#include <algorithm>

#include <iostream>

MatrixCCS::MatrixCCS(int nCollumns, int nRows, int nNotZero) {
    if (nCollumns <= 0 || nRows <= 0) {
        throw "Incorrect matrix size";
    } else if (nNotZero < 0 || nNotZero > nCollumns * nRows) {
        throw "Incorrect number of not zero elements";
    }

    _nCollumns = nCollumns;
    _nRows = nRows;
    _nNotZero = nNotZero;

    _collumnsIndexes.reserve(_nCollumns + 1);
    _rows.reserve(_nRows);
    _values.reserve(_nNotZero);
}

MatrixCCS::MatrixCCS(int nCollumns, int nRows,
                     std::vector<std::complex<int>> matrix) {
    if (nCollumns <= 0 || nRows <= 0) {
        throw "Incorrect matrix size";
    } else if (static_cast<int>(matrix.size()) != nCollumns * nRows) {
        throw "Size of matrix doesn't match given parameters";
    }

    _nCollumns = nCollumns;
    _nRows = nRows;

    _collumnsIndexes.reserve(_nCollumns + 1);
    _rows.reserve(_nRows);

    _collumnsIndexes.push_back(0);
    for (int j = 0; j < _nCollumns; j++) {
        for (int i = 0; i < _nRows; i++) {
            if (matrix[i * _nCollumns + j] != std::complex<int>(0, 0)) {
                _values.push_back(matrix[i * _nCollumns + j]);
                _rows.push_back(i);
            }
        }
        _collumnsIndexes.push_back(static_cast<int>(_values.size()));
    }

    _nNotZero = _collumnsIndexes[_nCollumns];
}

MatrixCCS::MatrixCCS(int nCollumns, int nRows,
                     std::vector<int> collumnsIndexes,
                     std::vector<int> rows,
                     std::vector<std::complex<int>> values) {
    if (nCollumns <= 0 || nRows <= 0) {
        throw "Incorrect matrix size";
    } else if (static_cast<int>(values.size()) > nCollumns * nRows) {
        throw "More elements when possible";
    } else if (static_cast<int>(collumnsIndexes.size()) != nCollumns + 1) {
        throw "Incorrect collumns indexes";
    } else if (std::find_if(rows.begin(), rows.end(),
               [&nRows](int i) { return i >= nRows; }) != rows.end()) {
        throw "Incorrect rows";
    }

    _nCollumns = nCollumns;
    _nRows = nRows;
    _nNotZero = static_cast<int>(values.size());
    _collumnsIndexes = collumnsIndexes;
    _rows = rows;
    _values = values;
}

MatrixCCS MatrixCCS::Transpose() const {
    MatrixCCS transposed{_nRows, _nCollumns, _nNotZero};

    transposed._collumnsIndexes.resize(_nRows + 1);
    for (int i = 0; i < _nNotZero; i++) {
        transposed._collumnsIndexes[_rows[i] + 1]++;
    }

    int tmp, sum = 0;
    for (int i = 1; i <= _nRows; i++) {
        tmp = transposed._collumnsIndexes[i];
        transposed._collumnsIndexes[i] = sum;
        sum += tmp;
    }

    transposed._rows.resize(_nNotZero);
    transposed._values.resize(_nNotZero);

    int j1, j2;
    int row, column, place;
    std::complex<int> value;
    for (int i = 0; i < _nCollumns; i++) {
        j1 = _collumnsIndexes[i];
        j2 = _collumnsIndexes[i + 1];
        row = i;
        for (int j = j1; j < j2; j++) {
            value = _values[j];
            column = _rows[j];
            place = transposed._collumnsIndexes[column + 1];
            transposed._rows[place] = row;
            transposed._values[place] = value;
            transposed._collumnsIndexes[column + 1]++;
        }
    }

    return transposed;
}

MatrixCCS SeqMultiplication(const MatrixCCS& A, const MatrixCCS& B) {
    if (A._nCollumns != B._nRows) {
        throw "Can't multiply matrix which not corresponding";
    }

    MatrixCCS AT = A.Transpose();

    std::vector<int>* rows = new std::vector<int>[B._nCollumns];
    std::vector<std::complex<int>>* values = new std::vector<std::complex<int>>[B._nCollumns];

    std::vector<int> collumnsIndexes(B._nCollumns + 1, 0);

    std::vector<int> tempRows(B._nRows);
    for (int i = 0; i < B._nCollumns; i++) {
        // Clear tempRows
        tempRows = std::vector<int>(B._nRows, -1);

        // Fill tempRows with rows indexes
        int ks = B._collumnsIndexes[i];
        int kf = B._collumnsIndexes[i + 1];
        for (int k = ks; k < kf; k++) {
            int row = B._rows[k];
            tempRows[row] = k;
        }

        // Multiply on every collumn of AT
        for (int j = 0; j < AT._nCollumns; j++) {
            std::complex<int> value{0, 0};

            int ls = AT._collumnsIndexes[j];
            int lf = AT._collumnsIndexes[j + 1];
            for (int l = ls; l < lf; l++) {
                int rowAT = AT._rows[l];
                int rowB = tempRows[rowAT];
                if (rowB != -1) {
                    value += B._values[rowB] * AT._values[l];
                }
            }

            if (value != std::complex<int>(0, 0)) {
                values[i].push_back(value);
                rows[i].push_back(j);
                collumnsIndexes[i]++;
            }
        }
    }

    int nNotZero = 0;
    for (int i = 0; i < B._nCollumns; i++) {
        int tmp = collumnsIndexes[i];
        collumnsIndexes[i] = nNotZero;
        nNotZero += tmp;
    }
    collumnsIndexes[B._nCollumns] = nNotZero;

    MatrixCCS product{B._nCollumns, A._nRows, nNotZero};
    product._collumnsIndexes = collumnsIndexes;
    for (int i = 0; i < B._nCollumns; i++) {
        product._rows.insert(product._rows.end(), rows[i].begin(), rows[i].end());
        product._values.insert(product._values.end(), values[i].begin(), values[i].end());
    }

    delete[] rows;
    delete[] values;

    return product;
}

MatrixCCS TBBMultiplication(const MatrixCCS& A, const MatrixCCS& B) {
    if (A._nCollumns != B._nRows) {
        throw "Can't multiply matrix which not corresponding";
    }

    MatrixCCS AT = A.Transpose();

    std::vector<int>* rows = new std::vector<int>[B._nCollumns];
    std::vector<std::complex<int>>* values = new std::vector<std::complex<int>>[B._nCollumns];

    std::vector<int> collumnsIndexes(B._nCollumns + 1, 0);

    TBBMatrixMultiplicator functor(AT, B,
                          rows,
                          values,
                          &collumnsIndexes);
    tbb::task_scheduler_init init;
    tbb::parallel_for(tbb::blocked_range<int>(0, B._nCollumns, 100), functor);
    init.terminate();

    int nNotZero = 0;
    for (int i = 0; i < B._nCollumns; i++) {
        int tmp = collumnsIndexes[i];
        collumnsIndexes[i] = nNotZero;
        nNotZero += tmp;
    }
    collumnsIndexes[B._nCollumns] = nNotZero;

    MatrixCCS product{B._nCollumns, A._nRows, nNotZero};
    product._collumnsIndexes = collumnsIndexes;
    for (int i = 0; i < B._nCollumns; i++) {
        product._rows.insert(product._rows.end(), rows[i].begin(), rows[i].end());
        product._values.insert(product._values.end(), values[i].begin(), values[i].end());
    }

    delete[] rows;
    delete[] values;

    return product;
}

void TBBMatrixMultiplicator::operator()(const tbb::blocked_range<int> &range) const {
    for (int i = range.begin(); i < range.end(); i++) {
        // Clear tempRows
        std::vector<int> tempRows = std::vector<int>(B._nRows, -1);

        // Fill tempRows with rows indexes
        int ks = B._collumnsIndexes[i];
        int kf = B._collumnsIndexes[i + 1];
        for (int k = ks; k < kf; k++) {
            int row = B._rows[k];
            tempRows[row] = k;
        }

        // Multiply on every collumn of AT
        for (int j = 0; j < AT._nCollumns; j++) {
            std::complex<int> value{0, 0};

            int ls = AT._collumnsIndexes[j];
            int lf = AT._collumnsIndexes[j + 1];
            for (int l = ls; l < lf; l++) {
                int rowAT = AT._rows[l];
                int rowB = tempRows[rowAT];
                if (rowB != -1) {
                    value += B._values[rowB] * AT._values[l];
                }
            }

            if (value != std::complex<int>(0, 0)) {
                values[i].push_back(value);
                rows[i].push_back(j);
                (*collumnsIndexes)[i]++;
            }
        }
    }
}

void MatrixCCS::FillRandom(unsigned seed, int min, int max) {
    _collumnsIndexes.clear();
    _rows.clear();
    _values.clear();

    std::mt19937 random;
    random.seed(seed);

    std::complex<int> value;
    const double probBarrier = 1.0 * _nNotZero / (_nCollumns * _nRows);

    std::uniform_real_distribution<> probDistrib(0.0, 1.0);
    std::uniform_int_distribution<int> valueDistrib(min, max);

    _collumnsIndexes.push_back(0);
    for (int collumn = 0; collumn < _nCollumns; collumn++) {
        for (int row = 0; row < _nRows; row++) {
            if (static_cast<int>(_values.size()) == _nNotZero) {
                break;
            } else if (_nNotZero - static_cast<int>(_values.size()) + collumn * _nRows + row ==
                       _nCollumns * _nRows ||
                       (static_cast<int>(_values.size()) < _nNotZero &&
                        probDistrib(random) < probBarrier)) {
                do {
                    value = std::complex<int>(valueDistrib(random), valueDistrib(random));
                } while (value == std::complex<int>(0, 0));
                _values.push_back(value);
                _rows.push_back(row);
            }
        }
        _collumnsIndexes.push_back(static_cast<int>(_values.size()));
        if (static_cast<int>(_values.size()) == _nNotZero) {
            while (static_cast<int>(_collumnsIndexes.size()) < _nCollumns + 1) {
                _collumnsIndexes.push_back(0);
            }
            break;
        }
    }
}

// void MatrixCCS::Print() {
//     for (int i = 0; i < _nCollumns; i++) {
//         for (int k = _collumnsIndexes[i];
//                 k < _collumnsIndexes[i + 1]; k++) {
//             std::cout << "Column " << i << "\tRow " << _rows[k] <<
//                 "\tValueR" << _values[k].real() << "\tValueI" << _values[k].imag() << "\n";
//         }
//     }
// }

bool operator==(const MatrixCCS& A, const MatrixCCS& B) {
    return A._nCollumns == B._nCollumns &&
           A._nRows == B._nRows &&
           A._nNotZero == B._nNotZero &&
           A._collumnsIndexes == B._collumnsIndexes &&
           A._rows == B._rows &&
           A._values == B._values;
}

\end{lstlisting}
\end{document}