\documentclass{report}

\usepackage[T2A]{fontenc}
\usepackage[utf8]{luainputenc}
\usepackage[english, russian]{babel}
\usepackage[pdftex]{hyperref}
\usepackage[14pt]{extsizes}
\usepackage{listings}
\usepackage{color}
\usepackage{geometry}
\usepackage{enumitem}
\usepackage{multirow}
\usepackage{graphicx}
\usepackage{indentfirst}
\usepackage{amsmath}

\geometry{a4paper,top=2cm,bottom=3cm,left=2cm,right=1.5cm}
\setlength{\parskip}{0.5cm}
\setlist{nolistsep, itemsep=0.3cm,parsep=0pt}

\lstset{language=C++,
		basicstyle=\footnotesize,
		keywordstyle=\color{blue}\ttfamily,
		stringstyle=\color{red}\ttfamily,
		commentstyle=\color{green}\ttfamily,
		morecomment=[l][\color{magenta}]{\#}, 
		tabsize=4,
		breaklines=true,
  		breakatwhitespace=true,
  		title=\lstname,       
}

\makeatletter
\renewcommand\@biblabel[1]{#1.\hfil}
\makeatother

\begin{document}

\begin{titlepage}

\begin{center}
Министерство науки и высшего образования Российской Федерации
\end{center}

\begin{center}
Федеральное государственное автономное образовательное учреждение высшего образования \\
Национальный исследовательский Нижегородский государственный университет им. Н.И. Лобачевского
\end{center}

\begin{center}
Институт информационных технологий, математики и механики
\end{center}

\vspace{4em}

\begin{center}
\textbf{\LargeОтчет по лабораторной работе} \\
\end{center}
\begin{center}
\textbf{\Large«Повышение контраста полутонового изображения посредством линейной растяжки гистограммы»} \\
\end{center}

\vspace{4em}

\newbox{\lbox}
\savebox{\lbox}{\hbox{text}}
\newlength{\maxl}
\setlength{\maxl}{\wd\lbox}
\hfill\parbox{7cm}{
\hspace*{5cm}\hspace*{-5cm}\textbf{Выполнил:} \\ студент группы 381808-2 \\ Шульман Е. А.\\
\\
\hspace*{5cm}\hspace*{-5cm}\textbf{Проверил:}\\ доцент кафедры МОСТ, \\ кандидат технических наук \\ Сысоев А. В. \\
}
\vspace{\fill}

\begin{center} Нижний Новгород \\ 2021 \end{center}

\end{titlepage}

\setcounter{page}{2}

% Содержание
\tableofcontents
\newpage

% Введение
\section*{Введение}
\addcontentsline{toc}{section}{Введение}
Слабый контраст – наиболее распространенный дефект фотографических, сканерных и телевизионных изображений, обусловленный ограниченностью диапазона воспроизводимых яркостей. Под контрастом обычно понимают разность максимального и минимального значений яркости. Путем цифровой обработки контраст можно повысить, изменяя яркость каждого элемента изображения и увеличивая диапазон яркостей. Существует три основных метода повышения контраста изображения:
\begin{itemize}
\item Линейная растяжка гистограммы (линейное контрастирование);
\item Нормализация гистограммы;
\item Выравнивание (линеаризация или эквализация, equalization) гистограммы.
\end{itemize}
\par Далее будет рассматриваться только линейная растяжка гистограммы.
\newpage

% Постановка задачи
\section*{Постановка задачи}
\addcontentsline{toc}{section}{Постановка задачи}
Целью данной лабораторной работы является разработка последовательной и параллельных реализаций алгоритма повышения контраста полутонового изображения посредством линейной растяжки гистограммы.
\par Линейная растяжка сводится к присваиванию новых значений интенсивности каждому пикселю изображения. Если интенсивности исходного изображения изменялись в диапазоне от $y_{min}$ до $y_{max}$, тогда необходимо линейно "растянуть" указанный диапазон так, чтобы значения изменялись от 0 до 255. Для этого достаточно пересчитать старые значения интенсивности для всех пикселей согласно формуле $f^{-1}(y)$ = (y - $y_{min}$) * ${\frac{255}{y_{max} - y_{min}}}$.
\par Исходные данные: изображение.
\par Для реализации параллельной версии необходимо использовать средства OMP, TBB и std::thread. Для проверки корректности работы алгоритмов требуется использовать Google C++ Testing Framework.
\newpage

% Описание алгоритма
\section*{Описание алгоритма}
\addcontentsline{toc}{section}{Описание алгоритма}
Алгоритм повышения контраста полутонового изображения посредством линейной растяжки гистограммы заключается в следующем:
\begin{itemize}
\item Шаг 1: Найти $y_{min}$ и $y_{max}$.
\item Шаг 2: Пересчитать старые значения интенсивности для всех пикселей согласно формуле $f^{-1}(y)$ = (y - $y_{min}$) * ${\frac{255}{y_{max} - y_{min}}}$.

\end{itemize}
\newpage

% Описание схемы распараллеливания
\section*{Описание схемы распараллеливания}
\addcontentsline{toc}{section}{Описание схемы распараллеливания}
Данный алгоритм является одним из простейших. Распараллеливание происходит на основе декомпозиции по данным. \par На первом шаге изображение разбивается на части, для которых находятся локальные $y_{min}$ и $y_{max}$. После нахождения всех локальных $y_{min}$ и $y_{max}$, находятся глобальные $y_{min}$ и $y_{max}$.
\par Следующим шагом каждый поток для своей части изображения пересчитывает старые значения интенсивности согласно формуле $f^{-1}(y)$.
\newpage

% Описание программной реализации
\section*{Описание программной реализации}
\addcontentsline{toc}{section}{Описание программной реализации}\
Последовательный алгоритм вызывается с помощью функции:
\begin{lstlisting}
std::vector<int> ContrastEnhancementSeq(const std::vector<int>& matrix);
\end{lstlisting}
\par В качестве входного параметра передается изображение (представленное в виде вектора).
\par OMP алгоритм вызывается с помощью функции:
\begin{lstlisting}
std::vector<int> ContrastEnhancementOmp(const std::vector<int>& matrix);
\end{lstlisting}
\par В качестве входного параметра передается изображение (представленное в виде вектора).
\par TBB алгоритм вызывается с помощью функции:
\begin{lstlisting}
std::vector<int> ContrastEnhancementTbb(const std::vector<int>& matrix);
\end{lstlisting}
\par В качестве входного параметра передается изображение (представленное в виде вектора).
\par STD::THREAD алгоритм вызывается с помощью функции:
\begin{lstlisting}
std::vector<int> ContrastEnhancementStd(const std::vector<int>& matrix);
\end{lstlisting}
\par В качестве входного параметра передается изображение (представленное в виде вектора).
\newpage

% Результаты экспериментов
\section*{Результаты экспериментов}
\addcontentsline{toc}{section}{Результаты экспериментов}
Вычислительные эксперименты проводились на оборудовании со следующей аппаратной конфигурацией:

\begin{itemize}
\item Процессор: Intel(R) Core(TM) i7-6700 CPU @ 3.40GHz, 3408 МГц, ядер: 4, логических процессоров: 8
\item Оперативная память: 8 ГБ DDR3;
\item ОС: Microsoft Windows 10 Home. Версия 10.0.19042. Сборка 19042.
\end{itemize}

\par Для проведения экспериментов генерируется изображение размером 10000 x 10000:
\\

\begin{table}[!h]
\begin{center}
\begin{tabular}{lllllll}
Реализация & Время & Ускорение   \\
SEQ        & 1.079 & 1           \\
OMP        & 0.538 & 2,006       \\
TBB        & 0.549 & 1,965       \\
STD        & 0.624 & 1,729        \\

\end{tabular}
\end{center}
\caption{Результаты вычислительных экспериментов}
\centering
\end{table}

\par По данным, полученным в результате экспериментов, можно сделать вывод о том, что параллельные реализации работают действительно быстрее, чем последовательная. Однако нужно заметить, что подобные задачи не поддаются эффективному распараллеливанию, производительность упирается в шину памяти из-за этого проблемы с ускорением.
\newpage

% Заключение
\section*{Заключение}
\addcontentsline{toc}{section}{Заключение}
В результате лабораторной работы были разработаны последовательная и параллельные реализации алгоритма повышения контраста полутонового изображения посредством линейной растяжки гистограммы.
\par Основной задачей данной лабораторной работы была реализация распараллеливания с использованием OpenMP, TBB и std::thread, которые должны были быть эффективнее последовательной. Эта задача была успешно достигнута, о чем говорят результаты экспериментов, проведенных в ходе работы. Они показывают, что параллельные варианты работают действительно быстрее, чем последовательный.
\par Кроме того, были разработаны и доведены до успешного выполнения тесты, созданные для данного программного проекта с использованием Google C++ Testing Framework и необходимые для подтверждения корректности работы программы.
\newpage

% Литература
\begin{thebibliography}{1}
\addcontentsline{toc}{section}{Литература}
\bibitem{Gorodetsky} Турлапов В. Е. «Обработка изображений. Часть 1».
\\URL:\url {http://www.graph.unn.ru/rus/materials/CG/CG03_ImageProcessing.pdf} (дата обращения: 22.05.2021)
\bibitem{} «Параллельное программирование на OpenMP»
\\URL:\url {http://ccfit.nsu.ru/arom/data/openmp.pdf} (дата обращения: 22.05.2021)
\bibitem{Sidnev} Сиднев А.А., Мееров И.Б., Сысоев А.В. «Разработка параллельных программ в системах с общей памятью с использованием библиотеки Intel Threading Building Blocks (TBB)».
\\URL:\url {http://hpc-education.ru/files/lectures/meerov/ppt06.pdf} (дата обращения: 22.05.2021)
\bibitem{} «Параллельное программирование. С++. Thread Support Library. Atomic Operations Library.»
\\URL:\url {https://ppt-online.org/184002} (дата обращения: 22.05.2021)
\end{thebibliography}
\newpage

% Приложение
\section*{Приложение}
\addcontentsline{toc}{section}{Приложение}
В данном разделе находится листинг всего кода, написанного в рамках лабораторной работы.
\begin{lstlisting}
// shulman_e_contrast_enhancement.h

// Copyright 2021 Shulman Egor

std::vector<int> getRandomMatrix(int n, int m);
std::vector<int> ContrastEnhancementSeq(const std::vector<int>& matrix);
std::vector<int> ContrastEnhancementOmp(const std::vector<int>& matrix);
std::vector<int> ContrastEnhancementTbb(const std::vector<int>& matrix);
std::vector<int> ContrastEnhancementStd(const std::vector<int>& matrix);

\end{lstlisting}
\newpage
\begin{lstlisting}
// shulman_e_contrast_enhancement.cpp

// Copyright 2021 Shulman Egor

std::vector<int> getRandomMatrix(int n, int m) {
    if (n < 0 || m < 0)
        throw "Rows or columns less than 0";
    std::mt19937 random;
    random.seed(static_cast<unsigned int>(time(0)));
    std::vector<int> matrix(n * m);
    for (int i = 0; i < n; i++) {
        for (int j = 0; j < m; j++) {
            matrix[i * n + j] = random() % 256;
        }
    }
    return matrix;
}

std::vector<int> ContrastEnhancementSeq(const std::vector<int>& matrix) {
    std::vector<int> result;
    int yMax = 0;
    int yMin = 255;
    for (int i = 0; i < static_cast<int>(matrix.size()); i++) {
        yMax = yMax < matrix[i] ? matrix[i] : yMax;
        yMin = yMin > matrix[i] ? matrix[i] : yMin;
    }
    for (int i = 0; i < static_cast<int>(matrix.size()); i++) {
        result.push_back(((matrix[i] - yMin) * 255) / (yMax - yMin));
    }
    return result;
}

std::vector<int> ContrastEnhancementOmp(const std::vector<int>& matrix) {
    std::vector<int> result(matrix.size());
    int yMax = 0;
    int yMin = 255;

#pragma omp parallel
    {
        int localMax = 0;
        int localMin = 255;
#pragma omp for
        for (int i = 0; i < static_cast<int>(matrix.size()); i++) {
            localMax = localMax < matrix[i] ? matrix[i] : localMax;
            localMin = localMin > matrix[i] ? matrix[i] : localMin;
        }
#pragma omp critical
        {
            yMax = yMax > localMax ? yMax : localMax;
            yMin = yMin < localMin ? yMin : localMin;
        }
#pragma omp barrier
#pragma omp for
        for (int i = 0; i < static_cast<int>(matrix.size()); i++) {
            result[i] = ((matrix[i] - yMin) * 255) / (yMax - yMin);
        }
    }
    return result;
}

std::vector<int> ContrastEnhancementTbb(const std::vector<int>& matrix) {
    std::vector<int> result(matrix.size());
    auto temp = std::minmax_element(matrix.begin(), matrix.end());
    int yMax = *(temp.second);
    int yMin = *(temp.first);
    tbb::parallel_for(0, static_cast<int>(matrix.size()), [&](const int i) {
        result[i] = ((matrix[i] - yMin) * 255) / (yMax - yMin);
    });
    return result;
}

std::vector<int> ContrastEnhancementStd(const std::vector<int>& matrix) {
    std::vector<int> result(matrix);

    auto temp = std::minmax_element(matrix.begin(), matrix.end());
    int yMax = *(temp.second);
    int yMin = *(temp.first);

    auto threadFill = [&](std::vector<int>::iterator first,
                          std::vector<int>::iterator last) {
        for (auto it = first; it != last; ++it) {
            *it = ((*it - yMin) * 255) / (yMax - yMin);
        }
    };

    const size_t size = std::thread::hardware_concurrency();
    std::vector<std::thread> threads(size);
    auto block = result.size() / size;
    auto work_iter = std::begin(result);

    for (auto it = std::begin(threads); it != std::end(threads) - 1; ++it) {
        *it = std::thread(threadFill, work_iter, work_iter + block);
        work_iter += block;
    }
    threads.back() = std::thread(threadFill, work_iter, std::end(result));

    for (auto&& i : threads) {
        i.join();
    }
    return result;
}

\end{lstlisting}
\newpage
\begin{lstlisting}
// main.cpp

// Copyright 2021 Shulman Egor

TEST(STD, testCreateMatrix) {
    int n = 100, m = 100;
    ASSERT_NO_THROW(getRandomMatrix(n, m));
}

TEST(SEQ, testRowsNegative) {
    int n = -100, m = 100;
    ASSERT_ANY_THROW(getRandomMatrix(n, m));
}

TEST(SEQ, testColumnsNegative) {
    int n = 100, m = -100;
    ASSERT_ANY_THROW(getRandomMatrix(n, m));
}

TEST(SEQ, testRowsAndColumnsNegative) {
    int n = -100, m = -100;
    ASSERT_ANY_THROW(getRandomMatrix(n, m));
}

TEST(SEQ, testContrastEnhancementSeq) {
    int n = 100, m = 100;
    std::vector<int> matrix = getRandomMatrix(n, m);
    ASSERT_NO_THROW(ContrastEnhancementSeq(matrix));
}

TEST(OMP, testContrastEnhancementOmp) {
    int n = 100, m = 100;
    std::vector<int> matrix = getRandomMatrix(n, m);
    ASSERT_NO_THROW(ContrastEnhancementOmp(matrix));
}

TEST(TBB, testContrastEnhancementTbb) {
    int n = 100, m = 100;
    std::vector<int> matrix = getRandomMatrix(n, m);
    ASSERT_NO_THROW(ContrastEnhancementTbb(matrix));
}

TEST(STD, testContrastEnhancementStd) {
    int n = 100, m = 100;
    std::vector<int> matrix = getRandomMatrix(n, m);
    ASSERT_NO_THROW(ContrastEnhancementStd(matrix));
}

TEST(SEQ, testComparisonSeqWithExpected) {
    int n = 5, m = 5;

    std::vector<int> matrix(n * m);
    std::vector<int> result(n * m);
    for (int i = 0; i < n * m; ++i) {
        matrix[i] = i + 10;
        result[i] = i + 10;
    }

    matrix = ContrastEnhancementSeq(matrix);

    int yMax = *std::max_element(result.begin(), result.end());
    int yMin = *std::min_element(result.begin(), result.end());
    for (size_t i = 0; i < result.size(); ++i)
        result[i] = ((result[i] - yMin) * 255) / (yMax - yMin);

    ASSERT_EQ(result, matrix);;
}

TEST(OMP, testComparisonOmpWithExpected) {
    int n = 5, m = 5;

    std::vector<int> matrix(n * m);
    std::vector<int> result(n * m);
    for (int i = 0; i < n * m; ++i) {
        matrix[i] = i + 10;
        result[i] = i + 10;
    }

    matrix = ContrastEnhancementOmp(matrix);

    int yMax = *std::max_element(result.begin(), result.end());
    int yMin = *std::min_element(result.begin(), result.end());
    for (size_t i = 0; i < result.size(); ++i)
        result[i] = ((result[i] - yMin) * 255) / (yMax - yMin);

    ASSERT_EQ(result, matrix);;
}

TEST(TBB, testComparisonTbbWithExpected) {
    int n = 5, m = 5;

    std::vector<int> matrix(n * m);
    std::vector<int> result(n * m);
    for (int i = 0; i < n * m; ++i) {
        matrix[i] = i + 10;
        result[i] = i + 10;
    }

    matrix = ContrastEnhancementTbb(matrix);

    int yMax = *std::max_element(result.begin(), result.end());
    int yMin = *std::min_element(result.begin(), result.end());
    for (size_t i = 0; i < result.size(); ++i)
        result[i] = ((result[i] - yMin) * 255) / (yMax - yMin);

    ASSERT_EQ(result, matrix);;
}

TEST(STD, testComparisonStdWithExpected) {
    int n = 5, m = 5;

    std::vector<int> matrix(n * m);
    std::vector<int> result(n * m);
    for (int i = 0; i < n * m; ++i) {
        matrix[i] = i + 10;
        result[i] = i + 10;
    }

    matrix = ContrastEnhancementStd(matrix);

    int yMax = *std::max_element(result.begin(), result.end());
    int yMin = *std::min_element(result.begin(), result.end());
    for (size_t i = 0; i < result.size(); ++i)
        result[i] = ((result[i] - yMin) * 255) / (yMax - yMin);

    ASSERT_EQ(result, matrix);
}

\end{lstlisting}
\end{document}