\documentclass{report}

\usepackage[T2A]{fontenc}
\usepackage[utf8]{luainputenc}
\usepackage[english, russian]{babel}
\usepackage[pdftex]{hyperref}
\usepackage[14pt]{extsizes}
\usepackage{listings}
\usepackage{color}
\usepackage{geometry}
\usepackage{enumitem}
\usepackage{multirow}
\usepackage{graphicx}
\usepackage{indentfirst}
\usepackage{amsmath}
\usepackage{xcolor}

\renewcommand{\thesection}{\arabic{section}}
\setlength{\parskip}{3mm}
\geometry{a4paper,top=20mm,bottom=20mm,left=20mm,right=15mm}
\setlist{nolistsep, itemsep=0.5cm,parsep=0pt}

\lstset{
  language=c++,
  backgroundcolor=\color{gray!5},
  basicstyle=\fontsize{12}{12}\ttfamily,
  columns=fullflexible,
  breakatwhitespace=false,
  breaklines=true,
  captionpos=b,
  commentstyle=\color{green},
  keepspaces=true,
  keywordstyle=\color{blue},
  numbers=none,
  stringstyle=\color{red},
  title=\lstname
}


\begin{document}

\begin{titlepage}

\begin{center}
Министерство науки и высшего образования Российской Федерации
\end{center}

\begin{center}
Федеральное государственное автономное образовательное учреждение высшего образования \\
Национальный исследовательский Нижегородский государственный университет им. Н.И. Лобачевского
\end{center}

\begin{center}
Институт информационных технологий, математики и механики
\end{center}

\vspace{4em}

\begin{center}
\textbf{\LargeОтчет по лабораторной работе} \\
\end{center}
\begin{center}
\textbf{\Large«Умножение плотных матриц. Алгоритм Штрассена.»} \\
\end{center}

\vspace{4em}

\newbox{\lbox}
\savebox{\lbox}{\hbox{text}}
\newlength{\maxl}
\setlength{\maxl}{\wd\lbox}
\hfill\parbox{7cm}{
\hspace*{5cm}\hspace*{-5cm}\textbf{Выполнил:} \\ студент группы 381806-1 \\ Емельховская Екатерина Евгеньевна\\
\\
\hspace*{5cm}\hspace*{-5cm}\textbf{Проверил:}\\ доцент кафедры МОСТ, \\ кандидат технических наук \\ Сысоев А. В.
}

\vspace{\fill}

\begin{center} Нижний Новгород \\ 2021 \end{center}

\end{titlepage}

\setcounter{page}{2}

\tableofcontents
\newpage


\section{Введение и постановка задачи}
\par Алгоритм Штрассена предназначен для быстрого умножения матриц. В отичие от обычного умножения его сложность равна примерно n в степени 2.81, а кол-во перемножений равно 7, а не 8, что дает существенный выйгрыш на больших матрицах.
\par В рамках данной лабораторной работы мы запрограммируем последовательную реализацию этого алгоритма, а также попробуем распараллелить его через tbb и openmp на языке прораммирования С++.

\newpage
\section{Описание алгоритма}
\par Изначально мы имеем 2 квадратные матрицы: A и B. Результатом их перемножения является также квадратная матрица C. Условно мы делим все эти матрицы на 4 равных блока и условно введем обозначения для каждой: например, для матрицы A: A11, A12, A21, A22 и т.д. для матриц B и C.
\par Как уже говорилось ранее, для обычного перемножения матриц требуется 8 перемножений, в то время как для алгоритма Штрассена требуется всего 7, вот они:
\par 
\verb|M1=(A11 + A22)*(B11 + B22)|

\verb|M2=(A21 + A22)*B11|

\verb|M3=A11*(B12 - B22)|

\verb|M4=A22*(B21 - B11)|

\verb|M5=(A11 + A12)*B22|

\verb|M6=(A21 - A11)*(B11 + B12)|

\verb|M7=(A12 - A22)*(B21 + B22)|

\par Через Mi мы вычисляем результируюзую матрицу C. А далее мы опять делим матрицы и рекурсивно вычисляем подматрицы Mi до тех пор, пока матрицы не станут маленькими.

\verb|C11=M1+M4-M5+M7|

\verb|C12=M3+M5|

\verb|C21=M2+M4|

\verb|C22=M1-M2+M3+M6|


\newpage
\section{Распараллеливание}
\par Алгоритм Штарссена можно распараллелить путем запуска параллельного выполнения операций для Mi. Эта идея будет актуальна и для tbb и для openmp.

\par OpenMp:
\par Для реализации алгоритма через openmp мы используем директиву 
\par
\verb|#pragma omp parallel sections shared()|. Она нужна для того, чтобы запустить подсчеты Mi параллельно, в скобочках через shared указываются части исходных матриц, которые являются общими данными для каждого из потоков. И уже в этой директиве мы для каждого умножения прописываем директиву \verb|#pragma omp section|. А для каждого маленького кусочка матрицы мы уже используем последовательное умножение матриц.

\par Tbb:
\par Подход для tbb аналогичен. Но вместо прошлых директив мы используем
\verb|tbb::task_group tasks| и \verb|tasks.run| для параллельного запуска tbb задач в отдельных потоках.


\newpage
\section{Результаты}
\par Тесты проводились на матрицах размерности 2048 на 2048

\begin{tabular}{|c|c|c|}
OpenMP & TBB &  Последовательная \\
2,34 & 2,85 & 6.54 
\end{tabular}








\newpage
\section{Заключение}
\par В результате выполнения лабораторной работы мы написали последовательную реализацию алгоритма Штрассена,  а также нам удалось распараллелить часть с умножениями через библиотеки tbb и openmp. В ходе замеров времени мы выяснили, что tbb и openmp дают существенный выйгрыш во времни. Стоит учесть что это работает особенно хорошо на матрицах начиная от размерности 64 на 64.



\newpage
\section{Литература}
\begin{enumerate}
\item Антонов А.С. "Параллельное программирование с использованием технологии OpenMP: Учебное пособие".-М.: Изд-во МГУ, 2009. - 77 с. 19.04.2020)
\end{enumerate}


\end{document}