\documentclass{report}

\usepackage[T2A]{fontenc}
\usepackage[utf8]{luainputenc}
\usepackage[english, russian]{babel}
\usepackage[pdftex]{hyperref}
\usepackage[14pt]{extsizes}
\usepackage{listings}
\usepackage{color}
\usepackage{geometry}
\usepackage{enumitem}
\usepackage{multirow}
\usepackage{graphicx}
\usepackage{indentfirst}
\usepackage[ruled,vlined,linesnumbered]{algorithm2e}

\geometry{a4paper,top=2cm,bottom=3cm,left=2cm,right=1.5cm}
\setlength{\parskip}{0.5cm}
\setlist{nolistsep, itemsep=0.3cm,parsep=0pt}

\newenvironment{pseudocode}[1][htb]
  {\renewcommand{\algorithmcfname}{Алгоритм}
   \begin{algorithm}[#1]%
  }{\end{algorithm}}


\lstset{language=C++,
		basicstyle=\footnotesize,
		keywordstyle=\color{blue}\ttfamily,
		stringstyle=\color{red}\ttfamily,
		commentstyle=\color{green}\ttfamily,
		morecomment=[l][\color{magenta}]{\#}, 
		tabsize=4,
		breaklines=true,
  		breakatwhitespace=true,
  		title=\lstname,       
}

\makeatletter

\renewcommand\@biblabel[1]{#1.\hfil}
\makeatother

\begin{document}

\begin{titlepage}

\begin{center}
Министерство науки и высшего образования Российской Федерации
\end{center}

\begin{center}
Федеральное государственное автономное образовательное учреждение высшего образования \\
Национальный исследовательский Нижегородский государственный университет им. Н.И. Лобачевского
\end{center}

\begin{center}
Институт информационных технологий, математики и механики
\end{center}

\vspace{4em}

\begin{center}
\textbf{\LargeОтчет по лабораторной работе} \\
\end{center}
\begin{center}
\textbf{\LargeПоиск кратчайших путей из одной вершины (алгоритм Дейкстры)} \\
\end{center}

\vspace{4em}

\newbox{\lbox}
\savebox{\lbox}{\hbox{text}}
\newlength{\maxl}
\setlength{\maxl}{\wd\lbox}
\hfill\parbox{7cm}{
\hspace*{5cm}\hspace*{-5cm}\textbf{Выполнил:} \\ студент группы 381808-3 \\ Стойчева Д.Д.\\
\\
\hspace*{5cm}\hspace*{-5cm}\textbf{Проверил:}\\ доцент кафедры МОСТ, \\ кандидат технических наук \\ Сысоев А. В.
}

\vspace{\fill}

\begin{center} Нижний Новгород \\ 2021 \end{center}

\end{titlepage}

\setcounter{page}{2}

\tableofcontents
\newpage

\section*{Введение}
\addcontentsline{toc}{section}{Введение}
Алгоритм Дейкстры - это алгоритм нахождения кратчайшего пути от одной заданной точки до всех остальных. Был создан голландским ученым Эдсгером Дейкстрой в 1959 году. Он является одним из наиболее часто используемых методов решения задач подобного рода. Его главным минусом является отсутствие возможности работы с графами, имеющими ребра отрицательного веса.
\par В данной лабораторной работе будет рассмотрен алгоритм в последовательной и параллельных реализациях.

\newpage

\section*{Постановка задачи}
\addcontentsline{toc}{section}{Постановка задачи}
В рамках данной лабораторной работы должны быть представлены версии алгоритма Дейкстры в разных вариациях:
\begin{itemize}
\item Последовательный алгоритм Дейкстры;
\item Параллельные алгоритмы Дейкстры, реализованные при помощи технологий OpenMP, TBB, std::threads;
\begin{sloppypar}
\item Набор автоматических тестов реализованных с использованием Google C++ Testing Framework. Прохождение этих тестов подтверждает корректность результатов.
\end{sloppypar}
\end{itemize}

\newpage

\section*{Метод решения}
\addcontentsline{toc}{section}{Метод решения}
Пусть дан неориентированный граф G, который имеет V - множество его вершин и Е - множество ребер. Необходимо найти кратчайшие пути от начальной вершины до всех остальных. В начале алгоритма все вершины помечаются как непосещенные и имеют расстояние до начальной вершины равное MAX. На первом шаге алгоритма обрабатывается начальная вершина, для всех ее соседей рассчитывается расстояние до начальной вершины как путь до текущей точки + расстояние между текущей точкой и соседом. Если этот путь меньше чем уже сохранненное расстояние у вершины соседа, то  это расстояние записывается для вершины соседа как расстояние до начальной точки. Попутно выбирается сосед с минимальным расстоянием до начальной точки. После посещения всех соседей текущая точка помечается как посещенная и в качестве следующей точки выбирается непосещенный сосед с минимальным расстоянием до начальной точки. Далее повторяются шаги для новой текущей точки. Если непосещенных соседей не осталось, то берется непосещенная точка с минимальным расстоянием до начальной.
\newpage

\section*{Схема распараллеливания}
\addcontentsline{toc}{section}{Схема распараллеливания}
В алгоритме Дейкстры можно расспараллелить два участка кода:
\begin{itemize}
\item нахождение непосещенной вершины с наименьшим расстоянием до начальной вершины;
\item расчет расстояний от текущей вершины до непройденных соседних вершин и обновление минимального расстояния до начальной точки;  
\end{itemize}
\par Соответсвтенно на каждой итерации цикла каждому потоку выделяется диапазон вершин для проверки на то, что они являются соседями и для обновления минимального расстояния у этих вершин. Попутно каждый поток находит непосещенного соседа с минимальным расстоянием. После этого из найденных каждым потоком минимальных соседей выбирается наименьший.  
\newpage

\section*{Описание программной реализации}
\addcontentsline{toc}{section}{Описание программной реализации}
В программе граф представлен в виде вектора. 
\begin{lstlisting}
const std::vector<int> graph;
\end{lstlisting}
В матрице вершин хранится расстояние от одной вершины до другой, если между ними есть связь, то значение положительное, если нет - отрицательное.
\begin{lstlisting}
std::vector<int> matrix(size);
\end{lstlisting}

Последовательный алгоритм реализован в функциях:
\begin{lstlisting}
int process_unprocessed_point(...)
int find_unprocessed_point_with_min_distance(...)
\end{lstlisting}
\begin{sloppypar}
Функции алгоритма Дейкстры, реализованные при помощи технологий OpenMP, TBB, std::threads в соответствующих функциях:
\end{sloppypar}
\begin{lstlisting}
int process_unprocessed_point_{omp|tbb|std}(...)
int find_unprocessed_point_with_min_distance_{omp|tbb|std}(...)
\end{lstlisting}

\subsection*{OpenMP}
\addcontentsline{toc}{subsection}{OpenMP}
Схема распараллеливания алгоритма:
\par Была создана параллельная секция с помощью следующей директивы:
\par \verb|#pragma omp parallel| 
\par В ней используется внешняя пеерменная min\_distance для сохранения конечного результата, а так же внутренняя для распаралеленного процесса переменаая local\_min\_distance, в которой сохраняется результат работы процесса. Далее с помощью директивы
\par \verb|#pragma omp for |
\par обработка всех точек с помощью библиотеки распараллеливается. После обработки в критической секции, созданной с помощью директивы \par \verb| #pragma omp critical| , \par полученный для процесса результат при необходимости сохраняется в качестве конечного. Критическая секция сделана для предотвращения конфликтов между процессами, чтобы доступ в один времени к этому блоку был только у одного процесса. Подобная схема используется для распараллеливания функции поиска необработанной вершины с минимальным расстоянием.


\subsection*{TBB}
\addcontentsline{toc}{subsection}{TBB}
\begin{sloppypar}
Для реализации алгоритма с помощью библиотеки TBB в функции обработки вершины \verb|process_unprocessed_point_tbb| и функции поиска необработанной вершины с минимальным расстоянием \\* \verb|find_unprocessed_point_with_min_distance_tbb| используется функция редукции \verb|tbb::parallel_reduce(...)|, в которую в виде лямбда-функций передаются функция обработки вершины и функция сравнения вершин для выполнения редукции, так же передается полуинтервал \verb|bb::blocked_range<int>| в котором общее количество точек делится между процессами по \verb|pointsPerThread| вершин па процесс.
\end{sloppypar}

\subsection*{std::threads}
\addcontentsline{toc}{subsection}{std::threads}
\begin{sloppypar}
При распараллеливании создаются массивы объектов \verb|std::promise<PointInfo>|, \verb|std::future<PointInfo>| и \verb|std::thread| с числом элементов равным количеству потоков. В цикле для каждого потока вычисляется диапазон точек, которые он должен обработать (start, end), создаются потоки с помощью шаблона библиотеки std::thread, куда передается имя вызываемой потоком функции, ссылки на вектора вершин, дистанций и обработанных точек. Ссылки передаеются с помощью std::cref|std::ref для исключения копирования исходных векторов в каждый поток. Так же с помощью std::move promise соответствующего потока копируется в потокобезопасное хранилище. Далее с помощью метода detach() запускается ассинхронное выполнение функции в созданном потоке. По завершению работы потоков с помощью future.get() результат работы потока используется в главном потоке.
\end{sloppypar}
\newpage


\section*{Подтверждение корректности}
\addcontentsline{toc}{section}{Подтверждение корректности}
Для подтверждения корректной работы программа запускает тесты с использованием Google C++ Testing Framework.
\par Используются следующие тесты:
\begin{itemize}
\item вычесление времени работы и расстояний до начальной вершины сгенерированного графа с большим количеством вершин;
\item запуск алгоритма для пустого графа;
\item запуск алгоритма для графа из одной вершины;
\item запуск алгоритма для графа из двух вершин;
\item запуск алгоритма для графа из шести вершин;
\item запуск алгоритма для графа из шести вершин с петлями;
\end{itemize}
\par Успешное прохождение тестов показывает корректность работы программы.

\newpage

\section*{Результаты экспериментов}
\addcontentsline{toc}{section}{Результаты экспериментов}
Программа разрабатывалась и тестировалась на оборудовании со следующими характеристиками:
\begin{itemize}
\item Процессор: Intel Core i5-3230M, 2600 MHz, ядер: 2, потоков: 4;
\item Оперативная память: 6 Гб, 800 MHz;
\item ОС: Microsoft Windows 10 Home, версия 1709 сборка 16299.431.
\end{itemize}

\par Тесты проводились на графе с \verb|10 000| вершинами.
\begin{table}[!h]
\centering
\begin{tabular}{lllll}
Кол-во п. & Посл.алг,(с) & OpenMP,(с) & TBB,(с) & std::thread,(с) \\
2 & 0.452521 & 0.308805 & 0.388285 &  2.17868\\
4 & 0.452521 & 0.225436 & 0.299492 &  2.17714\\
8 & 0.452521 & 0.273638 & 0.312993 &  2.23663
\end{tabular}
\caption{Результаты тестирования}
\end{table}

\par Из тестирования можно сделать вывод о том, что параллельный алгоритм работает немного быстрее, за исключением реализации std::thread, т.к. затраты на создание потоков по сравнению с временными затратами на выполнение значительно больше. Сам по себе алгорит Дейкстры достаточно плохо распараллеливается.

\newpage


\section*{Заключение}
\addcontentsline{toc}{section}{Заключение}
В данной лабораторной работе были представлены разные версии алгоритма Дейкстры: последовательная и параллельные с использованием технологий OpenMP, TBB и std::threads. Были проведены тесты с использованием Google C++ Testing Framework, которые показали корректность работы программы и сравнение времени работы всех версий, из чего можно сделать вывод, что OpenMP наиболее удачно подходит для данного алгоритма.

\newpage


\begin{thebibliography}{1}
\addcontentsline{toc}{section}{Список литературы}
\bibitem{Sysoev} Сысоев А.В., Мееров И.Б., Свистунов А.Н., Курылев А.Л., Сенин А.В., Шишков А.В., Корняков К.В., Сиднев А.А. «Параллельное программирование в системах с общей памятью. Инструментальная поддержка». Учебно-методические материалы по программе повышения квалификации «Технологии высокопроизводительных вычислений для обеспечения учебного процесса и научных исследований». Нижний Новгород, 2007, 110 с. 
\bibitem{Uilyams} Энтони Уильямс «Параллельное программирование на C++ в действии. Практика разработки многопоточных программ». ДМК Пресс, 2012, 672 с. 
\bibitem{Microsoft} Документации и учебные ресурсы Майкрософт для разработчиков и технических специалистов [Электронный ресурс] // URL: https://docs.microsoft.com
\end{thebibliography}
\newpage

\section*{Приложение}
\addcontentsline{toc}{section}{Приложение}
\par 1. Последовательная версия.
\begin{lstlisting}
int find_unprocessed_point_with_min_distance(const std::vector<int>& graph,
    const std::vector<int>& distances, const std::vector<bool>& processed) {
    int found_point = -1;
    int found_min_distance = std::numeric_limits<int>::max();
    for (size_t point = 0; point < processed.size(); point++) {
        if (!processed[point] && distances[point] < found_min_distance) {
            found_min_distance = distances[point];
            found_point = static_cast<int>(point);
        }
    }
    // std::cout << "FUP: Next  point with min distance " << found_min_distance
    //           << " from " << 0 << " is " << found_point << std::endl;
    return found_point;
}
int process_unprocessed_point(const std::vector<int>& graph,
    std::vector<int>* distances,
    std::vector<bool>* processed, int current_point) {

    int min_distance = std::numeric_limits<int>::max();
    int min_distance_point = -1;
    for (int point = 0; point < static_cast<int>(processed->size()); point++) {
        int start_row_index = current_point * static_cast<int>(processed->size());
        int distance = graph[start_row_index + point];
        if (!(*processed)[point] && distance > 0) {
            int *dp = distances->data() + point;
            int *dcp = distances->data() + current_point;
            *dp = std::min(*dp, *dcp + distance);
            if (min_distance > *dp) {
                min_distance = *dp;
                min_distance_point = point;
            }
        }
    }
    (*processed)[current_point] = true;
    // std::cout << "PUP: Next point with min distance " << min_distance << " is "
    //           << min_distance_point + 1 << std::endl;
    return min_distance_point;
}
std::vector<int> dijkstra(const std::vector<int>& graph, size_t start, size_t end) {
    if (graph.size() == 0) {
        throw "Error: empty graph";
    }

    size_t points_count = static_cast<size_t>(sqrt(graph.size()));
    if (points_count * points_count != graph.size()) {
        std::cout << "Illegal graph size: expected " << graph.size() << " calculated "
                  << points_count * points_count << std::endl;
        throw "Error: incorrect graph";
    }

    if (start < 1 || end < 1 || start > points_count || end > points_count) {
        std::cout << "Error: illegal start or end" << std::endl;
        throw "Error: illegal start or end";
    }
    start--; end--;

    if (points_count == 1) {
        return { static_cast<int>(start) + 1 };
    }

    constexpr int max_int = std::numeric_limits<int>::max();
    std::vector<int> distances = std::vector<int>(points_count, max_int);
    std::vector<bool> processed = std::vector<bool>(points_count, false);

    distances[start] = 0;

    int next_unprocessed_point = start;
    while (next_unprocessed_point >= 0) {
        next_unprocessed_point = process_unprocessed_point(graph,
            &distances, &processed, next_unprocessed_point);
        if (next_unprocessed_point < 0) {
            // std::cout << "- next unprocessed point not given, finding." << std::endl;
            next_unprocessed_point =
                find_unprocessed_point_with_min_distance(graph, distances,
                    processed);
        } else {
            // std::cout << "+ next unprocessed point given, ok." << std::endl;
        }
    }
    print_vector(distances, distances.size(), "distances");

    std::vector<int> path;
    path.push_back(end + 1);
    int weight = distances[end];
    int current = end;

    while (current != static_cast<int>(start)) {
        for (int i = 0; i < static_cast<int>(points_count); i++) {
            if (graph[current * points_count + i] > 0) {
                int tmp = weight - graph[current * points_count + i];
                if (distances[i] == tmp) {
                    weight = tmp;
                    current = i;
                    path.insert(path.begin(), i + 1);
                    break;
                }
            }
        }
    }

    return path;
}
\end{lstlisting}
\par 2. OpenMP версия.
\begin{lstlisting}
int find_unprocessed_point_with_min_distance_omp(const std::vector<int>& graph,
    const std::vector<int>& distances, const std::vector<bool>& processed) {
    int found_point = -1;
    int found_min_distance = std::numeric_limits<int>::max();
    #pragma omp parallel
    {
        int local_found_point = found_point;
        int local_found_min_distance = found_min_distance;
        #pragma omp for
        for (int point = 0; point < static_cast<int>(processed.size()); point++) {
            if (!processed[point] && distances[point] < local_found_min_distance) {
                local_found_min_distance = distances[point];
                local_found_point = point;
            }
        }
        #pragma omp critical
        {
            if (found_min_distance > local_found_min_distance) {
                found_min_distance = local_found_min_distance;
                found_point = local_found_point;
            }
        }
    }
    return found_point;
}
int process_unprocessed_point_omp(const std::vector<int>& graph,
    std::vector<int>* distances,
    std::vector<bool>* processed, int current_point) {

    int min_distance = std::numeric_limits<int>::max();
    int min_distance_point = -1;
    #pragma omp parallel
    {
        int local_min_distance = std::numeric_limits<int>::max();
        int local_min_distance_point = -1;
        #pragma omp for
        for (int point = 0; point < static_cast<int>(processed->size()); point++) {
            int start_row_index = current_point * static_cast<int>(processed->size());
            int distance = graph[start_row_index + point];
            if (!(*processed)[point] && distance > 0) {
                int* dp = distances->data() + point;
                int* dcp = distances->data() + current_point;
                *dp = std::min(*dp, *dcp + distance);
                if (local_min_distance > *dp) {
                    local_min_distance = *dp;
                    local_min_distance_point = point;
                }
            }
        }
        #pragma omp critical
        {
            if (min_distance > local_min_distance) {
                min_distance = local_min_distance;
                min_distance_point = local_min_distance_point;
            }
        }
    }
    (*processed)[current_point] = true;
    return min_distance_point;
}
std::vector<int> dijkstra_omp(const std::vector<int>& graph, size_t start, size_t end) {
    if (graph.size() == 0)
        throw "Error: empty graph";

    size_t points_count = static_cast<size_t>(sqrt(graph.size()));

    if (points_count * points_count != graph.size()) {
        std::cout << "Illegal graph size: expected " << graph.size() << " calculated "
                  << points_count * points_count << std::endl;
        throw "Error: incorrect graph";
    }

    if (start < 1 || end < 1 || start > points_count || end > points_count) {
        std::cout << "Error: illegal start or end" << std::endl;
        throw "Error: illegal start or end";
    }
    start--; end--;

    if (points_count == 1) {
        return { static_cast<int>(start) + 1 };
    }

    constexpr int max_int = std::numeric_limits<int>::max();
    std::vector<int> distances = std::vector<int>(points_count, max_int);
    std::vector<bool> processed = std::vector<bool>(points_count, false);

    distances[start] = 0;

    int next_unprocessed_point = start;
    while (next_unprocessed_point >= 0) {
            next_unprocessed_point = process_unprocessed_point_omp(graph,
                &distances, &processed, next_unprocessed_point);
            if (next_unprocessed_point < 0) {
                next_unprocessed_point =
                    find_unprocessed_point_with_min_distance_omp(graph, distances,
                        processed);
            }
    }
    print_vector(distances, distances.size(), "distances:");

    std::vector<int> path;
    path.push_back(end + 1);
    int distance = distances[end];
    int current = end;
    bool found = false;

    while (current != static_cast<int>(start)) {
        found = false;

        #pragma omp parallel for
        for (int i = 0; i < static_cast<int>(points_count); i++) {
            // if (found) {
            //     std::cout << "################ Found and not skipped #################" << std::endl;
            // }
            if (!found && graph[current * points_count + i] > 0) {
                int tmp_distance = distance - graph[current * points_count + i];
                if (distances[i] == tmp_distance) {
                    #pragma omp critical
                    {
                        distance = tmp_distance;
                        current = i;
                        path.insert(path.begin(), i + 1);
                        found = true;
                        // std::cout << "################ Found - now request for skip #################" << std::endl;
                    }
                }
            }
        }
    }

    return path;
}
\end{lstlisting}
\par 3. TBB версия.
\begin{lstlisting}
int find_unprocessed_point_with_min_distance_tbb(const std::vector<int>& graph,
    const std::vector<int>& distances, const std::vector<bool>& processed) {

    PointInfo minPoint;
    minPoint.distance = MAX_DISTANCE;
    minPoint.index = -1;
    size_t pointsCount = processed.size();
    size_t pointsPerThread = pointsCount / THREADS_COUNT;

    minPoint = tbb::parallel_reduce(
        tbb::blocked_range<int>(0, static_cast<int>(pointsCount), static_cast<int>(pointsPerThread)),
        minPoint,
        [&](const tbb::blocked_range<int>& range, PointInfo local_minPoint) {
            for (int point = range.begin(); point < range.end(); point++) {
                if (!processed[point] && distances[point] < local_minPoint.distance) {
                    local_minPoint.distance = distances[point];
                    local_minPoint.index = point;
                }
            }
            return local_minPoint;
        },
        [&](PointInfo p1, PointInfo p2) {
            if (p1.distance > p2.distance) {
                return p2;
            }
            return p1;
        });

    return minPoint.index;
}
int process_unprocessed_point_tbb(const std::vector<int>& graph,
    std::vector<int>* distances,
    std::vector<bool>* processed, int current_point) {

    PointInfo minPoint;
    minPoint.distance = MAX_DISTANCE;
    minPoint.index = -1;
    size_t pointsCount = processed->size();
    size_t pointsPerThread = pointsCount / THREADS_COUNT;

    minPoint = tbb::parallel_reduce(
        tbb::blocked_range<int>(0, static_cast<int>(pointsCount), static_cast<int>(pointsPerThread)),
        minPoint,
        [&](const tbb::blocked_range<int>& range, PointInfo local_minPoint) {
            for (int point = range.begin(); point < range.end(); point++) {
                int start_row_index = current_point * static_cast<int>(processed->size());
                int distance = graph[start_row_index + point];
                if (!(*processed)[point] && distance > 0) {
                    int* dp = distances->data() + point;
                    int* dcp = distances->data() + current_point;
                    *dp = *dp < *dcp + distance ? *dp : *dcp + distance;  // std::min(*dp, *dcp + distance);
                    if (local_minPoint.distance > *dp) {
                        local_minPoint.distance = *dp;
                        local_minPoint.index = point;
                    }
                }
            }
            return local_minPoint;
        },
        [&](PointInfo p1, PointInfo p2) {
            if (p1.distance > p2.distance) {
                return p2;
            }
            return p1;
        });

    (*processed)[current_point] = true;
    return minPoint.index;
}
std::vector<int> dijkstra_tbb(const std::vector<int>& graph, size_t start, size_t end) {
    if (graph.size() == 0)
        throw "Error: empty graph";

    size_t points_count = static_cast<size_t>(sqrt(graph.size()));

    if (points_count * points_count != graph.size()) {
        std::cout << "Illegal graph size: expected " << graph.size() << " calculated "
                  << points_count * points_count << std::endl;
        throw "Error: incorrect graph";
    }

    if (start < 1 || end < 1 || start > points_count || end > points_count) {
        std::cout << "Error: illegal start or end" << std::endl;
        throw "Error: illegal start or end";
    }
    start--; end--;

    if (points_count == 1) {
        return { static_cast<int>(start) + 1 };
    }

    constexpr int max_int = MAX_DISTANCE;
    std::vector<int> distances = std::vector<int>(points_count, max_int);
    std::vector<bool> processed = std::vector<bool>(points_count, false);

    distances[start] = 0;

    tbb::task_scheduler_init init(THREADS_COUNT);
    tbb::mutex mutex;

    int next_unprocessed_point = static_cast<int>(start);
    while (next_unprocessed_point >= 0) {
            next_unprocessed_point = process_unprocessed_point_tbb(graph,
                &distances, &processed, next_unprocessed_point);
            if (next_unprocessed_point < 0) {
                next_unprocessed_point =
                    find_unprocessed_point_with_min_distance_tbb(graph, distances,
                        processed);
            }
    }
    print_vector(distances, distances.size(), "distances:");

    std::vector<int> path;
    path.push_back(static_cast<int>(end + 1));
    int weight = distances[end];
    int current = static_cast<int>(end);

    while (current != static_cast<int>(start)) {
        for (int i = 0; i < static_cast<int>(points_count); i++) {
            if (graph[current * points_count + i] > 0) {
                int tmp = weight - graph[current * points_count + i];
                if (distances[i] == tmp) {
                    weight = tmp;
                    current = i;
                    path.insert(path.begin(), i + 1);
                    break;
                }
            }
        }
    }

    return path;
}
\end{lstlisting}
\par 4. std::thread версия.
\begin{lstlisting}
void atom_process_unprocessed_point_std(
    const std::vector<int>& graph, const int start, const int end,
    std::vector<int>* distances, std::vector<bool>* processed,
    const int current_point, std::promise<PointInfo>&& pr) {
    int start_row_index = current_point * static_cast<int>(processed->size());
    int* dcp = &(*distances)[current_point];
    PointInfo local_minPoint;
    local_minPoint.distance = MAX_DISTANCE;
    local_minPoint.index = -1;
    for (int point = start; point < end; point++) {
        int distance = graph[start_row_index + point];
        if (!(*processed)[point] && distance > 0) {
            int* dp = &(*distances)[point];
            std::lock_guard<std::mutex> my_lock(my_mutex);
            *dp = *dp < *dcp + distance ? *dp : *dcp + distance;  // std::min(*dp, *dcp + distance);
            if (local_minPoint.distance > *dp) {
                local_minPoint.distance = *dp;
                local_minPoint.index = point;
            }
        }
    }
    pr.set_value(local_minPoint);
}
int process_unprocessed_point_std(const std::vector<int>& graph,
    std::vector<int>* distances,
    std::vector<bool>* processed, int current_point) {

    const int nthreads = std::thread::hardware_concurrency();  //  THREADS_COUNT

    size_t pointsCount = processed->size();
    size_t pointsPerThread = pointsCount / nthreads;

    std::promise<PointInfo> *promises = new std::promise<PointInfo>[nthreads];
    std::future<PointInfo> *futures = new std::future<PointInfo>[nthreads];
    std::thread *threads = new std::thread[nthreads];

    PointInfo minPoint;
    minPoint.distance = MAX_DISTANCE;
    minPoint.index = -1;
    for (int i = 0; i < nthreads; i++) {
        futures[i] = promises[i].get_future();
        int start = i * pointsPerThread;
        int end = i < nthreads - 1 ? (i + 1) * pointsPerThread : pointsCount;
        threads[i] = std::thread(atom_process_unprocessed_point_std, std::cref(graph), start, end, std::ref(distances),
            std::ref(processed), current_point, std::move(promises[i]));
        threads[i].detach();
    }

    for (int i = 0; i < nthreads; i++) {
        PointInfo threadMinPoint = futures[i].get();
        if (minPoint.distance > threadMinPoint.distance) {
            minPoint.distance = threadMinPoint.distance;
            minPoint.index = threadMinPoint.index;
        }
    }

    delete []promises;
    delete []futures;
    delete []threads;

    (*processed)[current_point] = true;
    return minPoint.index;
}
int find_unprocessed_point_with_min_distance_std(const std::vector<int>& graph,
    const std::vector<int>& distances, const std::vector<bool>& processed) {


    size_t pointsCount = processed.size();


    const int nthreads = std::thread::hardware_concurrency();  //  THREADS_COUNT
    size_t pointsPerThread = pointsCount / nthreads;
    std::promise<PointInfo> *promises = new std::promise<PointInfo>[nthreads];
    std::future<PointInfo> *futures = new std::future<PointInfo>[nthreads];
    std::thread *threads = new std::thread[nthreads];

#ifdef DEBUG_PRINT
    std::cout << "Threads: " << nthreads << ", Points per thread: " << pointsPerThread
        << ", last: " << last << std::endl;
#endif


    PointInfo minPoint;
    minPoint.distance = MAX_DISTANCE;
    minPoint.index = -1;
    for (int i = 0; i < nthreads; i++) {
        futures[i] = promises[i].get_future();
        int start = i * pointsPerThread;
        int end = i < nthreads - 1 ? (i + 1) * pointsPerThread : pointsCount;
#ifdef DEBUG_PRINT
        std::cout << "Thread: " << i << ", start: " << start << ", end: " << end << std::endl;
#endif
        threads[i] = std::thread(atom_find_unprocessed_point_with_min_distance_std,
            std::ref(graph), start, end, std::ref(distances), std::ref(processed), std::move(promises[i]));
        threads[i].detach();
    }

    for (int i = 0; i < nthreads; i++) {
        PointInfo threadMinPoint = futures[i].get();
        if (minPoint.distance > threadMinPoint.distance) {
            minPoint.distance = threadMinPoint.distance;
            minPoint.index = threadMinPoint.index;
        }
    }

    delete []promises;
    delete []futures;
    delete []threads;

    return minPoint.index;
}
std::vector<int> dijkstra_std(const std::vector<int>& graph, size_t start, size_t end) {
    if (graph.size() == 0)
        throw "Error: empty graph";

    size_t points_count = static_cast<size_t>(sqrt(graph.size()));

    if (points_count * points_count != graph.size()) {
        std::cout << "Illegal graph size: expected " << graph.size() << " calculated "
                  << points_count * points_count << std::endl;
        throw "Error: incorrect graph";
    }

    if (start < 1 || end < 1 || start > points_count || end > points_count) {
        std::cout << "Error: illegal start or end" << std::endl;
        throw "Error: illegal start or end";
    }
    start--; end--;

    if (points_count == 1) {
        return { static_cast<int>(start) + 1 };
    }

    constexpr int max_int = MAX_DISTANCE;
    std::vector<int> distances = std::vector<int>(points_count, max_int);
    std::vector<bool> processed = std::vector<bool>(points_count, false);

    distances[start] = 0;
    int next_unprocessed_point = static_cast<int>(start);
    while (next_unprocessed_point >= 0) {
            next_unprocessed_point = process_unprocessed_point_std(graph,
                &distances, &processed, next_unprocessed_point);
            if (next_unprocessed_point < 0) {
                //  std::cout << "- next unprocessed point not given, finding." << std::endl;
                next_unprocessed_point =
                    find_unprocessed_point_with_min_distance_std(graph, distances,
                        processed);
            } else {
                //  std::cout << "+ next unprocessed point given, ok." << std::endl;
            }
    }
    print_vector(distances, distances.size(), "distances:");

    std::vector<int> path;
    path.push_back(static_cast<int>(end + 1));
    int weight = distances[end];
    int current = static_cast<int>(end);

    while (current != static_cast<int>(start)) {
        for (int i = 0; i < static_cast<int>(points_count); i++) {
            if (graph[current * points_count + i] > 0) {
                int tmp = weight - graph[current * points_count + i];
                if (distances[i] == tmp) {
                    weight = tmp;
                    current = i;
                    path.insert(path.begin(), i + 1);
                    break;
                }
            }
        }
    }

    return path;
}
\end{lstlisting}

\end{document}