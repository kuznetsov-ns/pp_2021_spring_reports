\documentclass{report}

\usepackage[T2A]{fontenc}
\usepackage[utf8]{luainputenc}
\usepackage[english, russian]{babel}
\usepackage[pdftex]{hyperref}
\usepackage[14pt]{extsizes}
\usepackage{listings}
\usepackage{color}
\usepackage{geometry}
\usepackage{enumitem}
\usepackage{multirow}
\usepackage{graphicx}
\usepackage{indentfirst}
\usepackage{amsmath}

\geometry{a4paper,top=2cm,bottom=3cm,left=2cm,right=1.5cm}
\setlength{\parskip}{0.5cm}
\setlist{nolistsep, itemsep=0.3cm,parsep=0pt}

\lstset{language=C++,
		basicstyle=\footnotesize,
		keywordstyle=\color{blue}\ttfamily,
		stringstyle=\color{red}\ttfamily,
		commentstyle=\color{green}\ttfamily,
		morecomment=[l][\color{magenta}]{\#}, 
		tabsize=4,
		breaklines=true,
  		breakatwhitespace=true,
  		title=\lstname,       
}

\addto\captionsrussian{\renewcommand \contentsname {\LARGE Содержание}}

\makeatletter
\renewcommand\@biblabel[1]{#1.\hfil}
\makeatother

\begin{document}

\begin{titlepage}

\begin{center}
МИНИСТЕРСТВО НАУКИ И ВЫСШЕГО ОБРАЗОВАНИЯ РОССИЙСКОЙ ФЕДЕРАЦИИ\\
Федеральное государственное автономное образовательное учреждение высшего образования \\
\textbf{"Национальный исследовательский Нижегородский государственный университет им. Н.И. Лобачевского" (ННГУ)}
\end{center}

\begin{center}
\textbf{Институт информационных технологий, математики и механики}
\end{center}

\vspace{4em}

\begin{center}
Отчет по лабораторной работе \\
\end{center}
\begin{center}
\textbf{\Large«Решение систем линейных уравнений методом сопряженных градиентов»} \\
\end{center}

\vspace{4em}

\newbox{\lbox}
\savebox{\lbox}{\hbox{text}}
\newlength{\maxl}
\setlength{\maxl}{\wd\lbox}
\hfill\parbox{7cm}{
\hspace*{5cm}\hspace*{-5cm}\textbf{Выполнил:} \\ студент группы 381808-1 \\ Горбунова В.И.\\
\\
\hspace*{5cm}\hspace*{-5cm}\textbf{Проверил:}\\ доцент кафедры МОСТ, \\ кандидат технических наук \\ Сысоев А. В.\\
}
\vspace{\fill}

\begin{center} Нижний Новгород \\ 2021 \end{center}

\end{titlepage}

\setcounter{page}{2}
% содержание
\tableofcontents{}
\clearpage

\begin{center}
\section*{Введение}
\end{center}
\addcontentsline{toc}{section}{Введение}
\par В практических задачах часто возникает необходимость решения систем линейных алгебраических уравнений (СЛАУ) с симметричной, положительно определенной матрицей. Такие матрицы возникают, например, при дискретизации эллиптических уравнений математической физики и порождают линейные системы большой или очень большой размерности. В связи с этим при реализации алгоритмов решения СЛАУ на однопроцессорном компьютере могут возникнуть трудности из-за нехватки памяти, либо решение займет много времени. В этом случае необходимо использовать нестандартные подходы к написанию алгоритмов, основанные на использовании технологии параллельного программирования.
\parЦелью данной работы является разработка параллельного алгоритма метода сопряженных градиентов для решения минимизации квадратичной функции.В данной работе было реализовано 3 решения данной задачи: Последовательная реализация, параллельная реализация с использованием OpenMP и параллельная реализация с использованием Tbb.
\newpage

\begin{center}
\section*{Постановка задач}
\end{center}
\addcontentsline{toc}{section}{Постановка задач}
В данной лабораторной работе необходимо:
\begin{enumerate}
\item Реализовать последовательный алгоритм решения системы линейных уравнений методом сопряженных градиентов.
\item Реализовать параллельный алгоритм решения системы линейных уравнений методом сопряженных градиентов с использованием технологии OpenMP.
\item Реализовать параллельный алгоритм решения системы линейных уравнений методом сопряженных градиентов с использованием технологии Tbb.
\item Провести вычислительные эксперементы и сравнить время работы трех алгоритмов.
\item Сделать вывод на основе полученных результатов.
\end{enumerate}
\newpage

\begin{center}
\section*{Описание алгоритма}
\end{center}
\addcontentsline{toc}{section}{Описание алгоритма}
\parМетод сопряженных градиентов–один из наиболее известных итерационных методов решения систем линейных уравнений. Он может быть применен для решения системы линейных уравнений с симметричной, положительноопределенной матрицей
\par$A = A^T $ - симметричная матрица A 
\par$x^T A x$ - положительно определённая матрица A 
\parПосле выполнения n итераций метода сопряженных градиентов(n есть порядок решаемой системы линейных уравнений), очередное приближение $X_n$ совпадает с точным решением.
\parЕсли матрица А симметричная и положительноопределена, то функция:
\begin {center}
$q(x) =  \frac{1}{2} x^T * Ax - x^Tb + c$
\end {center}
\parимеет единственный минимум который достигается в точке X*, совпадающий с решением системы линейных уравнений.
\parИтерация метода сопряженных градиентов состоит в вычислении очередного приближения к точному решению
где:
\begin {center}
$x^k = x^{k-1} + s^k d^k$
\end {center}

\par $x^k$ - очередное приближение
\par $x^{k-1}$- приближение, построенное на предыдущем шаге
\par $S^k$- скалярный шаг
\par $d^k$- вектор направления
\parПеред выполнением первой итерации $x^0$ и $d^0$ полагаются равными нулю, а для вектора $g^0$ устанавливается значение -b
\parАлогритм:

1. Вычисление градиента
$$g^k=A*X^{k-1}-b$$

2.   Вычисление вектора направления
$$d^k= -g^k + \frac{((g^k)^T, g^k)}{((g^{k-1})^T, g^{k-1})} * d^{k-1}$$


3.   Вычисление величины смещения по заданному направлению

$$s^k = \frac{(d^k, g^k)}{(d^k)^T * A * d^k}$$

4.   Вычисление нового приближения

$$x^k = x^{k-1} + s^k d^k$$

\newpage

\begin{center}
\section*{Схема распараллеливания}
\end{center}
\addcontentsline{toc}{section}{Схема распараллеливания}
При распараллеливании метода сопряженных градиентов создаются две функции для умножения матрицы на вектор и для вычисления скалярного произведения векторов. Они используются несколько раз в алгоритме решения задачи. В данных функциях матрица или вектор разделяются на несколько частей и вычисления происходят параллельно друг другу.
\newpage

\begin{center}
\section*{Описание программной реализации}
\end{center}
\addcontentsline{toc}{section}{Описание программной реализации}
\par Программа состоит из следующих функций:
\par Функция, которая создает симметричную положительноопределенную матрицу заданного размера :
\begin{lstlisting}
void Random_Matrix_A(int S, double** A);
\end{lstlisting}
\par Функция, которая создает рандомный вектор B (результат уравнений в СЛАУ)
\begin{lstlisting}
void Random_Matrix_B(int S, double* B);
\end{lstlisting}
\par Функция умножения матрицы на вектор(последовательная реализация):
\begin{lstlisting}
void MatrixMultiplicateS(double** A, int S, double* Ax, double* x1);
\end{lstlisting}
\par Функция умножения матрицы на вектор(параллельная реализация OpenMP):
\begin{lstlisting}
void MatrixMultiplicateP(double** A, int S, double* Ax, double* x1);
\end{lstlisting}
\par Функция умножения матрицы на вектор(параллельная реализация Tbb):
\begin{lstlisting}
void MatrixMultiplicateP(double** A, int S, double* Ax, double* x1, int grainsize);
\end{lstlisting}
\par Скалярное произведение векторов(последовательная реализация):
\begin{lstlisting}
double multiplicateS(int S, double* x1, double* x2);
\end{lstlisting}
\par Скалярное произведение векторов(параллельная реализация OpenMP):
\begin{lstlisting}
double multiplicateP(int S, double* x1, double* x2);
\end{lstlisting}
\par Скалярное произведение векторов(параллельная реализация Tbb):
\begin{lstlisting}
double multiplicateP(int S, double* x1, double* x2,int grainsize)
\end{lstlisting}
\par Функция с решением слау методом сопряженных градиентов(последовательная реализация):
\begin{lstlisting}
double conj_gradS(double** A, double* B, int S);
\end{lstlisting}
\par Функция с решением слау методом сопряженных градиентов(параллельная реализация OpenMP):
\begin{lstlisting}
double multiplicateP(int S, double* x1, double* x2);
\end{lstlisting}
\par Функция с решением слау методом сопряженных градиентов(параллельная реализация Tbb):
\begin{lstlisting}
double multiplicateP(int S, double* x1, double* x2,int grainsize);
\end{lstlisting}

\newpage

\begin{center}
\section*{Подтверждение корректности}
\end{center}
\addcontentsline{toc}{section}{Подтверждение корректности}
\par Для подтверждения корректности в программе реализована система тестов, разработанная с помощью использования Google C++ Testing Framework.
\par Эта система включает в себя тесты, проверяющие как работает программа на матрицах разного размера, от 5*5 до 1000*1000. Все тесты проходят успешно что подтверждает работоспособность программы.
\newpage

\begin{center}
\section*{Результаты эксперементов}
\end{center}
\addcontentsline{toc}{section}{Результаты эксперементов}
\par Вычислительные эксперименты для оценки эффективности сортировки Шелла с четно-нечетным слиянием Бэтчера проводились на оборудовании со следующей аппаратной конфигурацией::
\begin{enumerate}
\item Процессор: AMD Ryzon 7 3700X - 8 Core Processor
\item Оперативная память:32 Gb;
\item Операционная система: Windows 10 Домашняя;
\item Среда разработки: Visual Studio 2019.
\end{enumerate}
\begin{table}[!h]
\caption{Результаты вычислительных экспериментов. Сравнение последовательного алгоритма с OpenMP}
\centering
\begin{tabular}{|p{4cm}|p{4cm}|p{4cm}|p{4cm}|}
\hline
Количество процессов & Размер матрицы & Последовательный алгоритм,сек & Параллельный алгоритм,сек \\\hline
4  & 500*500 & 1.485732 & 0.349583   \\\hline
10  & 500*500 & 1.569384 & 0.195423  \\\hline
4 & 1000*1000 & 2.38623 & 0.654635  \\\hline
10  & 1000*1000 & 2.745632 & 0.728755  \\
\hline
\end{tabular}
\end{table}

\begin{table}[!h]
\caption{Результаты вычислительных экспериментов. Сравнение последовательного алгоритма с TBB}
\centering
\begin{tabular}{|p{4cm}|p{4cm}|p{4cm}|p{4cm}|}
\hline
Количество процессов & Размер матрицы & Последовательный алгоритм,сек & Параллельный алгоритм,сек \\\hline
4  & 500*500 & 1.504938 & 0.259605   \\\hline
10  & 500*500 & 1.598695 & 0.194857  \\\hline
4 & 1000*1000 & 2.948273 & 0.306948 \\\hline
10  & 1000*1000 & 3.495823 & 0.548573  \\
\hline
\end{tabular}
\end{table}
\newpage

\begin{center}
\section*{Заключение}
\end{center}
\addcontentsline{toc}{section}{Заключение}
\par В ходе данной лабораторной работы мною были реализованы три версии алгоритма решения СЛАУ методом сопряженных градиентов: последовательная реализация, параллельнная реализация с импользованием технологии OpenMP и параллельнная реализаци\ с использованием технологии TBB. А также были разработаны и доведены до успешного выполнения тесты, созданные для данного программного проекта с использованием Google C++ Testing Framework и необходимые для подтверждения корректности работы программы.
\par Основной целью работы было создание параллельной версии метода, который способен выполнять вычисления в разы быстрее чем последовательная версия.
\newpage

\addcontentsline{toc}{section}{Литература}
\begin{thebibliography}{}
\bibitem{1} Гергель В. П. Теория и практика параллельных вычислений. – 2007.
\bibitem{2} Гергель В. П., Стронгин Р. Г. Основы параллельных вычислений для многопроцессорных вычислительных систем. – 2003.
\bibitem{3}http://www.hpcc.unn.ru/?dir=847


\end{thebibliography}{}
\newpage

\begin{center}
\section*{Приложение}
\end{center}
\addcontentsline{toc}{section}{Приложение}
В данном разделе находятся две версии кода, OMP реализация и Tbb реализация,содержащая код последовательной программы.
\begin{lstlisting}
//gorbunova_v_conjugate_grad_omp.h

// Copyright 2021 Gorbunova Valeria
#ifndef MODULES_TASK_2_GORBUNOVA_V_CONJUGATE_GRAD_OMP_CONJUGATE_GRADIENT_H_
#define MODULES_TASK_2_GORBUNOVA_V_CONJUGATE_GRAD_OMP_CONJUGATE_GRADIENT_H_

double conj_grad(double** A, double* B, int S, int proc);
void MatrixMultiplicate(double** A, int S, double* Ax, double* x1);
double multiplicate(int S, double* x1, double* x2);
void Random_Matrix_A(int S, double** A);
void Random_Matrix_B(int S, double* B);
#endif  // MODULES_TASK_2_GORBUNOVA_V_CONJUGATE_GRAD_OMP_CONJUGATE_GRADIENT_H_


//gorbunova_v_conjugate_grad_omp.cpp

// Copyright 2021 Gorbunova Valeria
#include <stdio.h>
#include <omp.h>
#include <stdlib.h>
#include <iostream>
#include "../../../modules/task_2/gorbunova_v_conjugate_grad_omp/conjugate_gradient.h"

void Random_Matrix_A(int S, double** A) {
    for (int i = 0; i < S; i++) {
        for (int j = 0; j < S; j++) {
            A[i][i] = std::rand() % 2;
            if ((i != j) && (i < j)) {
                A[i][j] = std::rand() % 100 - std::rand() % 100;
            }
            A[j][i] = A[i][j];
        }
    }
}

void Random_Matrix_B(int S, double* B) {
    for (int i = 0; i < S; i++) {
        B[i] = std::rand() % 100 - std::rand() % 100;
    }
}

void MatrixMultiplicate(double** A, int S, double* Ax, double* x1) {
#pragma omp parallel for
    for (int i = 0; i < S; i++) {
        Ax[i] = 0;
        for (int j = 0; j < S; j++)
            Ax[i] += A[i][j] * x1[j];
    }
}

double multiplicate(int S, double* x1, double* x2) {
    double result = 0;
#pragma omp parallel for reduction(+:result)
    for (int i = 0; i < S; i++) {
        result += x1[i] * x2[i];
    }
    return result;
}

double conj_grad(double** A, double* B, int S, int proc) {
    double t1,t2;
    double* x = new double[S];
    double* d = new double[S];
    double* r = new double[S];
    double* Ax = new double[S];
    double alpha, beta, sumB, num, R, den;
    int i, IterNum = 1;
    t1 = omp_get_wtime();
    double MaxIteration = 100000;
    for (i = 0; i < S; i++) {
        x[i] = 1;
    }
    omp_set_num_threads(proc);
    MatrixMultiplicate(A, S, Ax, x);
    for (int i = 0; i < S; i++) {
        r[i] = B[i] - Ax[i];
        d[i] = r[i];
    }
    for (sumB = 0, i = 0; i < S; i++) {
        sumB += B[i] * B[i];
    }

    int Iteration = 0;
    do {
        Iteration++;
        MatrixMultiplicate(A, S, Ax, d);
        num = multiplicate(S, r, r);
        den = multiplicate(S, Ax, d);
        alpha = num / den;
        for (i = 0; i < S; i++) {
            x[i] += alpha * d[i];
            r[i] -= alpha * Ax[i];
        }
        IterNum++;
        R = multiplicate(S, r, r);
        beta = R / num;
        for (i = 0; i < S; i++)
            d[i] = r[i] + beta * d[i];
    } while (R / sumB > 0.01 * 0.01 && Iteration < MaxIteration);

    int res = 0;
    double* A1 = new double[S];
    int F = S / 10;
    if (S / 10 == 0) {
        F = 1;
    }
    double fault = 0.2 * F;
    for (i = 0; i < S; i++) {
        A1[i] = 0;
        for (int j = 0; j < S; j++) {
            A1[i] += A[i][j]*x[j];
        }
        if (((abs(A1[i]) > abs(B[i])) && (abs(A1[i]) - abs(B[i]) > fault))||
        ((abs(A1[i]) <= abs(B[i])) && (abs(B[i]) - abs(A1[i]) > fault))) {
            res = 1;
            break;
        }
    }
    t2 = omp_get_wtime();
    std::cout << "Omp time: " << t1 - t2 << std::endl;
    delete[] Ax;
    for (int i = 0; i < S; i++)
        delete[] A[i];
    delete[] A;
    delete[] A1;
    delete[] B;
    delete[] d;
    delete[] r;
    delete[] x;
    return res;
}



//main.cpp

// Copyright 2021 Gorbunova Valeria
#include <stdio.h>
#include <gtest/gtest.h>
#include <omp.h>
#include "./conjugate_gradient.h"

TEST(Conj_Grad, test_1_Rand_Matrix_5x5_5_proc) {
    int S = 5;
    double** A = new double* [S];
    for (int count = 0; count < S; count++)
        A[count] = new double[S];
    Random_Matrix_A(S, A);
    double* B = new double[S];
    Random_Matrix_B(S, B);
    for (int i = 0; i < S; i++) {
        for (int j = 0; j < S; j++) {
            std::cout << A[i][j] << "  ";
        }
        std::cout <<"  " << B[i] << std::endl;
    }
    ASSERT_EQ(0, conj_grad(A, B, S, 3));
}

TEST(Conj_Grad, test_2_Rand_Matrix_500x500_4_proc) 
    int S = 500;
    double** A = new double* [S];
    for (int count = 0; count < S; count++)
        A[count] = new double[S];
    Random_Matrix_A(S, A);
    double* B = new double[S];
    Random_Matrix_B(S, B);
    ASSERT_EQ(0, conj_grad(A, B, S, 4));
}

TEST(Conj_Grad, test_3_Rand_Matrix_500x500_10_proc) {
    int S = 500;
    double** A = new double* [S];
    for (int count = 0; count < S; count++)
        A[count] = new double[S];
    Random_Matrix_A(S, A);
    double* B = new double[S];
    Random_Matrix_B(S, B);
    ASSERT_EQ(0, conj_grad(A, B, S, 10));
}


TEST(Conj_Grad, test_4_Rand_Matrix_1000x1000_4_proc) {
    int S = 1000;
    double** A = new double* [S];
    for (int count = 0; count < S; count++)
        A[count] = new double[S];
    Random_Matrix_A(S, A);
    double* B = new double[S];
    Random_Matrix_B(S, B);
    ASSERT_EQ(0, conj_grad(A, B, S, 4));
}

TEST(Conj_Grad, test_5_Rand_Matrix_1000x1000_10_proc) {
    int S = 1000;
    double** A = new double* [S];
    for (int count = 0; count < S; count++)
        A[count] = new double[S];
    Random_Matrix_A(S, A);
    double* B = new double[S];
    Random_Matrix_B(S, B);
    ASSERT_EQ(0, conj_grad(A, B, S, 10));
}

int main(int argc, char **argv) {
    ::testing::InitGoogleTest(&argc, argv);
    return RUN_ALL_TESTS();


//gorbunova_v_conjugate_grad_tbb.h

// Copyright 2021 Gorbunova Valeria
#ifndef MODULES_TASK_2_GORBUNOVA_V_CONJUGATE_GRAD_OMP_CONJUGATE_GRADIENT_H_
#define MODULES_TASK_2_GORBUNOVA_V_CONJUGATE_GRAD_OMP_CONJUGATE_GRADIENT_H_

double conj_grad(double** A, double* B, int S, int proc);
void MatrixMultiplicate(double** A, int S, double* Ax, double* x1);
double multiplicate(int S, double* x1, double* x2);
void Random_Matrix_A(int S, double** A);
void Random_Matrix_B(int S, double* B);
#endif  // MODULES_TASK_2_GORBUNOVA_V_CONJUGATE_GRAD_OMP_CONJUGATE_GRADIENT_H_

//gorbunova_v_conjugate_grad_tbb.cpp

/// Copyright 2021 Gorbunova Valeria
#include <stdio.h>
#include <tbb/tbb.h>
#include <tbb/blocked_range.h>
#include "tbb/blocked_range2d.h"
#include <stdlib.h>
#include <thread>
#include <iostream>
#include "../../../modules/task_3/gorbunova_v_conjugate_grad_tbb/conjugate_gradient.h"

using namespace tbb;

void Random_Matrix_A(int S, double** A) {
    for (int i = 0; i < S; i++) {
        for (int j = 0; j < S; j++) {
            A[i][i] = std::rand() % 2;
            if ((i != j) && (i < j)) {
                A[i][j] = std::rand() % 100 - std::rand() % 100;
            }
            A[j][i] = A[i][j];
        }
    }
}

void Random_Matrix_B(int S, double* B) {
    for (int i = 0; i < S; i++) {
        B[i] = std::rand() % 100 - std::rand() % 100;
    }
}

void MatrixMultiplicateP(double** A, int S, double* Ax, double* x1,int grainsize) {
    tbb::parallel_for(tbb::blocked_range2d<int>(0 , S, grainsize , 0, S, S),
        [&](tbb::blocked_range2d<int> r) {
            for (int i = r.rows().begin(); i < r.rows().end(); i++) {
                Ax[i] = 0;
                for (int j = r.cols().begin(); j < r.cols().end(); j++)
                    Ax[i] += A[i][j] * x1[j];
            } 
        });
}

double multiplicateP(int S, double* x1, double* x2,int grainsize) {
    double result = 0;
    std::thread t( [=]{
        for (int i = r.begin(); i < r.end(); i++) {
                result += x1[i] * x2[i];
            }
    });
    return result;
}

double conj_gradP(double** A, double* B, int S, int proc) {
    double* x = new double[S];
    double* d = new double[S];
    double* r = new double[S];
    double* Ax = new double[S];
    double alpha, beta, sumB, num, R, den;
    int i, IterNum = 1;
    int grainsize = S / proc;
    tbb::task_scheduler_init init(proc);
    tbb::tick_count start2 = tbb::tick_count::now();

    double MaxIteration = 100000;
    for (i = 0; i < S; i++) {
        x[i] = 1;
    }

    MatrixMultiplicateP(A, S, Ax, x, grainsize);
    for (int i = 0; i < S; i++) {
        r[i] = B[i] - Ax[i];
        d[i] = r[i];
    }
    for (sumB = 0, i = 0; i < S; i++) {
        sumB += B[i] * B[i];
    }

    int Iteration = 0;
    do {
        Iteration++;
        MatrixMultiplicateP(A, S, Ax, d, grainsize);
        num = multiplicateS(S, r, r);
        den = multiplicateS(S, Ax, d);
        alpha = num / den;
        for (i = 0; i < S; i++) {
            x[i] += alpha * d[i];
            r[i] -= alpha * Ax[i];
        }
        IterNum++;
        R = multiplicateS(S, r, r);
        beta = R / num;
        for (i = 0; i < S; i++)
            d[i] = r[i] + beta * d[i];
    } while (R / sumB > 0.01 * 0.01 && Iteration < MaxIteration);

    int res = 0;
    double* A1 = new double[S];
    int F = S / 10;
    if (S / 10 == 0) {
        F = 1;
    }
    double fault = 0.2 * F;
    for (i = 0; i < S; i++) {
        A1[i] = 0;
        for (int j = 0; j < S; j++) {
            A1[i] += A[i][j] * x[j];
        }
        if (((abs(A1[i]) > abs(B[i])) && (abs(A1[i]) - abs(B[i]) > fault)) ||
            ((abs(A1[i]) <= abs(B[i])) && (abs(B[i]) - abs(A1[i]) > fault))) {
            res = 1;
            break;
        }
    }
    tbb::tick_count end2 = tbb::tick_count::now();
    std::cout << "Tbb time: " << (end2 - start2).seconds() << std::endl;
    delete[] Ax;
    for (int i = 0; i < S; i++)
        delete[] A[i];
    delete[] A;
    delete[] A1;
    delete[] B;
    delete[] d;
    delete[] r;
    delete[] x;
    return res;
}




void MatrixMultiplicateS(double** A, int S, double* Ax, double* x1) {
    for (int i = 0; i < S; i++) {
        Ax[i] = 0;
        for (int j = 0; j < S; j++)
            Ax[i] += A[i][j] * x1[j];
    }
}

double multiplicateS(int S, double* x1, double* x2) {
    double result = 0;
    for (int i = 0; i < S; i++) {
        result += x1[i] * x2[i];
    }
    return result;
}

double conj_gradS(double** A, double* B, int S) {
    double* x = new double[S];
    double* d = new double[S];
    double* r = new double[S];
    double* Ax = new double[S];
    double alpha, beta, sumB, num, R, den;
    int i, IterNum = 1;
    double MaxIteration = 100000;
    tbb::tick_count start2 = tbb::tick_count::now();;
    for (i = 0; i < S; i++) {
        x[i] = 1;
    }

    MatrixMultiplicateS(A, S, Ax, x);
    for (int i = 0; i < S; i++) {
        r[i] = B[i] - Ax[i];
        d[i] = r[i];
    }
    for (sumB = 0, i = 0; i < S; i++) {
        sumB += B[i] * B[i];
    }

    int Iteration = 0;
    do {
        Iteration++;
        MatrixMultiplicateS(A, S, Ax, d);
        num = multiplicateS(S, r, r);
        den = multiplicateS(S, Ax, d);
        alpha = num / den;
        for (i = 0; i < S; i++) {
            x[i] += alpha * d[i];
            r[i] -= alpha * Ax[i];
        }
        IterNum++;
        R = multiplicateS(S, r, r);
        beta = R / num;
        for (i = 0; i < S; i++)
            d[i] = r[i] + beta * d[i];
    } while (R / sumB > 0.01 * 0.01 && Iteration < MaxIteration);

    int res = 0;
    double* A1 = new double[S];
    int F = S / 10;
    if (S / 10 == 0) {
        F = 1;
    }
    double fault = 0.2 * F;
    for (i = 0; i < S; i++) {
        A1[i] = 0;
        for (int j = 0; j < S; j++) {
            A1[i] += A[i][j] * x[j];
        }
        if (((abs(A1[i]) > abs(B[i])) && (abs(A1[i]) - abs(B[i]) > fault)) ||
            ((abs(A1[i]) <= abs(B[i])) && (abs(B[i]) - abs(A1[i]) > fault))) {
            res = 1;
            break;
        }
    }
    tbb::tick_count end2 = tbb::tick_count::now();
    std::cout << " time: " << (end2 - start2).seconds() << std::endl;
    return res;
}

//main.cpp

// Copyright 2021 Gorbunova Valeria
#include <stdio.h>
#include <gtest/gtest.h>
#include <omp.h>
#include "./conjugate_gradient.h"

TEST(Conj_Grad, test_1_Rand_Matrix_5x5_5_proc) {
    int S = 5;
    double** A = new double* [S];
    for (int count = 0; count < S; count++)
        A[count] = new double[S];
    Random_Matrix_A(S, A);
    double* B = new double[S];
    Random_Matrix_B(S, B);
    for (int i = 0; i < S; i++) {
        for (int j = 0; j < S; j++) {
            std::cout << A[i][j] << "  ";
        }
        std::cout <<"  " << B[i] << std::endl;
    }
    ASSERT_EQ(0, conj_grad(A, B, S, 3));
}

TEST(Conj_Grad, test_2_Rand_Matrix_500x500_4_proc) 
    int S = 500;
    double** A = new double* [S];
    for (int count = 0; count < S; count++)
        A[count] = new double[S];
    Random_Matrix_A(S, A);
    double* B = new double[S];
    Random_Matrix_B(S, B);
    ASSERT_EQ(0, conj_grad(A, B, S, 4));
}

TEST(Conj_Grad, test_3_Rand_Matrix_500x500_10_proc) {
    int S = 500;
    double** A = new double* [S];
    for (int count = 0; count < S; count++)
        A[count] = new double[S];
    Random_Matrix_A(S, A);
    double* B = new double[S];
    Random_Matrix_B(S, B);
    ASSERT_EQ(0, conj_grad(A, B, S, 10));
}


TEST(Conj_Grad, test_4_Rand_Matrix_1000x1000_4_proc) {
    int S = 1000;
    double** A = new double* [S];
    for (int count = 0; count < S; count++)
        A[count] = new double[S];
    Random_Matrix_A(S, A);
    double* B = new double[S];
    Random_Matrix_B(S, B);
    ASSERT_EQ(0, conj_grad(A, B, S, 4));
}

TEST(Conj_Grad, test_5_Rand_Matrix_1000x1000_10_proc) {
    int S = 1000;
    double** A = new double* [S];
    for (int count = 0; count < S; count++)
        A[count] = new double[S];
    Random_Matrix_A(S, A);
    double* B = new double[S];
    Random_Matrix_B(S, B);
    ASSERT_EQ(0, conj_grad(A, B, S, 10));
}

int main(int argc, char **argv) {
    ::testing::InitGoogleTest(&argc, argv);
    return RUN_ALL_TESTS();
}
\end{lstlisting}
\end{document}
