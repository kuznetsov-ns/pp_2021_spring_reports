\documentclass{report}

\usepackage[T2A]{fontenc}
\usepackage[utf8]{luainputenc}
\usepackage[english, russian]{babel}
\usepackage[pdftex]{hyperref}
\usepackage[14pt]{extsizes}
\usepackage{listings}
\usepackage{color}
\usepackage{geometry}
\usepackage{enumitem}
\usepackage{multirow}
\usepackage{graphicx}
\usepackage{indentfirst}

\geometry{a4paper,top=2cm,bottom=3cm,left=2cm,right=1.5cm}
\setlength{\parskip}{0.5cm}
\setlist{nolistsep, itemsep=0.3cm,parsep=0pt}

\lstset{language=C++,
		basicstyle=\footnotesize,
		keywordstyle=\color{blue}\ttfamily,
		stringstyle=\color{red}\ttfamily,
		commentstyle=\color{green}\ttfamily,
		morecomment=[l][\color{magenta}]{\#}, 
		tabsize=4,
		breaklines=true,
  		breakatwhitespace=true,
  		title=\lstname,       
}

\makeatletter
\renewcommand\@biblabel[1]{#1.\hfil}
\makeatother

\begin{document}

\begin{titlepage}

\begin{center}
Министерство науки и высшего образования Российской Федерации
\end{center}

\begin{center}
Федеральное государственное автономное образовательное учреждение высшего образования \\
Национальный исследовательский Нижегородский государственный университет им. Н.И. Лобачевского
\end{center}

\begin{center}
Институт информационных технологий, математики и механики
\end{center}

\vspace{4em}

\begin{center}
\textbf{\LargeОтчет по лабораторной работе} \\
\end{center}
\begin{center}
\textbf{\Large«Сортировка Шелла простым слиянием»} \\
\end{center}

\vspace{4em}

\newbox{\lbox}
\savebox{\lbox}{\hbox{text}}
\newlength{\maxl}
\setlength{\maxl}{\wd\lbox}
\hfill\parbox{7cm}{
\hspace*{5cm}\hspace*{-5cm}\textbf{Выполнил:} \\ студент группы 381808-1 \\ Кочанков Илья\\
\\
\hspace*{5cm}\hspace*{-5cm}\textbf{Проверил:}\\ доцент кафедры МОСТ, \\ кандидат технических наук \\ Сысоев А.В.\\
}
\vspace{\fill}

\begin{center} Нижний Новгород \\ 2021 \end{center}

\end{titlepage}

\setcounter{page}{2}

% Введение
\section*{Введение}
\addcontentsline{toc}{section}{Введение}
Сортировка Шелла (англ. Shell sort) — алгоритм сортировки, являющийся усовершенствованным вариантом сортировки вставками. Идея метода Шелла состоит в сравнении элементов, стоящих не только рядом, но и на определённом расстоянии друг от друга.
\newpage

% Постановка задачи
\section*{Постановка задачи}
\addcontentsline{toc}{section}{Постановка задачи}
В данной работе необходимо реализовать сортировку Шелла с простым слиянием. Данная работа должна содержать последовательную и несколько параллельных версий, с использованием следующих технологий: OpenMP, TBB, API std::thread. 
\par
На основе разработанной программы необходимо провести вычислительные эксперименты, сравнив время работы последовательного и параллельных алгоритмов на разных входных данных, затем сделать вывод об эффективности параллельных версий.
\par
Также следует экспериментально проверить корректность разработанных версий алгоритма с помощью Google тестов.
\newpage

% Описание алгоритма
\section*{Описание алгоритма}
\addcontentsline{toc}{section}{Описание алгоритма}
При сортировке Шелла сначала сравниваются и сортируются между собой значения, стоящие один от другого на некотором расстоянии d. После этого процедура повторяется для некоторых меньших значений d, а завершается сортировка Шелла упорядочиванием элементов при d=1 (то есть обычной сортировкой вставками). 
\par
Эффективность сортировки Шелла в определённых случаях обеспечивается тем, что элементы «быстрее» встают на свои места (в простых методах сортировки, например, пузырьковой, каждая перестановка двух элементов уменьшает количество инверсий в списке максимум на 1, а при сортировке Шелла это число может быть больше).

Невзирая на то, что сортировка Шелла во многих случаях медленнее, чем быстрая сортировка, она имеет ряд преимуществ:

\begin{enumerate}
	\item отсутствие потребности в памяти под стек
	\item отсутствие деградации при неудачных наборах данных — быстрая сортировка легко деградирует до O(n²), что хуже, чем худшее гарантированное время для сортировки Шелла
\end{enumerate}

\par
Среднее время работы алгоритма зависит от длин промежутков — d, на которых будут находиться сортируемые элементы исходного массива ёмкостью N на каждом шаге алгоритма. В своей работе я использовал   $d=N/2,d_{i}=d_{i-1}/2,d_{k}=1$
\newpage

% Схема распараллеливания
\section*{Схема распараллеливания}
\addcontentsline{toc}{section}{Схема распараллеливания}
Для распараллеливания алгоритма поразрядной сортировки необходимо разделить массив на равные подмассивы, количество которых равно количеству потоков. Если массив не делится на цело на число потоков, необходимо записать остаток от деления с один из подмассивов (в моем случае в последний).
 Каждый поток берет соответствующую часть массива и разупскает локальную поразрядную сортировку для своего подмассива.
\par Затем отсортированные подмассивы сливаются алгоритмом простого слияния. На каждом шаге мы берём меньший из двух первых элементов подмассивов и записываем его в результирующий массив. Счётчики номеров элементов результирующего массива и подмассива, из которого был взят элемент, увеличиваем на 1. Когда один из подмассивов закончился, мы добавляем все оставшиеся элементы второго подмассива в результирующий массив.
\newpage

% Описание программной реализации
\section*{Описание программной реализации}
\addcontentsline{toc}{section}{Описание программной реализации}
В программе имеются следующие функции:

\begin{itemize}
\item Алгоритм последовательной сортировки Шелла вызывается с помощью функции:
\end{itemize}
\begin{lstlisting}
	std::vector<double> shell_sort(const std::vector<double>& vec);
\end{lstlisting}
\begin{itemize}
\item Алгоритм слияния отсортированных массивов вызывается с помощью функции: 
\end{itemize}
\begin{lstlisting}
	std::vector<double> merge(const std::vector<double>& a, const std::vector<double>& b);
\end{lstlisting}
\begin{itemize}
\item Функция, реализующая OpenMP версию паралельной сортировки Шелла: 
\end{itemize}
\begin{lstlisting}
	std::vector<double> shell_sort_omp(const std::vector<double>& vec, int num_threads = 4);
\end{lstlisting}
\begin{itemize}
\item Функция, реализующая TBB версию паралельной сортировки Шелла: 
\end{itemize}
\begin{lstlisting}
	std::vector<double> shell_sort_tbb(const std::vector<double>& vec, int num_threads = 4);
\end{lstlisting}
\begin{itemize}
\item Функция, реализующая std::thread версию паралельной сортировки Шелла: 
\end{itemize}
\begin{lstlisting}
	std::vector<double> shell_sort_std(const std::vector<double> &vec, int num_threads = 4);
\end{lstlisting}
\newpage

% Описание экспериментов
\section*{Описание экспериментов}
\addcontentsline{toc}{section}{Описание экспериментов}
Эксперименты проводились на 4-х ядерном компьютере с операционной системой – Windows 10, который имеет 8 логических процессоров. Результаты экспериментов приведены в таблице:
\begin{table}[!h]
\caption{Результаты вычислительных экспериментов}
\centering
\begin{tabular}{llllll}
Размер & Количество потоков & OpenMp & TBB & Std & Послед. \\
1000000 & 4 & 0.58993 & 0.586713 & 0.57126 & 1.08169\\
1000000 & 2 & 0.6234 & 0.61983 & 0.63985 & 0.91023\\
\end{tabular}
\end{table}
\par
По результатам эксперимента, можно сделать вывод о том, что параллельнык версии работают быстрее последовательной. Самой быстрой оказалась TBB версия, на втором месте OpenMp версия, и затем идёт std::thread реализация.
\newpage

% Заключение
\section*{Заключение}
\addcontentsline{toc}{section}{Заключение}
В ходе выполнения работы была успешно реализован алгорим сортировки Шелла с простым слиянием.
\par 
Основной задачей лабораторной работы была реализация эффективной параллельной версии. Эта цель была успешно достигнута, что подтверждается результатами экспериментов, проведенных в ходе работы. 
\par 
Работоспособность программы была проверена с помощью библиотеки для модульного тестирования.
\newpage

% Список литературы
\begin{thebibliography}{1}
\addcontentsline{toc}{section}{Список литературы}
\bibitem{Gergel}
Гергель В. П., Стронгин Р. Г. Основы параллельных вычислений для многопроцессорных вычислительных систем. – 2003.
\bibitem{Shell}
Сортировка Шелла: \url {https://en.wikipedia.org/wiki/Shellsort }
\end{thebibliography}
\newpage

% Приложение
\section*{Приложение}
\addcontentsline{toc}{section}{Приложение}
\begin{lstlisting}
							// main.cpp
// Copyright 2021 Kochankov Ilya
#include <gtest/gtest.h>
#include <vector>
#include <algorithm>
#include "./shell_sort_simple.h"

TEST(getRandomVector, no_exceptions) {
    ASSERT_NO_THROW(getRandomVector(100));
}

TEST(getRandomVector, wrong_size) {
    ASSERT_ANY_THROW(getRandomVector(-100));
}

TEST(getRandomVector, works) {
    auto vect = getRandomVector(100);
    ASSERT_EQ(static_cast<int>(vect.size()), 100);
}

TEST(shell_sort, no_exceptions) {
    auto vect = getRandomVector(100);
    ASSERT_NO_THROW(shell_sort(vect));
}

TEST(shell_sort, size_one_no_exceptions) {
    auto vect = getRandomVector(1);
    ASSERT_NO_THROW(shell_sort(vect));
}

TEST(shell_sort, size_two_no_exceptions) {
    auto vect = getRandomVector(2);
    ASSERT_NO_THROW(shell_sort(vect));
}

TEST(shell_sort, works_size_two) {
    auto vect = std::vector<double>({2, 1});
    auto sorted = shell_sort(vect);
    ASSERT_EQ(sorted, std::vector<double>({1, 2}));
}

TEST(shell_sort, works_size_three) {
    auto vect = std::vector<double>({3, 2, 1});
    auto sorted = shell_sort(vect);
    ASSERT_EQ(sorted, std::vector<double>({1, 2, 3}));
}

TEST(shell_sort, works_size_five) {
    auto vect = std::vector<double>({3, 2, 1, 6, 7});
    auto sorted = shell_sort(vect);
    ASSERT_EQ(sorted, std::vector<double>({1, 2, 3, 6, 7}));
}

TEST(shell_sort, works_size_zero) {
    auto vect = std::vector<double>();
    auto sorted = shell_sort(vect);
    ASSERT_EQ(sorted, std::vector<double>());
}

TEST(shell_sort, works_random_size_100) {
    auto vect = getRandomVector(100);
    auto sorted = shell_sort(vect);
    std::sort(vect.begin(), vect.end());
    ASSERT_EQ(vect, sorted);
}

TEST(shell_sort, works_random_size_1000) {
    auto vect = getRandomVector(1000);
    auto sorted = shell_sort(vect);
    std::sort(vect.begin(), vect.end());
    ASSERT_EQ(vect, sorted);
}

TEST(shell_sort, works_random_size_10000) {
    auto vect = getRandomVector(10000);
    auto sorted = shell_sort(vect);
    std::sort(vect.begin(), vect.end());
    ASSERT_EQ(vect, sorted);
}

TEST(merge, no_exceptions) {
    auto vect_a = getRandomVector(100);
    auto vect_b = getRandomVector(100);
    ASSERT_NO_THROW(merge(vect_a, vect_b));
}

TEST(merge, no_exceptions_different_size) {
    auto vect_a = getRandomVector(10);
    auto vect_b = getRandomVector(100);
    ASSERT_NO_THROW(merge(vect_a, vect_b));
}

TEST(merge, no_exceptions_null_size) {
    auto vect_a = getRandomVector(0);
    auto vect_b = getRandomVector(0);
    ASSERT_NO_THROW(merge(vect_a, vect_b));
}

TEST(merge, works_size_2) {
    auto vect_a = std::vector<double>({6, 7});
    auto vect_b = std::vector<double>({1, 2});
    auto merged = merge(vect_a, vect_b);
    ASSERT_EQ(merged, std::vector<double>({1, 2, 6, 7}));
}

TEST(merge, works_size_5) {
    auto vect_a = std::vector<double>({6, 7, 8, 9, 10});
    auto vect_b = std::vector<double>({1, 2, 3, 4, 5});
    auto merged = merge(vect_a, vect_b);
    ASSERT_EQ(merged, std::vector<double>({1, 2, 3, 4, 5, 6, 7, 8, 9, 10}));
}

TEST(merge, works_random_size_1) {
    auto vect_a = getRandomVector(1);
    auto vect_b = getRandomVector(1);
    auto merged = merge(vect_a, vect_b);
    for (int i = 0; i < static_cast<int>(vect_a.size()) - 1; ++i) {
        ASSERT_TRUE(merged[i] <= merged[i+1]);
    }
}

TEST(merge, works_random_size_5) {
    auto vect_a = getRandomVector(5);
    auto vect_b = getRandomVector(5);
    vect_a = shell_sort(vect_a);
    vect_b = shell_sort(vect_b);
    auto merged = merge(vect_a, vect_b);
    for (int i = 0; i < static_cast<int>(vect_a.size()) - 1; ++i) {
        ASSERT_TRUE(merged[i] <= merged[i+1]);
    }
}

TEST(merge, works_random_size_100) {
    auto vect_a = getRandomVector(100);
    auto vect_b = getRandomVector(100);
    vect_a = shell_sort(vect_a);
    vect_b = shell_sort(vect_b);
    auto merged = merge(vect_a, vect_b);
    for (int i = 0; i < static_cast<int>(vect_a.size()) - 1; ++i) {
        ASSERT_TRUE(merged[i] <= merged[i+1]);
    }
}

TEST(merge, works_random_size_1000) {
    auto vect_a = getRandomVector(1000);
    auto vect_b = getRandomVector(1000);
    vect_a = shell_sort(vect_a);
    vect_b = shell_sort(vect_b);
    auto merged = merge(vect_a, vect_b);
    for (int i = 0; i < static_cast<int>(vect_a.size()) - 1; ++i) {
        ASSERT_TRUE(merged[i] <= merged[i+1]);
    }
}

int main(int argc, char **argv) {
    ::testing::InitGoogleTest(&argc, argv);
    return RUN_ALL_TESTS();
}

\end{lstlisting}
\begin{lstlisting}
							//shell_sort_simple.cpp
// Copyright 2021 Kochankov Ilya
#include <random>
#include <vector>
#include <algorithm>
#include "../../../modules/task_1/kochankov_i_shell_sort_simple/shell_sort_simple.h"


std::vector<double> getRandomVector(int sz) {
    std::random_device dev;
    std::mt19937 gen(dev());
    std::vector<double> vec(sz);
    for (int  i = 0; i < sz; i++) { vec[i] = gen() % 100; }
    return vec;
}

std::vector<double> shell_sort(const std::vector<double>& vec) {
    std::vector<double> copy(vec);
    for (int step = copy.size() / 2; step != 0; step /= 2) {
        for (int i = step; i < static_cast<int>(copy.size()); i++) {
            for (int j = i - step; j >= 0 && copy[j] > copy[j + step]; j -= step) {
                double x = copy[j];
                copy[j] = copy[j + step];
                copy[j + step] = x;
            }
        }
    }
    return copy;
}

std::vector<double> merge(const std::vector<double>& a, const std::vector<double>& b) {
    std::vector<double> result((a.size() + b.size()));

    int i = 0, j = 0, k = 0;
    while (i < static_cast<int>(a.size()) && j < static_cast<int>(b.size())) {
        if (a[i] < b[j])
            result[k] = a[i++];
        else
            result[k] = b[j++];
        k++;
    }
    while (i < static_cast<int>(a.size())) {
        result[k++] = a[i++];
    }
    while (j < static_cast<int>(b.size())) {
        result[k++] = b[j++];
    }

    return result;
}

\end{lstlisting}
\begin{lstlisting}
							// shell_sort_simple.h
// Copyright 2021 Kochankov Ilya
#ifndef MODULES_TASK_1_KOCHANKOV_I_SHELL_SORT_SIMPLE_SHELL_SORT_SIMPLE_H_
#define MODULES_TASK_1_KOCHANKOV_I_SHELL_SORT_SIMPLE_SHELL_SORT_SIMPLE_H_

#include <vector>


std::vector<double> getRandomVector(int sz);
std::vector<double> shell_sort(const std::vector<double>& vec);
std::vector<double> merge(const std::vector<double>& a, const std::vector<double>& b);


#endif  // MODULES_TASK_1_KOCHANKOV_I_SHELL_SORT_SIMPLE_SHELL_SORT_SIMPLE_H_

\end{lstlisting}
\begin{lstlisting}
							// shell_sort_simple_omp.cpp
// Copyright 2021 Kochankov Ilya
#include <omp.h>
#include <random>
#include <vector>
#include <algorithm>
#include <iostream>
#include "../../../modules/task_2/kochankov_i_shell_sort_simple_omp/shell_sort_simple_omp.h"


std::vector<double> getRandomVector(int sz) {
    std::random_device dev;
    std::mt19937 gen(dev());
    std::vector<double> vec(sz);
    for (int i = 0; i < sz; i++) { vec[i] = gen() % 100; }
    return vec;
}

std::vector<double> shell_sort(const std::vector<double>& vec) {
    std::vector<double> copy(vec);
    for (int step = copy.size() / 2; step != 0; step /= 2) {
        for (int i = step; i < static_cast<int>(copy.size()); i++) {
            for (int j = i - step; j >= 0 && copy[j] > copy[j + step]; j -= step) {
                double x = copy[j];
                copy[j] = copy[j + step];
                copy[j + step] = x;
            }
        }
    }
    return copy;
}

std::vector<double> merge(const std::vector<double>& a, const std::vector<double>& b) {
    std::vector<double> result((a.size() + b.size()));

    int i = 0, j = 0, k = 0;
    while (i < static_cast<int>(a.size()) && j < static_cast<int>(b.size())) {
        if (a[i] < b[j])
            result[k] = a[i++];
        else
            result[k] = b[j++];
        k++;
    }
    while (i < static_cast<int>(a.size())) {
        result[k++] = a[i++];
    }
    while (j < static_cast<int>(b.size())) {
        result[k++] = b[j++];
    }

    return result;
}


std::vector<double> shell_sort_omp(const std::vector<double>& vec, int num_threads) {
    if (static_cast<int>(vec.size()) < num_threads) {
        return shell_sort(vec);
    }
    std::vector<double> copy(vec);
    if (vec.size() == 1) {
        return copy;
    }
    omp_set_num_threads(num_threads);

    int delta = vec.size() / num_threads;
    std::vector<double> local_vec;

#pragma omp parallel private(local_vec) shared(copy, delta)
    {
        int thread_num = omp_get_thread_num();
        if (thread_num == num_threads - 1) {
            local_vec = std::vector<double>(copy.size() - delta * (num_threads - 1));
        } else {
            local_vec = std::vector<double>(delta);
        }

        if (thread_num == num_threads - 1) {
            std::copy(copy.begin() + delta * (num_threads - 1), copy.end(), local_vec.begin());
        } else {
            std::copy(copy.begin() + delta * thread_num,
                copy.begin() + delta * (thread_num + 1), local_vec.begin());
        }
        local_vec = shell_sort(local_vec);

#pragma omp barrier
        if (thread_num == 0) {
            copy = local_vec;
        }

#pragma omp barrier
#pragma omp critical
        if (thread_num != 0) {
            copy = merge(copy, local_vec);
        }
    }
    return copy;
}

\end{lstlisting}
\begin{lstlisting}
							// shell_sort_simple_omp.h

// Copyright 2021 Kochankov Ilya
#ifndef MODULES_TASK_2_KOCHANKOV_I_SHELL_SORT_SIMPLE_OMP_SHELL_SORT_SIMPLE_OMP_H_
#define MODULES_TASK_2_KOCHANKOV_I_SHELL_SORT_SIMPLE_OMP_SHELL_SORT_SIMPLE_OMP_H_

#include <vector>


std::vector<double> getRandomVector(int sz);
std::vector<double> shell_sort(const std::vector<double>& vec);
std::vector<double> shell_sort_omp(const std::vector<double>& vec, int num_threads = 4);
std::vector<double> merge(const std::vector<double>& a, const std::vector<double>& b);


#endif  // MODULES_TASK_2_KOCHANKOV_I_SHELL_SORT_SIMPLE_OMP_SHELL_SORT_SIMPLE_OMP_H_

\end{lstlisting}
\begin{lstlisting}
							// shell_sort_simple_tbb.cpp

// Copyright 2021 Kochankov Ilya
#include <tbb/tbb.h>
#include <omp.h>
#include <random>
#include <vector>
#include <algorithm>
#include <iostream>
#include "../../../modules/task_3/kochankov_i_shell_sort_simple_tbb/shell_sort_simple_tbb.h"


std::vector<double> getRandomVector(int sz) {
    std::random_device dev;
    std::mt19937 gen(dev());
    std::vector<double> vec(sz);
    for (int i = 0; i < sz; i++) { vec[i] = gen() % 100; }
    return vec;
}

std::vector<double> shell_sort(const std::vector<double>& vec) {
    std::vector<double> copy(vec);
    for (int step = copy.size() / 2; step != 0; step /= 2) {
        for (int i = step; i < static_cast<int>(copy.size()); i++) {
            for (int j = i - step; j >= 0 && copy[j] > copy[j + step]; j -= step) {
                double x = copy[j];
                copy[j] = copy[j + step];
                copy[j + step] = x;
            }
        }
    }
    return copy;
}

std::vector<double> merge(const std::vector<double>& a, const std::vector<double>& b) {
    std::vector<double> result((a.size() + b.size()));

    int i = 0, j = 0, k = 0;
    while (i < static_cast<int>(a.size()) && j < static_cast<int>(b.size())) {
        if (a[i] < b[j])
            result[k] = a[i++];
        else
            result[k] = b[j++];
        k++;
    }
    while (i < static_cast<int>(a.size())) {
        result[k++] = a[i++];
    }
    while (j < static_cast<int>(b.size())) {
        result[k++] = b[j++];
    }

    return result;
}


std::vector<double> shell_sort_tbb(const std::vector<double>& vec, int num_threads) {
    if (static_cast<int>(vec.size()) < num_threads) {
        return shell_sort(vec);
    }
    std::vector<double> copy(vec);
    if (vec.size() == 1) {
        return copy;
    }

    int delta = vec.size() / num_threads;

    std::vector<std::vector<double>> split_vec;
    std::vector<double> tmp;

    for (int thread_num = 0; thread_num < num_threads; thread_num++) {
        if (thread_num == num_threads - 1) {
            tmp.insert(tmp.end(), copy.begin() + delta * thread_num, copy.end());
        } else {
            tmp.insert(tmp.end(), copy.begin() + delta * thread_num,
                copy.begin() + delta * (thread_num + 1));
        }
        split_vec.push_back(tmp);
        tmp.clear();
    }

    tbb::task_scheduler_init init(num_threads);

    tbb::parallel_for(tbb::blocked_range<size_t>(0, split_vec.size(), 1),
    [&split_vec](const tbb::blocked_range<size_t>& range){
        for (size_t i = range.begin(); i != range.end(); ++i) {
            split_vec[i] = shell_sort(split_vec[i]);
        }
    }, tbb::simple_partitioner());

    std::vector<double> merged = split_vec[0];
    // std::copy(merged.begin(), merged.end(), std::ostream_iterator<double>(std::cout, " "));
    for (size_t i = 1; i < split_vec.size(); ++i) {
        merged = merge(merged, split_vec[i]);
    }

    return merged;
}

\end{lstlisting}
\begin{lstlisting}
							// shell_sort_simple_tbb.h

// Copyright 2021 Kochankov Ilya
#ifndef MODULES_TASK_3_KOCHANKOV_I_SHELL_SORT_SIMPLE_TBB_SHELL_SORT_SIMPLE_TBB_H_
#define MODULES_TASK_3_KOCHANKOV_I_SHELL_SORT_SIMPLE_TBB_SHELL_SORT_SIMPLE_TBB_H_

#include <vector>


std::vector<double> getRandomVector(int sz);
std::vector<double> shell_sort(const std::vector<double>& vec);
std::vector<double> shell_sort_tbb(const std::vector<double>& vec, int num_threads = 4);
std::vector<double> merge(const std::vector<double>& a, const std::vector<double>& b);


#endif  // MODULES_TASK_3_KOCHANKOV_I_SHELL_SORT_SIMPLE_TBB_SHELL_SORT_SIMPLE_TBB_H_

\end{lstlisting}
\begin{lstlisting}
							// shell_sort_simple_std.cpp

// Copyright 2021 Kochankov Ilya
#include <algorithm>
#include <random>
#include <vector>
#include "../../../modules/task_4/kochankov_i_shell_sort_simple_std/shell_sort_simple_std.h"
#include "../../../3rdparty/unapproved/unapproved.h"

std::vector<double> getRandomVector(int sz) {
  std::random_device dev;
  std::mt19937 gen(dev());
  std::vector<double> vec(sz);
  for (int i = 0; i < sz; i++) {
    vec[i] = gen() % 100;
  }
  return vec;
}

std::vector<double> shell_sort(const std::vector<double> &vec) {
  std::vector<double> copy(vec);
  for (int step = copy.size() / 2; step != 0; step /= 2) {
    for (int i = step; i < static_cast<int>(copy.size()); i++) {
      for (int j = i - step; j >= 0 && copy[j] > copy[j + step]; j -= step) {
        double x = copy[j];
        copy[j] = copy[j + step];
        copy[j + step] = x;
      }
    }
  }
  return copy;
}

std::vector<double> merge(const std::vector<double> &a,
                          const std::vector<double> &b) {
  std::vector<double> result((a.size() + b.size()));

  int i = 0, j = 0, k = 0;
  while (i < static_cast<int>(a.size()) && j < static_cast<int>(b.size())) {
    if (a[i] < b[j])
      result[k] = a[i++];
    else
      result[k] = b[j++];
    k++;
  }
  while (i < static_cast<int>(a.size())) {
    result[k++] = a[i++];
  }
  while (j < static_cast<int>(b.size())) {
    result[k++] = b[j++];
  }

  return result;
}

std::vector<double> shell_sort_std(const std::vector<double> &vec,
                                   int num_threads) {
  if (static_cast<int>(vec.size()) < num_threads) {
    return shell_sort(vec);
  }
  std::vector<double> copy(vec);
  if (vec.size() == 1) {
    return copy;
  }

  int delta = vec.size() / num_threads;

  std::vector<std::vector<double>> split_vec;
  std::vector<double> tmp;

  for (int thread_num = 0; thread_num < num_threads; thread_num++) {
    if (thread_num == num_threads - 1) {
      tmp.insert(tmp.end(), copy.begin() + delta * thread_num, copy.end());
    } else {
      tmp.insert(tmp.end(), copy.begin() + delta * thread_num,
                 copy.begin() + delta * (thread_num + 1));
    }
    split_vec.push_back(tmp);
    tmp.clear();
  }

  std::vector<std::thread> threads;
  for (size_t i = 0; i < 4; i++) {
    threads.emplace_back(
        [&split_vec, i](const std::vector<double> &vec) {
          split_vec[i] = shell_sort(vec);
        },
        split_vec[i]);
  }
  for (size_t i = 0; i < 4; i++) {
    threads[i].join();
  }

  std::vector<double> merged = split_vec[0];
  // std::copy(merged.begin(), merged.end(),
  // std::ostream_iterator<double>(std::cout, " "));
  for (size_t i = 1; i < split_vec.size(); ++i) {
    merged = merge(merged, split_vec[i]);
  }

  return merged;
}

\end{lstlisting}
\begin{lstlisting}
							// shell_sort_simple_std.h

// Copyright 2021 Kochankov Ilya
#ifndef MODULES_TASK_4_KOCHANKOV_I_SHELL_SORT_SIMPLE_STD_SHELL_SORT_SIMPLE_STD_H_
#define MODULES_TASK_4_KOCHANKOV_I_SHELL_SORT_SIMPLE_STD_SHELL_SORT_SIMPLE_STD_H_

#include <vector>

std::vector<double> getRandomVector(int sz);
std::vector<double> shell_sort(const std::vector<double> &vec);
std::vector<double> shell_sort_std(const std::vector<double> &vec,
                                   int num_threads = 4);
std::vector<double> merge(const std::vector<double> &a,
                          const std::vector<double> &b);

#endif  // MODULES_TASK_4_KOCHANKOV_I_SHELL_SORT_SIMPLE_STD_SHELL_SORT_SIMPLE_STD_H_

\end{lstlisting}
\end{document}