\documentclass{report}

\usepackage[T2A]{fontenc}
\usepackage[utf8]{luainputenc}
\usepackage[english, russian]{babel}
\usepackage[pdftex]{hyperref}
\usepackage[14pt]{extsizes}
\usepackage{listings}
\usepackage{color}
\usepackage{geometry}
\usepackage{enumitem}
\usepackage{multirow}
\usepackage{graphicx}
\usepackage{indentfirst}
\usepackage{amsmath}


\geometry{a4paper,top=2cm,bottom=3cm,left=2cm,right=1.5cm}
\setlength{\parskip}{0.5cm}
\setlist{nolistsep, itemsep=0.3cm,parsep=0pt}
\lstset{language=C++,
        basicstyle=\footnotesize,
        keywordstyle=\color{blue}\ttfamily,
        stringstyle=\color{green}\ttfamily,
        commentstyle=\color{green}\ttfamily,
        morecomment=[l][\color{red}]{\#}, 
        tabsize=2,
        breaklines=true,
        breakatwhitespace=true,
        title=\lstname,
}

\makeatletter
\renewcommand\@biblabel[1]{#1.\hfil}
\makeatother

\begin{document}

\begin{titlepage}

\begin{center}
Министерство науки и высшего образования Российской Федерации
\end{center}

\begin{center}
Федеральное государственное автономное образовательное учреждение высшего образования \\
Национальный исследовательский Нижегородский государственный университет им. Н.И. Лобачевского
\end{center}

\begin{center}
Институт информационных технологий, математики и механики
\end{center}

\vspace{4em}

\begin{center}
\textbf{\LargeОтчет по лабораторной работе} \\
\end{center}
\begin{center}
\textbf{\Large«Умножение разреженных матриц. Элементы типа double. Формат хранения матрицы - строковый (CRS).»} \\
\end{center}

\vspace{4em}

\newbox{\lbox}
\savebox{\lbox}{\hbox{text}}
\newlength{\maxl}
\setlength{\maxl}{\wd\lbox}
\hfill\parbox{7cm}{
\hspace*{5cm}\hspace*{-5cm}\textbf{Выполнил:} \\ студентка группы 381806-2 \\ Прокофьева Е.А.\\
\\
\hspace*{5cm}\hspace*{-5cm}\textbf{Проверил:}\\ доцент кафедры МОСТ, \\ кандидат технических наук \\ Сысоев А. В.\\
}
\vspace{\fill}

\begin{center} Нижний Новгород \\ 2021 \end{center}

\end{titlepage}

\setcounter{page}{2}
\tableofcontents
\newpage

\section* {Введение}
Матриц размера N*N называется разреженной, если количество её ненулевых элементов есть O(N). Для хранения такой матрицы важно учитывать факт её разреженности. Один из наиболее простых для понимания форматов хранения разреженных матриц - координатный формат. Данный формат обеспечивает медленный доступ к элементам матрицы, и является затратным по используемой памяти. Формат хранения CSR (Compressed Sparse Rows) или CRS (Compressed Row Storage) призван устранить недоработки координатного представления. Эта схема предъявляет минимальные требования к памяти и в то же время оказывается очень удобной для нескольких важных операций над разреженными матрицами: сложения, умножения, перестановок строк и столбцов, транспонирования, решения линейных систем с разреженными матрицами коэффициентов как прямыми, так и итерационными методами и т. д. Далее будет подробнее рассмотрен формат хранения CRS.
\addcontentsline{toc}{section}{Введение}
\newpage
\section* {Постановка задачи}
 В данной лабораторной работе необходимо: \newline\labelitemi\quad  создать структуру для хранения матрицы в формате CRS \newline\labelitemi\quad  реализовать алгоритм транспонирования матрицы \newline\labelitemi\quad реализовать последовательный алгоритм умножения матриц
\newline\labelitemi\quad реализовать параллельные версии алгоритма умножения с использованием OpenMP и TBB \newline\labelitemi\quad  сгенерировать случайную разреженную матрицу для корректной оценки эффективности параллельных версий алгоритма умножения \newline\labelitemi\quad  провести тестирование с использованием Google C++ Testing Framework для проверки корректности работы реализованных алгоритмов \newline\labelitemi\quad замерить время работы последовательного и параллельных версий алгоритма умножения для оценки эффективности распараллеливания
\addcontentsline{toc}{section}{Постановка задачи}
\newpage
\section* {Описание алгоритма}
Для хранения матрицы в формате CRS используется структура, содержащая поля: размер матрицы и три вектора. Первый вектор хранит значения элементов построчно по порядку сверху вниз, второй - номера столбцов для каждого элемента, а третий - индекс начала каждой строки.
\par
Тестовые разреженные матрицы создаются с помощью функции, которая из вектора формирует саму матрицу.
\par Генерация случайных матриц осуществляется функцией, которая создаёт матрицу с одним ненулевым элементом в строке.
\par Алгоритм транспонирования:
\newline\labelitemi\quad Сформировать одномерные вектора для хранения целых чисел и вектора для хранения вещественных чисел размером, соответствующим числу столбцов исходной матрицы. \newline\labelitemi\quad В цикле просмотреть все строки исходной матрицы, для каждой строки - все её элементы. Для текущего элемента добавить число, соответствующее номеру строки, и число, соответствующее номеру столбца, в вектор для хранения целых чисел и его значение в вектор для хранения вещественных чисел. Тем самым формируя в векторах строки транспонированной матрицы.
\newline\labelitemi\quad Последовательно скопировать данные из векторов в CRS-структуру транспонированной матрицы, формируя массив индекса начала строки.
\par Алгоритм умножения разреженных матриц А и В:
\newline\labelitemi\quad Создать результирующую матрицу C. \newline\labelitemi\quad Транспонировать матрицу В. \newline\labelitemi\quad В цикле по i от 0 до N-1 перебрать все строки матрицы А. Для каждого i в цикле по j от 0 до N-1 перебрать все строки транспонированной матрицы В. Вычислить скалярное произведение векторов-строк $A_i$ и $B_j$. Если значение скалярного произведения не равен нулю, добавить его в вектор для значений элементов матрицы C, а в вектор для номера столбцов каждого элемента матрицы С - элемент j. При окончании цикла по j, записать в вектор для индекса начала строки  матрицы C текущее значение числа ненулевых элементов.
\par Алгоритм скалярного произведения матрицы А и транспонированной матрицы В:
\newline Для каждого элемента строки матрицы А перебрать все элементы строки матрицы В, пока не будет найден элемент с таким же значением в векторе для номера столбцов каждого элемента или не кончится строка.
\addcontentsline{toc}{section}{Описание алгоритма}
\newpage
\section* {Схема распараллеливания}
Для того, чтобы получить наиболее эффективную реализацию параллельных версий алгоритма умножения, оптимизируем алгоритм скалярного умножения:
\newline\labelitemi\quad Создать дополнительный целочисленный вектор длины N. Инициализировать его числом -1. \newline\labelitemi\quad Просмотреть в цикле все ненулевые элементы первого вектора. Пусть такой элементы с номером i расположен в столбце j. В этом случае запишем i в j-ю ячейку дополнительного массива.\newline\labelitemi\quad Просмотреть в цикле все ненулевые элементы второго вектора. Пусть элемент с номером k расположен в столбце с номером z. Проверим z-е значение дополнительного массива, если оно равно -1, умножение выполнять не нужно. Иначе умножаем и накапливаем в сумму.\par
Идея распаралелливания алгоритма состоит в перебирании во внешнем цикле строк матрицы А и дальнейшей независимой работе с ними. Таким образом, произойдет распределение строк между потоками. 
\par В такой реализации есть некоторые проблемы. Когда необходимо записать новые значение в вектора структуры CRS-матрицы, необходимо выполнить операцию \verb|push_back()|, потоки могут помешать друг другу сделать это корректно. Разрешение этой проблемы - продублировать все структры по числу строк, собрать в них данные отдельно по каждой строке и в конце одним потоком их просуммировать. С дополнительном вектором для скалярного умножение это не работает, поэтому поместим его объявление в параллельную секцию. 
\addcontentsline{toc}{section}{Схема распараллеливания}
\newpage
\section* {Описание программной реализации}
Структура для хранения CRS-матрицы:
\begin{lstlisting}
struct crs_matrix {
    int N;
    std::vector<double> values;
    std::vector<int> cols;
    std::vector<int> row_index; };
\end{lstlisting}
N - размерность матрицы, вектор values хранит значения элементов, вектор cols хранит номера столбцов, вектор \verb|row_index| хранит индекс начала строки.
\par Функция для транспонирования матрицы:
\begin{lstlisting}
crs_matrix transpose(crs_matrix matrix);
\end{lstlisting}
На входе подается матрица, которую необходимо транспонировать.
\par Функция создания матрицы из вектора:
\begin{lstlisting}
crs_matrix create(int size, std::vector<double> matrix);
\end{lstlisting}
Первым аргументом передаётся размерность матрицы, вторым - вектор, из которого создаётся матрица. 
\par Функция умножения двух разреженных матриц:
\begin{lstlisting}
crs_matrix mult(crs_matrix first, crs_matrix second);
\end{lstlisting}
Входными параметрами являются две матрицы, которые необходимо перемножить. Выходным параметром является результирующая разреженная матрица.
\par Функция скалярного умножения:
\begin{lstlisting}
double scalar_mult(crs_matrix first, crs_matrix second, int i, int j);
\end{lstlisting}
Первые два параметра - матрицы, которые необходимо перемножить, вторые два - элементы строк этих матриц.
\par Функция генерации случайной разреженной матрицы:
\begin{lstlisting}
crs_matrix generate(int size);
\end{lstlisting}
Входным параметром является размерность матрицы, выходным - разреженная матрица с одним ненулевым элементом в строке.
\par Функция параллельного умножения разреженных матриц с использованием OpenMP:
\begin{lstlisting}
crs_matrix parallel_mult(crs_matrix first, crs_matrix second);
\end{lstlisting}
Входные параметры - две матрицы, к которым применяется алгоритм умножения. Выходной параметр - результирующая матрица.\newline Распараллеливание происходит с помощью встроенной директивы  \verb|#pragma omp parallel for|, которая отвечает за распараллеливание циклов.
\par Функция параллельного умножения разреженных матриц с использованием TBB:
\begin{lstlisting}
crs_matrix mult_tbb(crs_matrix first, crs_matrix second);
\end{lstlisting}
Входные параметры - две матрицы, к которым применяется алгоритм умножения. Выходной параметр - результирующая матрица.\newline  Для возможностей использования TBB в функции инициализируется экземпляр класса \\\verb|tbb::task_scheduler_init init()|. Распараллеливание происходит с помощью функции \verb|tbb::parallel_for()|, в которую передается интервал цикла и лямбда функция, в которую помещается внутреннее содержание цикла. 
\addcontentsline{toc}{section}{Описание программной реализации}
\newpage
\section* {Результаты экспериментов}
Для подтверждения корректности работы алгоритмов в программе реализован набор тестов, созданных с помощью использования Google C++ Testing Framework.
\par В качестве проверки алгоритмов, реализованных в последовательной версии, были созданы тесты, которые сравнивают реальные значения полей структуры CRS-матрицы с ожидаемыми. В параллельной версии сравниваются результаты последовательного алгоритма умножения разреженных матриц с параллельным.
\par Успешное прохождение всех тестов подтверждает корректность работы написанной программы.
\par Вычислительные эксперименты для оценки эффективности параллельного алгоритма умножения разреженных матриц проводились на оборудовании со следующей аппаратной конфигурацией:
\begin{itemize}
\item Процессор: AMD Ryzen 5 3500U, 2,10 GHz, ядер: 4, логических процессоров: 8;
\item Оперативная память: 8 ГБ (DDR4);
\item ОС: Microsoft Windows 10.
\end{itemize}
\begin{table}[!h]
\caption{Результаты вычислительных экспериментов.}
\begin{tabular}{ | l | l | l | l |}
\hline
Размер матрицы & Последовательный алгоритм & OpenMP & Ускорение \\ \hline
1000*1000 & 0.0114982 & 0.0033757 & 3.4 \\
2000*2000 & 0.0491158 & 0.0112372 & 4.37 \\
3000*3000 & 0.108181 & 0.0139448 & 7.757 \\
4000*4000 & 0.190988 & 0.0389736 & 4.9 \\
5000*5000 & 0.298922 & 0.0990666 & 3.017 \\
10000*10000 & 1.3475 & 0.218422 &  6.169 \\ \hline
\end{tabular}
\end{table}
\newpage
\begin{table}[!h]
\caption{Результаты вычислительных экспериментов.}
\begin{tabular}{ | l | l | l | l |}
\hline
Размер матрицы & Последовательный алгоритм & TBB & Ускорение \\ \hline
2000*3000 & 0,0289775 & 0,0172942 & 1,675 \\
3000*3000 & 0,0755932 & 0,0242689 &  3,114\\
4000*4000 & 0,137411 & 0,0439593 & 3,125\\ 
5000*5000 & 0,230505 & 0,106673 & 2,16\\ 
10000*10000 & 1,16008 & 0,217172 & 5,34\\ 
15000*15000 & 3,06412 & 0,484046 &  6,33 \\ \hline
 \end{tabular}
\end{table}
\par Исходя из результатов тестирования, которые представлены в таблице, можно сделать вывод о том, что OpenMP эффективнее, чем TBB. Максимальное ускорение, которое может быть достигнуто - 8, так как процессор поддерживает 8 потоков. OpenMP позволяет максимально к нему приблизиться. Также эффективность в значительной степени зависит от размера матрицы. В теории с ростом вычислительной нагрузки, то есть с увлечением размера матрицы, ускорение должно возрастать.
\addcontentsline{toc}{section}{Результаты экспериментов}
\newpage
\section* {Заключение}
В результате лабораторной работы была достигнута основная цель - реализовать и распараллелить умножение разреженных матриц, хранящихся в строковом формате. Тестирование с помощью Google C++ Testing Framework показало правильность и эффективность реализованных алгоритмов.
\addcontentsline{toc}{section}{Заключение}
\newpage
\addcontentsline{toc}{section}{Список литературы}
\begin{thebibliography}{1}
		\bibitem{1} Гергель В.П., Стронгин Р.Г. Основы параллельных вычислений для многопроцессорных вычислительных систем. Учебное пособие – Нижний Новгород: Изд-во ННГУ им. Н.И. Лобачевского, 2003. 184 с. ISBN 5-85746-602-4.

		\bibitem{2}Баркалов К.А. Параллельные численные методы. Решение разреженных СЛАУ. Н. Новгород: Изд-во Нижегородского госуниверситета им. Н.И. Лобачевского, 2011
		\bibitem{3}Х. Д. Икрамов, “Разреженные матрицы”, Итоги науки и техн. Сер. Мат. анал., 20, ВИНИТИ, М., 1982, 179–260; J. Soviet Math., 34:3 (1986), 1697–1749

	\end{thebibliography}
\newpage
\section*{Приложение}
Последовательная версия:
\newline Файл \verb|sparse_crs.h|
\begin{lstlisting}
// Copyright 2021 Prokofeva Elizaveta
#ifndef MODULES_TASK_1_PROKOFEVA_E_SPARSE_CRS_SPARSE_CRS_H_
#define MODULES_TASK_1_PROKOFEVA_E_SPARSE_CRS_SPARSE_CRS_H_

#include<vector>

struct crs_matrix {
    int N;
    std::vector<double> values;
    std::vector<int> cols;
    std::vector<int> row_index; };

crs_matrix transpose(crs_matrix matrix);
crs_matrix create(int size, std::vector<double> matrix);
crs_matrix mult(crs_matrix first, crs_matrix second);
double scalar_mult(crs_matrix first, crs_matrix second, int i, int j);

#endif  // MODULES_TASK_1_PROKOFEVA_E_SPARSE_CRS_SPARSE_CRS_H_
\end{lstlisting}
 
\begin{lstlisting}
// Copyright 2021 Prokofeva Elizaveta
#include "../../../modules/task_1/prokofeva_e_sparse_crs/sparse_crs.h"
#include <random>
#include <cmath>
#include <vector>
#include <algorithm>
#include <iostream>

crs_matrix create(int size, std::vector<double> matrix) {
    crs_matrix result;
    result.N = size;
    int tmp = 0;

    result.row_index.push_back(tmp);
    for (int i = 0; i < size; i++) {
        int count = 0;
        for (int j = 0; j < size; j++) {
            if (matrix[result.N * i + j] != 0) {
                count++;
                result.cols.push_back(j);
                result.values.push_back(matrix[result.N * i + j]);
            }
        }
        count += result.row_index.back();
        result.row_index.push_back(result.cols.size());
    }

    return result;
}

crs_matrix transpose(crs_matrix matrix) {
    crs_matrix Tmatrix;
    int N = Tmatrix.N = matrix.N;
    Tmatrix.row_index = std::vector<int>(static_cast<size_t>(N + 1));
    Tmatrix.values = std::vector<double>(static_cast<size_t>(matrix.values.size()));
    Tmatrix.cols = std::vector<int>(static_cast<size_t>(matrix.cols.size()));
    size_t val = matrix.values.size();
    for (size_t i = 0; i < val; i++) {
        int col = matrix.cols[i];
        Tmatrix.row_index[col + 1]++;
    }

    double s = 0;
    for (int i = 1; i <= N; i++) {
        int temp = Tmatrix.row_index[i];
        Tmatrix.row_index[i] = s;
        s += temp;
    }

    for (int i = 0; i < N; i++) {
        for (int j = matrix.row_index[i]; j < matrix.row_index[i + 1]; j++) {
            Tmatrix.values[Tmatrix.row_index[matrix.cols[j] + 1]] = matrix.values[j];
            Tmatrix.cols[Tmatrix.row_index[matrix.cols[j] + 1]] = i;
            Tmatrix.row_index[matrix.cols[j] + 1]++;
        }
    }
    return Tmatrix;
}

double scalar_mult(crs_matrix first, crs_matrix second, int i, int j) {
    double summ = 0.0;
    int a = first.row_index[i];
    int a1 = first.row_index[i + 1];
    int b = second.row_index[j];
    int b1 = second.row_index[j + 1];
    for (int k = a; k < a1; k++) {
        for (int l = b; l < b1; l++) {
            if (first.cols[k] == second.cols[l]) {
                summ = summ + first.values[k] * second.values[l];
                break;
            }
        }
    }
    return summ;
}

crs_matrix mult(crs_matrix first, crs_matrix second) {
    int N = first.N;
    crs_matrix result;
    second = transpose(second);
    double summ;
    int rowNZ = 0;
    int tmp = 0;

    result.row_index.emplace_back(tmp);
    for (int i = 0; i < N; i++) {
        for (int j = 0; j < N; j++) {
            summ = scalar_mult(first, second, i, j);
            if (summ != 0) {
                rowNZ++;
                result.cols.emplace_back(j);
                result.values.emplace_back(summ);
            }
        }
        result.row_index.emplace_back(result.row_index[i] + rowNZ);
    }
    return result;
}

\end{lstlisting}
\begin{lstlisting}
// Copyright 2021 Prokofeva Elizaveta
#include <gtest/gtest.h>
#include <vector>
#include <algorithm>
#include "../../../modules/task_1/prokofeva_e_sparse_crs/sparse_crs.h"


TEST(Sparse_crs_matrix_1, Create) {
    std::vector<double> f = { 0, 3, 0, 7,
                              0, 0, 8, 0,
                              0, 0, 0, 0,
                              9, 0, 15, 16};
    crs_matrix first = create(4, f);
    std::vector<double> value = { 3, 7, 8, 9, 15, 16 };
    std::vector<int> col = { 1, 3, 2, 0, 2, 3 };
    std::vector<int> rowIndex = { 0, 2, 3, 3, 6 };
    ASSERT_EQ(value, first.values);
    ASSERT_EQ(col, first.cols);
    ASSERT_EQ(rowIndex, first.row_index);
}
TEST(Sparse_crs_matrix_2, Transposition) {
    std::vector<double> f = { 0, 3, 0, 7,
                              0, 0, 8, 0,
                              0, 0, 0, 0,
                              9, 0, 15, 16 };
    crs_matrix first = create(4, f);
    first = transpose(first);
    std::vector<double> value = { 9, 3, 8, 15, 7, 16 };
    std::vector<int> col = { 3, 0, 1, 3, 0, 3 };
    std::vector<int> rowIndex = { 0, 1, 2, 4, 6 };
    ASSERT_EQ(value, first.values);
    ASSERT_EQ(col, first.cols);
    ASSERT_EQ(rowIndex, first.row_index);
}
TEST(Sparse_crs_matrix_3, Multiplication_3x3) {
    std::vector<double> f = { 0, 2, 0,
                             10, 0, 0,
                              0, 0, 6};
    crs_matrix first = create(3, f);
    std::vector<double> s = { 1, 0, 0,
                              0, 5, 0,
                              0, 14, 3};
    crs_matrix second = create(3, s);
    crs_matrix result = mult(first, second);
    std::vector<double> r = { 10, 10, 84, 18 };
    ASSERT_EQ(r, result.values);
}
TEST(Sparse_crs_matrix_4, Multiplication_3x3) {
    std::vector<double> f = { 0, 0.5, 1.5,
                              2, 0, 0,
                              0, 6.5, 0};
    crs_matrix first = create(3, f);
    std::vector<double> s = { 3.5, 0, 0,
                               0, 1, 4.5,
                               0, 0, 0};
    crs_matrix second = create(3, s);
    crs_matrix result = mult(first, second);
    std::vector<double> r = { 0.5, 2.25, 7, 6.5, 29.25 };
    ASSERT_EQ(r, result.values);
}
TEST(Sparse_crs_matrix_5, Multiplication_4x4) {
    std::vector<double> f = { 0, 11, 0, 25,
                              96, 0, 0, 0,
                              0, 0, 15, 0,
                              0, 44, 0, 0};
    crs_matrix first = create(4, f);
    std::vector<double> s = { 57, 63, 0, 0,
                              0, 0, 41, 0,
                              0, 78, 0, 0,
                              0, 0, 0, 22};
    crs_matrix second = create(4, s);
    crs_matrix result = mult(first, second);
    std::vector<double> r = { 451, 550, 5472, 6048, 1170, 1804 };
    ASSERT_EQ(r, result.values);
}
TEST(Sparse_crs_matrix_6, Multiplication_zero) {
    std::vector<double> f = { 0, 1, 0,
                              1, 0, 0,
                              0, 0, 1 };
    crs_matrix first = create(3, f);
    std::vector<double> s = { 0, 0, 0,
                              0, 0, 0,
                              0, 0, 0 };
    crs_matrix second = create(3, s);
    crs_matrix result = mult(first, second);
    ASSERT_EQ(result.values.empty(), true);
}

int main(int argc, char** argv) {
    ::testing::InitGoogleTest(&argc, argv);
    return RUN_ALL_TESTS();
}

\end{lstlisting}
Параллельная версия с использованием OpenMP:
\newline Файл \verb|main.cpp|
\begin{lstlisting}
// Copyright 2021 Prokofeva Elizaveta
#include <gtest/gtest.h>
#include <omp.h>
#include <vector>
#include <algorithm>
#include <iostream>
#include "../../../modules/task_2/prokofeva_e_sparse_crs_omp/sparse_crs.h"

TEST(Sparse_crs_matrix, Test1) {
    crs_matrix first = generate(1000);
    crs_matrix second = generate(1000);
    double t1 = omp_get_wtime();
    crs_matrix res = mult(first, second);
    double t2 = omp_get_wtime();
    std::cout << t2 - t1 << std::endl;
    t1 = omp_get_wtime();
    crs_matrix resp = parallel_mult(first, second);
    t2 = omp_get_wtime();
     std::cout << t2 - t1 << std::endl;
    ASSERT_EQ(resp.values, res.values);
    ASSERT_EQ(resp.cols, res.cols);
    ASSERT_EQ(resp.row_index, res.row_index);
}
TEST(Sparse_crs_matrix, Test2) {
    crs_matrix first = generate(2000);
    crs_matrix second = generate(2000);
    double t1 = omp_get_wtime();
    crs_matrix res = mult(first, second);
    double t2 = omp_get_wtime();
    std::cout << t2 - t1 << std::endl;
    t1 = omp_get_wtime();
    crs_matrix resp = parallel_mult(first, second);
    t2 = omp_get_wtime();
     std::cout << t2 - t1 << std::endl;
    ASSERT_EQ(resp.values, res.values);
    ASSERT_EQ(resp.cols, res.cols);
    ASSERT_EQ(resp.row_index, res.row_index);
}
TEST(Sparse_crs_matrix, Test3) {
    crs_matrix first = generate(3000);
    crs_matrix second = generate(3000);
   double t1 = omp_get_wtime();
    crs_matrix res = mult(first, second);
    double t2 = omp_get_wtime();
    std::cout << t2 - t1 << std::endl;
    t1 = omp_get_wtime();
    crs_matrix resp = parallel_mult(first, second);
    t2 = omp_get_wtime();
     std::cout << t2 - t1 << std::endl;
    ASSERT_EQ(resp.values, res.values);
    ASSERT_EQ(resp.cols, res.cols);
    ASSERT_EQ(resp.row_index, res.row_index);
}
TEST(Sparse_crs_matrix, Test4) {
    crs_matrix first = generate(4000);
    crs_matrix second = generate(4000);
   double t1 = omp_get_wtime();
    crs_matrix res = mult(first, second);
    double t2 = omp_get_wtime();
    std::cout << t2 - t1 << std::endl;
    t1 = omp_get_wtime();
    crs_matrix resp = parallel_mult(first, second);
    t2 = omp_get_wtime();
     std::cout << t2 - t1 << std::endl;
    ASSERT_EQ(resp.values, res.values);
    ASSERT_EQ(resp.cols, res.cols);
    ASSERT_EQ(resp.row_index, res.row_index);
}
TEST(Sparse_crs_matrix, Test5) {
    crs_matrix first = generate(5000);
    crs_matrix second = generate(5000);
   double t1 = omp_get_wtime();
    crs_matrix res = mult(first, second);
    double t2 = omp_get_wtime();
    std::cout << t2 - t1 << std::endl;
    t1 = omp_get_wtime();
    crs_matrix resp = parallel_mult(first, second);
    t2 = omp_get_wtime();
     std::cout << t2 - t1 << std::endl;
    ASSERT_EQ(resp.values, res.values);
    ASSERT_EQ(resp.cols, res.cols);
    ASSERT_EQ(resp.row_index, res.row_index);
}
TEST(Sparse_crs_matrix, Test5) {
    crs_matrix first = generate(10000);
    crs_matrix second = generate(10000);
   double t1 = omp_get_wtime();
    crs_matrix res = mult(first, second);
    double t2 = omp_get_wtime();
    std::cout << t2 - t1 << std::endl;
    t1 = omp_get_wtime();
    crs_matrix resp = parallel_mult(first, second);
    t2 = omp_get_wtime();
     std::cout << t2 - t1 << std::endl;
    ASSERT_EQ(resp.values, res.values);
    ASSERT_EQ(resp.cols, res.cols);
    ASSERT_EQ(resp.row_index, res.row_index);
}

int main(int argc, char** argv) {
    ::testing::InitGoogleTest(&argc, argv);
    return RUN_ALL_TESTS();
}
\end{lstlisting}
\begin{lstlisting}
// Copyright 2021 Prokofeva Elizaveta
#include "../../../modules/task_2/prokofeva_e_sparse_crs_omp/sparse_crs.h"
#include <omp.h>
#include <random>
#include <ctime>
#include <vector>
#include <algorithm>
#include <iostream>

crs_matrix generate(int size) {
    crs_matrix result;
    result.N = size;
    std::mt19937 gen;
    gen.seed(static_cast<unsigned int>(time(0)));
    std::uniform_real_distribution<double> dis(-100, 100);
    result.row_index.emplace_back(0);
    for (int i = 0; i < size; i++) {
        result.values.emplace_back(dis(gen));
        result.cols.emplace_back(gen() % size);
        result.row_index.emplace_back(static_cast<int>(result.values.size()));
    }
    return result;
}

crs_matrix create(int size, std::vector<double> matrix) {
    crs_matrix result;
    result.N = size;
    int tmp = 0;

    result.row_index.push_back(tmp);
    for (int i = 0; i < size; i++) {
        int count = 0;
        for (int j = 0; j < size; j++) {
            if (matrix[result.N * i + j] != 0) {
                count++;
                result.cols.push_back(j);
                result.values.push_back(matrix[result.N * i + j]);
            }
        }
        count += result.row_index.back();
        result.row_index.push_back(result.cols.size());
    }
    return result;
}

crs_matrix transpose(crs_matrix matrix) {
    crs_matrix Tmatrix;
    int N = Tmatrix.N = matrix.N;
    Tmatrix.row_index = std::vector<int>(static_cast<size_t>(N + 1));
    Tmatrix.values = std::vector<double>(static_cast<size_t>(matrix.values.size()));
    Tmatrix.cols = std::vector<int>(static_cast<size_t>(matrix.cols.size()));
    size_t val = matrix.values.size();
    for (size_t i = 0; i < val; i++) {
        int col = matrix.cols[i];
        Tmatrix.row_index[col + 1]++;
    }

    double s = 0;
    for (int i = 1; i <= N; i++) {
        int temp = Tmatrix.row_index[i];
        Tmatrix.row_index[i] = s;
        s += temp;
    }

    for (int i = 0; i < N; i++) {
        for (int j = matrix.row_index[i]; j < matrix.row_index[i + 1]; j++) {
            Tmatrix.values[Tmatrix.row_index[matrix.cols[j] + 1]] = matrix.values[j];
            Tmatrix.cols[Tmatrix.row_index[matrix.cols[j] + 1]] = i;
            Tmatrix.row_index[matrix.cols[j] + 1]++;
        }
    }
    return Tmatrix;
}

double scalar_mult(crs_matrix first, crs_matrix second, int i, int j) {
    double summ = 0.0;
    int a = first.row_index[i];
    int a1 = first.row_index[i + 1];
    int b = second.row_index[j];
    int b1 = second.row_index[j + 1];
    for (int k = a; k < a1; k++) {
        for (int l = b; l < b1; l++) {
            if (first.cols[k] == second.cols[l]) {
                int v1 = first.values[k];
                int v2 = second.values[l];
                summ = summ + v1 * v2;
                break;
            }
        }
    }
    return summ;
}

crs_matrix mult(crs_matrix first, crs_matrix second) {
    crs_matrix result;
    second = transpose(second);
    double summ;

    result.row_index.push_back(0);
    for (int i = 0; i < first.N; i++) {
        for (int j = 0; j < second.N; j++) {
            summ = scalar_mult(first, second, i, j);
            if (summ != 0) {
                result.values.push_back(summ);
                result.cols.push_back(j);
            }
        }
        result.row_index.push_back(static_cast<int>(result.values.size()));
    }
    return result;
}

crs_matrix parallel_mult(crs_matrix first, crs_matrix second) {
    crs_matrix res;
    int N = res.N = first.N;
    std::vector<std::vector<double>> val(N);
    std::vector<std::vector<int>> col(N);
    std::vector<int> rind;
    second = transpose(second);
    res.row_index.emplace_back(0);
    int n, j;
#pragma omp parallel
    {
        std::vector<int> tmp(N);
        #pragma omp for private(j, n) schedule(static)
        for (int i= 0; i < N; i++) {
            tmp.assign(N, -1);

            for (j = first.row_index[i]; j < first.row_index[i + 1]; j++) {
                int coln = first.cols[j];
                tmp[coln] = j;
            }

            for (j = 0; j < N; j++) {
                double sum = 0.0;
                int a = second.row_index[j];
                int b = second.row_index[j + 1];
                for (n = a; n < b; n++) {
                    int bcol = second.cols[n];
                    int ind = tmp[bcol];
                    if (ind != -1) {
                        int v1 = first.values[ind];
                        int v2 = second.values[n];
                        sum = sum + v1 * v2;
                    }
                }
                if (sum != 0) {
                    val[i].emplace_back(sum);
                    col[i].emplace_back(j);
                }
            }
        }
    }
    for (int i = 0; i < N; ++i) {
        int size = static_cast<int>(col[i].size());
        res.row_index.push_back(res.row_index.back() + size);
        res.values.insert(res.values.end(), val[i].begin(), val[i].end());
        res.cols.insert(res.cols.end(), col[i].begin(), col[i].end());
    }
    return res;
}
\end{lstlisting}

\begin{lstlisting}
// Copyright 2021 Prokofeva Elizaveta
#ifndef MODULES_TASK_2_PROKOFEVA_E_SPARSE_CRS_OMP_SPARSE_CRS_H_
#define MODULES_TASK_2_PROKOFEVA_E_SPARSE_CRS_OMP_SPARSE_CRS_H_

#include<vector>

struct crs_matrix {
    int N;
    std::vector<double> values;
    std::vector<int> cols;
    std::vector<int> row_index;
};

crs_matrix transpose(crs_matrix matrix);
crs_matrix create(int size, std::vector<double> matrix);
crs_matrix mult(crs_matrix first, crs_matrix second);
double scalar_mult(crs_matrix first, crs_matrix second, int i, int j);
crs_matrix generate(int size);
crs_matrix parallel_mult(crs_matrix first, crs_matrix second);

#endif  // MODULES_TASK_2_PROKOFEVA_E_SPARSE_CRS_OMP_SPARSE_CRS_H_
\end{lstlisting}
 Параллельная версия с использованием TBB:
\newline Файл \verb|sparse_crs.h|
\begin{lstlisting}
// Copyright 2021 Prokofeva Elizaveta
#ifndef MODULES_TASK_3_PROKOFEVA_E_SPARSE_CRS_TBB_SPARSE_CRS_H_
#define MODULES_TASK_3_PROKOFEVA_E_SPARSE_CRS_TBB_SPARSE_CRS_H_

#include<vector>
using std::vector;

struct crs_matrix {
    int N;
    std::vector<double> values;
    std::vector<int> cols;
    std::vector<int> row_index;
};
crs_matrix transpose(crs_matrix matrix);
crs_matrix mult(crs_matrix first, crs_matrix second);
crs_matrix generate(int size);
crs_matrix mult_tbb(crs_matrix first, crs_matrix second);

#endif  // MODULES_TASK_3_PROKOFEVA_E_SPARSE_CRS_TBB_SPARSE_CRS_H_
\end{lstlisting}

\begin{lstlisting}
// Copyright 2021 Prokofeva Elizaveta
#include "../../../modules/task_3/prokofeva_e_sparse_crs_tbb/sparse_crs.h"
#include <tbb/tbb.h>
#include <random>
#include <ctime>
#include <vector>
#include <algorithm>
#include <iostream>

crs_matrix generate(int size) {
    crs_matrix result;
    result.N = size;
    std::mt19937 gen;
    gen.seed(static_cast<unsigned int>(time(0)));
    std::uniform_real_distribution<double> dis(-100, 100);
    result.row_index.emplace_back(0);
    for (int i = 0; i < size; i++) {
        result.values.emplace_back(dis(gen));
        result.cols.emplace_back(gen() % size);
        int size = static_cast<int>(result.values.size());
        result.row_index.emplace_back(size);
    }
    return result;
}

crs_matrix transpose(crs_matrix matrix) {
    crs_matrix Tmatrix;
    int N = Tmatrix.N = matrix.N;
    Tmatrix.row_index = std::vector<int>(static_cast<size_t>(N + 1));
    Tmatrix.values = std::vector<double>(static_cast<size_t>(matrix.values.size()));
    Tmatrix.cols = std::vector<int>(static_cast<size_t>(matrix.cols.size()));

    for (size_t i = 0; i < matrix.values.size(); i++) {
        Tmatrix.row_index[matrix.cols[i] + 1]++;
    }

    double s = 0;
    for (int i = 1; i <= N; i++) {
        int temp = Tmatrix.row_index[i];
        Tmatrix.row_index[i] = s;
        s += temp;
    }

    for (int i = 0; i < N; i++) {
        int j1 = matrix.row_index[i];
        int j2 = matrix.row_index[i + 1];
        int col = i;
        for (int j = j1; j < j2; j++) {
            double V = matrix.values[j];
            int r_index = matrix.cols[j];
            int i_index = Tmatrix.row_index[r_index + 1];
            Tmatrix.values[i_index] = V;
            Tmatrix.cols[i_index] = col;
            Tmatrix.row_index[r_index + 1]++;
        }
    }
    return Tmatrix;
}

crs_matrix mult(crs_matrix first, crs_matrix second) {
    crs_matrix res;
    second = transpose(second);
    int N = first.N;

    std::vector<int> tmp(N);
    int j, n;

    res.row_index.push_back(0);
    for (int i = 0; i < N; i++) {
        tmp.assign(N, -1);

        for (j = first.row_index[i]; j < first.row_index[i + 1]; j++) {
            int coln = first.cols[j];
            tmp[coln] = j;
        }

        for (j = 0; j < N; j++) {
            double sum = 0;
            for (n = second.row_index[j]; n < second.row_index[j + 1]; n++) {
                int bcol = second.cols[n];
                int ind = tmp[bcol];
                if (ind != -1) {
                    sum += first.values[ind] * second.values[n];
                }
            }

            if (sum != 0) {
                res.values.emplace_back(sum);
                res.cols.emplace_back(j);
            }

        }
        res.row_index.push_back(static_cast<int>(res.values.size()));
    }
    return res;
}

crs_matrix mult_tbb(crs_matrix first, crs_matrix second) {
    crs_matrix res;
    int N = res.N = first.N;
    std::vector<std::vector<double>> val(N);
    std::vector<std::vector<int>> col(N);
    second = transpose(second);
    res.row_index.emplace_back(0);
    tbb::task_scheduler_init init();
    int grainsize = 10;
    tbb::parallel_for(tbb::blocked_range<int>(0, first.N, 10),
        [&](const tbb::blocked_range<int>& r) {
            std::vector<int> tmp(N);

            for (int row1 = r.begin(); row1 < r.end(); row1++) {
                tmp.assign(N, -1);
                std::vector<double> temp_sum;
                std::vector<int> temp_row;

                for (int j = first.row_index[row1]; j < first.row_index[row1 + 1]; j++) {
                    int coln = first.cols[j];
                    tmp[coln] = j;
                }

                for (int i = 0; i < N; i++) {
                    double sum = 0.0;
                    for (int n = second.row_index[i]; n < second.row_index[i + 1]; n++) {
                        int bcol = second.cols[n];
                        int ind = tmp[bcol];
                        if (ind != -1) {
                            sum += first.values[ind] * second.values[n];
                        }
                    }
                    if (sum != 0) {
                        val[row1].emplace_back(sum);
                        col[row1].emplace_back(i);
                    }
                }

            }
        });
    for (int i = 0; i < N; ++i) {
        int size = static_cast<int>(col[i].size());
        res.row_index.push_back(res.row_index.back() + size);
        res.values.insert(res.values.end(), val[i].begin(), val[i].end());
        res.cols.insert(res.cols.end(), col[i].begin(), col[i].end());
    }
    return res;
}

\end{lstlisting}
\begin{lstlisting}
// Copyright 2021 Prokofeva Elizaveta
#include <gtest/gtest.h>
#include <tbb/tbb.h>
#include <vector>
#include <algorithm>
#include <iostream>
#include "../../../modules/task_3/prokofeva_e_sparse_crs_tbb/sparse_crs.h"

TEST(Sparse_crs_matrix, Test2000x2000) {
    crs_matrix first = generate(2000);
    crs_matrix second = generate(2000);
    auto t1 = tbb::tick_count::now();
    crs_matrix res = mult(first, second);
    auto t2 = tbb::tick_count::now();
    std::cout << "Sequential time = " << (t2 - t1).seconds() << "\n";
    t1 = tbb::tick_count::now();
    crs_matrix resp = mult_tbb(first, second);
    t2 = tbb::tick_count::now();
    std::cout << "TBB time = " << (t2 - t1).seconds() << "\n";
    ASSERT_EQ(resp.values, res.values);
    ASSERT_EQ(resp.cols, res.cols);
    ASSERT_EQ(resp.row_index, res.row_index);
}
TEST(Sparse_crs_matrix, Test3000x3000) {
    crs_matrix first = generate(3000);
    crs_matrix second = generate(3000);
    auto t1 = tbb::tick_count::now();
    crs_matrix res = mult(first, second);
    auto t2 = tbb::tick_count::now();
    std::cout << "Sequential time = " << (t2 - t1).seconds() << "\n";
    t1 = tbb::tick_count::now();
    crs_matrix resp = mult_tbb(first, second);
    t2 = tbb::tick_count::now();
    std::cout << "TBB time = " << (t2 - t1).seconds() << "\n";
    ASSERT_EQ(resp.values, res.values);
    ASSERT_EQ(resp.cols, res.cols);
    ASSERT_EQ(resp.row_index, res.row_index);
}
TEST(Sparse_crs_matrix, Test4000x4000) {
    crs_matrix first = generate(4000);
    crs_matrix second = generate(4000);
    auto t1 = tbb::tick_count::now();
    crs_matrix res = mult(first, second);
    auto t2 = tbb::tick_count::now();
    std::cout << "Sequential time = " << (t2 - t1).seconds() << "\n";
    t1 = tbb::tick_count::now();
    crs_matrix resp = mult_tbb(first, second);
    t2 = tbb::tick_count::now();
    std::cout << "TBB time = " << (t2 - t1).seconds() << "\n";
    ASSERT_EQ(resp.values, res.values);
    ASSERT_EQ(resp.cols, res.cols);
    ASSERT_EQ(resp.row_index, res.row_index);
}
TEST(Sparse_crs_matrix, Test5000x5000) {
    crs_matrix first = generate(5000);
    crs_matrix second = generate(5000);
    auto t1 = tbb::tick_count::now();
    crs_matrix res = mult(first, second);
    auto t2 = tbb::tick_count::now();
    std::cout << "Sequential time = " << (t2 - t1).seconds() << "\n";
    t1 = tbb::tick_count::now();
    crs_matrix resp = mult_tbb(first, second);
    t2 = tbb::tick_count::now();
    std::cout << "TBB time = " << (t2 - t1).seconds() << "\n";
    ASSERT_EQ(resp.values, res.values);
    ASSERT_EQ(resp.cols, res.cols);
    ASSERT_EQ(resp.row_index, res.row_index);
}
TEST(Sparse_crs_matrix, Test10000x10000) {
    crs_matrix first = generate(10000);
    crs_matrix second = generate(10000);
    auto t1 = tbb::tick_count::now();
    crs_matrix res = mult(first, second);
    auto t2 = tbb::tick_count::now();
    std::cout << "Sequential time = " << (t2 - t1).seconds() << "\n";
    t1 = tbb::tick_count::now();
    crs_matrix resp = mult_tbb(first, second);
    t2 = tbb::tick_count::now();
    std::cout << "TBB time = " << (t2 - t1).seconds() << "\n";
    ASSERT_EQ(resp.values, res.values);
    ASSERT_EQ(resp.cols, res.cols);
    ASSERT_EQ(resp.row_index, res.row_index);
}
TEST(Sparse_crs_matrix, Test15000x15000) {
    crs_matrix first = generate(15000);
    crs_matrix second = generate(15000);
    auto t1 = tbb::tick_count::now();
    crs_matrix res = mult(first, second);
    auto t2 = tbb::tick_count::now();
    std::cout << "Sequential time = " << (t2 - t1).seconds() << "\n";
    t1 = tbb::tick_count::now();
    crs_matrix resp = mult_tbb(first, second);
    t2 = tbb::tick_count::now();
    std::cout << "TBB time = " << (t2 - t1).seconds() << "\n";
    ASSERT_EQ(resp.values, res.values);
    ASSERT_EQ(resp.cols, res.cols);
    ASSERT_EQ(resp.row_index, res.row_index);
}

int main(int argc, char** argv) {
    ::testing::InitGoogleTest(&argc, argv);
    return RUN_ALL_TESTS();
}
\end{lstlisting}
\addcontentsline{toc}{section}{Приложение}
\end{document}