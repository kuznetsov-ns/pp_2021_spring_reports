\documentclass{report}

\usepackage[T2A]{fontenc}
\usepackage[utf8]{luainputenc}
\usepackage[english, russian]{babel}
\usepackage[pdftex]{hyperref}
\usepackage[14pt]{extsizes}
\usepackage{listings}
\usepackage{color}
\usepackage{geometry}
\usepackage{enumitem}
\usepackage{multirow}
\usepackage{graphicx}
\usepackage{indentfirst}

\geometry{a4paper,top=2cm,bottom=3cm,left=2cm,right=1.5cm}
\setlength{\parskip}{0.5cm}
\setlist{nolistsep, itemsep=0.3cm,parsep=0pt}

\lstset{language=C++,
		basicstyle=\footnotesize,
		keywordstyle=\color{blue}\ttfamily,
		stringstyle=\color{red}\ttfamily,
		commentstyle=\color{green}\ttfamily,
		morecomment=[l][\color{magenta}]{\#}, 
		tabsize=4,
		breaklines=true,
  		breakatwhitespace=true,
  		title=\lstname,       
}

\makeatletter
\renewcommand\@biblabel[1]{#1.\hfil}
\makeatother

\begin{document}

\begin{titlepage}

\begin{center}
Министерство науки и высшего образования Российской Федерации
\end{center}

\begin{center}
Федеральное государственное автономное образовательное учреждение высшего образования \\
Национальный исследовательский Нижегородский государственный университет им. Н.И. Лобачевского
\end{center}

\begin{center}
Институт информационных технологий, математики и механики
\end{center}

\vspace{4em}

\begin{center}
\textbf{\LargeОтчет по лабораторной работе} \\
\end{center}
\begin{center}
\textbf{\Large«Линейная фильтрация изображений (блочное разбиение). Ядро Гаусса 3*3»} \\
\end{center}

\vspace{4em}

\newbox{\lbox}
\savebox{\lbox}{\hbox{text}}
\newlength{\maxl}
\setlength{\maxl}{\wd\lbox}
\hfill\parbox{7cm}{
\hspace*{5cm}\hspace*{-5cm}\textbf{Выполнил:} \\ студент группы 381808-1 \\ Кустова Анастасия\\
\\
\hspace*{5cm}\hspace*{-5cm}\textbf{Проверил:}\\ доцент кафедры МОСТ, \\ кандидат технических наук \\ Сысоев А.В.\\
}
\vspace{\fill}

\begin{center} Нижний Новгород \\ 2021 \end{center}

\end{titlepage}

\setcounter{page}{2}

% Введение
\section*{Введение}
\addcontentsline{toc}{section}{Введение}
В электронике и обработке сигналов, фильтром Гаусса называют фильтр, чья импульсная характеристика является функцией Гаусса. Фильтр Гаусса обычно используется в цифровом виде для обработки двумерных сигналов с целью снижения уровня шума. Визуально данных эффект представляет собой лёгкое размытие, как при наблюдении через мутное стекло. 
\par
Стоит отметить весьма ограниченную скорость фильтра Гаусса при реализации с помощью явного метода, особенно заметную на больших объёмах данных. Поэтому имеет смысл разделять обработку изображения между процессами или потоками.
\newpage

% Постановка задачи
\section*{Постановка задачи}
\addcontentsline{toc}{section}{Постановка задачи}
В данной работе необходимо реализовать линейную фильтрацию изображений «Фильтр Гаусса» с блочным разбиением. Данная работа должна содержать последовательную и несколько параллельных версий, с использованием следующих технологий: OpenMP, TBB, API std::thread. 
\par
На основе разработанной программы необходимо провести вычислительные эксперименты, сравнив время работы последовательного и параллельных алгоритмов на разных входных данных, затем сделать вывод об эффективности параллельных версий.
\par
Также следует экспериментально проверить корректность разработанных версий алгоритма с помощью Google тестов и проиллюстрировать работу на реальных изображениях.
\newpage

% Описание алгоритма
\section*{Описание алгоритма}
\addcontentsline{toc}{section}{Описание алгоритма}
Алгоритм фильтрации Гаусса состоит из двух этапов.
\par
На первом шаге необходимо создать ядро Гаусса, в создании которого участвуют два параметра: радиус и сигма. При помощи формулы инициализируем ячейки ядра Гаусса. Затем это ядро необходимо нормировать.
\par
На втором шаге необходимо применить ядро, полученное на первом шаге к каждому пикселю изображения:
\par
    1) Для каждого пикселя исходного изображения рассматривается область соседних пикселей. Размер этой области зависит от размера ядра Гаусса.
\par 

    2) Далее мы обходим все значения из этой области. Значение каждого соседнего пикселя умножается на значение из соответствующей ячейки в ядре Гаусса, и полученный результат от произведения прибавляется к результирующему значению цвета пикселя.
\newpage

% Схема распараллеливания
\section*{Схема распараллеливания}
\addcontentsline{toc}{section}{Схема распараллеливания}
Для начала необходимо распределить какому потоку какие пиксели достанутся для обработки. Для этого исходя из количества потоков, на котором планируется запустить программу вычисляется оптимальный размер блока. Пусть n – количество потоков, которые будут участвовать в обработке изображения. Тогда оптимальный размер блока будет вычисляться как квадратный корень из n, округлённый в большую сторону.
\par
Затем создаётся вектор векторов “array”, в котором будут храниться координаты начала очередного блока, длина и ширина блока. И в цикле по длине и ширине изображения эти координаты вычисляются.
\par
Затем этот вектор векторов передаётся в цикл, который распараллеливается различными способами (OpenMP, TBB, API std::thread). В этом цикле определяется, какому потоку какой блок нужно обрабатывать.
\begin{itemize}
\item В случае OpenMP реализации распараллеливание происходит с помощью 
 \begin{lstlisting}
 #pragma omp parallel for schedule(dynamic)
\end{lstlisting}
\item В случае TBB реализации сначала инициализируется объект:
\begin{lstlisting}
tbb::task_scheduler_init init(count_thread);
\end{lstlisting}
Затем такие параметры, как исходное изображение, ссылка на результирующее изображение, ядро Гаусса, ширина изображения, высота изображения, радиус и размеры блока передаются в класс GaussianParallel:
\begin{lstlisting}
GaussianParallel tmp(img, &res, kernel, array1, width, height, radius, block_width, block_height);
\end{lstlisting}
Распараллеливание происходит с помощью функции
\begin{lstlisting}
tbb::parallel_for(tbb::blocked_range<int>(0, static_cast<int>(array1.size())), tmp);
\end{lstlisting}
\item В API std::thread реализации в цикле создаются потоки с помощью функции std::thread.
\end{itemize}
\newpage

% Описание программной реализации
\section*{Описание программной реализации}
\addcontentsline{toc}{section}{Описание программной реализации}
В программе имеются следующие функции:

\begin{itemize}
\item Функция, реализующая TBB версию, в которой вычисляются координаты и размеры блоков, а также вызываются функции создания ядра и распараллеливания: 
\end{itemize}
\begin{lstlisting}
	std::vector<int> gaussianFilterTbb(const std::vector<int> & img, int width,
    int height, int radius, float sigma, int count_thread));
\end{lstlisting}
\begin{itemize}
\item Функция, реализующая OpenMP версию, в которой вычисляются координаты и размеры блоков, а также вызываются функции создания ядра и распараллеливания: 
\end{itemize}
\begin{lstlisting}
	std::vector<int> gaussianFilterOmp(const std::vector<int> & img, int width, int height, int radius, float sigma, int count_thread);
\end{lstlisting}
\begin{itemize}
\item Функция, реализующая API std::thread версию, в которой вычисляются координаты и размеры блоков, а также вызываются функции создания ядра и распараллеливания: 
\end{itemize}
\begin{lstlisting}
	std::vector<int> gaussianFilterStd(const std::vector<int> & img, int width, int height, int radius, float sigma, int count_thread);
\end{lstlisting}
\begin{itemize}
\item Функция, вызываемая при создании нового потока в API std::thread реализации: 
\end{itemize}
\begin{lstlisting}
	void threadFunction(std::vector<std::vector<int>> array, int i, int count_thread, int width, int height, const std::vector<int> & img, const std::vector<float> & kernel, int radius);
\end{lstlisting}

\begin{itemize}
\item Функция, реализующая последовательную версию, в которой вычисляются координаты и размеры блоков, а также вызываются функции создания ядра и распараллеливания: 
\end{itemize}
\begin{lstlisting}
	std::vector<int> gaussianFilterSeq(const std::vector<int> & img, int width, int height, int radius, float sigma);
\end{lstlisting}

\begin{itemize}
\item Функция, которая создаёт ядро Гаусса :
\end{itemize}
\begin{lstlisting}
	std::vector<float> createGaussianKernel(int radius, float sigma);
\end{lstlisting}
\begin{itemize}
\item Функция, которая вычисляет цвет пикселя: 
\end{itemize}
\begin{lstlisting}
	int calculateNewPixelColor(std::vector<int> img, int width, int height,
    int x, int y, int radius, std::vector<float> kernel);
\end{lstlisting}
\newpage

% Описание экспериментов
\section*{Описание экспериментов}
\addcontentsline{toc}{section}{Описание экспериментов}
Эксперименты проводились на 4-х ядерном компьютере с операционной системой – Windows 10, который имеет 8 логических процессоров. Результаты экспериментов приведены в таблице:
\begin{table}[!h]
\caption{Результаты вычислительных экспериментов}
\centering
\begin{tabular}{llllll}
Размер & Количество потоков & OpenMp & TBB & Std & Послед. \\
600x400 & 4 & 3.202 & 3.110 & 3.3 & 12.06\\
600x400 & 2 & 6.5 & 6.421 & 6.53 & 12.568\\
\end{tabular}
\end{table}
\par
По результатам эксперимента, можно сделать вывод о том, что параллельнык версии работают быстрее последовательной. Самой быстрой оказалась TBB версия, на втором месте OpenMp версия, и затем идёт API std::thread реализация.
\par
Kорректность работы программы могут результаты гугл-тестов и наглядный пример картинки.
\newpage

% Заключение
\section*{Заключение}
\addcontentsline{toc}{section}{Заключение}
В ходе выполнения работы была успешна реализована фильтрация изображения методом Гаусса в последовательном и параллельных вариантах. 
\par 
Основной задачей лабораторной работы была реализация эффективной параллельной версии. Эта цель была успешно достигнута, что подтверждается результатами экспериментов, проведенных в ходе работы. 
\par 
Работоспособность программы была проверена с помощью библиотеки для модульного тестирования и с помощью реального изображения.
\newpage

% Список литературы
\begin{thebibliography}{1}
\addcontentsline{toc}{section}{Список литературы}
\bibitem{Gergel}
Гергель В. П., Стронгин Р. Г. Основы параллельных вычислений для многопроцессорных вычислительных систем. – 2003.
\bibitem{Gaussian}
Фильтр Гаусса: URL: https://habr.com/ru/post/142818/
\end{thebibliography}
\newpage

% Приложение
\section*{Приложение}
\addcontentsline{toc}{section}{Приложение}
\begin{lstlisting}
							// main.cpp
// Copyright 2021 Kustova Anastasiya
#include <gtest/gtest.h>
#include <math.h>
#include <tbb/tbb.h>
#include <omp.h>
#include<vector>
#include<ctime>
#include<iostream>
#include"./filter_gaussa_block.h"

TEST(Gaussian_Method, Test_can_gaussian_filter) {
    int radius = 1;
    float sigma = 5.0;
    int height = 8;
    int count_thread = 3;
    int width = 8;
    std::vector<int> image1_decimal(height * width);
    image1_decimal[0] = 61; image1_decimal[1] = 2; image1_decimal[2] = 3; image1_decimal[3] = 4;
    image1_decimal[4] = 6; image1_decimal[5] = 8; image1_decimal[6] = 240; image1_decimal[7] = 2;
    image1_decimal[8] = 84; image1_decimal[9] = 6; image1_decimal[10] = 250; image1_decimal[11] = 3;
    image1_decimal[12] = 2; image1_decimal[13] = 8; image1_decimal[14] = 7; image1_decimal[15] = 3;
    image1_decimal[16] = 6;    image1_decimal[17] = 68; image1_decimal[18] = 2; image1_decimal[19] = 4;
    image1_decimal[20] = 3;    image1_decimal[21] = 9; image1_decimal[22] = 1; image1_decimal[23] = 4;
    image1_decimal[24] = 6;    image1_decimal[25] = 68; image1_decimal[26] = 2; image1_decimal[27] = 4;
    image1_decimal[28] = 3;    image1_decimal[29] = 9; image1_decimal[30] = 1; image1_decimal[31] = 4;
    image1_decimal[32] = 250;    image1_decimal[33] = 182; image1_decimal[34] = 90; image1_decimal[35] = 30;
    image1_decimal[36] = 23;    image1_decimal[37] = 4; image1_decimal[38] = 24; image1_decimal[39] = 1;
    image1_decimal[40] = 6;    image1_decimal[41] = 68; image1_decimal[42] = 2; image1_decimal[43] = 4;
    image1_decimal[44] = 3;    image1_decimal[45] = 9; image1_decimal[46] = 1; image1_decimal[47] = 4;
    image1_decimal[48] = 89;    image1_decimal[49] = 68; image1_decimal[50] = 5; image1_decimal[51] = 4;
    image1_decimal[52] = 3;    image1_decimal[53] = 9; image1_decimal[54] = 1; image1_decimal[55] = 4;
    image1_decimal[56] = 6;    image1_decimal[57] = 50; image1_decimal[58] = 2; image1_decimal[59] = 4;
    image1_decimal[60] = 3;    image1_decimal[61] = 9; image1_decimal[62] = 1; image1_decimal[63] = 4;
    ASSERT_NO_THROW(gaussianFilterTbb(image1_decimal, width, height, radius, sigma, count_thread));
}
TEST(Gaussian_Method, Test_big_matrix) {
    int radius = 1;
    float sigma = 5.0;
    int count_thread = 4;
    int height = 400;
    int width = 400;
    std::vector<int> image1_decimal(height * width);
    std::vector<int> image2_seq(height * width);
    std::vector<int> image2_omp(height * width);
    std::vector<int> image2_tbb(height * width);
    std::vector<int> image2_std(height * width);
    for (int i = 0; i < width * height; i++) {
        image1_decimal[i] = 25;
    }
    double t1, t2;
    t1 = omp_get_wtime();
    image2_seq = gaussianFilterSeq(image1_decimal, width, height, radius, sigma);
    t2 = omp_get_wtime();
    std::cout << "Seq: " << t2 - t1 << std::endl;
    double t3, t4;
    t3 = omp_get_wtime();
    image2_omp = gaussianFilterOmp(image1_decimal, width, height, radius, sigma, count_thread);
    t4 = omp_get_wtime();
    std::cout << "Omp: " << t4 - t3 << std::endl;
    tbb::tick_count t5, t6;
    t5 = tbb::tick_count::now();
    image2_tbb = gaussianFilterTbb(image1_decimal, width, height, radius, sigma, count_thread);
    t6 = tbb::tick_count::now();
    std::cout << "Tbb: " << (t6 - t5).seconds() << std::endl;
    double t7, t8;
    t7 = omp_get_wtime();
    image2_std = gaussianFilterStd(image1_decimal, width, height, radius, sigma, count_thread);
    t8 = omp_get_wtime();
    std::cout << "Std: " << t8 - t7 << std::endl;
}
TEST(Gaussian_Method, Test_check_work_with_rectangle_matrix) {
    int radius = 1;
    float sigma = 5.0;
    int count_thread = 8;
    int height = 8;
    int width = 9;
    std::vector<int> image1_decimal(height * width);
    std::vector<int> image2_decimal(height * width);
    std::vector<int> check(height * width);
    image1_decimal[0] = 61; image1_decimal[1] = 2; image1_decimal[2] = 3; image1_decimal[3] = 4;
    image1_decimal[4] = 6; image1_decimal[5] = 8; image1_decimal[6] = 240; image1_decimal[7] = 2;
    image1_decimal[8] = 84; image1_decimal[9] = 6; image1_decimal[10] = 250; image1_decimal[11] = 3;
    image1_decimal[12] = 2; image1_decimal[13] = 8; image1_decimal[14] = 7; image1_decimal[15] = 3;
    image1_decimal[16] = 6;    image1_decimal[17] = 68; image1_decimal[18] = 2; image1_decimal[19] = 4;
    image1_decimal[20] = 3;    image1_decimal[21] = 9; image1_decimal[22] = 1; image1_decimal[23] = 4;
    image1_decimal[24] = 6;    image1_decimal[25] = 68; image1_decimal[26] = 2; image1_decimal[27] = 4;
    image1_decimal[28] = 3;    image1_decimal[29] = 9; image1_decimal[30] = 1; image1_decimal[31] = 4;
    image1_decimal[32] = 250;    image1_decimal[33] = 182; image1_decimal[34] = 90; image1_decimal[35] = 30;
    image1_decimal[36] = 23;    image1_decimal[37] = 4; image1_decimal[38] = 24; image1_decimal[39] = 1;
    image1_decimal[40] = 6;    image1_decimal[41] = 68; image1_decimal[42] = 2; image1_decimal[43] = 4;
    image1_decimal[44] = 3;    image1_decimal[45] = 9; image1_decimal[46] = 1; image1_decimal[47] = 4;
    image1_decimal[48] = 89;    image1_decimal[49] = 68; image1_decimal[50] = 5; image1_decimal[51] = 4;
    image1_decimal[52] = 3;    image1_decimal[53] = 9; image1_decimal[54] = 1; image1_decimal[55] = 4;
    image1_decimal[56] = 6;    image1_decimal[57] = 50; image1_decimal[58] = 2; image1_decimal[59] = 4;
    image1_decimal[60] = 3;    image1_decimal[61] = 9; image1_decimal[62] = 1; image1_decimal[63] = 4;
    image1_decimal[64] = 6;    image1_decimal[65] = 50; image1_decimal[66] = 2; image1_decimal[67] = 4;
    image1_decimal[68] = 3;    image1_decimal[69] = 9; image1_decimal[70] = 1; image1_decimal[71] = 4;
    image2_decimal = gaussianFilterTbb(image1_decimal, width, height, radius, sigma, count_thread);
    check[0] = 56; check[1] = 43; check[2] = 29; check[3] = 4;
    check[4] = 5; check[5] = 58; check[6] = 58; check[7] = 80;
    check[8] = 53; check[9] = 43; check[10] = 37; check[11] = 31;
    check[12] = 4; check[13] = 5; check[14] = 31; check[15] = 38;
    check[16] = 52;    check[17] = 42; check[18] = 30; check[19] = 31;
    check[20] = 31;    check[21] = 4; check[22] = 31; check[23] = 51;
    check[24] = 68;    check[25] = 50; check[26] = 40; check[27] = 7;
    check[28] = 8;    check[29] = 6; check[30] = 6; check[31] = 38;
    check[32] = 59;    check[33] = 75; check[34] = 43; check[35] = 25;
    check[36] = 8;    check[37] = 9; check[38] = 15; check[39] = 22;
    check[40] = 54;    check[41] = 65; check[42] = 67; check[43] = 36;
    check[44] = 19;    check[45] = 8; check[46] = 8; check[47] = 20;
    check[48] = 28;    check[49] = 32; check[50] = 18; check[51] = 11;
    check[52] = 4;    check[53] = 4; check[54] = 4; check[55] = 9;
    check[56] = 23;    check[57] = 30; check[58] = 25; check[59] = 11;
    check[60] = 4;    check[61] = 4; check[62] = 4; check[63] = 3;
    check[64] = 14;    check[65] = 19; check[66] = 18; check[67] = 8;
    check[68] = 4;    check[69] = 4; check[70] = 4; check[71] = 3;
    for (int i = 0; i < height; i++) {
        for (int j = 0; j < width; j++) {
            ASSERT_EQ(image2_decimal[i * width + j], check[i * width + j]);
        }
    }
}

TEST(Gaussian_Method, Test_check_gaussian_filter) {
    int radius = 1;
    float sigma = 5.0;
    int count_thread = 3;
    int height = 8;
    int width = 8;
    std::vector<int> image1_decimal(height * width);
    std::vector<int> image2_decimal(height * width);
    std::vector<int> check(height * width);
    image1_decimal[0] = 61; image1_decimal[1] = 2; image1_decimal[2] = 3; image1_decimal[3] = 4;
    image1_decimal[4] = 6; image1_decimal[5] = 8; image1_decimal[6] = 240; image1_decimal[7] = 2;
    image1_decimal[8] = 84; image1_decimal[9] = 6; image1_decimal[10] = 250; image1_decimal[11] = 3;
    image1_decimal[12] = 2; image1_decimal[13] = 8; image1_decimal[14] = 7; image1_decimal[15] = 3;
    image1_decimal[16] = 6;    image1_decimal[17] = 68; image1_decimal[18] = 2; image1_decimal[19] = 4;
    image1_decimal[20] = 3;    image1_decimal[21] = 9; image1_decimal[22] = 1; image1_decimal[23] = 4;
    image1_decimal[24] = 6;    image1_decimal[25] = 68; image1_decimal[26] = 2; image1_decimal[27] = 4;
    image1_decimal[28] = 3;    image1_decimal[29] = 9; image1_decimal[30] = 1; image1_decimal[31] = 4;
    image1_decimal[32] = 250;    image1_decimal[33] = 182; image1_decimal[34] = 90; image1_decimal[35] = 30;
    image1_decimal[36] = 23;    image1_decimal[37] = 4; image1_decimal[38] = 24; image1_decimal[39] = 1;
    image1_decimal[40] = 6;    image1_decimal[41] = 68; image1_decimal[42] = 2; image1_decimal[43] = 4;
    image1_decimal[44] = 3;    image1_decimal[45] = 9; image1_decimal[46] = 1; image1_decimal[47] = 4;
    image1_decimal[48] = 89;    image1_decimal[49] = 68; image1_decimal[50] = 5; image1_decimal[51] = 4;
    image1_decimal[52] = 3;    image1_decimal[53] = 9; image1_decimal[54] = 1; image1_decimal[55] = 4;
    image1_decimal[56] = 6;    image1_decimal[57] = 50; image1_decimal[58] = 2; image1_decimal[59] = 4;
    image1_decimal[60] = 3;    image1_decimal[61] = 9; image1_decimal[62] = 1; image1_decimal[63] = 4;
    // double t1, t2, dt;
    // t1 = omp_get_wtime();
    image2_decimal = gaussianFilterTbb(image1_decimal, width, height, radius, sigma, count_thread);
    // t2 = omp_get_wtime();
    // dt = t2 - t1;
    // std::cout << dt;
    check[0] = 46; check[1] = 51; check[2] = 30; check[3] = 30;
    check[4] = 5; check[5] = 58; check[6] = 58; check[7] = 55;
    check[8] = 42; check[9] = 53; check[10] = 38; check[11] = 30;
    check[12] = 5; check[13] = 31; check[14] = 31; check[15] = 29;
    check[16] = 37;    check[17] = 54; check[18] = 45; check[19] = 29;
    check[20] = 4;    check[21] = 4; check[22] = 5; check[23] = 3;
    check[24] = 93;    check[25] = 74; check[26] = 49; check[27] = 17;
    check[28] = 9;    check[29] = 8; check[30] = 6; check[31] = 4;
    check[32] = 94;    check[33] = 75; check[34] = 50; check[35] = 18;
    check[36] = 9;    check[37] = 8; check[38] = 6; check[39] = 4;
    check[40] = 111;    check[41] = 84; check[42] = 50; check[43] = 18;
    check[44] = 9;    check[45] = 8; check[46] = 6; check[47] = 4;
    check[48] = 43;    check[49] = 33; check[50] = 22; check[51] = 3;
    check[52] = 5;    check[53] = 4; check[54] = 4; check[55] = 3;
    check[56] = 40;    check[57] = 30; check[58] = 20; check[59] = 3;
    check[60] = 5;    check[61] = 4; check[62] = 4; check[63] = 3;
    for (int i = 0; i < height; i++) {
        for (int j = 0; j < width; j++) {
            ASSERT_EQ(image2_decimal[i * width + j], check[i * width + j]);
        }
    }
}
TEST(Gaussian_Method, Test_can_Clamp) {
    int val1 = -1;
    ASSERT_NO_THROW(Clamp(val1, 0, 255));
}
TEST(Gaussian_Method, Test_check_Clamp) {
    int val1 = -1;
    ASSERT_EQ(Clamp(val1, 0, 255), 0);
    ASSERT_EQ(Clamp(val1, -5, 30), -1);
    ASSERT_EQ(Clamp(val1, -5, -3), -3);
}

TEST(Gaussian_Method, Can_creating_Gaussian_Kernel) {
    int radius = 1;
    float sigma = 5.0;
    ASSERT_NO_THROW(createGaussianKernel(radius, sigma));
}

\end{lstlisting}
\begin{lstlisting}
							//filter_gaussa_block.cpp
// Copyright 2021 Kustova Anastasiya
#include <math.h>
#include <tbb/tbb.h>
#include <omp.h>
#include <iostream>
#include <vector>
#include <tuple>
#include "../../../modules/task_3/kustova_a_gauss_tbb/filter_gaussa_block.h"

std::vector<int> resultImage1;

class GaussianParallel {
 private:
    const std::vector<int> &img;
    std::vector<int> *res;
    const std::vector<float> &kernel;
    const std::vector<std::vector<int>> &array;
    int width, height, radius, block_width, block_height;

 public:
    void operator()(const tbb::blocked_range<int> &r) const {
        for (int i = r.begin(); i != r.end(); ++i) {
            int j_start = array[i][0];
            int j_finish = array[i][0] + array[i][2];
            int i_start = array[i][1];
            int i_finish = array[i][1] + array[i][3];
            for (int i = i_start; i < i_finish && i < width; i++) {
                for (int j = j_start; j < j_finish && j < height; j++) {
                    int color = calculateNewPixelColor(img, width, height, i, j, radius, kernel);
                    res->at(j * width + i) = color;
                }
            }
        }
    }

    GaussianParallel(const std::vector<int> &img, std::vector<int> *res, const std::vector<float> &kernel,
        const std::vector<std::vector<int>> &array, int width, int height, int radius,
        int block_width, int block_height) :
        img(img),
        res(res),
        kernel(kernel),
        array(array),
        width(width),
        height(height),
        radius(radius),
        block_width(block_width),
        block_height(block_height){}
};
std::vector<int> gaussianFilterTbb(const std::vector<int> & img, int width,
    int height, int radius, float sigma, int count_thread) {
    std::vector<int> res(img);
    int size = 2 * radius + 1;
    std::vector<float> kernel(size * size);
    int l = 0;
    int k = 0;
    kernel = createGaussianKernel(radius, sigma);
    std::vector<std::vector<int>> array1;
    int grid_size = ceil(static_cast<double>(sqrt(count_thread)));
    int block_height = height / grid_size;
    int block_width = width / grid_size;
    while (l < height) {
        k = 0;
        int block_w = block_width;
        int block_h = block_height;
        while (k < width) {
            if (width - k < block_width + block_width / 2) {
                block_w = width - k;
            } else {
                block_w = block_width;
            }
            if (height - l < block_height / 2) {
                block_h = height - l;
            } else {
                block_h = block_height;
            }
            std::vector<int> tup = { l, k , block_h, block_w };
            array1.push_back(tup);
            k += block_w;
        }
        l += block_h;
    }
    tbb::task_scheduler_init init(count_thread);
    GaussianParallel tmp(img, &res, kernel, array1, width, height, radius, block_width, block_height);
    tbb::parallel_for(tbb::blocked_range<int>(0, static_cast<int>(array1.size())), tmp);
    return res;
}
std::vector<int> gaussianFilterOmp(const std::vector<int> & img, int width, int height, int radius, float sigma, int count_thread) {
    std::vector<int> resultImage(height * width);
    int size = 2 * radius + 1;
    int color = 0;
    int l = 0;
    int k = 0;
    std::vector<float> kernel(size * size);
    kernel = createGaussianKernel(radius, sigma);
    std::vector<std::vector<int>> array;
    int grid_size = ceil(static_cast<double>(sqrt(count_thread)));
    int block_height = height / grid_size;
    int block_width = width / grid_size;
    while (l < height) {
        k = 0;
        int block_w = block_width;
        int block_h = block_height;
        while (k < width) {
            if (width - k < block_width + block_width / 2) {
                block_w = width - k;

            }
            else {
                block_w = block_width;
            }
            if (height - l < block_height / 2) {
                block_h = height - l;

            }
            else {
                block_h = block_height;
            }
            std::vector<int> tup = { l, k , block_h, block_w };
            array.push_back(tup);
            k += block_w;

        }
        l += block_h;
    }

    omp_set_num_threads(count_thread);
#pragma omp parallel for schedule(dynamic) shared(img, width, height, radius, kernel, array, resultImage)
    for (int t = 0; t < array.size(); t++) {
        int j_start = array[t][0];
        int j_finish = array[t][0] + array[t][2];
        int i_start = array[t][1];
        int i_finish = array[t][1] + array[t][3];
        for (int i = i_start; i < i_finish && i < width; i++) {
            for (int j = j_start; j < j_finish && j < height; j++) {
                int color = calculateNewPixelColor(img, width, height, i, j, radius, kernel);
                resultImage[j * width + i] = color;
            }
        }
    }
    return resultImage;
}
std::vector<int> gaussianFilterSeq(const std::vector<int> & img, int width, int height, int radius, float sigma) {
    std::vector<int> resultImage(height * width);
    int size = 2 * radius + 1;
    std::vector<float> kernel(size * size);
    kernel = createGaussianKernel(radius, sigma);
    for (int i = 0; i < width; i++) {
        for (int j = 0; j < height; j++) {
            int color = calculateNewPixelColor(img, width, height, i, j, radius, kernel);
            resultImage[j * width + i] = color;
        }
    }
    return resultImage;
}

std::vector<float> createGaussianKernel(int radius, float sigma) {
    int size = 2 * radius + 1;
    std::vector<float> kernel(size * size);
    float norm = 0;
    for (int i = -radius; i <= radius; i++) {
        for (int j = -radius; j <= radius; j++) {
            kernel[(i + radius)* size + j + radius] = static_cast<float>(exp(-(i * i + j * j) / (2 * sigma * sigma)));
            norm = norm + kernel[(i + radius) * size + j + radius];
        }
    }
    for (int i = 0; i < size; i++) {
        for (int j = 0; j < size; j++) {
            kernel[i * size + j] = kernel[i * size + j] / norm;
        }
    }
    return kernel;
}

int calculateNewPixelColor(std::vector<int> img, int width, int height,
    int x, int y, int radius, std::vector<float> kernel) {
    int size = radius * 2 + 1;
    double sumColor = 0;
    for (int l = -radius; l <= radius; l++) {
        for (int k = -radius; k <= radius; k++) {
            int idX = Clamp(x + k, 0, width - 1);
            int idY = Clamp(y + l, 0, height - 1);
            int neighborColor = img[idY * width + idX];
            sumColor += neighborColor * kernel[(k + radius) * size + l + radius];
        }
    }
    return Clamp(static_cast<int>(sumColor), 0, 255);
}
int Clamp(int value, int min, int max) {
    if (value < min) {
        return min;
    }
    if (value > max) {
        return max;
    }
    return value;
}
void threadFunction(std::vector<std::vector<int>> array, int i, int count_thread, int width, int height, const std::vector<int> & img, const std::vector<float> & kernel, int radius) {
    for (int t = i; t < static_cast<int>(array.size()); t++) {
        if ((t % count_thread) == i) {
            int j_start = array[t][0];
            int j_finish = array[t][0] + array[t][2];
            int i_start = array[t][1];
            int i_finish = array[t][1] + array[t][3];
            for (int i = i_start; i < i_finish && i < width; i++) {
                for (int j = j_start; j < j_finish && j < height; j++) {
                    int color = calculateNewPixelColor(img, width, height, i, j, radius, kernel);
                    resultImage1[j * width + i] = color;
                }
            }
            t += count_thread - 1;
        }
    }
}

std::vector<int> gaussianFilterStd(const std::vector<int> & img, int width, int height,
    int radius, float sigma, int count_thread) {
    resultImage1 = img;
    int size = 2 * radius + 1;
    int l = 0;
    int k = 0;
    std::vector<float> kernel(size * size);
    kernel = createGaussianKernel(radius, sigma);
    std::vector<std::vector<int>> array;
    int grid_size = ceil(static_cast<double>(sqrt(count_thread)));
    int block_height = height / grid_size;
    int block_width = width / grid_size;
    while (l < height) {
        k = 0;
        int block_w = block_width;
        int block_h = block_height;
        while (k < width) {
            if (width - k < block_width + block_width / 2) {
                block_w = width - k;
            } else {
                block_w = block_width;
            }
            if (height - l < block_height / 2) {
                block_h = height - l;
            } else {
                block_h = block_height;
            }
            std::vector<int> tup = { l, k , block_h, block_w };
            array.push_back(tup);
            k += block_w;
        }
        l += block_h;
    }
    std::vector<std::thread> threads;
    for (int i = 0; i < count_thread; i++) {
        threads.push_back(std::thread(threadFunction, array, i, count_thread,
            width, height, std::ref(img), std::ref(kernel), radius));
    }
    for (int i = 0; i < count_thread; i++) {
        threads[i].join();
    }
    return resultImage1;
}

\end{lstlisting}
\begin{lstlisting}
							// filter_gaussa_block.h
// Copyright 2021 Kustova Anastasiya
#ifndef MODULES_TASK_3_KUSTOVA_A_GAUSS_TBB_FILTER_GAUSSA_BLOCK_H_
#define MODULES_TASK_3_KUSTOVA_A_GAUSS_TBB_FILTER_GAUSSA_BLOCK_H_
#include <vector>

std::vector<int> gaussianFilterTbb(const std::vector<int> & img, int width,
    int height, int radius, float sigma, int count_thread);
std::vector<int> gaussianFilterOmp(const std::vector<int> & img, int width,
    int height, int radius, float sigma, int count_thread);
std::vector<int> gaussianFilterSeq(const std::vector<int> & img, int width,
    int height, int radius, float sigma);
std::vector<int> gaussianFilterStd(const std::vector<int> & img, int width, int height,
    int radius, float sigma, int count_thread);
void threadFunction(std::vector<std::vector<int>> array, int i, int count_thread, int width, int height, const std::vector<int> & img, const std::vector<float> & kernel, int radius);
std::vector<float> createGaussianKernel(int radius, float sigma);
int calculateNewPixelColor(std::vector<int> img, int width, int height,
    int x, int y, int radius, std::vector<float> kernel);
int Clamp(int value, int min, int max);
#endif  // MODULES_TASK_3_KUSTOVA_A_GAUSS_TBB_FILTER_GAUSSA_BLOCK_H_

\end{lstlisting}
\end{document}
