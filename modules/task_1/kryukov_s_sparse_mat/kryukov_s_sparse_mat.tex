\documentclass{report}

\usepackage[T2A]{fontenc}
\usepackage[utf8]{luainputenc}
\usepackage[english, russian]{babel}
\usepackage[pdftex]{hyperref}
\usepackage[14pt]{extsizes}
\usepackage{listings}
\usepackage{color}
\usepackage{geometry}
\usepackage{enumitem}
\usepackage{multirow}
\usepackage{graphicx}
\usepackage{indentfirst}
\usepackage{amsmath}


\geometry{a4paper,top=2cm,bottom=3cm,left=2cm,right=1.5cm}
\setlength{\parskip}{0.5cm}
\setlist{nolistsep, itemsep=0.3cm,parsep=0pt}

\lstset{language=C++,
		basicstyle=\footnotesize,
		keywordstyle=\color{blue}\ttfamily,
		stringstyle=\color{red}\ttfamily,
		commentstyle=\color{green}\ttfamily,
		morecomment=[l][\color{magenta}]{\#}, 
		tabsize=4,
		breaklines=true,
  		breakatwhitespace=true,
  		title=\lstname,       
}

\makeatletter
\renewcommand\@biblabel[1]{#1.\hfil}
\makeatother

\begin{document}

\begin{titlepage}

\begin{center}
Министерство науки и высшего образования Российской Федерации
\end{center}

\begin{center}
Федеральное государственное автономное образовательное учреждение высшего образования \\
Национальный исследовательский Нижегородский государственный университет им. Н.И. Лобачевского
\end{center}

\begin{center}
Институт информационных технологий, математики и механики
\end{center}

\vspace{4em}

\begin{center}
\textbf{\LargeОтчет по лабораторной работе} \\
\end{center}
\begin{center}
\textbf{\Large«Умножение разреженных матриц. Элементы комплексного типа. Формат хранения матрицы - строковый (CRS).»} \\
\end{center}

\vspace{4em}

\newbox{\lbox}
\savebox{\lbox}{\hbox{text}}
\newlength{\maxl}
\setlength{\maxl}{\wd\lbox}
\hfill\parbox{7cm}{
\hspace*{5cm}\hspace*{-5cm}\textbf{Выполнил:} \\ студент группы 381808-1 \\ Крюков С. А.\\
\\
\hspace*{5cm}\hspace*{-5cm}\textbf{Проверил:}\\ доцент кафедры МОСТ, \\ кандидат технических наук \\ Сысоев А. В.\\
}
\vspace{\fill}

\begin{center} Нижний Новгород \\ 2021 \end{center}

\end{titlepage}

\setcounter{page}{2}

% Содержание
\tableofcontents
\newpage

% Введение
\section*{Введение}
\addcontentsline{toc}{section}{Введение}
Алгебра разреженных матриц -важный раздел математики, имеющий очевидное практическое применение. Разреженные матрицы возникают естественным образом при постановке и решении задач из разных научных и инженерных областей: при численном решении дифференциальных уравнений в частных производных, в теории графов, при решении оптимизационных задач.

\par Разреженной называют матрицу, имеющую малый процент ненулевых элементов.

\par Эффективные методы хранения и обработки разреженных матриц на протяжении последних десятилетий вызывают интерес у широкого круга исследователей.  Для учета разреженной структуры приходится существенно усложнять как методы хранения, так и алгоритмы обработки. В этой работе будет рассмотрена операция умножения разреженных матриц с элементами комплексного типа. Хранение их будет осуществлено в строковом формате.

\newpage

% Постановка задачи
\section*{Постановка задачи}
\addcontentsline{toc}{section}{Постановка задачи}
В рамках лабораторной работы нужно реализовать алгоритм умножения разреженных матриц с элементами комплексного типа, используя строковый формат хранения.

\par Для этого нужно выполнить следующую последовательность действий:

\begin{enumerate}
\item Изучить способы хранения разреженных матриц.
\item Разработать программную инфраструктуру для проведения экспериментов - генератора разреженных матриц, метод для проверки корректности умножения матриц.
\item Разработать последовательную версию алгоритма умножения разреженных матриц.
\item Распараллеливание разреженного матричного умножения с использованием OpenMP и TBB.
\item Осуществить проверку корректности работы алгоритмов с помощью тестов, созданных с использованием Google C++ Testing Framework.
\item Провести вычислительные эксперименты и сделать соответствующие выводы.
\end{enumerate}
\newpage

% Метод решения
\section*{Метод решения}
\addcontentsline{toc}{section}{Метод решения}
Хранение разреженной матрицы осуществляется с помощью структуры, которая содержит следующие поля: размерность матрицы, массив для хранения элементов построчно, массив хранящий номера столбцов для каждого элемента, массив, с номерами строк для каждого элемента. В качестве тестовых матриц будут использоваться диагональные матрицы разной размерности. Проверка корректности алгоритма умножения проводится с помощью метода, который вычисляет скалярное произведение векторов.

\par Метод реализующий алгоритм умножения принимает на вход две матрицы A и B в строковом формате хранения и возвращает их произведение - матрицу С. Опишем его шаги:
\begin{enumerate}
\item Создаем три вспомогательных вектора (matC\_val, matC\_colNum, matC\_rowNum) для хранения результирующей матрицы.
\item Транспонируем с помощью алгоритма Густавсона матрицу, переданную в качестве второго параметра.
\item В цикле перебираем все строки матрицы A, и для каждой из них перебираем все строки транспонированной матрицы B.
\item Вычисляем скалярное произведение векторов-строк матриц A и транспонированной B, если оно отлично от 0, тогда
добавляем его в вектор matC\_val и в вектор matC\_colNum номер текущей строки матрицы B.
\item после прохода по каждой строке из транспонированной B записываем в вектор matC\_rowNum текущее количество ненулевых элементов.
\end{enumerate}
По окончании работы алгоритма, заполним поля структуры матрицы С данными из вспомогательных векторов.

\newpage

% Схема распараллеливания
\section*{Схема распараллеливания}
\addcontentsline{toc}{section}{Схема распараллеливания}
Рассмотрим параллельную реализацию алгоритма умножения разреженных матриц.

\par Идея распараллеливания алгоритма достаточно очевидна. Во внешнем цикле перебираем строки матрицы А и далее работаем с ними независимо. Таким образом получаем естественный вариант параллелизма - распределение строк между потоками.

\newpage

% Описание программной реализации
\section*{Описание программной реализации}
\addcontentsline{toc}{section}{Описание программной реализации}
Разреженная матрица хранится в виде следующей структуры:

\begin{lstlisting}
struct crs_mat {
    int size;
    std::vector<std::complex<double>> val;
    std::vector<int> colNum, rowNum;
};
\end{lstlisting}
В программе используются функции, представленные ниже.
\par Метод для изменения формата хранения матрицы из обычного вектора значений в разреженный:
\begin{lstlisting}
crs_mat createSparseMat(int size_, const std::vector<std::complex<double>> &vectMat);
\end{lstlisting}
\par Входными параметрами являются размерность матрицы и вектор с элементами матрицы.
\par Метод для генерации диагональной матрицы заданной размерности:
\begin{lstlisting}
crs_mat genDiagonalSparseMat(int size_);
\end{lstlisting}
\par Метод для вычисления скалярного произведения векторов:
\begin{lstlisting}
std::vector<std::complex<double>> MultipluValues(const std::vector<std::complex<double>> & first, const std::vector<std::complex<double>> & second, const int size_);
\end{lstlisting}
\par Метод для транспонирования матрицы:
\begin{lstlisting}
crs_mat transposeMatrixGustavson(crs_mat matr);
\end{lstlisting}
\par Реализация параллельного алгоритма умножения разреженных матриц представлена в функциях:
\begin{lstlisting}
crs_mat multiplicateMatrix(crs_mat A, crs_mat B);
crs_mat multiplicateMatrixTBB(crs_mat A, crs_mat B);
\end{lstlisting}
\par В качестве входных параметров передаются матрицы, которые нужно перемножить, а выходным является результирующая матрица.
\newpage

% Подтверждение корректности
\section*{Подтверждение корректности}
\addcontentsline{toc}{section}{Подтверждение корректности}
Для проверки эффективности и корректности работы программы используются тесты, написанные с помощью Google Testing Framework.
\par Они проверяют эффективность работы параллельной схемы (путем сравнения времени выполнения параллельной и последовательной реализации) и корректность вычислений (проверкой правильности формирования массивов разреженной матрицы и проверкой точности - операцией скалярного умножения векторов).

\par Успешное прохождение всех тестов подтверждает корректность работы написанной программы.

\newpage

% Результаты экспериментов
\section*{Результаты экспериментов}
\addcontentsline{toc}{section}{Результаты экспериментов}
Вычислительные эксперименты для оценки эффективности параллельного алгоритма умножения разреженных матриц проводились на оборудовании со следующей аппаратной конфигурацией:

\begin{itemize}
\item Процессор: AMD A10-9620P RADEON R5, 2.50 GHz, 4 ядра;
\item Оперативная память: 8192 МБ;
\item ОС: Майкрософт Windows 10.
\end{itemize}

\par В рамках эксперимента было вычислено время работы параллельного и последовательно алгоритмов умножения разреженных матриц размерностью 1000 и 3000. Тесты запускались на четырех потоках.
\par Результаты экспериментов представлены ниже.

\begin{table}[!h]
\caption{Результаты экспериментов. Сравнение с OpenMP.}
\centering
\begin{tabular}{lllll}
Размерность матрицы & Последовательно & Параллельно & Ускорение  \\
1000        & 0.811266         & 0.203862     & 3.979       \\
3000        & 6.70915         & 1.32474     & 5.065     
\end{tabular}
\end{table}

\begin{table}[!h]
\caption{Результаты экспериментов. Сравнение с TBB.}
\centering
\begin{tabular}{lllll}
Размерность матрицы & Последовательно & Параллельно & Ускорение  \\
1000        & 0.93158         & 0.311613     & 2.99       \\
3000        & 5.72898         & 1.16366     & 4.923     
\end{tabular}
\end{table}

\par По данным экспериментов можно сделать вывод, что параллельный алгоритм работает намного быстрее, чем последовательный.
\newpage

% Заключение
\section*{Заключение}
\addcontentsline{toc}{section}{Заключение}
В ходе выполнения лабораторной работы был реализован последовательный и параллельный вариант алгоритма умножения разреженных матриц, хранящихся в формате CRS.
\par По данным вычислительных экспериментов удалось сравнить время работы параллельной и последовательной версии, а также показать эффективность реализованного алгоритма.

\newpage

% Список литературы
\begin{thebibliography}{1}
\addcontentsline{toc}{section}{Список литературы}
\bibitem{wiki} Википедия: Разреженная матрица // URL \url {https://en.wikipedia.org/wiki/Sparse\_Matrix}
\bibitem{Gergel}
Гергель В. П. Теория и практика параллельных вычислений. – 2007. 
\bibitem{Gergel}
Гергель В. П., Стронгин Р. Г. Основы параллельных вычислений для многопроцессорных вычислительных систем. – 2003.

\end{thebibliography}
\newpage

% Приложение
\section*{Приложение}
\addcontentsline{toc}{section}{Приложение}
В этом разделе находится листинг всего кода, написанного в рамках лабораторной работы.
\par Реализация с использованием OpenMP
\begin{lstlisting}
// sparsemat_omp.h

// Copyright 2021 Kryukov Sergey
#ifndef MODULES_TASK_2_KRYUKOV_S_SPARSE_MAT_OMP_SPARSEMAT_OMP_H_
#define MODULES_TASK_2_KRYUKOV_S_SPARSE_MAT_OMP_SPARSEMAT_OMP_H_
#include <omp.h>
#include <vector>
#include <complex>

struct crs_mat {
    int size;
    std::vector<std::complex<double>> val;
    std::vector<int> colNum, rowNum;
};
crs_mat createSparseMat(int size_,
    const std::vector<std::complex<double>> &vectMat);

crs_mat genDiagonalSparseMat(int size_);
std::vector<std::complex<double>> MultipluValues(
    const std::vector<std::complex<double>> & first,
    const std::vector<std::complex<double>> & second, const int size_);

crs_mat transposeMatrixGustavson(crs_mat matr);
crs_mat multiplicateMatrix(crs_mat A, crs_mat B);
crs_mat omp_multiplicateMatrix(crs_mat A, crs_mat B);
#endif  // MODULES_TASK_2_KRYUKOV_S_SPARSE_MAT_OMP_SPARSEMAT_OMP_H_
\end{lstlisting}
\begin{lstlisting}
// sparsemat_omp.cpp

// Copyright 2021 Kryukov Sergey
#include<float.h>
#include <omp.h>
#include <random>
#include <ctime>
#include <vector>
#include <string>
#include "../../../modules/task_2/kryukov_s_sparse_mat_omp/sparsemat_omp.h"

crs_mat createSparseMat(int size_,
    const std::vector<std::complex<double>> &vectMat) {
    crs_mat matrix;
    matrix.size = size_;
    matrix.rowNum.push_back(0);
    for (int i = 0; i < size_; i++) {
        for (int j = 0; j < size_; j++) {
            if (std::abs(vectMat[i*size_ + j]) < DBL_MIN)
                continue;
            matrix.val.push_back(vectMat[i*size_ + j]);
            matrix.colNum.push_back(j);
        }
        matrix.rowNum.push_back(matrix.colNum.size());
    }
    return matrix;
}

crs_mat genDiagonalSparseMat(int size_) {
    crs_mat resultMat;
    resultMat.size = size_;
    std::vector<std::complex<double>> values;
    std::vector<int> colNum, rowNum;
    rowNum.push_back(0);

    std::mt19937 gen;
    gen.seed(static_cast<unsigned int>(time(0)));
    for (int i = 0; i < size_; i++) {
        values.push_back(std::complex<double>(
            static_cast<double>(gen() % 100000),
            static_cast<double>(gen() % 100000)));
        colNum.push_back(i);
        int a = i + 1;
        rowNum.push_back(a);
    }

    resultMat.val = values;
    resultMat.rowNum = rowNum;
    resultMat.colNum = colNum;

    return resultMat;
}

std::vector<std::complex<double>> MultipluValues(
    const std::vector<std::complex<double>>& first,
    const std::vector<std::complex<double>> & second,
    const int size_) {

    std::vector<std::complex<double>> resultVec;
    std::complex<double> sum = { 0.0, 0.0 };
    for (int i = 0; i < size_; i++) {
        sum = first[i] * second[i];
        resultVec.push_back(sum);
    }

    return resultVec;
}

crs_mat transposeMatrixGustavson(crs_mat matr) {
    crs_mat TrMat;
    std::vector<std::vector<std::complex<double>>> storageValue(matr.size);
    std::vector<std::vector<int>> storageCol(matr.size);
    TrMat.rowNum.push_back(0);
    for (int i = 0; i < matr.size; i++) {
        for (int j = matr.rowNum[i]; j < matr.rowNum[i + 1]; j++) {
            storageValue[matr.colNum[j]].push_back(matr.val[j]);
            storageCol[matr.colNum[j]].push_back(i);
        }
    }
    for (int i = 0; i < matr.size; i++)
        for (int j = 0; j < static_cast<int>(storageCol[i].size()); j++) {
            TrMat.colNum.push_back(storageCol[i][j]);
            TrMat.val.push_back(storageValue[i][j]);
        }
    int sum = 0;
    for (int i = 0; i < matr.size; i++) {
        sum += static_cast<int>(storageCol[i].size());
        TrMat.rowNum.push_back(sum);
    }
    return TrMat;
}

crs_mat multiplicateMatrix(crs_mat A, crs_mat B) {
    crs_mat C;
    if (A.size != B.size)
        throw(std::string)"incorrect size";
    C.size = A.size;
    std::vector<std::complex<double>> matC_val;
    std::vector<int> matC_colNum;
    std::vector<int> matC_rowNum;
    B = transposeMatrixGustavson(B);
    matC_rowNum.push_back(0);
    for (int rowA = 0; rowA < A.size; rowA++) {
        for (int rowB = 0; rowB < A.size; rowB++) {
            std::complex<double> scalsum = { 0.0, 0.0 };
            for (int k = A.rowNum[rowA]; k < A.rowNum[rowA + 1]; k++) {
                for (int n = B.rowNum[rowB]; n < B.rowNum[rowB + 1]; n++) {
                    if (A.colNum[k] == B.colNum[n]) {
                        scalsum += A.val[k] * B.val[n];
                        break;
                    }
                }
            }
            if (std::abs(scalsum) >= DBL_MIN) {
                matC_val.push_back(scalsum);
                matC_colNum.push_back(rowB);
            }
        }
        matC_rowNum.push_back(matC_val.size());
    }
    for (int j = 0; j < static_cast<int>(matC_colNum.size()); j++) {
        C.colNum.push_back(matC_colNum[j]);
        C.val.push_back(matC_val[j]);
    }
    for (int i = 0; i <= A.size; i++) {
        C.rowNum.push_back(matC_rowNum[i]);
    }
    return C;
}

crs_mat omp_multiplicateMatrix(crs_mat A, crs_mat B) {
    crs_mat C;
    if (A.size != B.size)
        throw(std::string)"incorrect size";
    C.size = A.size;
    std::vector<std::vector<std::complex<double>>> matC_val(A.size);
    std::vector<std::vector<int>> matC_colNum(A.size);
    std::vector<int> matC_rowNum;
    B = transposeMatrixGustavson(B);
    matC_rowNum.push_back(0);
    int n, j;

    #pragma omp parallel
    {
        std::vector<int> tmp(C.size, -1);
        #pragma omp for private(j, n) schedule(static)
        for (int rowA = 0; rowA < A.size; rowA++) {
            tmp.assign(A.size, -1);
            int indexfirst = A.rowNum[rowA];
            int indexSecond = A.rowNum[rowA + 1];

            for (j = indexfirst; j < indexSecond; j++) {
                int column = A.colNum[j];
                tmp[column] = j;
            }

            for (j = 0; j < A.size; j++) {
                std::complex<double> scalsum = { 0.0, 0.0 };
                for (n = B.rowNum[j]; n < B.rowNum[j + 1]; n++) {
                    if (tmp[B.colNum[n]] != -1) {
                        scalsum += A.val[tmp[B.colNum[n]]] * B.val[n];
                    }
                }

                if (std::abs(scalsum) >= DBL_MIN) {
                    matC_val[rowA].push_back(scalsum);
                    matC_colNum[rowA].push_back(j);
                }
            }
        }
    }

    for (int j = 0; j < static_cast<int>(A.size); j++) {
        for (int i = 0; i < static_cast<int>(matC_colNum[j].size()); i++) {
            C.colNum.push_back(matC_colNum[j][i]);
            C.val.push_back(matC_val[j][i]);
        }
        matC_rowNum.push_back(C.val.size());
    }
    for (int i = 0; i <= A.size; i++) {
        C.rowNum.push_back(matC_rowNum[i]);
    }
    return C;
}
\end{lstlisting}
\begin{lstlisting}
//main.cpp

// Copyright 2021 Kryukov Sergey
#include <gtest/gtest.h>
#include <vector>
#include <algorithm>
#include <iostream>
#include "./sparsemat_omp.h"

TEST(omp_version, correct_transpose_mat) {
    int size = 3;
    std::vector<std::complex<double>> standartMat = {
        {0, 0}, {0, 2}, {0, 0},
        {0, 1}, {0, 0}, {0, 0},
        {0, 0}, {0, 1}, {0, 0}
    };
    crs_mat TransponationMat, sparseMat;
    sparseMat = createSparseMat(size, standartMat);
    TransponationMat = transposeMatrixGustavson(sparseMat);
    std::vector<int> rightRowNum = { 0, 1, 3, 3 };
    ASSERT_EQ(TransponationMat.rowNum, rightRowNum);
}

TEST(omp_version, correct_transpose_mat2) {
    int size = 4;
    std::vector<std::complex<double>> standartMat = {
        {0, 0}, {0, 2}, {0, 0}, {0, 0},
        {0, 1}, {0, 0}, {0, 0}, {0, 3},
        {0, 0}, {0, 1}, {0, 0}, {0, 0},
        {0, 0}, {0, 0}, {0, 0}, {0, 0}
    };
    crs_mat TransponationMat, sparseMat;
    sparseMat = createSparseMat(size, standartMat);
    TransponationMat = transposeMatrixGustavson(sparseMat);
    std::vector<int> rightRowNum = { 0, 1, 3, 3, 4 };
    ASSERT_EQ(TransponationMat.rowNum, rightRowNum);
}

TEST(omp_version, check_multiplic) {
    int size = 2;
    std::vector<std::complex<double>> standartMat1 = {
        {0, 0}, {2, 0},
        {1, 0}, {0, 0}
    };
    std::vector<std::complex<double>> standartMat2 = {
        {0, 0}, {2, 0},
        {2, 0}, {0, 0}
    };

    crs_mat SparseMat1, SparseMat2, SparseMatResult;
    SparseMat1 = createSparseMat(size, standartMat1);
    SparseMat2 = createSparseMat(size, standartMat2);
    SparseMatResult = multiplicateMatrix(SparseMat1, SparseMat2);
    std::vector<std::complex<double>> rightVal = { {4, 0}, {2, 0} };
    ASSERT_EQ(SparseMatResult.val, rightVal);
}

TEST(omp_version, check_multiplic2) {
    int size = 2;
    std::vector<std::complex<double>> standartMat1 = {
        {0, 0}, {0, 0},
        {0, 0}, {0, 0}
    };
    std::vector<std::complex<double>> standartMat2 = {
        {0, 0}, {2, 0},
        {2, 0}, {0, 0}
    };

    crs_mat SparseMat1, SparseMat2, SparseMatResult;
    SparseMat1 = createSparseMat(size, standartMat1);
    SparseMat2 = createSparseMat(size, standartMat2);
    SparseMatResult = multiplicateMatrix(SparseMat1, SparseMat2);
    std::vector<std::complex<double>> rightVal = {};
    ASSERT_EQ(SparseMatResult.val, rightVal);
}

TEST(omp_version, check_multiplic_time) {
    int size = 3000;
    double begin, end;
    crs_mat SparseMat1 = genDiagonalSparseMat(size);
    crs_mat SparseMat2 = genDiagonalSparseMat(size);
    begin = omp_get_wtime();
    crs_mat SparseMatResult = multiplicateMatrix(SparseMat1, SparseMat2);
    end = omp_get_wtime();
    std::cout << "Seq time = " << end - begin << "\n";

    begin = omp_get_wtime();
    crs_mat SparseMatResult_omp = omp_multiplicateMatrix(SparseMat1,
        SparseMat2);
    end = omp_get_wtime();
    std::cout << "OMP time = " << end - begin << "\n";

    EXPECT_EQ(SparseMatResult.val, SparseMatResult_omp.val);
    EXPECT_EQ(SparseMatResult.rowNum, SparseMatResult_omp.rowNum);
    EXPECT_EQ(SparseMatResult.colNum, SparseMatResult_omp.colNum);
}

int main(int argc, char **argv) {
    ::testing::InitGoogleTest(&argc, argv);
    return RUN_ALL_TESTS();
}
\end{lstlisting}

\par Реализация с использованием TBB
\begin{lstlisting}
//sparsemat_tbb.h

// Copyright 2021 Kryukov Sergey
#ifndef MODULES_TASK_3_KRYUKOV_S_SPARSE_MAT_TBB_SPARSEMAT_TBB_H_
#define MODULES_TASK_3_KRYUKOV_S_SPARSE_MAT_TBB_SPARSEMAT_TBB_H_

#include <tbb/tbb.h>
#include <vector>
#include <complex>
#include "tbb/task_scheduler_init.h"
#include "tbb/blocked_range.h"
#include "tbb/parallel_for.h"


struct crs_mat {
    int size;
    std::vector<std::complex<double>> val;
    std::vector<int> colNum, rowNum;
};
crs_mat createSparseMat(int size_,
    const std::vector<std::complex<double>> &vectMat);

crs_mat genDiagonalSparseMat(int size_);
std::vector<std::complex<double>> MultipluValues(
    const std::vector<std::complex<double>> & first,
    const std::vector<std::complex<double>> & second, const int size_);

crs_mat transposeMatrixGustavson(crs_mat matr);
crs_mat multiplicateMatrix(crs_mat A, crs_mat B);
crs_mat multiplicateMatrixTBB(crs_mat A, crs_mat B);
#endif  // MODULES_TASK_3_KRYUKOV_S_SPARSE_MAT_TBB_SPARSEMAT_TBB_H_
\end{lstlisting}
\begin{lstlisting}
//sparsemat_tbb.cpp
// Copyright 2021 Kryukov Sergey
#include<float.h>
#include <random>
#include <ctime>
#include <vector>
#include <string>
#include "../../../modules/task_3/kryukov_s_sparse_mat_tbb/sparsemat_tbb.h"

crs_mat createSparseMat(int size_,
    const std::vector<std::complex<double>> &vectMat) {
    crs_mat matrix;
    matrix.size = size_;
    matrix.rowNum.push_back(0);
    for (int i = 0; i < size_; i++) {
        for (int j = 0; j < size_; j++) {
            if (std::abs(vectMat[i*size_ + j]) < DBL_MIN)
                continue;
            matrix.val.push_back(vectMat[i*size_ + j]);
            matrix.colNum.push_back(j);
        }
        matrix.rowNum.push_back(matrix.colNum.size());
    }
    return matrix;
}

crs_mat genDiagonalSparseMat(int size_) {
    crs_mat resultMat;
    resultMat.size = size_;
    std::vector<std::complex<double>> values;
    std::vector<int> colNum, rowNum;
    rowNum.push_back(0);

    std::mt19937 gen;
    gen.seed(static_cast<unsigned int>(time(0)));
    for (int i = 0; i < size_; i++) {
        values.push_back(std::complex<double>(
            static_cast<double>(gen() % 100000),
            static_cast<double>(gen() % 100000)));
        colNum.push_back(i);
        int a = i + 1;
        rowNum.push_back(a);
    }

    resultMat.val = values;
    resultMat.rowNum = rowNum;
    resultMat.colNum = colNum;

    return resultMat;
}

std::vector<std::complex<double>> MultipluValues(
    const std::vector<std::complex<double>>& first,
    const std::vector<std::complex<double>> & second,
    const int size_) {

    std::vector<std::complex<double>> resultVec;
    std::complex<double> sum = { 0.0, 0.0 };
    for (int i = 0; i < size_; i++) {
        sum = first[i] * second[i];
        resultVec.push_back(sum);
    }

    return resultVec;
}

crs_mat transposeMatrixGustavson(crs_mat matr) {
    crs_mat TrMat;
    std::vector<std::vector<std::complex<double>>> storageValue(matr.size);
    std::vector<std::vector<int>> storageCol(matr.size);
    TrMat.rowNum.push_back(0);
    for (int i = 0; i < matr.size; i++) {
        for (int j = matr.rowNum[i]; j < matr.rowNum[i + 1]; j++) {
            storageValue[matr.colNum[j]].push_back(matr.val[j]);
            storageCol[matr.colNum[j]].push_back(i);
        }
    }
    for (int i = 0; i < matr.size; i++)
        for (int j = 0; j < static_cast<int>(storageCol[i].size()); j++) {
            TrMat.colNum.push_back(storageCol[i][j]);
            TrMat.val.push_back(storageValue[i][j]);
        }
    int sum = 0;
    for (int i = 0; i < matr.size; i++) {
        sum += static_cast<int>(storageCol[i].size());
        TrMat.rowNum.push_back(sum);
    }
    return TrMat;
}

crs_mat multiplicateMatrix(crs_mat A, crs_mat B) {
    crs_mat C;
    if (A.size != B.size)
        throw(std::string)"incorrect size";
    C.size = A.size;
    std::vector<std::complex<double>> matC_val;
    std::vector<int> matC_colNum;
    std::vector<int> matC_rowNum;
    B = transposeMatrixGustavson(B);
    matC_rowNum.push_back(0);
    for (int rowA = 0; rowA < A.size; rowA++) {
        for (int rowB = 0; rowB < A.size; rowB++) {
            std::complex<double> scalsum = { 0.0, 0.0 };
            for (int k = A.rowNum[rowA]; k < A.rowNum[rowA + 1]; k++) {
                for (int n = B.rowNum[rowB]; n < B.rowNum[rowB + 1]; n++) {
                    if (A.colNum[k] == B.colNum[n]) {
                        scalsum += A.val[k] * B.val[n];
                        break;
                    }
                }
            }
            if (std::abs(scalsum) >= DBL_MIN) {
                matC_val.push_back(scalsum);
                matC_colNum.push_back(rowB);
            }
        }
        matC_rowNum.push_back(matC_val.size());
    }
    for (int j = 0; j < static_cast<int>(matC_colNum.size()); j++) {
        C.colNum.push_back(matC_colNum[j]);
        C.val.push_back(matC_val[j]);
    }
    for (int i = 0; i <= A.size; i++) {
        C.rowNum.push_back(matC_rowNum[i]);
    }
    return C;
}

crs_mat multiplicateMatrixTBB(crs_mat A, crs_mat B) {
    crs_mat C;
    if (A.size != B.size)
        throw(std::string)"incorrect size";
    C.size = A.size;
    tbb::task_scheduler_init init;
    std::vector<std::vector<std::complex<double>>> matC_val(A.size);
    std::vector<std::vector<int>> matC_colNum(A.size);
    std::vector<int> matC_rowNum;
    B = transposeMatrixGustavson(B);
    matC_rowNum.push_back(0);

    tbb::parallel_for(tbb::blocked_range<int>(0, A.size, 50),
        [&](const tbb::blocked_range<int> & r) {
        int start = r.begin();
        int finish = r.end();
        std::vector<int>tmp(A.size);

        for (int rowA = start; rowA < finish; rowA++) {
            tmp.assign(A.size, -1);

            int indexfirst = A.rowNum[rowA];
            int indexSecond = A.rowNum[rowA + 1];

            for (int j = indexfirst; j < indexSecond; j++) {
                 int column = A.colNum[j];
                 tmp[column] = j;
            }

            for (int j = 0; j < A.size; j++) {
                std::complex<double> scalsum = { 0.0, 0.0 };
                for (int n = B.rowNum[j]; n < B.rowNum[j + 1]; n++) {
                    if (tmp[B.colNum[n]] != -1) {
                    scalsum += A.val[tmp[B.colNum[n]]] * B.val[n];
                    }
                }

                if (std::abs(scalsum) >= DBL_MIN) {
                    matC_val[rowA].push_back(scalsum);
                    matC_colNum[rowA].push_back(j);
                }
            }
        }
    });
    init.terminate();
    for (int j = 0; j < static_cast<int>(A.size); j++) {
        for (int i = 0; i < static_cast<int>(matC_colNum[j].size()); i++) {
            C.colNum.push_back(matC_colNum[j][i]);
            C.val.push_back(matC_val[j][i]);
        }
        matC_rowNum.push_back(C.val.size());
    }
    for (int i = 0; i <= A.size; i++) {
        C.rowNum.push_back(matC_rowNum[i]);
    }
    return C;
}
\end{lstlisting}
\begin{lstlisting}
//main.cpp

// Copyright 2021 Kryukov Sergey
#include <gtest/gtest.h>
#include <vector>
#include <algorithm>
#include <iostream>
#include "tbb/tick_count.h"
#include "./sparsemat_tbb.h"

TEST(tbb_version, correct_transpose_mat) {
    int size = 3;
    std::vector<std::complex<double>> standartMat = {
        {0, 0}, {0, 2}, {0, 0},
        {0, 1}, {0, 0}, {0, 0},
        {0, 0}, {0, 1}, {0, 0}
    };
    crs_mat TransponationMat, sparseMat;
    sparseMat = createSparseMat(size, standartMat);
    TransponationMat = transposeMatrixGustavson(sparseMat);
    std::vector<int> rightRowNum = { 0, 1, 3, 3 };
    ASSERT_EQ(TransponationMat.rowNum, rightRowNum);
}

TEST(tbb_version, correct_transpose_mat2) {
    int size = 4;
    std::vector<std::complex<double>> standartMat = {
        {0, 0}, {0, 2}, {0, 0}, {0, 0},
        {0, 1}, {0, 0}, {0, 0}, {0, 3},
        {0, 0}, {0, 1}, {0, 0}, {0, 0},
        {0, 0}, {0, 0}, {0, 0}, {0, 0}
    };
    crs_mat TransponationMat, sparseMat;
    sparseMat = createSparseMat(size, standartMat);
    TransponationMat = transposeMatrixGustavson(sparseMat);
    std::vector<int> rightRowNum = { 0, 1, 3, 3, 4 };
    ASSERT_EQ(TransponationMat.rowNum, rightRowNum);
}

TEST(tbb_version, check_multiplic) {
    int size = 2;
    std::vector<std::complex<double>> standartMat1 = {
        {0, 0}, {2, 0},
        {1, 0}, {0, 0}
    };
    std::vector<std::complex<double>> standartMat2 = {
        {0, 0}, {2, 0},
        {2, 0}, {0, 0}
    };

    crs_mat SparseMat1, SparseMat2, SparseMatResult;
    SparseMat1 = createSparseMat(size, standartMat1);
    SparseMat2 = createSparseMat(size, standartMat2);
    SparseMatResult = multiplicateMatrix(SparseMat1, SparseMat2);
    std::vector<std::complex<double>> rightVal = { {4, 0}, {2, 0} };
    ASSERT_EQ(SparseMatResult.val, rightVal);
}

TEST(tbb_version, check_multiplic2) {
    int size = 2;
    std::vector<std::complex<double>> standartMat1 = {
        {0, 0}, {0, 0},
        {0, 0}, {0, 0}
    };
    std::vector<std::complex<double>> standartMat2 = {
        {0, 0}, {2, 0},
        {2, 0}, {0, 0}
    };

    crs_mat SparseMat1, SparseMat2, SparseMatResult;
    SparseMat1 = createSparseMat(size, standartMat1);
    SparseMat2 = createSparseMat(size, standartMat2);
    SparseMatResult = multiplicateMatrix(SparseMat1, SparseMat2);
    std::vector<std::complex<double>> rightVal = {};
    ASSERT_EQ(SparseMatResult.val, rightVal);
}

TEST(tbb_version, check_multiplic_time) {
    int size = 3000;
    crs_mat SparseMat1 = genDiagonalSparseMat(size);
    crs_mat SparseMat2 = genDiagonalSparseMat(size);

    auto begin = tbb::tick_count::now();
    crs_mat SparseMatResult = multiplicateMatrix(SparseMat1, SparseMat2);
    auto end = tbb::tick_count::now();
    std::cout << "Seq time = " << (end - begin).seconds() << "\n";

    begin = tbb::tick_count::now();
    crs_mat SparseMatResult_tbb = multiplicateMatrixTBB(SparseMat1,
        SparseMat2);
    end = tbb::tick_count::now();
    std::cout << "TBB time = " << (end - begin).seconds() << "\n";

    EXPECT_EQ(SparseMatResult.val, SparseMatResult_tbb.val);
    EXPECT_EQ(SparseMatResult.rowNum, SparseMatResult_tbb.rowNum);
    EXPECT_EQ(SparseMatResult.colNum, SparseMatResult_tbb.colNum);
}

int main(int argc, char **argv) {
    ::testing::InitGoogleTest(&argc, argv);
    return RUN_ALL_TESTS();
}
\end{lstlisting}

\end{document}
