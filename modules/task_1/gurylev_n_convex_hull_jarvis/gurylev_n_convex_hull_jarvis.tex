\documentclass{report}

\usepackage[T2A]{fontenc}
\usepackage[utf8]{luainputenc}
\usepackage[english, russian]{babel}
\usepackage[pdftex]{hyperref}
\usepackage[14pt]{extsizes}
\usepackage{listings}
\usepackage{color}
\usepackage{geometry}
\usepackage{enumitem}
\usepackage{multirow}
\usepackage{graphicx}
\usepackage{indentfirst}
\usepackage{amsmath}

\geometry{a4paper,top=2cm,bottom=3cm,left=2cm,right=1.5cm}
\setlength{\parskip}{0.5cm}
\setlist{nolistsep, itemsep=0.3cm,parsep=0pt}

\lstset{language=C++,
		basicstyle=\footnotesize,
		keywordstyle=\color{blue}\ttfamily,
		stringstyle=\color{red}\ttfamily,
		commentstyle=\color{green}\ttfamily,
		morecomment=[l][\color{magenta}]{\#}, 
		tabsize=4,
		breaklines=true,
  		breakatwhitespace=true,
  		title=\lstname,       
}

\addto\captionsrussian{\renewcommand \contentsname {\LARGE Содержание}}

\makeatletter
\renewcommand\@biblabel[1]{#1.\hfil}
\makeatother

\begin{document}

\begin{titlepage}

\begin{center}
МИНИСТЕРСТВО НАУКИ И ВЫСШЕГО ОБРАЗОВАНИЯ РОССИЙСКОЙ ФЕДЕРАЦИИ\\
Федеральное государственное автономное образовательное учреждение высшего образования \\
\textbf{"Национальный исследовательский Нижегородский государственный университет им. Н.И. Лобачевского" (ННГУ)}
\end{center}

\begin{center}
\textbf{Институт информационных технологий, математики и механики}
\end{center}

\vspace{4em}

\begin{center}
Отчет по лабораторной работе \\
\end{center}
\begin{center}
\textbf{\Large«Построение выпуклой оболочки - проход Джарвиса.»} \\
\end{center}

\vspace{4em}

\newbox{\lbox}
\savebox{\lbox}{\hbox{text}}
\newlength{\maxl}
\setlength{\maxl}{\wd\lbox}
\hfill\parbox{7cm}{
\hspace*{5cm}\hspace*{-5cm}\textbf{Выполнил:} \\ студент группы 381806-3 \\ Гурылев Н. С.\\
\\
\hspace*{5cm}\hspace*{-5cm}\textbf{Проверил:}\\ доцент кафедры МОСТ, \\ кандидат технических наук \\ Сысоев А. В.\\
}
\vspace{\fill}

\begin{center} Нижний Новгород \\ 2021 \end{center}

\end{titlepage}

\setcounter{page}{2}
% содержание
\tableofcontents{}
\clearpage

\begin{center}
\section*{Введение}
\end{center}
\addcontentsline{toc}{section}{Введение}
\parАлгоритм Джарвиса (или алгоритм обхода Джарвиса, или алгоритм заворачивания подарка) определяет последовательность элементов множества, образующих выпуклую оболочку для этого множества. Метод можно представить как обтягивание верёвкой множества вбитых в доску гвоздей. Алгоритм работает за время O(nh), где n — общее число точек на плоскости, h — число точек в выпуклой оболочке.
\par Для практических целей выпуклые оболочки полезны тем, что они компактно хранят всю необходимую информацию о множестве точек, что позволяет быстро отвечать на разнообразные запросы на этом множестве.
\par Выпуклые оболочки можно рассматривать в любом пространстве, но мы в этой лабораторной  работе ограничимся двумерным.
\newpage

\begin{center}
\section*{Постановка задач}
\end{center}
\addcontentsline{toc}{section}{Постановка задач}
В рамках данной лабораторной работы необходимо:
\begin{enumerate}
\item Реализовать последовательный алгоритм построения выпуклой оболочки методом Джарвиса.
\item Реализовать параллельный алгоритм построения выпуклой оболочки методом Джарвиса с использованием технологии OpenMP.
\item Реализовать параллельный алгоритм построения выпуклой оболочки методом Джарвиса с использованием технологии TBB.
\item Провести вычислительные эксперементы, а так же проверить корректность работы при помощи тестов, создданых с помощью Google C ++ Testing Framework.
\item Сделать вывод на основе полученных результатов.
\end{enumerate}
\newpage

\begin{center}
\section*{Описание алгоритма}
\end{center}
\addcontentsline{toc}{section}{Описание алгоритма}
\textbf{Алгоритм построения выпуклой оболочки методом Джарвиса:}
\par Выберем какую-нибудь точку p0, которая гарантированно попадёт в выпуклую оболочку. Например, нижнюю, а если таких несколько, то самую левую из них.
\par Дальше будем действовать так: найдём самую «правую» точку от последней добавленной (то есть точку с минимальным полярным углом) и добавим её в оболочку. Будем так итеративно добавлять точки, пока не «замкнёмся», то есть пока самой правой точкой не станет p0.
\par Алгоритм Джарвиса также называют алгоритмом заворачивания подарка: мы каждый раз находим самый близкий «угол».
\par Для каждой точки выпуклой оболочки (обозначим их количество за h) мы из всех оставшихся O(n) точек будем искать оптимальную, что суммарно будет работать за O(nh).
\par Для краткости, будем считать, что все точки различны, имеют целочисленные координаты, а также что нет трёх точек на одной прямой.
\newpage

\begin{center}
\section*{Схема распараллеливания}
\end{center}
\addcontentsline{toc}{section}{Схема распараллеливания}
\par Основной идеей распараллеливания алгоритма построения выпуклой оболочки методом Джарвиса является разбиение массива точек на блоки. Кол-вл блоков будет равняться кол-ву потоков.
\par При условии, что наш массив точек не будет делиться нацело на число потоков, то у нас появится остаток. Для того, чтобы посчитать остаток, мы учитываем эти точки в блоке последнего потока.
\par Далее мы говорим каждому потоку искать свою локальную точку и после завершения, сравнивать с точкой главного потока.
\newpage

\begin{center}
\section*{Описание программной реализации}
\end{center}
\addcontentsline{toc}{section}{Описание программной реализации}
\par Реализация последовательного алгоритма построения выпуклой оболочки методом Джарвиса представлена функцией:
\begin{lstlisting}
  std::vector < std::pair<int, int>> JarvisAlg(const std::vector<std::pair<int, int>>& points);
\end{lstlisting}
В качетве входного параметра передается вектор наших точек. На выходе вектор, содержащий МВО.
\par Реализация параллельных алгоритмов построения выпуклой оболочки методом Джарвиса представления функциями:
\begin{lstlisting}
  std::vector < std::pair<int, int>> JarvisAlgTbb(const std::vector<std::pair<int, int>>&points);
\end{lstlisting}
\begin{lstlisting}
  std::vector < std::pair<int, int>> JarvisAlgOmp(const std::vector<std::pair<int, int>>&points);
\end{lstlisting}
На входе и выходе аналогичные параметры по сравнению с последовательной функцией.
\newpage

\begin{center}
\section*{Подтверждение корректности}
\end{center}
\addcontentsline{toc}{section}{Подтверждение корректности}
\par Для подтверждения корректности в программе реализована система тестов, разработанная с помощью использования Google C++ Testing Framework.
\par Она включает в себя тесты, проверяющие все алгоритмы написанной программы. А в паралльленой версии сравниваются значения последовательного алгоритма построения выпуклой оболочки методом Джарвиса с параллельным.
\par Успешное прохождение данных тестов доказывает корректность работы всей программы.
\par Дополнительно для подтверждения корректности в программе реализова-на демонстрация работы алгоритма визуально с использованием библиотекиOpenCV.
\newpage

\begin{center}
\section*{Результаты эксперементов}
\end{center}
\addcontentsline{toc}{section}{Результаты эксперементов}
\par Вычислительные эксперименты для оценки эффективности алгоритма умножения двух разреженных матриц в столбцовом формате хранения проводились на оборудовании со следующей аппаратной конфигурацией:
\begin{enumerate}
\item Процессор: Intel(R) Core(TM) i5-4460 CPU @ 3.20GHz 3.20GHz, 4 ядра.
\item Оперативная память: 8 Gb;
\item Операционная система: Windows 10;
\end{enumerate}
В рамках нашего теста происходило вычисление на разном кол-ве точек: 1500000, 2000000 и 2500000.
\begin{table}[!h]
\caption{Сравнение последовательного алгоритма с OpenMP}
\centering
\begin{tabular}{|p{4cm}|p{4cm}|p{4cm}|p{3cm}|}
\hline
Кол-во точек & Последовательный алгоритм,сек & Параллельный алгоритм,сек & Ускорение  \\\hline
1500000 & 0.032486 & 0.0333403 & 1.0877  \\\hline
2000000  & 0.425559 & 0.1280879 & 3.351  \\\hline
2500000  & 0.708678 & 0.3276799 & 3.926  \\\hline
\end{tabular}
\end{table}

\begin{table}[!h]
\caption{Сравнение последовательного алгоритма с TBB}
\centering
\begin{tabular}{|p{4cm}|p{4cm}|p{4cm}|p{3cm}|}
\hline
Размер матрицы & Последовательный алгоритм,сек & Параллельный алгоритм,сек & Ускорение  \\\hline
1500000  & 0.232486 & 0.0769763 & 1.238  \\\hline
2000000  & 0.300169 & 0.0869188 & 3.726  \\\hline
2500000  & 0.413325 & 0.1247168 & 3.882  \\\hline
\end{tabular}
\end{table}

\par Исходя из результатов таблицы можно смело сделать вывод, что при распараллеливании мы выигрываем во времени. Самое максимальное ускокрение показало реузльтат в 4 раза.
\newpage

\begin{center}
\section*{Заключение}
\end{center}
\addcontentsline{toc}{section}{Заключение}
\par В ходе данной лабораторной работы было реализовано 3 алгоритма: последовательная версия построения выпуклой оболочки методом Джарвиса; паралельное с использованием технологии OpenMP; параллельное с использование технологии Tbb. Также были разработаны тесты, созданные для данного программного проекта с использованием Google C++ Testing Framework, необходимые для подтверждения корректности работы программы.
\par Перед мной стояла основная задача - преимущество параллельной реализации перед последовательной. Как можно заметить из реузльтатов показательных значений таблицы, это удалось выполнить.
\newpage

\addcontentsline{toc}{section}{Литература}
\begin{thebibliography}{}
\bibitem{1} Гергель В. П., Стронгин Р. Г. Основы параллельных вычислений для многопроцессорных вычислительных систем. – 2003.
\bibitem{2} https://ru.wikipedia.org/wiki/АлгоритмДжарвиса (дата обращения: 20.03.2021).
\bibitem{3} https://algorithmica.org/ru/convex-hulls (дата обращения: 20.03.2021).
\end{thebibliography}{}
\newpage

\begin{center}
\section*{Приложение}
\end{center}
\addcontentsline{toc}{section}{Приложение}
В этом разделе находится листинг всего кода, написанного в рамках лабораторной работы.
\begin{lstlisting}
// convex_hull_jarvis.h

// Copyright 2021 Gurylev Nikita

#ifndef MODULES_TASK_3_GURYLEV_N_CONVEX_HULL_JARVIS_CONVEX_HULL_JARVIS_H_
#define MODULES_TASK_3_GURYLEV_N_CONVEX_HULL_JARVIS_CONVEX_HULL_JARVIS_H_

#include <tbb/tbb.h>
#include <vector>
#include <utility>

std::vector<std::pair<int, int>> getRandomPoint(int n);
std::pair<int, int> SLeftMPoint(const std::vector<std::pair<int, int>>& points);
bool Dist(const std::pair<int, int> current,
    const std::pair<int, int> next, const std::pair<int, int> temp);
double Around(const std::pair<int, int>& A, const std::pair<int, int>& B,
    const std::pair<int, int>& C);
std::vector < std::pair<int, int>> JarvisAlg(const std::vector<std::pair<int,
    int>>& points);
std::vector < std::pair<int, int>> JarvisAlgOmp(const std::vector<std::pair<int,
    int>>&points);
std::vector < std::pair<int, int>> JarvisAlgTbb(const std::vector<std::pair<int,
    int>>&points);

#endif  // MODULES_TASK_3_GURYLEV_N_CONVEX_HULL_JARVIS_CONVEX_HULL_JARVIS_H_


// convex_hull_jarvis.cpp

// Copyright 2021 Gurylev Nikita

#include <random>
#include <vector>
#include <ctime>
#include <utility>

#include "tbb/parallel_reduce.h"
#include "tbb/blocked_range.h"
#include "../../../modules/task_3/gurylev_n_convex_hull_jarvis/convex_hull_jarvis.h"

std::vector<std::pair<int, int>> getRandomPoint(int size) {
    if (size <= 0) {
        throw("Invalid size!");
    }
    std::vector<std::pair<int, int>> points(size);
    std::default_random_engine random;
    random.seed(static_cast<unsigned int>(std::time(0)));
    for (int i = 0; i < size; i++) {
        int x = random() % 100;
        int y = random() % 100;
        points[i] = std::make_pair(x, y);
    }
    return points;
}

std::pair<int, int> SLeftMPoint(const std::vector<std::pair<int,
    int>>& points) {
    std::pair<int, int> point_left = points[0];
    for (size_t i = 1; i < points.size(); i++) {
        if (points[i] < point_left)
            point_left = points[i];
    }
    return point_left;
}

bool Dist(const std::pair<int, int> curr,
    const std::pair<int, int> next, const std::pair<int,
    int> tmp) {
    int dist_next_curr = (next.first - curr.first) * (next.first - curr.first) +
        (next.second - curr.second) * (next.second - curr.second);
    int dist_tmp_curr = (tmp.first - curr.first) * (tmp.first - curr.first) +
        (tmp.second - curr.second) * (tmp.second - curr.second);
    if (dist_next_curr < dist_tmp_curr)
        return true;
    return false;
}

double Around(const std::pair<int, int>& A,
    const std::pair<int, int>& B, const std::pair<int, int>& C) {
    return (C.first - A.first) * (B.second - A.second) -
        (C.second - A.second) * (B.first - A.first);
}

std::vector < std::pair<int, int>> JarvisAlg(const std::vector<std::pair<int,
    int>>& points) {
    size_t count_points = points.size();
    if (count_points == 0) {
        throw("Error");
    }
    std::pair<int, int> base = SLeftMPoint(points);
    std::vector < std::pair<int, int>> convex_hull;
    convex_hull.push_back(base);

    std::pair<int, int> curr = base;
    std::pair<int, int> next;
    do {
        next = curr == points[0] ? points[1] : points[0];

        for (size_t i = 0; i < count_points; i++) {
            double tmp = Around(curr, next, points[i]);
            if (tmp > 0) {
                next = points[i];
            } else if (tmp == 0) {
                if (Dist(curr, next, points[i])) {
                    next = points[i];
                }
            }
        }
        curr = next;
        convex_hull.push_back(next);
    } while (curr != base);
    convex_hull.pop_back();
    return convex_hull;
}

std::vector < std::pair<int, int>> JarvisAlgTbb(const std::vector<std::pair<int,
    int>>&points) {
    int count_points = static_cast<int>(points.size());
    if (count_points == 0)
        throw("Error");
    std::pair<int, int> base = tbb::parallel_reduce(
        tbb::blocked_range<size_t>(1, count_points),
        points[0],
        [&points](tbb::blocked_range<size_t> r, std::pair<int,
            int> local_base) -> const std::pair<int, int> {
            auto begin = r.begin(), end = r.end();
            for (auto i = begin; i != end; i++) {
                if (points[i] < local_base)
                    local_base = points[i];
            }
            return local_base;
        },
        [](const std::pair<int, int>& a, const std::pair<int,
            int>& b) -> const std::pair<int, int> {
            return a < b ? a : b;
        });

    std::vector<std::pair<int, int>> convex_hull;
    convex_hull.push_back(base);
    std::pair<int, int> curr = base;
    std::pair<int, int> next;
    do {
        next = curr == points[0] ? points[1] : points[0];

        curr = tbb::parallel_reduce(
            tbb::blocked_range<size_t>(0, count_points),
            next,
            [&curr, &points](tbb::blocked_range<size_t> r, std::pair<int,
                int> local_next) -> const std::pair<int, int> {
                auto begin = r.begin(), end = r.end();
                for (auto i = begin; i != end; i++) {
                    int temp = Around(curr, local_next, points[i]);
                    if (temp > 0) {
                        local_next = points[i];
                    } else if (temp == 0) {
                        if (Dist(curr, local_next, points[i]))
                            local_next = points[i];
                    }
                }
                return local_next;
            },
            [&curr](const std::pair<int, int>& next, const std::pair<int, 
                int>& local_next) -> const std::pair<int, int> {
                int temp = Around(curr, next, local_next);
                if (temp > 0) {
                    return local_next;
                } else if (temp == 0) {
                    if (Dist(curr, next, local_next))
                        return local_next;
                }
                return next;
            });
        next = curr;
        convex_hull.push_back(next);
    } while (curr != base);
    convex_hull.pop_back();
    return convex_hull;
}

std::vector < std::pair<int, int>> JarvisAlgOmp(const std::vector<std::pair<int,
    int>>&points) {
    int count_points = static_cast<int>(points.size());
    if (count_points == 0)
        throw("Error");
    std::pair<int, int> base = points[0];
#pragma omp parallel shared(points)
    {
        std::pair<int, int> local_base(base);
#pragma omp for
        for (int i = 1; i < count_points; i++) {
            if (points[i] < local_base)
                local_base = points[i];
        }
#pragma omp critical
        {
            if (local_base < base)
                base = local_base;
        }
    }
    std::vector<std::pair<int, int>> convex_hull;
    convex_hull.push_back(base);
    std::pair<int, int> curr = base;
    std::pair<int, int> next;
    do {
        next = curr == points[0] ? points[1] : points[0];
#pragma omp parallel shared(points)
        {
            std::pair<int, int> local_next = next;
#pragma omp for
            for (int i = 0; i < count_points; i++) {
                int tmp = Around(curr, local_next, points[i]);
                if (tmp > 0) {
                    local_next = points[i];
                } else if (tmp == 0) {
                    if (Dist(curr, local_next, points[i])) {
                        local_next = points[i];
                    }
                }
            }
#pragma omp critical
            {
                int tmp = Around(curr, next, local_next);
                if (tmp > 0) {
                    next = local_next;
                } else if (tmp == 0) {
                    if (Dist(curr, next, local_next)) {
                        next = local_next;
                    }
                }
            }
        }
        curr = next;
        convex_hull.push_back(next);
    } while (curr != base);
    convex_hull.pop_back();
    return convex_hull;
}


// main.cpp

// Copyright 2021 Gurylev Nikita

#include <gtest/gtest.h>
#include <vector>
#include <utility>

#include "./convex_hull_jarvis.h"
#include "tbb/tick_count.h"

TEST(ConvexHull, test1) {
    std::vector<std::pair<int, int>> points = getRandomPoint(1550000);
    tbb::tick_count time1 = tbb::tick_count::now();
    std::vector<std::pair<int, int>> alg_seq = JarvisAlg(points);
    tbb::tick_count time2 = tbb::tick_count::now();
    std::cout << "Seq time: " <<
        static_cast<double>((time2 - time1).seconds()) << std::endl;
    time1 = tbb::tick_count::now();
    std::vector<std::pair<int, int>> alg_tbb = JarvisAlgTbb(points);
    time2 = tbb::tick_count::now();
    std::cout << "Tbb time: " <<
        static_cast<double>((time2 - time1).seconds()) << std::endl;
    ASSERT_EQ(alg_seq, alg_tbb);
}

TEST(ConvexHull, test2) {
    std::vector<std::pair<int, int>> points = getRandomPoint(2000000);
    tbb::tick_count time1 = tbb::tick_count::now();
    std::vector<std::pair<int, int>> alg_seq = JarvisAlg(points);
    tbb::tick_count time2 = tbb::tick_count::now();
    std::cout << "Seq time: " <<
        static_cast<double>((time2 - time1).seconds()) << std::endl;
    time1 = tbb::tick_count::now();
    std::vector<std::pair<int, int>> alg_tbb = JarvisAlgTbb(points);
    time2 = tbb::tick_count::now();
    std::cout << "Tbb time: " <<
        static_cast<double>((time2 - time1).seconds()) << std::endl;
    ASSERT_EQ(alg_seq, alg_tbb);
}

TEST(ConvexHull, test3) {
    std::vector<std::pair<int, int>> points = { std::pair<int, int>(0, 0),
        std::pair<int, int>(3, 3), std::pair<int, int>(6, 6),
        std::pair<int, int>(7, 7), std::pair<int, int>(9, 9) };
    std::vector<std::pair<int, int>> alg_seq = JarvisAlg(points);
    std::vector<std::pair<int, int>> alg_tbb = JarvisAlgTbb(points);
    ASSERT_EQ(alg_seq, alg_tbb);
}

TEST(ConvexHull, test4) {
    std::vector<std::pair<int, int>> points = { std::pair<int, int>(0, 0),
        std::pair<int, int>(34, 26), std::pair<int, int>(21, 39),
        std::pair<int, int>(0, 50), std::pair<int, int>(4, 15),
        std::pair<int, int>(36, 41), std::pair<int, int>(16, 35),
        std::pair<int, int>(8, 27), std::pair<int, int>(50, 50),
        std::pair<int, int>(50, 0) };
    std::vector<std::pair<int, int>> alg_seq = JarvisAlg(points);
    std::vector<std::pair<int, int>> alg_tbb = JarvisAlgTbb(points);
    ASSERT_EQ(alg_seq, alg_tbb);
}

TEST(ConvexHull, test5) {
    std::vector<std::pair<int, int>> points = { std::pair<int, int>(3, 8),
        std::pair<int, int>(60, 11), std::pair<int, int>(90, 40),
        std::pair<int, int>(60, 55),
        std::pair<int, int>(13, 33) };
    std::vector<std::pair<int, int>> alg_seq = JarvisAlg(points);
    std::vector<std::pair<int, int>> alg_tbb = JarvisAlgTbb(points);
    ASSERT_EQ(alg_seq, alg_tbb);
}
\end{lstlisting}
\end{document}
