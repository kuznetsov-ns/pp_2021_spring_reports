\documentclass{report}

\usepackage[T2A]{fontenc}
\usepackage[utf8]{luainputenc}
\usepackage[english, russian]{babel}
\usepackage[pdftex]{hyperref}
\usepackage[14pt]{extsizes}
\usepackage{listings}
\usepackage{color}
\usepackage{geometry}
\usepackage{indentfirst}
\usepackage{pgfplots}
\usepackage{amsmath}
\usepackage{mathtools}

\geometry{a4paper,top=2cm,bottom=3cm,left=2cm,right=1.5cm}
\setlength{\parskip}{0.5cm}

\lstset{language=C++,
		basicstyle=\footnotesize,
		keywordstyle=\color{blue}\ttfamily,
		stringstyle=\color{red}\ttfamily,
		commentstyle=\color{green}\ttfamily,
		morecomment=[l][\color{magenta}]{\#}, 
		tabsize=4,
		breaklines=true,
  		breakatwhitespace=true,
  		title=\lstname,       
}
%Переименовываем Оглавление в Содержание
\addto\captionsrussian{\renewcommand \contentsname {\LARGE Содержание}}
\DeclareUnicodeCharacter{2212}{-}
\DeclareUnicodeCharacter{2217}{+}
\DeclareUnicodeCharacter{22C5}{+}
\begin{document}
\begin{titlepage}
\begin{center}
Министерство образования и науки Российской федерации \\
Федеральное государственное автономное образовательное учреждение высшего образования \\
Национальный исследовательский Нижегородский государственный университет им. Н.И. Лобачевского \\
Институт информационных технологий, математики и механики \\
Направление подготовки: «Фундаментальная информатика и информационные технологии»
\end{center}

\vspace{4em}

\begin{center}
\textbf{\Large Отчет по лабораторной работе}
\end{center}
\begin{center}
\textbf{\Large «Выделение ребер на изображении с использованием оператора Собеля}
\end{center}

\vspace{4em}

\newbox{\lbox}
\savebox{\lbox}{\hbox{text}}
\newlength{\maxl}
\setlength{\maxl}{\wd\lbox}
\hfill\parbox{7cm}{
\textbf{Выполнил:} \\ 
студент группы 381806-1 \\ 
Лютова Т.С.\\
\\
\textbf{Проверил:}\\ 
доцент кафедры МОСТ, \\ 
кандидат технических наук \\ 
Сысоев А. В.\\ }
\vspace{\fill}

\begin{center} Нижний Новгород \\ 2021 \end{center}

\end{titlepage}

\setcounter{page}{2}
% Содержание
\tableofcontents
\newpage

% Введение
\section*{Введение}
\addcontentsline{toc}{section}{Введение}
Одним из наиболее важных операторов в обнаружении краев пиксельных изображений является оператор Собеля. Он играет ключевую роль в таких областях информационных технологий, как машинное обучение, цифровые носители и компьютерное зрение. Технически это дискретный оператор разности первого порядка, который используется для вычисления приблизительного значения шага функции яркости изображения. Использование этого оператора в любой точке изображения сгенерирует вектор градиента, соответствующий этой точке или ее вектору нормали.
\par Как уже было сказано, оператор Собеля вычисляет градиент яркости изображения в каждой точке. То есть находится направление наибольшего увеличения яркости и величина её изменения в этом направлении. Результат показывает, насколько «резко» или «плавно» меняется яркость изображения в каждой точке.

Математически, градиент функции двух переменных для каждой точки изображения  — двумерный вектор, компонентами которого являются производные яркости изображения по горизонтали и вертикали. В каждой точке изображения градиентный вектор ориентирован в направлении наибольшего увеличения яркости, а его длина соответствует величине изменения яркости. Это означает, что результатом оператора Собеля в точке области постоянной яркости будет нулевой вектор, а в точке, лежащей на границе областей различной яркости — вектор, пересекающий границу в направлении увеличения яркости.

\par В пространственной области оператор Собеля легко реализовать, имеет высокую скорость выполнения, сглаживает некоторые шумы и может предоставить более точную информацию о направлении кромки. Недостатком является недостаточная точность позиционирования кромок. Край относится к границе между объектом и другим объектом. Как правило, значения серого внутри и снаружи края будут различаться. Оператор Собеля вычисляется с помощью операции взвешивания по шкале серого соседних точек пиксельного пространства вверх и вниз, влево и вправо.

\newpage

% Постановка задачи
\section*{Постановка задачи}
\addcontentsline{toc}{section}{Постановка задачи}
В рамках лабораторной работы необходимо реализовать последовательный и параллельный алгоритмы выделения ребер на изображении с использованием оператора Собеля.Реализацию парааллельного алгоритма необходимо выполнить с помощью технологий OpenMP и TBB. Кроме того, необходимо  проверить корректность выполения работы алгоритмов, а также оценить их эффективность и сделать выводы на основе полученных результатов.
\newpage

% Описание алгоритма
\section*{Описание алгоритма}
\addcontentsline{toc}{section}{Описание алгоритма}
Оператор Собеля основан на свёртке изображения небольшими сепарабельными целочисленными фильтрами в вертикальном и горизонтальном направлениях, поэтому его относительно легко вычислять. С другой стороны, используемая им аппроксимация градиента достаточно грубая,что особенно сказывается на высокочастотных колебаниях изображения. 

\par  Оператор использует ядра 3×3, с которыми свёртывают исходное изображение для вычисления приближенных значений производных по горизонтали и по вертикали.\\
\par Пусть A исходное изображение, а $G_x и G_y$ — два изображения, где каждая точка содержит приближенные производные по x и по y. Они вычисляются следующим образом (матрицы могут заполняться рандомно): \\
$G_x$ = A *  
$
\begin{bmatrix}
    -1 & -2 & -1\\
     0 & 0 & 0\\
    +1 & +2 & +1
\end{bmatrix}
$
  $G_y$ = A * 
$
\begin{bmatrix}
    -1 & 0 & +1\\
    -2 & 0 & +2\\
    -1 & 0 & +1
\end{bmatrix} \\
$

где * обозначает двумерную операцию свёртки.
\par  Операция свёртки (по опредениею) -  операция над парой матриц A (размера $n_x$ × $n_y$) и B (размера  $m_x$ × $m_y$), результатом которой является матрица C = A ∗ B размера ($n_x$ - $m_x$ + 1) × ($n_y$ − $m_y$ + 1). Каждый элемент результата вычисляется как скалярное произведение матрицы B и некоторой подматрицы A такого же размера.\\ 
\par Координата х в нашем примере возрастает «направо», а y — «вниз». В каждой точке изображения приближённое значение градиента можно посчитать, используя приближенные значения производных:
G = $\sqrt {G_x^2 + G_y^2} $ \\
Все описанные операции выполняются за O(M⋅N) на одном изображении

\newpage

% Описание схемы распараллеливания
\section*{Описание схемы распараллеливания}
\addcontentsline{toc}{section}{Схема распараллеливания}
Данная задача реализует достаточно простой алгоритм, но при этом работает с большим изображением, поэтому распараллеливание было осуществлено с помощью
декомпозиции по данным. Для этого общее количество строк изображения разбивалось на требуемое
количество потоков, и каждый поток получал свои граничные значения, исходя из которых и производил обработку.

\newpage

\newpage
% Описание программной реализации
\section*{Описание программной реализации}
\addcontentsline{toc}{section}{Описание программной реализации}

Для начала мы создаем структуру, в которой задаём наше изображение и перегружаем некоторые операции, необходимые нам для дальнейшей реализации.
\begin{lstlisting}
 struct Image {...}
\end{lstlisting}

Для того, чтобы не вводить значения матриц строк и столбцов вручную, была реализована функция рандомного заполнения:
\begin{lstlisting}
Image createRandomImage(int rows, int cols);
\end{lstlisting}
 В качестве входных параметров передаются соответсвенно строки и столбцы\\
 
 Алгоритм последовательной реализации выделения границ на изображении с помощью оператора Собеля вызывается с помощью функции:
 \begin{lstlisting}
Image sobel(const Image &image);
\end{lstlisting}
В качестве входного параметра передается изображение, на котором необходимо реализовать оператор Собеля.\\

Алгоритм параллельной реализации вызывается с помощью функции:
 \begin{lstlisting}
Image sobelParallel(const Image &image);
\end{lstlisting}
Аналогично функции для последовательной реализации в качестве аргумента передается изображение.
\newpage

% Подтверждение корректности
\section*{Подтверждение корректности}
\addcontentsline{toc}{section}{Подтверждение корректности}
Для подтверждения корректности в программе представлен набор тестов, разработанных с помощью использования Google C++ Testing Framework.
\par Набор представляет из себя тесты, которые проверяют корректность входных данных: Например, если была некорректно задача матрица (с нулевой размерностью, прямоугольного вида и т.д.), то срабатывает исключение, а также эффективность выполнения последовательной и параллельной реализаций программы.
\par Успешное прохождение всех тестов доказывает корректность работы программного комплекса.
\newpage


% Результаты экспериментов
\section*{Результаты экспериментов}
\addcontentsline{toc}{section}{Результаты экспериментов}
Вычислительные эксперименты для оценки эффективности параллельной быстрой сортировки проводились на оборудовании со следующей аппаратной конфигурацией:
\begin{itemize}
\item Процессор:Intel(R) Core(TM) i5-8250 CPU, тактовая частота 1.60GHz и 1.80GHz, ядер:4;
\item Оперативная память: 8 ГБ (DDR4);
\item ОС: Microsoft Windows $10$
\end{itemize}
\par Чтобы сравнить скорость выполнения различных реализаций программы, была выбрана матрица размера 5000 * 5000

Результаты экспериментов представлены в Таблице 1.

\begin{table}[h]
    \begin{tabular}{ | p{4cm} | p{4cm} | p{4cm} | p{4cm} | }
    \hline
    Реализация & Последовательная реализация, сек & Параллельная \par реализация, сек & Ускорение\\ \hline
    OpenMP & 1.35  & 0.57  & 2.37 \\ \hline
    Tbb    & 1.18  & 0.35  & 3.37 \\ \hline 
   
    \end{tabular}
    \caption{Результаты экспериментов}
\end{table}
\par По данным, полученным в результате экспериментов, можно сделать вывод о том, что параллельная реализация работает быстрее, чем последовательная. Это можно увидеть из таблицы по ускорению. Паларрельная реализация оказалась быстрее последовательной почти в 3 раза.
\newpage
 
 % Заключение
\section*{Заключение}
\addcontentsline{toc}{section}{Заключение}
В результате лабораторной работы были разработаны последовательная и параллельная реализации выделения границ на изображении с помощью оператора Собеля.
\par Основной задачей данной лабораторной работы была реализация параллельной версии, которая должна была быть эффективнее последовательной. Эта задача была успешно достигнута, о чем говорят результаты экспериментов, проведенных в ходе работы. 
\par Кроме того, были разработаны и доведены до успешного выполнения тесты, созданные для данного программного проекта с использованием Google C++ Testing Framework и необходимые для подтверждения корректности работы программы.
\newpage

% Список литературы
\begin{thebibliography}{1}
\addcontentsline{toc}{section}{Список литературы}
\bibitem{Algorithm} 
Алгоритмы выделения контуров изображений URL: https://habr.com/ru/post/114452/
\bibitem{Gergel} 
Гергель В. П. Теория и практика параллельных вычислений. – 2007.
\bibitem{Gergel} 
Гергель В. П., Стронгин Р. Г. Основы параллельных вычислений для многопроцессорных вычислительных систем. – 2003.
\bibitem{Korneev} 
Корнеев В. Д. Параллельное программирование в MPI //Новосибирск: Изд-во СО РАН. – 2000.
\end{thebibliography}
\newpage

% Приложение
\section*{Приложение}
\addcontentsline{toc}{section}{Приложение}
В данном разделе находится листинг всего кода, написанного в рамках лабораторной работы.
\par Реализация OpenMP
\begin{lstlisting}
//lyutova_t_sobel.h

// Copyright 2021 Lyutova Tanya
#ifndef MODULES_TASK_2_LYUTOVA_T_SOBEL_LYUTOVA_T_SOBEL_H_
#define MODULES_TASK_2_LYUTOVA_T_SOBEL_LYUTOVA_T_SOBEL_H_

#include <cmath>
#include <vector>

struct Image {
  std::vector<int> pixels;
  int rows = 0;
  int cols = 0;

  Image(const std::vector<int> &pixels, int rows, int cols)
      : pixels(pixels), rows(rows), cols(cols) {}

  Image(int rows, int cols) : pixels(rows * cols, 0), rows(rows), cols(cols) {}
  const int &operator()(int x, int y) const { return pixels[x * cols + y]; }
  int &operator()(int x, int y) { return pixels[x * cols + y]; }

  bool operator==(const Image &_img) const {
    if (rows == _img.rows && cols == _img.cols)
      return true;
    return false;
  }
};

Image createRandomImage(int rows, int cols);
Image sobel(const Image &image);
Image sobelParallel(const Image &image);

#endif  // MODULES_TASK_2_LYUTOVA_T_SOBEL_LYUTOVA_T_SOBEL_H_

//lyutova_t_sobel.cpp

// Copyright 2021 Lyutova Tanya
#include <random>
#include <vector>
#include "../../modules/task_2/lyutova_t_sobel/lyutova_t_sobel.h"

inline int clamp(int value, int min, int max) {
  if (value > max)
    return max;
  if (value < min)
    return min;
  return value;
}

Image sobel(const Image &image) {
  static const std::vector<int> matrixX = {-1, 2, -1, 0, 0, 0, 1, 2, 1};
  static const std::vector<int> matrixY = {-1, 0, 1, -2, 0, 2, -1, 0, 1};
  static const int matrixSize = 3;
  static const int matrixRadius = 1;
  Image filteredImage(image.rows, image.cols);
  for (int x = 0; x < image.rows; x++)
    for (int y = 0; y < image.cols; y++) {
      int valX = 0, valY = 0;
      for (int i = -matrixRadius; i <= matrixRadius; i++)
        for (int j = -matrixRadius; j <= matrixRadius; j++) {
          int index = (i + matrixRadius) * matrixSize + j + matrixRadius;
          int nearX = clamp(x + j, 0, image.rows - 1);
          int nearY = clamp(y + i, 0, image.cols - 1);
          valX += image(nearX, nearY) * matrixX[index];
          valY += image(nearX, nearY) * matrixY[index];
        }
      filteredImage(x, y) =
          clamp(static_cast<int>(std::sqrt(valX * valX + valY * valY)), 0, 255);
    }
  return filteredImage;
}

Image sobelParallel(const Image &image) {
  static const std::vector<int> matrixX = {-1, 2, -1, 0, 0, 0, 1, 2, 1};
  static const std::vector<int> matrixY = {-1, 0, 1, -2, 0, 2, -1, 0, 1};
  static const int matrixSize = 3;
  static const int matrixRadius = 1;
  Image filteredImage(image.rows, image.cols);
#pragma omp parallel for
  for (int x = 0; x < image.rows; x++)
    for (int y = 0; y < image.cols; y++) {
      int valX = 0, valY = 0;
      for (int i = -matrixRadius; i <= matrixRadius; i++)
        for (int j = -matrixRadius; j <= matrixRadius; j++) {
          int index = (i + matrixRadius) * matrixSize + j + matrixRadius;
          int nearX = clamp(x + j, 0, image.rows - 1);
          int nearY = clamp(y + i, 0, image.cols - 1);
          valX += image(nearX, nearY) * matrixX[index];
          valY += image(nearX, nearY) * matrixY[index];
        }
      filteredImage(x, y) =
          clamp(static_cast<int>(std::sqrt(valX * valX + valY * valY)), 0, 255);
    }
  return filteredImage;
}

Image createRandomImage(int rows, int cols) {
  Image image(rows, cols);
  std::random_device rd;
  std::mt19937 generator(rd());
  for (size_t i = 0; i < image.pixels.size(); i++)
    image.pixels[i] = std::abs(static_cast<int>(generator())) % 256;
  return image;
}

//main.cpp
// Copyright 2020 Lyutova Tanya
#include <gtest/gtest.h>
#include <omp.h>
#include "../../modules/task_2/lyutova_t_sobel/lyutova_t_sobel.h"

TEST(OMP_Sobel, RandomImage) {
    Image image = createRandomImage(100, 100);
    ASSERT_EQ(image.cols, 100);
    ASSERT_EQ(image.rows, 100);
}

TEST(OMP_Sobel, SameAsSequential) {
  Image image = createRandomImage(100, 100);
  Image seq = sobel(image);
  Image par = sobelParallel(image);
  ASSERT_EQ(seq, par);
}

TEST(OMP_Sobel, DifferentImages) {
    Image image = createRandomImage(100, 100);
    Image res = sobelParallel(image);
    ASSERT_NE(image.pixels, res.pixels);
}

TEST(OMP_Sobel, SameResult) {
    Image image(4, 4);
    Image seq = sobel(image);
    Image par = sobelParallel(image);
    ASSERT_EQ(seq, par);
}

TEST(OMP_Sobel, Different) {
    Image image = createRandomImage(100, 100);
    Image image2(100, 100);
    ASSERT_NE(image.pixels, image2.pixels);
}


int main(int argc, char **argv) {
  ::testing::InitGoogleTest(&argc, argv);
  return RUN_ALL_TESTS();
}
\end{lstlisting}

\par Рефлизация TBB
\begin{lstlisting}
// lyutova_t_sobel.h
// Copyright 2021 Lyutova Tanya
#ifndef MODULES_TASK_3_LYUTOVA_T_SOBEL_LYUTOVA_T_SOBEL_H_
#define MODULES_TASK_3_LYUTOVA_T_SOBEL_LYUTOVA_T_SOBEL_H_

#include <cmath>
#include <vector>

struct Image {
  std::vector<int> pixels;
  int rows = 0;
  int cols = 0;

  Image(const std::vector<int> &pixels, int rows, int cols)
      : pixels(pixels), rows(rows), cols(cols) {}

  Image(int rows, int cols) : pixels(rows * cols, 0), rows(rows), cols(cols) {}
  const int &operator()(int x, int y) const { return pixels[x * cols + y]; }
  int &operator()(int x, int y) { return pixels[x * cols + y]; }

  bool operator==(const Image &_img) const {
    if (rows == _img.rows && cols == _img.cols)
      return true;
    return false;
  }
};

Image createRandomImage(int rows, int cols);
Image sobel(const Image &image);
Image sobelParallel(const Image &image);

#endif  // MODULES_TASK_3_LYUTOVA_T_SOBEL_LYUTOVA_T_SOBEL_H_

//lyutova_t_sobel.cpp
// Copyright 2021 Lyutova Tanya
#include <tbb/blocked_range.h>
#include <tbb/parallel_for.h>
#include <random>
#include <cmath>
#include <vector>
#include "../../modules/task_3/lyutova_t_sobel/lyutova_t_sobel.h"

inline int clamp(int value, int min, int max) {
    if (value > max)
        return max;
    if (value < min)
        return min;
    return value;
}

Image sobel(const Image &image) {
    static const std::vector<int> matrixX = { -1, 2, -1, 0, 0, 0, 1, 2, 1 };
    static const std::vector<int> matrixY = { -1, 0, 1, -2, 0, 2, -1, 0, 1 };
    static const int matrixSize = 3;
    static const int matrixRadius = 1;
    Image filteredImage(image.rows, image.cols);
    for (int x = 0; x < image.rows; x++)
        for (int y = 0; y < image.cols; y++) {
            int valX = 0, valY = 0;
            for (int i = -matrixRadius; i <= matrixRadius; i++)
                for (int j = -matrixRadius; j <= matrixRadius; j++) {
                    int index = (i + matrixRadius) * matrixSize + j +
                        matrixRadius;
                    int nearX = clamp(x + j, 0, image.rows - 1);
                    int nearY = clamp(y + i, 0, image.cols - 1);
                    valX += image(nearX, nearY) * matrixX[index];
                    valY += image(nearX, nearY) * matrixY[index];
                }
            filteredImage(x, y) =
                clamp(static_cast<int>(std::sqrt(valX * valX +
                    valY * valY)), 0, 255);
        }
    return filteredImage;
}

Image sobelParallel(const Image &image) {
    static const std::vector<int> matrixX = { -1, 2, -1, 0, 0, 0, 1, 2, 1 };
    static const std::vector<int> matrixY = { -1, 0, 1, -2, 0, 2, -1, 0, 1 };
    static const int matrixSize = 3;
    static const int matrixRadius = 1;
    Image filteredImage(image.rows, image.cols);
    tbb::parallel_for(tbb::blocked_range<int>(0, image.rows),
        [&](const tbb::blocked_range<int>& r) {
        for (int x = r.begin(); x < r.end(); x++)
            for (int y = 0; y < image.cols; y++) {
                int valX = 0, valY = 0;
                for (int i = -matrixRadius; i <= matrixRadius; i++)
                    for (int j = -matrixRadius; j <= matrixRadius; j++) {
                        int index = (i + matrixRadius) * matrixSize + j +
                            matrixRadius;
                        int nearX = clamp(x + j, 0, image.rows - 1);
                        int nearY = clamp(y + i, 0, image.cols - 1);
                        valX += image(nearX, nearY) * matrixX[index];
                        valY += image(nearX, nearY) * matrixY[index];
                    }
                filteredImage(x, y) = clamp(
                    static_cast<int>(std::sqrt(valX * valX +
                        valY * valY)), 0, 255);
            }
        });
    return filteredImage;
}

Image createRandomImage(int rows, int cols) {
    Image image(rows, cols);
    std::random_device rd;
    std::mt19937 generator(rd());
    for (size_t i = 0; i < image.pixels.size(); i++)
        image.pixels[i] = std::abs(static_cast<int>(generator())) % 256;
    return image;
}

//main.cpp
// Copyright 2021 Lyutova Tanya
#include <gtest/gtest.h>
#include <tbb/tick_count.h>
#include "../../modules/task_3/lyutova_t_sobel/lyutova_t_sobel.h"

TEST(TBB_Sobel, Performance) {
    Image image = createRandomImage(5000, 5000);
    tbb::tick_count t1, t2;
    t1 = tbb::tick_count::now();
    Image seq = sobel(image);
    t2 = tbb::tick_count::now();
    std::cout << "seq " << (t2 - t1).seconds() << std::endl;
    t1 = tbb::tick_count::now();
    Image par = sobelParallel(image);
    t2 = tbb::tick_count::now();
    std::cout << "par " << (t2 - t1).seconds() << std::endl;
    ASSERT_EQ(seq, par);
}

TEST(TBB_Sobel, RandomImage) {
    Image image = createRandomImage(100, 100);
    ASSERT_EQ(image.cols, 100);
    ASSERT_EQ(image.rows, 100);
}

TEST(TBB_Sobel, SameAsSequential) {
  Image image = createRandomImage(100, 100);
  Image seq = sobel(image);
  Image par = sobelParallel(image);
  ASSERT_EQ(seq, par);
}

TEST(TBB_Sobel, DifferentImages) {
    Image image = createRandomImage(100, 100);
    Image res = sobelParallel(image);
    ASSERT_NE(image.pixels, res.pixels);
}

TEST(TBB_Sobel, SameResult) {
    Image image(4, 4);
    Image seq = sobel(image);
    Image par = sobelParallel(image);
    ASSERT_EQ(seq, par);
}

TEST(TBB_Sobel, Different) {
    Image image = createRandomImage(100, 100);
    Image image2(100, 100);
    ASSERT_NE(image.pixels, image2.pixels);
}


int main(int argc, char **argv) {
  ::testing::InitGoogleTest(&argc, argv);
  return RUN_ALL_TESTS();
}
\end{lstlisting}

\end{document}