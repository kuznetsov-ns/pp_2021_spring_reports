\documentclass{report}

\usepackage[T2A]{fontenc}
\usepackage[utf8]{luainputenc}
\usepackage[english, russian]{babel}
\usepackage[pdftex]{hyperref}
\usepackage[14pt]{extsizes}
\usepackage{listings}
\usepackage{color}
\usepackage{geometry}
\usepackage{enumitem}
\usepackage{multirow}
\usepackage{graphicx}
\usepackage{indentfirst}
\usepackage{amsmath}
\usepackage{esint}

\geometry{a4paper,top=2cm,bottom=3cm,left=2cm,right=1.5cm}
\setlength{\parskip}{0.5cm}
\setlist{nolistsep, itemsep=0.3cm,parsep=0pt}

\lstset{language=C++,
		basicstyle=\footnotesize,
		keywordstyle=\color{blue}\ttfamily,
		stringstyle=\color{red}\ttfamily,
		commentstyle=\color{green}\ttfamily,
		morecomment=[l][\color{magenta}]{\#}, 
		tabsize=4,
		breaklines=true,
  		breakatwhitespace=true,
  		title=\lstname,       
}

\makeatletter
\renewcommand\@biblabel[1]{#1.\hfil}
\makeatother

\begin{document}

\begin{titlepage}

\begin{center}
Министерство науки и высшего образования Российской Федерации
\end{center}

\begin{center}
Федеральное государственное автономное образовательное учреждение высшего образования \\
Национальный исследовательский Нижегородский государственный университет им. Н.И. Лобачевского
\end{center}

\begin{center}
Институт информационных технологий, математики и механики
\end{center}

\vspace{4em}

\begin{center}
\textbf{\LargeОтчет по лабораторной работе} \\
\end{center}
\begin{center}
\textbf{\Large«Построение выпуклой оболочки – проход Грэхема»} \\
\end{center}

\vspace{4em}

\newbox{\lbox}
\savebox{\lbox}{\hbox{text}}
\newlength{\maxl}
\setlength{\maxl}{\wd\lbox}
\hfill\parbox{7cm}{
\hspace*{5cm}\hspace*{-5cm}\textbf{Выполнил:} \\ студент группы 381806-1 \\ Бакаева М.И.\\
\\
\hspace*{5cm}\hspace*{-5cm}\textbf{Проверил:}\\ доцент кафедры МОСТ, \\ Нестеров А.Ю.\\
}
\vspace{\fill}

\begin{center} Нижний Новгород \\ 2021 \end{center}

\end{titlepage}

\setcounter{page}{2}

% Содержание
\tableofcontents
\newpage

% Введение
\section*{Введение}
\addcontentsline{toc}{section}{Введение}
Существует множество задач вычислительной геометрии, связанных с построением выпуклой оболочки. На сегодняшний день задача хороше исследована и имеет широкое применение в обработке изображений, базах данных, распознавании объектов.
\par Понятие выпуклой оболочки является интуитивно понятным и простым - выпуклое множество точек, что все точки фигуры также лежат в нем. Существует несколько методов для построения оболочки: Грэхэма, Джарвиса и другие.
\par В данном лабораторной работе будет рассмотрен метод Грэхэма построения выпуклой оболочки. В ходе работы будет представлен подход к распараллеливанию задачи с помощью технологий OMP, TBB, STD Threads.
\newpage

% Постановка задачи
\section*{Постановка задачи}
\addcontentsline{toc}{section}{Постановка задачи}
В лабораторной работе необходимо реализовать последовательную и параллельную реализации построения выпуклой оболочки методом Грэхэма, проверить корректность работы алгоритмов, провести тесты для оценки времени. По полученным результатам сделать выводы.
\par Для реализации параллельной версии будут использоваться средства OMP, TBB, STD Threads. Для проверки корректности работы алгоритмов будут использзоваться Google C++ Testing Framework.
\newpage

% Метод решения
\section*{Метод решения}
\addcontentsline{toc}{section}{Метод решения}
Алгоритм Грэхэма:
\begin{itemize}
\item Ищем любую точку в множестве точек, которая точно входит в минимальную выпуклую оболочку.Это будет самая правая или самая левая точка из множества.
\item Стартовую точку перемещаем в начало списка всех точек
\item Сортируем все точки кроме первой по степени их левизны относительно стартовой точки А. B < C, если точка C находится по левую сторону от вектора AB.
\item После предыдущего шага мы получим невыпуклый прямоугольник. Нам необходимо проитись по всем вершинам и удалить те из них, в которых выполняется правый поворот (угол больше развернутого).
\end{itemize}
\newpage

% Схема распараллеливания
\section*{Схема распараллеливания}
\addcontentsline{toc}{section}{Схема распараллеливания базового алгоритма}
Зная количество потоков, на котором будет запущена программа, разделяем множество точек на равные части, при этом последнему потоку может достаться больше точек с учетом возможного остатка при неравном делении. В итоге каждый поток работает со своим множеством точек. Каждое такое множество пропускается через последовательный алгоритм. Таким образом мы избавляемся от внутренних ненужных точек.
\\ Далее необходимо воспользоваться критической секцией, чтобы собрать оставшиеся точки вместе в одном векторе. Последнм шагом необходимо пропустить итоговое множество точек через последовательный алгоритм Грэхэма. В итоге получаем минимальную выпуклую оболочку для исходного множества точек.
\newpage

% Описание программной реализации
\section*{Описание программной реализации}
\addcontentsline{toc}{section}{Описание программной реализации}
Алгоритм последовательного построения выпуклой оболочки методом Грэхэма вызывается функцией:
\begin{lstlisting}
std::vector<std::pair<double, double>> grahamScan(std::vector<std::pair<double, double>> setOfPoints);
\end{lstlisting}
\par В качестве входных параметров принимает - множество точек для построения оболочки. Возвращает множество точек выпуклой оболочки.
\par Алгоритм параллельного построения выпуклой оболочки методом Грэхэм вызывается функцией:
\begin{lstlisting}
std::vector<std::pair<double, double> > grahamScanParallel(std::vector<std::pair<double, double> >::iterator b, std::vector<std::pair<double, double> >::iterator e, int numberOfThreads);
\end{lstlisting}
\par В качестве входных параметров принимает - итераторы на начало и конец вектора точек, количество потоков. Возвращает множество точек выпуклой оболочки.

\newpage

% Подтверждение корректности
\section*{Подтверждение корректности}
\addcontentsline{toc}{section}{Подтверждение корректности}
Для подтверждения корректности в программе существует набор тестов, разработанных с использованием Google C++ Testing Framework.
\par 
Представленные тесты проверяют корректность вычислений, а именно сравниваются результаты линейной и параллельной реализаций. Также проверяется эффективность с помощью измерения времени, полученные результаты выводятся в консоль.
\par Успешное прохождение всех тестов доказывает корректность работы программы.
\newpage

% Результаты экспериментов
\section*{Результаты экспериментов}
\addcontentsline{toc}{section}{Результаты экспериментов}
Вычислительные эксперименты для оценки эффективности параллельного вычисления многомерных интегралов методом прямоугольников проводились на оборудовании со следующей аппаратной конфигурацией:

\begin{itemize}
\item Процессор: Intel Core i3-7130U, 2700 MHz, ядер: 2;
\item Оперативная память: 7880 МБ (DDR3), 1600 MHz;
\item ОС: Microsoft Windows 10 Home, версия 1903 сборка 18362.535.
\end{itemize}

\par Для проведения экспериментов производилось построение выпуклой оболочки для 10 000 000 точек.
\par Результаты экспериментов представлены в Таблице 1.

\begin{table}[!h]
\caption{Результаты вычислительных экспериментов}
\centering
\begin{tabular}{lllll}
Реализация & Последовательно & Параллельно & Ускорение  \\
OMP        & 16.6772          & 9.1245     & 1.82       \\
TBB       & 16.3392         & 8.9315     & 1.82       \\
STD        & 16.5849         & 8.9963     & 1.84       \\
\end{tabular}
\end{table}

\par По результатам экспериментов видно, что параллельная реализация работает быстрее линейной.
\newpage

% Заключение
\section*{Заключение}
\addcontentsline{toc}{section}{Заключение}
В результате лабораторной работы были разработаны последовательная и параллельная реализации построения выпуклой оболочки методом Грэхэма.
\par Основная задача лабораторной работы была достигнута. Параллельная версия работает быстрее, что было доказано в ходе экспериментов.
\par Тесты созданные с помощью Google C++ Testing Framework подтвердили корректность работы программы и помогли в проведении экспериментов.
\newpage

% Список литературы
\begin{thebibliography}{1}
\addcontentsline{toc}{section}{Список литературы}
\bibitem{Sidnev} Redminer: Материалы лекций [Электронный ресурс] // URL: \url{http://redmine.software.unn.ru/issues/2180}
\bibitem{parallel} Habr:[Электронный ресурс] // URL: \url {https://habr.com/ru/post/144921/}
\end{thebibliography}
\newpage

% Приложение
\section*{Приложение}
\addcontentsline{toc}{section}{Приложение}
Листинг кода лабораторной работы.
\begin{lstlisting}
// convex_hull_graham.h
// Copyright 2021 Bakaeva Maria
#ifndef MODULES_TASK_2_BAKAEVA_M_CONVEX_HULL_GRAHAM_CONVEX_HULL_GRAHAM_H_
#define MODULES_TASK_2_BAKAEVA_M_CONVEX_HULL_GRAHAM_CONVEX_HULL_GRAHAM_H_
#include <vector>
#include <utility>

std::vector<std::pair<double, double>> generateRandomPoints(size_t count);
size_t getLeftmostPoint(std::vector<std::pair<double, double>> setOfPoints);
double rotate(std::pair<double, double> A,
    std::pair<double, double> B, std::pair<double, double> C);
std::vector<std::pair<double, double>>
sortPoints(std::vector<std::pair<double, double>> setOfPoints);
std::vector<std::pair<double, double>>
grahamScan(std::vector<std::pair<double, double>> setOfPoints);
size_t getLeftmostPointOmp
(std::vector < std::pair<double, double>> setOfPoints);
std::vector<std::pair<double, double>>
grahamScanOmp(std::vector<std::pair<double, double> >::iterator b,
    std::vector<std::pair<double, double> >::iterator e);
std::vector<std::pair<double, double> > grahamScanParallel(
    std::vector<std::pair<double, double> >::iterator b,
    std::vector<std::pair<double, double> >::iterator e,
    int numberOfThreads);
#endif  // MODULES_TASK_2_BAKAEVA_M_CONVEX_HULL_GRAHAM_CONVEX_HULL_GRAHAM_H_
\end{lstlisting}
\begin{lstlisting}
// convex_hull_graham.cpp
// Copyright 2021 Bakaeva Maria
#include <omp.h>
#include <random>
#include <ctime>
#include <cmath>
#include <vector>
#include <utility>
#include <algorithm>
#include <stack>

#include "../../../modules/task_2/bakaeva_m_convex_hull_graham/convex_hull_graham.h"

std::vector<std::pair<double, double>> generateRandomPoints(size_t count) {
    std::mt19937 gen;
    gen.seed(static_cast<unsigned int>(time(0)));
    std::uniform_real_distribution<double> dis(50, 550);

    std::vector<std::pair<double, double> > points(count);
    for (std::size_t i = 0; i < count; ++i) {
        points[i] = std::make_pair(dis(gen), dis(gen));
    }
    return points;
}

size_t getLeftmostPoint(std::vector<std::pair<double, double>> setOfPoints) {
    size_t n = setOfPoints.size();
    size_t leftmostPoint = 0;

    for (size_t i = 0; i < n; i++) {
        if (setOfPoints[i] < setOfPoints[leftmostPoint]) {
            leftmostPoint = i;
        }
    }

    return leftmostPoint;
}

double rotate(std::pair<double, double> A,
    std::pair<double, double> B, std::pair<double, double> C) {
    return (B.first - A.first) * (C.second - B.second)
        - (B.second - A.second) * (C.first - B.first);
}

std::vector<std::pair<double, double>>
sortPoints(std::vector<std::pair<double, double>> setOfPoints) {
    std::vector<std::pair<double, double>> res(setOfPoints);
    size_t n = res.size();
    size_t j;

    for (size_t i = 2; i < n; i++) {
        j = i;

        while (j > 1 && (rotate(res[0], res[j - 1], res[j])) < 0) {
            std::swap(res[j], res[j - 1]);
            j--;
        }
    }

    return res;
}

size_t getLeftmostPointOmp
(std::vector < std::pair<double, double>> setOfPoints) {
    size_t n = setOfPoints.size();
    size_t leftmostPoint = 0;
    size_t commonLeftmostPoint = 0;

#pragma omp parallel firstprivate(leftmostPoint)
    {
#pragma omp for
        for (int i = 0; i < static_cast<int>(n); i++) {
            if (setOfPoints[i] < setOfPoints[leftmostPoint]) {
                leftmostPoint = i;
            }
        }

#pragma omp critical
        if (setOfPoints[leftmostPoint] < setOfPoints[commonLeftmostPoint]) {
            commonLeftmostPoint = leftmostPoint;
        }
    }

    return commonLeftmostPoint;
}

std::vector<std::pair<double, double>>
grahamScan(std::vector<std::pair<double, double>> setOfPoints) {
    // 1 - Find the leftmost point
    size_t leftmostPoint = getLeftmostPoint(setOfPoints);
    std::swap(setOfPoints[0], setOfPoints[leftmostPoint]);

    // 2 - Sorting all points to the left relative to the starting point
    auto sortingPoints = sortPoints(setOfPoints);

    // 3 - Cut corners
    std::stack<std::pair<double, double>> stackOfPoints;
    stackOfPoints.push(sortingPoints[0]);
    stackOfPoints.push(sortingPoints[1]);

    size_t n = sortingPoints.size();
    size_t stackN = stackOfPoints.size();

    std::pair<double, double> A, B, C;

    for (size_t i = 2; i < n; i++) {
        stackN = stackOfPoints.size();
        B = stackOfPoints.top();
        stackOfPoints.pop();
        A = stackOfPoints.top();
        C = sortingPoints[i];

        if (rotate(A, B, C) > 0) {
            stackOfPoints.push(B);
            stackOfPoints.push(C);
        } else if (stackN < 3) {
            stackOfPoints.push(C);
        } else {
            i--;
        }
    }

    std::vector<std::pair<double, double>> convexHull(stackOfPoints.size());
    size_t iterator = stackOfPoints.size() - 1;
    while (!stackOfPoints.empty()) {
        convexHull[iterator] = stackOfPoints.top();
        stackOfPoints.pop();
        iterator--;
    }

    return convexHull;
}

std::vector<std::pair<double, double>>
grahamScanOmp(std::vector<std::pair<double, double> >::iterator b,
    std::vector<std::pair<double, double> >::iterator e) {

    std::vector<std::pair<double, double> > setOfPoints(e - b);
    std::copy(b, e, setOfPoints.begin());

    // 1 - Find the leftmost point (omp)
    size_t leftmostPoint = getLeftmostPointOmp(setOfPoints);
    std::swap(setOfPoints[0], setOfPoints[leftmostPoint]);

    // 2 - Sorting all points to the left relative to the starting point
    auto sortingPoints = sortPoints(setOfPoints);

    // 3 - Cut corners
    std::stack<std::pair<double, double>> stackOfPoints;
    stackOfPoints.push(sortingPoints[0]);
    stackOfPoints.push(sortingPoints[1]);

    size_t n = sortingPoints.size();
    size_t stackN = stackOfPoints.size();

    std::pair<double, double> A, B, C;

    for (size_t i = 2; i < n; i++) {
        stackN = stackOfPoints.size();
        B = stackOfPoints.top();
        stackOfPoints.pop();
        A = stackOfPoints.top();
        C = sortingPoints[i];

        if (rotate(A, B, C) > 0) {
            stackOfPoints.push(B);
            stackOfPoints.push(C);
        } else if (stackN < 3) {
            stackOfPoints.push(C);
        } else {
            i--;
        }
    }

    std::vector<std::pair<double, double>> convexHull(stackOfPoints.size());
    size_t iterator = stackOfPoints.size() - 1;
    while (!stackOfPoints.empty()) {
        convexHull[iterator] = stackOfPoints.top();
        stackOfPoints.pop();
        iterator--;
    }

    return convexHull;
}

std::vector<std::pair<double, double> > grahamScanParallel(
    std::vector<std::pair<double, double> >::iterator b,
    std::vector<std::pair<double, double> >::iterator e,
    int numberOfThreads) {

    if (numberOfThreads == 0) {
        throw "Number of threads is incorrect";
    }

    int st = static_cast<int>((e - b) / numberOfThreads);
    std::vector<std::pair<double, double> > lastPoints;

#pragma omp parallel num_threads(numberOfThreads)
    {
        size_t thread = omp_get_thread_num();
        std::vector<std::pair<double, double> > localHull;

        if (static_cast<int>(thread) == numberOfThreads - 1) {
            localHull = grahamScanOmp(b + st * thread, e);
        } else {
            localHull
                = grahamScanOmp(b + st * thread, b + st * (thread + 1));
        }

#pragma omp critical
        {
            for (size_t i = 0; i < localHull.size(); i++) {
                lastPoints.push_back(localHull[i]);
            }
        }
    }

    return grahamScanOmp(lastPoints.begin(), lastPoints.end());
}
\end{lstlisting}
\begin{lstlisting}
// main.cpp
// Copyright 2021 Bakaeva Maria
#include <gtest/gtest.h>
#include <omp.h>
#include <vector>
#include <utility>
#include <algorithm>
#include <numeric>
#include "./convex_hull_graham.h"

TEST(ConvexHull, DISABLED_timeTest) {
    size_t size = 100000;
    std::vector<std::pair<double, double> > points(size);
    points = generateRandomPoints(size);

    double t1 = omp_get_wtime();
    auto resultSeq = grahamScan(points);
    double t2 = omp_get_wtime();

    std::cout << " Sequential time " << t2 - t1 << std::endl;

    t1 = omp_get_wtime();
    auto resultParallel = grahamScanParallel(points.begin(), points.end(), 4);
    t2 = omp_get_wtime();

    std::cout << " Parallel time " << t2 - t1 << std::endl;

    ASSERT_EQ(resultParallel, resultSeq);
}

TEST(ConvexHull, exceptionThreadsTest) {
  std::vector<std::pair<double, double> > result(12);
  result[0] = std::make_pair(-1, 2);
  result[1] = std::make_pair(-2, 6);
  result[2] = std::make_pair(1, 7);
  result[3] = std::make_pair(-4, 0);
  result[4] = std::make_pair(2, 4);
  result[5] = std::make_pair(5, 5);
  result[6] = std::make_pair(-1, -1);
  result[7] = std::make_pair(1, 1);
  result[8] = std::make_pair(3, 3);
  result[9] = std::make_pair(1, -3);
  result[10] = std::make_pair(4, -2);
  result[11] = std::make_pair(4, 1);

  ASSERT_ANY_THROW(grahamScanParallel(result.begin(), result.end(), 0));
}

TEST(ConvexHull, getLeftmostPointOmpTest) {
  std::vector<std::pair<double, double> > setOfPoints(5);

  setOfPoints[0] = std::make_pair(-1, -9);
  setOfPoints[1] = std::make_pair(0, 0);
  setOfPoints[2] = std::make_pair(9, 3);
  setOfPoints[3] = std::make_pair(4, 12);
  setOfPoints[4] = std::make_pair(-0.5, -4);

  std::pair<double, double> leftmostPoint = setOfPoints[0];
  std::swap(setOfPoints[0], setOfPoints[time(0) % setOfPoints.size()]);
  size_t leftmost = getLeftmostPointOmp(setOfPoints);

  ASSERT_EQ(setOfPoints[leftmost], leftmostPoint);
}

TEST(ConvexHull, getConvexHullGrahamOmp_1) {
  std::vector<std::pair<double, double> > result(12);
  result[0] = std::make_pair(-1, 2);
  result[1] = std::make_pair(-2, 6);
  result[2] = std::make_pair(1, 7);
  result[3] = std::make_pair(-4, 0);
  result[4] = std::make_pair(2, 4);
  result[5] = std::make_pair(5, 5);
  result[6] = std::make_pair(-1, -1);
  result[7] = std::make_pair(1, 1);
  result[8] = std::make_pair(3, 3);
  result[9] = std::make_pair(1, -3);
  result[10] = std::make_pair(4, -2);
  result[11] = std::make_pair(4, 1);

  result = grahamScanParallel(result.begin(), result.end(), 4);

  std::vector<std::pair<double, double> > check(result.size());
  check[0] = std::make_pair(-4, 0);
  check[1] = std::make_pair(1, -3);
  check[2] = std::make_pair(4, -2);
  check[3] = std::make_pair(5, 5);
  check[4] = std::make_pair(1, 7);
  check[5] = std::make_pair(-2, 6);

  ASSERT_EQ(result, check);
}

TEST(ConvexHull, getConvexHullGrahamOmp_2) {
  std::vector<std::pair<double, double> > result(20);
  result[0] = std::make_pair(-3, 1);
  result[1] = std::make_pair(-2, 8);
  result[2] = std::make_pair(-5, 3);
  result[3] = std::make_pair(5, 7);
  result[4] = std::make_pair(-7, -3);
  result[5] = std::make_pair(8, 3);
  result[6] = std::make_pair(1, 3);
  result[7] = std::make_pair(-1, 2);
  result[8] = std::make_pair(-4, 5);
  result[9] = std::make_pair(-8, 1);
  result[10] = std::make_pair(4, -2);
  result[11] = std::make_pair(-6, 1);
  result[12] = std::make_pair(2, -3);
  result[13] = std::make_pair(5, 4);
  result[14] = std::make_pair(-7, 1);
  result[15] = std::make_pair(1, 6);
  result[16] = std::make_pair(0, -6);
  result[17] = std::make_pair(9, 0);
  result[18] = std::make_pair(2, 1);
  result[19] = std::make_pair(2, 9);

  result = grahamScanParallel(result.begin(), result.end(), 4);

  std::vector<std::pair<double, double> > check(result.size());
  check[0] = std::make_pair(-8, 1);
  check[1] = std::make_pair(-7, -3);
  check[2] = std::make_pair(0, -6);
  check[3] = std::make_pair(9, 0);
  check[4] = std::make_pair(8, 3);
  check[5] = std::make_pair(5, 7);
  check[6] = std::make_pair(2, 9);
  check[7] = std::make_pair(-2, 8);

  ASSERT_EQ(result, check);
}

TEST(ConvexHull, getConvexHullGrahamOmp_3) {
  size_t size = 30;
  std::vector<std::pair<double, double> > points(size);

  points = generateRandomPoints(size);

  auto resultParallel = grahamScanParallel(points.begin(), points.end(), 4);
  auto resultSeq = grahamScan(points);

  ASSERT_EQ(resultParallel, resultSeq);
}

int main(int argc, char **argv) {
    ::testing::InitGoogleTest(&argc, argv);
    return RUN_ALL_TESTS();
}
\end{lstlisting}
\begin{lstlisting}
//TBB
std::vector<std::pair<double, double> > grahamScanParallel(
    std::vector<std::pair<double, double> >::iterator b,
    std::vector<std::pair<double, double> >::iterator e,
    int numberOfThreads) {

    if (numberOfThreads == 0) {
        throw "Number of threads is incorrect";
    }

    int st = static_cast<int>((e - b) / numberOfThreads);
    std::vector<std::pair<double, double> > lastPoints;

    // An instance of the class to create threads
    tbb::task_scheduler_init init(numberOfThreads);
    tbb::spin_mutex mutex;
    tbb::task_group g;

    for (int i = 0; i < numberOfThreads - 1; i++) {
        // Create a task
        g.run([&lastPoints, &mutex, i, b, st] () {
            auto l = b + st * i;
            auto r = b + st * (i + 1);
            auto localHull = grahamScan(l, r);

            for (size_t j = 0; j < localHull.size(); j++) {
                // Create capture object
                tbb::spin_mutex::scoped_lock lock;
                // Grab a mutex
                lock.acquire(mutex);

                lastPoints.push_back(localHull[j]);
                // Release mutex
                lock.release();
            }
        });
    }

    g.run([&lastPoints, b, e, st, numberOfThreads, &mutex]() {
        auto localHull = grahamScan(b + st * (numberOfThreads - 1), e);

        for (size_t i = 0; i < localHull.size(); i++) {
            // Create capture object
            tbb::spin_mutex::scoped_lock lock;
            // Grab a mutex
            lock.acquire(mutex);

            lastPoints.push_back(localHull[i]);
            // Release mutex
            lock.release();
        }
    });

    // Wait for task to complete
    g.wait();

    return grahamScan(lastPoints.begin(), lastPoints.end());
}
\end{lstlisting}
\begin{lstlisting}
//std::threads
std::vector<std::pair<double, double> > grahamScanParallel(
    std::vector<std::pair<double, double> >::iterator b,
    std::vector<std::pair<double, double> >::iterator e,
    int numberOfThreads) {

    if (numberOfThreads == 0) {
        throw "Number of threads is incorrect";
    }

    std::thread* threadsPull = new std::thread[numberOfThreads];

    int st = static_cast<int>((e - b) / numberOfThreads);
    std::vector<std::pair<double, double> > lastPoints;

    std::mutex mutex;

    for (int i = 0; i < numberOfThreads - 1; i++) {
        threadsPull[i] = std::thread([&lastPoints, i, b, st, &mutex] () {
            auto localHull = grahamScan(b + st * i, b + st * (i + 1));

            for (size_t j = 0; j < localHull.size(); j++) {
                std::lock_guard<std::mutex> lock(mutex);
                lastPoints.push_back(localHull[j]);
            }
        });
    }

    threadsPull[numberOfThreads - 1] =
    std::thread([&lastPoints, b, e, numberOfThreads, st, &mutex] () {
        auto localHull = grahamScan(b + st * (numberOfThreads - 1), e);

        for (size_t i = 0; i < localHull.size(); i++) {
            std::lock_guard<std::mutex> lock(mutex);
            lastPoints.push_back(localHull[i]);
        }
    });

    // Wait for threads to complete
    for (int i = 0; i < numberOfThreads; i++) {
        threadsPull[i].join();
    }

    delete[] threadsPull;
    return grahamScan(lastPoints.begin(), lastPoints.end());
}
\end{lstlisting}

\end{document}