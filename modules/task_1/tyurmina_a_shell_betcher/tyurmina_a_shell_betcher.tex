\documentclass{report}

\usepackage[T2A]{fontenc}
\usepackage[utf8]{luainputenc}
\usepackage[english, russian]{babel}
\usepackage[pdftex]{hyperref}
\usepackage[14pt]{extsizes}
\usepackage{listings}
\usepackage{color}
\usepackage{geometry}
\usepackage{enumitem}
\usepackage{multirow}
\usepackage{graphicx}
\usepackage{indentfirst}
\usepackage{amsmath}

\geometry{a4paper,top=2cm,bottom=3cm,left=2cm,right=1.5cm}
\setlength{\parskip}{0.5cm}
\setlist{nolistsep, itemsep=0.3cm,parsep=0pt}

\lstset{language=C++,
		basicstyle=\footnotesize,
		keywordstyle=\color{blue}\ttfamily,
		stringstyle=\color{red}\ttfamily,
		commentstyle=\color{green}\ttfamily,
		morecomment=[l][\color{magenta}]{\#}, 
		tabsize=4,
		breaklines=true,
  		breakatwhitespace=true,
  		title=\lstname,       
}

\addto\captionsrussian{\renewcommand \contentsname {\LARGE Содержание}}

\makeatletter
\renewcommand\@biblabel[1]{#1.\hfil}
\makeatother

\begin{document}

\begin{titlepage}

\begin{center}
МИНИСТЕРСТВО НАУКИ И ВЫСШЕГО ОБРАЗОВАНИЯ РОССИЙСКОЙ ФЕДЕРАЦИИ\\
Федеральное государственное автономное образовательное учреждение высшего образования \\
\textbf{"Национальный исследовательский Нижегородский государственный университет им. Н.И. Лобачевского" (ННГУ)}
\end{center}

\begin{center}
\textbf{Институт информационных технологий, математики и механики}
\end{center}

\vspace{4em}

\begin{center}
Отчет по лабораторной работе \\
\end{center}
\begin{center}
\textbf{\Large«Сортировка Шелла с чётно-нечётным слиянием Бэтчера.»} \\
\end{center}

\vspace{4em}

\newbox{\lbox}
\savebox{\lbox}{\hbox{text}}
\newlength{\maxl}
\setlength{\maxl}{\wd\lbox}
\hfill\parbox{7cm}{
\hspace*{5cm}\hspace*{-5cm}\textbf{Выполнил:} \\ студент группы 381806-3 \\ Тюрмина А.Н.\\
\\
\hspace*{5cm}\hspace*{-5cm}\textbf{Проверил:}\\ доцент кафедры МОСТ, \\ кандидат технических наук \\ Сысоев А. В.\\
}
\vspace{\fill}

\begin{center} Нижний Новгород \\ 2021 \end{center}

\end{titlepage}

\setcounter{page}{2}
% содержание
\tableofcontents{}
\clearpage

\begin{center}
\section*{Введение}
\end{center}
\addcontentsline{toc}{section}{Введение}
\par Сортировка — это упорядочивание набора однотипных данных по возрастанию или убыванию. Она является одной из наиболее распространненых операций обработки данных и используется во многих задачах. 
\par Для сортировки массива в данной работе мы используем сортировку "Шелла" - алгоритм сортировки, являющийся усовершенствованным вариантом сортировки вставками. Идея метода Шелла состоит в сравнении элементов, стоящих не только рядом, но и на определённом расстоянии друг от друга. \\
Алгоритм четно-нечетной сортировки слиянием (odd-even mergesort) был разработан Бэтчером в 1968 году. Алгоритм не слишком популярный и не слишком известный. Однако его реализация не очень сложна.
\par Для достижения большей производительности нам необходимо разбить массив на меньшие размеры, отсортировать его и в конце слить воедино. Поэтому, в данной работе рассматривается сортировка Шелла с четно-нечетным слиянием Бэтчера с использованием параллельных вычислений.
\newpage

\begin{center}
\section*{Постановка задач}
\end{center}
\addcontentsline{toc}{section}{Постановка задач}
В данной лабораторной работе необходимо:
\begin{enumerate}
\item Реализовать последовательный алгоритм сортировки Шелла с четно-нечетным слиянием Бэтчера.
\item Реализовать параллельный алгоритм сортировки Шелла с четно-нечетным слиянием Бэтчера с использованием технологии OpenMP.
\item Реализовать параллельный алгоритм сортировки Шелла с четно-нечетным слиянием Бэтчера с использованием технологии TBB.
\item Провести вычислительные эксперементы и сравнить время работы трех алгоритмов.
\item Сделать вывод на основе полученных результатов.
\end{enumerate}
\newpage

\begin{center}
\section*{Описание алгоритма}
\end{center}
\addcontentsline{toc}{section}{Описание алгоритма}
Сортировка Шелла:
\par Идея метода заключается в сравнение разделенных на группы элементов последовательности, находящихся друг от друга на некотором расстоянии. Изначально это расстояние равно $d$ или $N/2$, где $N$ — общее число элементов. На первом шаге каждая группа включает в себя два элемента расположенных друг от друга на расстоянии $N/2$; они сравниваются между собой, и, в случае необходимости, меняются местами. На последующих шагах также происходят проверка и обмен, но расстояние $d$ сокращается на $d/2$, и количество групп, соответственно, уменьшается. Постепенно расстояние между элементами уменьшается, и на $d=1$ проход по массиву происходит в последний раз.\\
\\
Четно-нечетное слияние Бэтчера:
\par Идея метода заключается в том, что мы делим массив пополам и сортируем. Далее элементы, стоящие на четных позициях (0,2,4, ...) в двух полученных массивах сравниваются  и по возрастанию складываются в первый массив, а стоящие на нечетных (1,3,5, ...) во второй массив. После этого мы сливаем оба массива вместе: берем по одному элементу из первого и второго массива и складываем друг за другом (четное, нечетное, четное, нечетное и т.д.).
\newpage

\begin{center}
\section*{Схема распараллеливания}
\end{center}
\addcontentsline{toc}{section}{Схема распараллеливания}
Общая схема работы параллельного алгоритма:
\begin{enumerate}
\item Исходный массив размера $size$, который нужно отсортировать, в зависимости от числа потоков $n$, заданных пользователем разбивается на  подмассивы размера $\frac{size}{n}$.
\item Каждый из потоков выполняет алгоритм сортировки Шелла для своего подмассива.
\item Отсортированные части сливаются в один массив при помощи алгоритма четно-нечетного слияния Бэтчера.
\end{enumerate}
\newpage

\begin{center}
\section*{Описание программной реализации}
\end{center}
\addcontentsline{toc}{section}{Описание программной реализации}
\par Программа состоит из следующих функций:
\par Функция, которая заполняет массив рандомными числами:
\begin{lstlisting}
std::vector<int> Random(int);
\end{lstlisting}
В качестве входных данных - размер массива. На выходе массив с рандомными числами.
\par Функция, в которой реализована сортировка Шелла
\begin{lstlisting}
std::vector<int> ShellSort(const std::vector<int>&, int);
\end{lstlisting}
В качетве входных данных передается массив и его размер. На выходе получаем отсортированный массив.
\par Функция, которая делит вектор на части:
\begin{lstlisting}
std::vector<std::vector<int>> Separat(const std::vector<int>&, size_t);
\end{lstlisting}
В качетве входных данных передается массив и его размер. На выходе получаем разделенный массив.
\par Функция, которая содержит массив чисел с четными позициями:
\begin{lstlisting}
std::vector<int> BetcherEven(const std::vector<int>&, const std::vector<int>&);
\end{lstlisting}
В качетве входных данных передается два массива. На выходе получаем массив, в котором собраны числа из двух массивов, стоящие на четных позициях.
\par Функция, которая содержит массив чисел с нечетными позициями:
\begin{lstlisting}
std::vector<int> BetcherOdd(const std::vector<int>&, const std::vector<int>&);
\end{lstlisting}
В качетве входных данных передается два массива. На выходе получаем массив, в котором собраны числа из двух массивов, стоящие на нечетных позициях.
\par Функция слияния массивов:
\begin{lstlisting}
std::vector<int> BetcherMerge(const std::vector<int>&, const std::vector<int>&);
\end{lstlisting}
В качетве входных данных передается два массива. На выходе получаем массив, в котором собраны четные и нечетные позиции в установленном порядке.
\par Функция, в которой происходит сортировка Шелла-Бэтчера:
\begin{lstlisting}
std::vector<int> ShellBetcher(const std::vector<int>&, int);
\end{lstlisting}
В качетве входных данных передается массив и его размер. На выходе получаем отсортированный массив с помощью сортировки Шелла и слияния Бэтчера.
\par Параллельная часть:
\par Функция, в которой происходит сортировка Шелла-Бэтчера(с использованием OpenMP):
\begin{lstlisting}
std::vector<int> ShellBetcherOmp(const std::vector<int>&, const int, int);
\end{lstlisting}
В качетве входных данных передается массив, его размер и количество потоков. На выходе получаем отсортированный массив с помощью сортировки Шелла и слияния Бэтчера.
\par Функция слияния массивов(с использованием OpenMP):
\begin{lstlisting}
std::vector<int> MergeOmp(const std::vector<std::vector<int>>&, const int, int);
\end{lstlisting}
В качетве входных данных передается массив, его размер и количество потоков. На выходе получаем массив,  в котором собраны четные и нечетные позиции в установленном порядке.
\par Функция слияния массивов(с использованием TBB):
\begin{lstlisting}
std::vector<int> MergeTBB(const std::vector<int>&, const int, int);
\end{lstlisting}
В качетве входных данных передается массив, его размер и количество потоков. На выходе получаем массив,  в котором собраны четные и нечетные позиции в установленном порядке.
\newpage

\begin{center}
\section*{Подтверждение корректности}
\end{center}
\addcontentsline{toc}{section}{Подтверждение корректности}
\par Для подтверждения корректности в программе реализована система тестов, разработанная с помощью использования Google C++ Testing Framework.
\par Эта система включает в себя тесты, проверяющие все алгоритмы написанной программы. Тесты проверяют сортировку Шелла, слияние Бэтчера, Шелла-Бэтчера. А в паралльленой версии сравниваются значение последовательного алгоритма сортировки Шелла с четно-нечетным слиянием Бэтчера с параллельным.
\par Успешное прохождение данных тестов доказывает корректность работы всей программы.
\newpage

\begin{center}
\section*{Результаты эксперементов}
\end{center}
\addcontentsline{toc}{section}{Результаты эксперементов}
\par Вычислительные эксперименты для оценки эффективности сортировки Шелла с четно-нечетным слиянием Бэтчера проводились на оборудовании со следующей аппаратной конфигурацией::
\begin{enumerate}
\item Процессор: Intel(R) Core(TM) i5-7200U CPU @ 2.50GHz 2701МГц, ядер: 2, логических процессоров: 4;
\item Оперативная память:8 Gb;
\item Операционная система: Windows 10 Домашняя;
\item Среда разработки: Visual Studio 2019.
\end{enumerate}
В тестах сортировались массивы заполненные вручную с небольшими размерами от 5 - 20 и заполненные рандомно с размерами от 50 - 10000.
\begin{table}[!h]
\caption{Результаты вычислительных экспериментов. Сравнение последовательного алгоритма с OpenMP}
\centering
\begin{tabular}{|p{4cm}|p{4cm}|p{4cm}|p{3cm}|}
\hline
Количество процессов & Последовательный алгоритм,сек & Параллельный алгоритм,сек & Ускорение  \\\hline
1  & 0.152486 & 0.131001 & 1.164  \\\hline
2  & 0.105847 & 0.062594 & 1.691  \\\hline
3  & 0.08623 & 0.032235 & 2.675  \\\hline
4  & 0.052659 & 0.016162 & 3.258  \\\hline
5  & 0.032564 & 0.007311 & 4.454  \\
\hline
\end{tabular}
\end{table}

\begin{table}[!h]
\caption{Результаты вычислительных экспериментов. Сравнение последовательного алгоритма с TBB}
\centering
\begin{tabular}{|p{4cm}|p{4cm}|p{4cm}|p{3cm}|}
\hline
Размер матрицы & Последовательный алгоритм,сек & Параллельный алгоритм,сек & Ускорение  \\\hline
1  & 0.118434 & 0.0719903 & 1.645  \\\hline
2  & 0.0618067 & 0.0375135 & 1.647  \\\hline
3  & 0.0352304 & 0.0207673 & 1.696  \\\hline
4  & 0.0302765 & 0.0137396 & 2.204  \\\hline
5  & 0.0239192 & 0.0109999 & 2.174  \\
\hline
\end{tabular}
\end{table}

\par На основе полученных данных, можно сделать вывод, что в большинстве случаев параллельный алгоритм с использованием OpenMP более эффективен. Максимальное ускорение в 4 раза.
\newpage

\begin{center}
\section*{Заключение}
\end{center}
\addcontentsline{toc}{section}{Заключение}
\par В ходе данной лабораторной работы мною были реализованы три алгоритма сортировки Шелла с четно-нечетным слиянием Бэтчера: последовательная сортировка, параллельнная сортировка с импользованием технологии OpenMP и параллельнная сортировка с использованием технологии TBB. А также были разработаны и доведены до успешного выполнения тесты, созданные для данного программного проекта с использованием Google C++ Testing Framework и необходимые для подтверждения корректности работы программы.
\par Основной целью работы была реализация параллельной версии алгоритма сортировки, которая должна работать быстрее по времени, чем последовательная.
Вычислительные эксперименты подтвердили это.
\newpage

\addcontentsline{toc}{section}{Литература}
\begin{thebibliography}{}
\bibitem{1} Гергель В. П. Теория и практика параллельных вычислений. – 2007.
\bibitem{2} Гергель В. П., Стронгин Р. Г. Основы параллельных вычислений для многопроцессорных вычислительных систем. – 2003.
\bibitem{3}Wikipedia: the free encyclopedia [Электронный ресурс] // URL: https://en.wikipedia.org/wiki/Shellsort
\bibitem{4}Kvodo: [Электронный ресурс] // URL: https://kvodo.ru/sortirovka-shella.html 
\bibitem{5}Habr: [Электронный ресурс] // https://habr.com/ru/post/261777/
\bibitem{6}Wikipedia: the free encyclopedia [Электронный ресурс] // URL: https://en.wikipedia.org/wiki/\verb|Batcher_odd-even_mergesort|


\end{thebibliography}{}
\newpage

\begin{center}
\section*{Приложение}
\end{center}
\addcontentsline{toc}{section}{Приложение}
В этом разделе находится листинг всего кода, написанного в рамках лабораторной работы.
\begin{lstlisting}
// shell_betcher.h

// Copyright 2021 Tyurmina Alexandra
#ifndef MODULES_TASK_3_TYURMINA_A_SHELL_BETCHER_SHELL_BETCHER_H_
#define MODULES_TASK_3_TYURMINA_A_SHELL_BETCHER_SHELL_BETCHER_H_

#include <iostream>
#include <vector>
#include <utility>
#include <functional>

std::vector<int> Random(int size);
std::vector<std::vector<int>> Separat(const std::vector<int>& massiv, size_t n);
std::vector<int> ShellSort(const std::vector<int>&, int size);
std::vector<int> BetcherEven(const std::vector<int>&, const std::vector<int>&);
std::vector<int> BetcherOdd(const std::vector<int>&, const std::vector<int>&);
std::vector<int> BetcherMerge(const std::vector<int>&, const std::vector<int>&);
std::vector<int> ShellBetcherSeq(const std::vector<int>&, const int, int);
std::vector<int> ShellBetcherMerge(const std::vector<std::vector<int>>&, const int, int);
std::vector<int> MergeTBB(const std::vector<int>&, const int, int);
std::vector<int> ShellBetcherOmp(const std::vector<int>&, const int, int);
std::vector<int> MergeOmp(const std::vector<std::vector<int>>&, const int, int);

#endif  // MODULES_TASK_3_TYURMINA_A_SHELL_BETCHER_SHELL_BETCHER_H_

//shell_betcher.cpp

// Copyright 2021 Tyurmina Alexandra
#include <tbb/tbb.h>
#include <vector>
#include <string>
#include <random>
#include <ctime>
#include <algorithm>
#include <iterator>
#include <utility>
#include "../../../modules/task_3/tyurmina_a_shell_betcher/shell_betcher.h"

std::vector<int> Random(int size) {
    std::mt19937 gen;
    gen.seed(static_cast<unsigned int>(time(0)));
    std::vector<int> vec1(size);
    for (int i = 0; i < size; i++) {
        vec1[i] = gen() % 100;
    }
    return vec1;
}

std::vector<std::vector<int>> Separat(const std::vector<int>& massiv, size_t n) {
    size_t remainder = massiv.size() % n;
    size_t integer = massiv.size() / n;
    size_t finish = 0;
    size_t start = 0;
    std::vector<std::vector<int>> result;

    for (size_t i = 0; i < n; ++i) {
        if (remainder > 0) {
            finish = finish + (integer + !!(remainder--));
        } else {
            finish = finish + integer;
        }
        std::vector<int> hal(massiv.begin() + start, massiv.begin() + finish);
        result.push_back(hal);
        start = finish;
    }
    return result;
}

std::vector<int> ShellSort(const std::vector<int>& massiv, int size) {
    int part;
    int i;
    int j;
    int temp;
    std::vector<int>mass(massiv);

    for (part = size / 2; part > 0; part = part / 2) {
        for (i = part; i < size; i++) {
            for (j = i - part; j >= 0; j = j - part) {
                if (mass[j] > mass[j + part]) {
                    temp = mass[j];
                    mass[j] = mass[j + part];
                    mass[j + part] = temp;
                }
            }
        }
    }
    return mass;
}

std::vector<int> BetcherEven(const std::vector<int>& massiv1, const std::vector<int>& massiv2) {
    int size1 = massiv1.size();
    int size2 = massiv2.size();
    int size_result = size1 / 2 + size2 / 2 + size1 % 2 + size2 % 2;
    std::vector <int> mass_result(size_result);
    int i1 = 0;
    int i2 = 0;
    int i = 0;

    while ((i1 < size1) && (i2 < size2)) {
        if (massiv1[i1] <= massiv2[i2]) {
            mass_result[i] = massiv1[i1];
            i1 += 2;
        } else {
            mass_result[i] = massiv2[i2];
            i2 += 2;
        }
        i++;
    }

    if (i1 >= size1) {
        for (int j = i2; j < size2; j += 2) {
            mass_result[i] = massiv2[j];
            i++;
        }
    }
    if (i2 >= size2) {
        for (int j = i1; j < size1; j += 2) {
            mass_result[i] = massiv1[j];
            i++;
        }
    }

    return mass_result;
}

std::vector<int> BetcherOdd(const std::vector<int>& massiv1, const std::vector<int>& massiv2) {
    int size1 = massiv1.size();
    int size2 = massiv2.size();
    int size_result = size1 / 2 + size2 / 2;
    std::vector <int> mass_result(size_result);
    int i1 = 1;
    int i2 = 1;
    int i = 0;

    while ((i1 < size1) && (i2 < size2)) {
        if (massiv1[i1] <= massiv2[i2]) {
            mass_result[i] = massiv1[i1];
            i1 += 2;
        } else {
            mass_result[i] = massiv2[i2];
            i2 += 2;
        }
        i++;
    }

    if (i1 >= size1) {
        for (int j = i2; j < size2; j += 2) {
            mass_result[i] = massiv2[j];
            i++;
        }
    }
    if (i2 >= size2) {
        for (int j = i1; j < size1; j += 2) {
            mass_result[i] = massiv1[j];
            i++;
        }
    }
    return mass_result;
}

std::vector<int> BetcherMerge(const std::vector<int>& massiveven, const std::vector<int>& massivodd) {
    int size1 = massiveven.size();
    int i1 = 0;
    int size2 = massivodd.size();
    int i2 = 0;
    int i = 0;
    int temp;
    int size_result = massiveven.size() + massivodd.size();
    std::vector <int> mass_result(size_result);

    while ((i1 < size1) && (i2 < size2)) {
        mass_result[i] = massiveven[i1];
        mass_result[i + 1] = massivodd[i2];
        i1 = i1 + 1;
        i2 = i2 + 1;
        i = i + 2;
    }
    if ((i1 < size1) && (i2 >= size2)) {
        for (int a = i; a < size_result; a++) {
            mass_result[a] = massiveven[i1];
            i1 = i1 + 1;
        }
    } else {
        for (int a = i; a < size_result; a++) {
            mass_result[a] = massiveven[i2];
            i2 = i2 + 1;
        }
    }
    for (int i = 1; i < size_result; i++) {
        if (mass_result[i] < mass_result[i - 1]) {
            temp = mass_result[i];
            mass_result[i] = mass_result[i - 1];
            mass_result[i - 1] = temp;
        }
    }
    return mass_result;
}

std::vector<int> ShellBetcherSeq(const std::vector<int>& massiv, const int n, int size) {
    std::vector<std::vector<int>> mass = Separat(massiv, n);
    const int n1 = n;

    for (int i = 0; i < static_cast<int>(mass.size()); i++) {
        mass[i] = ShellSort(mass[i], mass[i].size());
    }
    return ShellBetcherMerge(mass, n1, size);
}

std::vector<int> ShellBetcherMerge(const std::vector<std::vector<int>>& massiv, const int n, int size) {
    std::vector<int> ev;
    std::vector<int> od;
    std::vector<int> result = massiv[0];

    for (int i = 1; i < n; i++) {
        od = BetcherOdd(result, massiv[i]);
        ev = BetcherEven(result, massiv[i]);
        result = BetcherMerge(ev, od);
    }
    return result;
}

std::vector<int> MergeTBB(const std::vector<int>& massiv, const int n, int size) {
    std::vector<std::vector<int>> mass = Separat(massiv, n);

    tbb::task_scheduler_init init(n); 
    tbb::parallel_for(tbb::blocked_range<size_t>(0, mass.size(), 2), 
        [&mass](const tbb::blocked_range<size_t>& r) {
            int start = r.begin(), finish = r.end();
            for (int i = start; i != finish; ++i)
                mass[i] = ShellSort(mass[i], mass[i].size());
        }, tbb::simple_partitioner()); 

    init.terminate(); 
    return ShellBetcherMerge(mass, n, size);
}

std::vector<int> ShellBetcherOmp(const std::vector<int>& massiv, const int n, int size) {
    std::vector<std::vector<int>> mass = Separat(massiv, n);

#pragma omp parallel shared(mass)
    {
#pragma omp for
        for (int i = 0; i < static_cast<int> (mass.size()); i++) {
            mass[i] = ShellSort(mass[i], mass[i].size());
        }
    }
    return MergeOmp(mass, n, size);
}

std::vector<int> MergeOmp(const std::vector<std::vector<int>>& massiv, const int n, int size) {
    std::vector<int> result = massiv[0];
    std::vector<int> od;
    std::vector<int> ev;

    for (int i = 1; i < n; i++) {
#pragma omp parallel sections num_threads(1)
        {
#pragma omp section
            {
                od = BetcherOdd(result, massiv[i]);
            }
#pragma omp section
            {
                ev = BetcherEven(result, massiv[i]);
            }
#pragma omp section
            {
                result = BetcherMerge(ev, od);
            }
        }
    }
    return result;
}


//main.cpp

// Copyright 2021 Tyurmina Alexandra
#include <gtest/gtest.h>
#include <omp.h>
#include <vector>
#include <utility>
#include <functional>
#include <cmath>
#include <algorithm>
#include "./shell_betcher.h"

TEST(shell_betcher, Shell_5) {
    std::vector<int> vec = { 3, -1, 12, 8, 5 };
    std::vector<int> res = ShellSort(vec, 5);
    std::vector<int> correct_res = { -1, 3, 5, 8, 12 };

    ASSERT_EQ(res, correct_res);
}

TEST(shell_betcher, Shell_rand_10) {
    std::vector<int> vec = Random(10);
    std::vector<int> res = ShellSort(vec, 10);
    std::sort(vec.begin(), vec.end());

    ASSERT_EQ(res, vec);
}

TEST(shell_betcher, betcher_merge_8) {
    std::vector<int> vec1 = { 15, 47, 5, 91 };
    std::vector<int> vec2 = { 62, 35, 78, 94 };
    std::vector<int> correct_res = { 15, 47, 35, 5, 62, 78, 91, 94 };
    std::vector<int> res = BetcherMerge(vec1, vec2);

    ASSERT_EQ(res, correct_res);
}

TEST(shell_betcher, shell_betcher_seq_50) {
    std::vector<int> vec = Random(50);
    std::vector<int> sort = ShellBetcherSeq(vec, 4, 50);
    std::sort(vec.begin(), vec.end());

    ASSERT_EQ(vec, sort);
}

TEST(shell_betcher, shell_betcher_tbb_50) {
    std::vector<int> vec = Random(50);
    std::vector<int> sort = MergeTBB(vec, 4, 50);
    std::sort(vec.begin(), vec.end());

    ASSERT_EQ(vec, sort);
}

TEST(shell_betcher, shell_betcher_seq_20) {
    std::vector<int> vec = Random(20);
    std::vector<int> res = ShellBetcherSeq(vec, 4, 20);
    std::sort(vec.begin(), vec.end());
    ASSERT_EQ(vec, res);
}

TEST(shell_betcher, shell_betcher_tbb_20) {
    std::vector<int> vec = Random(20);
    std::vector<int> res = MergeTBB(vec, 4, 20);
    std::sort(vec.begin(), vec.end());
    ASSERT_EQ(vec, res);
}

TEST(shell_betcher, shell_b_1000) {
    std::vector<int> vec1 = Random(1000);
    std::vector<int> vec2 = ShellBetcherOmp(vec1, 4, 1000);
    std::sort(vec1.begin(), vec1.end());

    ASSERT_EQ(vec1, vec2);
}

int main(int argc, char** argv) {
    ::testing::InitGoogleTest(&argc, argv);
    return RUN_ALL_TESTS();
}

\end{lstlisting}
\end{document}