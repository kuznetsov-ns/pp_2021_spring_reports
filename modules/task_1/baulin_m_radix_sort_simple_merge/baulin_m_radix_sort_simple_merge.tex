\documentclass{report}

\usepackage[T2A]{fontenc}
\usepackage[utf8]{luainputenc}
\usepackage[english, russian]{babel}
\usepackage[pdftex]{hyperref}
\usepackage[14pt]{extsizes}
\usepackage{listings}
\usepackage{color}
\usepackage{geometry}
\usepackage{enumitem}
\usepackage{multirow}
\usepackage{graphicx}
\usepackage{indentfirst}
\usepackage{amsmath}

\geometry{a4paper,top=2cm,bottom=3cm,left=2cm,right=1.5cm}
\setlength{\parskip}{0.5cm}
\setlist{nolistsep, itemsep=0.3cm,parsep=0pt}

\lstset{language=C++,
    basicstyle=\footnotesize,
    keywordstyle=\color{blue}\ttfamily,
    stringstyle=\color{red}\ttfamily,
    commentstyle=\color{green}\ttfamily,
    morecomment=[l][\color{magenta}]{\#},
    tabsize=4,
    breaklines=true,
    breakatwhitespace=true,
    title=\lstname,
}

\makeatletter
\renewcommand\@biblabel[1]{#1.\hfil}
\makeatother

\begin{document}

    \begin{titlepage}

        \begin{center}
            Министерство науки и высшего образования Российской Федерации
        \end{center}

        \begin{center}
            Федеральное государственное автономное образовательное учреждение высшего образования \\
            Национальный исследовательский Нижегородский государственный университет им. Н.И. Лобачевского
        \end{center}

        \begin{center}
            Институт информационных технологий, математики и механики
        \end{center}

        \vspace{4em}

        \begin{center}
            \textbf{\LargeОтчет по лабораторной работе} \\
        \end{center}
        \begin{center}
            \textbf{\Large«Поразрядная сортировка целых чисел с простым слиянием»} \\
        \end{center}

        \vspace{4em}

        \newbox{\lbox}
        \savebox{\lbox}{\hbox{text}}
        \newlength{\maxl}
        \setlength{\maxl}{\wd\lbox}
        \hfill\parbox{7cm}{
            \hspace*{5cm}\hspace*{-5cm}\textbf{Выполнил:} \\ студент группы 381808-2 \\ Баулин М.А. \\
            \\
            \hspace*{5cm}\hspace*{-5cm}\textbf{Проверил:}\\ доцент кафедры МОСТ, \\ кандидат технических наук \\ Сысоев А. В. \\
        }
        \vspace{\fill}

        \begin{center} Нижний Новгород \\ 2021 \end{center}

    \end{titlepage}

    \setcounter{page}{2}

    % Содержание
    \tableofcontents
    \newpage

    % Введение
    \section*{Введение}
    \addcontentsline{toc}{section}{Введение}
    Существуют десятки различных способов расположить элементы данных по порядку. Среди них выделяют те, что работают быстро, такие, что, например, могут в максимум за минуты отсортировать любой массив данных, размещенный в оперативной памяти компьютера. Более конкретно можно сказать, что сортировка, работающая быстро, упорядочивает на хорошем современном персональном компьютере миллиард целых чисел за менее, чем сто секунд. Одной из таких как раз и является поразрядная сортировка. Она очень чувствительна к типу данных для сортировки, например, нужно иметь свой вариант такой сортировки для данных каждого размера, нужно делать специальный варианты для беззнаковых и знаковых целых чисел, а поддержка работы с вещественными числами может потребовать совсем немаленьких усилий.
    \newpage

    % Постановка задачи
    \section*{Постановка задачи}
    \addcontentsline{toc}{section}{Постановка задачи}
    Целью данной лабораторной работы является разработка последовательной и параллельной версий алгоритма поразрядной сортировки вещественных чисел простым слиянием, а также сравнение скорости работы последовательной и параллельной версий.
    \par Исходные данные - неупорядоченный набор вещественных чисел. Результат работы - отсортированный по возрастанию набор элементов.
    \par Необходимо реализовать последовательную и параллельную версию программы. Для реализации параллельной версии необходимо использовать средства OMP, TBB. Для проверки корректности работы программы необходимо использовать GTest.
    \newpage

    % Описание алгоритма
    \section*{Описание алгоритма}
    \addcontentsline{toc}{section}{Описание алгоритма}
    На вход алгоритма поступает неупорядоченный набор вещественных чисел.
    \par Описание алгоритма:
    \begin{itemize}
        \item Числа представляются как массивы символов (char) для рассмотрения каждого байта по отдельности
        \item Производится сортировка чисел на основе байта, соответствующего итерации сортировки (от 0 до 6)
        \item Далее происходит обработка байта, содержашего в себе знаковый бит, и сортировка с учетом знаковой части

    \end{itemize}

    \newpage

    % Описание схемы распараллеливания
    \section*{Описание схемы распараллеливания}
    \addcontentsline{toc}{section}{Описание схемы распараллеливания}
    Для распаралеливания используется подход разбиения массива входных данных на куски с последующей обработкой внутри подчиненных потоков. Для слияния результата обработки используется simple merge.
    \par Для достижения данного результата в OpenMP существует деректива препроцессора $pragma omp parallel for$
    \newpage

    % Описание программной реализации
    \section*{Описание программной реализации}
    \addcontentsline{toc}{section}{Описание программной реализации}\
    Входными параметрами всех реализаций алгоритма являются:
    \begin{itemize}
        \item std::vector<double> source
        \item std::vector<double> tmp
        \item const int offset
    \end{itemize}

    \par Для корректной работы программы используются следующие методы:

    \par GetRandomVector(int sz) - получение вектора с рандомными значениями
    \par CountingSort(std::vector<double> source, std::vector<double> tmp, const int offset) - непосредственно для сортировки элементов на основе значения конкретного байта.
    \par SignSort(std::vector<double> source) - непосредственно для сортировки элементов на основе значения байта, хранящего знаковый бит, и сортировки с учетом этого бита.
    \par RadixSort(const std::vector<double> source) - Метод, представляющий собой саму поразрядную сортировку

    \newpage

    % Результаты экспериментов
    \section*{Результаты экспериментов}
    \addcontentsline{toc}{section}{Результаты экспериментов}
    Вычислительные эксперименты проводились на оборудовании со следующей аппаратной конфигурацией:

    \begin{itemize}
        \item Процессор: Intel(R) Core(TM) i5-7200 CPU @ 2.4GHz, ядер: 4
        \item Оперативная память: 12 ГБ DDR4;
        \item ОС: Microsoft Windows 10 Pro
    \end{itemize}

    \begin{table}[!h]
        \begin{center}
            \begin{tabular}{l l l l l l l}
                Реализация & Время & Ускорение   \\
                SEQ        & 1.2 & 1           \\
                OMP        & 1.4 & 0.85      \\


            \end{tabular}
        \end{center}
        \caption{Результаты вычислительных экспериментов}
        \centering
    \end{table}

    \par Из полученных результатов видно, что не удалось получить значительное ускорение в параллельной реализациях метода из-за неправильно выбранного метода распаралеливания.
    \newpage

    % Заключение
    \section*{Заключение}
    \addcontentsline{toc}{section}{Заключение}
    В результате лабораторной работы были разработаны последовательная и параллельная реализация метода поразрядной сортировки вещественных чисел простым слиянием.
    \par Из результатов вычислительных экспериментов видно, что параллельные версии программы не работают быстрее, если их плохо написать.
    \newpage

    % Литература
    \begin{thebibliography}{1}
        \addcontentsline{toc}{section}{Литература}
        \bibitem{} «Параллельное программирование на OpenMP»
        \\URL:\url {http://ccfit.nsu.ru/arom/data/openmp.pdf}
 \bibitem{Sidnev} Сиднев А.А., Мееров И.Б., Сысоев А.В. «Разработка параллельных программ в системах с общей памятью с использованием библиотеки Intel Threading Building Blocks (TBB)».
        \\URL:\url {http://hpc-education.ru/files/lectures/meerov/ppt06.pdf}
 \bibitem{} «Параллельное программирование. С++. Thread Support Library. Atomic Operations Library.»
        \\URL:\url {https://ppt-online.org/184002}
 \end{thebibliography}
    \newpage

    % Приложение
    \section*{Приложение}
    \addcontentsline{toc}{section}{Приложение}
    В данном разделе находится реализация методов последовательной и параллельной реализции алгоритма
    \begin{lstlisting}

 // Copyright 2021 Baulin Mikhail

#include "../../../modules/task_1/baulin_m_radix_sort_simple_merge/radix_sort.h"

#include <algorithm>
#include <random>
#include <vector>

std::vector<double> GetRandomVector(int sz) {
  std::random_device dev;
  std::mt19937 gen(dev());
  std::vector<double> vec(sz);
  for (int i = 0; i < sz; i++) {
    vec[i] = gen() % 100;
  }
  return vec;
}

std::vector<double> CountingSort(std::vector<double> source, std::vector<double> tmp,
                         const int offset) {
  auto const size = static_cast<int>(source.size());
  auto byteArr = (unsigned char*)source.data();
  auto acc = 0;
  int counters[256] = {0};

  for (auto i = 0; i < size; i++) {
    counters[byteArr[offset + 8 * i]]++;
  }

  for (int i = 0; i < 256; i++) {
    int tmp = counters[i];
    counters[i] = acc;
    acc += tmp;
  }

  for (auto i = 0; i < size; i++) {
    auto byteIndex = offset + 8 * i;
    auto byte = byteArr[byteIndex];
    auto counter = counters[byte];
    tmp[counter] = source[i];
    counters[byte]++;
  }

  return tmp;
}

std::vector<double> SignSort(std::vector<double> source) {
  auto const offset = 7;
  auto const size = static_cast<int>(source.size());
  auto byteArr = (unsigned char*)source.data();
  auto acc = 0;

  auto result = std::vector<double>(source);
  int counters[256] = {0};

  for (auto i = 0; i < size; i++) {
    counters[byteArr[7 + 8 * i]]++;
  }

  for (int i = 255; i > 127; i--) {
    counters[i] += acc;
    acc = counters[i];
  }

  for (int i = 0; i < 128; i++) {
    int tmp = counters[i];
    counters[i] = acc;
    acc += tmp;
  }

  for (int i = 0; i < size; i++) {
    auto byteIndex = offset + 8 * i;
    auto byte = byteArr[byteIndex];
    if (byte < 128) {
      result[counters[byte]] = source[i];
      counters[byte]++;
    } else {
      counters[byte]--;
      result[counters[byte]] = source[i];
    }
  }

  return result;
}

std::vector<double> RadixSort(const std::vector<double>& source) {
  auto tmp = std::vector<double>(source.size());
  auto tmpForSwap = std::vector<double>(source.size());
  auto copy = std::vector<double>(source);

  for (int i = 0; i < 8; i++) {
    tmp = CountingSort(copy, tmp, i);
    tmpForSwap = copy;
    copy = tmp;
    tmp = tmpForSwap;
  }

  return SignSort(copy);
}

std::vector<double> Merge(const std::vector<double>& left,
                          const std::vector<double>& right) {
  std::vector<double> result((left.size() + right.size()));

  auto const leftSize = static_cast<int>(left.size());
  auto const rightSize = static_cast<int>(right.size());

  auto i = 0, j = 0, k = 0;
  for (k = 0; k < leftSize + rightSize - 1; k++) {
    if (left[i] < right[j])
      result[k] = left[i++];
    else
      result[k] = right[j++];
  }

  while (i < leftSize) {
    result[k++] = left[i++];
  }
  while (j < rightSize) {
    result[k++] = right[j++];
  }

  return result;
}

    \end{lstlisting}

    \begin{lstlisting}
 // Copyright 2021 Baulin Mikhail

#include "../../../modules/task_2/baulin_m_radix_sort_simple_merge_omp/radix_sort.h"

#include <omp.h>

#include <algorithm>
#include <iterator>
#include <random>
#include <vector>

std::vector<double> GetRandomVector(int sz) {
  std::random_device dev;
  std::mt19937 gen(dev());
  std::vector<double> vec(sz);
  for (int i = 0; i < sz; i++) {
    vec[i] = gen() % 100;
  }
  return vec;
}

std::vector<double> CountingSort(std::vector<double> source,
                                 std::vector<double> tmp, const int offset) {
  auto const size = static_cast<int>(source.size());
  auto byteArr = (unsigned char*)source.data();
  auto acc = 0;
  int counters[256] = {0};

  for (auto i = 0; i < size; i++) {
    counters[byteArr[offset + 8 * i]]++;
  }

  for (auto i = 0; i < 256; i++) {
    int tmp = counters[i];
    counters[i] = acc;
    acc += tmp;
  }

  for (auto i = 0; i < size; i++) {
    auto byteIndex = offset + 8 * i;
    auto byte = byteArr[byteIndex];
    auto counter = counters[byte];
    tmp[counter] = source[i];
    counters[byte]++;
  }

  return tmp;
}

std::vector<double> SignSort(std::vector<double> source) {
  auto const offset = 7;
  auto const size = static_cast<int>(source.size());
  auto byteArr = (unsigned char*)source.data();
  auto acc = 0;

  auto result = std::vector<double>(source);
  int counters[256] = {0};

  for (auto i = 0; i < size; i++) {
    counters[byteArr[7 + 8 * i]]++;
  }

  for (int i = 255; i > 127; i--) {
    counters[i] += acc;
    acc = counters[i];
  }

  for (int i = 0; i < 128; i++) {
    int tmp = counters[i];
    counters[i] = acc;
    acc += tmp;
  }

  for (int i = 0; i < size; i++) {
    auto byteIndex = offset + 8 * i;
    auto byte = byteArr[byteIndex];
    if (byte < 128) {
      result[counters[byte]] = source[i];
      counters[byte]++;
    } else {
      counters[byte]--;
      result[counters[byte]] = source[i];
    }
  }

  return result;
}

std::vector<double> serialRadixSort(const std::vector<double>& source) {
  auto tmp = std::vector<double>(source.size());
  auto tmpForSwap = std::vector<double>(source.size());
  auto copy = std::vector<double>(source);

  for (int i = 0; i < 8; i++) {
    tmp = CountingSort(copy, tmp, i);
    tmpForSwap = copy;
    copy = tmp;
    tmp = tmpForSwap;
  }

  return SignSort(copy);
}

std::vector<double> RadixSort(const std::vector<double>& source) {
  int threadCount = 4;
  std::vector<double> resultVector;
  std::vector<std::vector<double>> sortedVectors;

  int bunch_size = (source.size() - 1) / threadCount + 1;
  std::vector<std::vector<double>> bunches((source.size() + bunch_size) /
                                           bunch_size);
  int bunchesCount = bunches.size();

  for (size_t i = 0; i < source.size(); i += bunch_size) {
    auto last = std::min(source.size(), i + bunch_size);
    auto index = i / bunch_size;
    auto& vec = bunches[index];
    vec.reserve(last - i);
    move(source.begin() + i, source.begin() + last, back_inserter(vec));
  }

  bunchesCount = bunches.size();

#pragma omp parallel num_threads(threadCount)
  {
#pragma omp for
    for (auto i = 0; i < bunchesCount; i++) {
      auto sortedBunch = serialRadixSort(bunches[i]);
#pragma omp critical
      sortedVectors.push_back(sortedBunch);
      sortedBunch.clear();
    }
  }

  resultVector = sortedVectors[0];
  if (bunchesCount != 1) {
    for (auto i = 0; i < bunchesCount - 1; i++) {
      resultVector = Merge(resultVector, sortedVectors[i + 1]);
    }
  }
  sortedVectors.clear();
  bunches.clear();

  return resultVector;
}

std::vector<double> Merge(const std::vector<double>& left,
                          const std::vector<double>& right) {
  std::vector<double> result((left.size() + right.size()));

  auto const leftSize = static_cast<int>(left.size());
  auto const rightSize = static_cast<int>(right.size());

  auto i = 0, j = 0, k = 0;
  for (k = 0; k < (leftSize + rightSize - 1) && i < leftSize && j < rightSize;
       k++) {
    if (left[i] < right[j])
      result[k] = left[i++];
    else
      result[k] = right[j++];
  }

  while (i < leftSize) {
    result[k++] = left[i++];
  }
  while (j < rightSize) {
    result[k++] = right[j++];
  }

  return result;
}
    \end{lstlisting}

\end{document}