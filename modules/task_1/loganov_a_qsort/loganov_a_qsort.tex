\documentclass{report}

\usepackage[T2A]{fontenc}
\usepackage[utf8]{luainputenc}
\usepackage[english, russian]{babel}
\usepackage[pdftex]{hyperref}
\usepackage[14pt]{extsizes}
\usepackage{listings}
\usepackage{color}
\usepackage{geometry}
\usepackage{enumitem}
\usepackage{multirow}
\usepackage{graphicx}
\usepackage{indentfirst}
\usepackage[ruled,vlined,linesnumbered]{algorithm2e}

\geometry{a4paper,top=2cm,bottom=3cm,left=2cm,right=1.5cm}
\setlength{\parskip}{0.5cm}
\setlist{nolistsep, itemsep=0.3cm,parsep=0pt}

\newenvironment{pseudocode}[1][htb]
  {\renewcommand{\algorithmcfname}{Алгоритм}
   \begin{algorithm}[#1]%
  }{\end{algorithm}}


\lstset{language=C++,
		basicstyle=\footnotesize,
		keywordstyle=\color{blue}\ttfamily,
		stringstyle=\color{red}\ttfamily,
		commentstyle=\color{green}\ttfamily,
		morecomment=[l][\color{magenta}]{\#}, 
		tabsize=4,
		breaklines=true,
  		breakatwhitespace=true,
  		title=\lstname,       
}

\makeatletter
\renewcommand\@biblabel[1]{#1.\hfil}
\makeatother

\begin{document}

\begin{titlepage}

\begin{center}
Министерство науки и высшего образования Российской Федерации
\end{center}

\begin{center}
Федеральное государственное автономное образовательное учреждение высшего образования \\
Национальный исследовательский Нижегородский государственный университет им. Н.И. Лобачевского
\end{center}

\begin{center}
Институт информационных технологий, математики и механики
\end{center}

\vspace{4em}

\begin{center}
\textbf{\LargeОтчет по лабораторной работе} \\
\end{center}
\begin{center}
\textbf{\LargeСортировка Хоара с простым слиянием.} \\
\end{center}

\vspace{4em}

\newbox{\lbox}
\savebox{\lbox}{\hbox{text}}
\newlength{\maxl}
\setlength{\maxl}{\wd\lbox}
\hfill\parbox{7cm}{
\hspace*{5cm}\hspace*{-5cm}\textbf{Выполнил:} \\ студент группы 381806-2 \\ Логанов А.С.\\
\\
\hspace*{5cm}\hspace*{-5cm}\textbf{Проверил:}\\ доцент кафедры МОСТ, \\ кандидат технических наук \\ Сысоев А. В.
}

\vspace{\fill}

\begin{center} Нижний Новгород \\ 2021 \end{center}

\end{titlepage}

\setcounter{page}{2}

\tableofcontents
\newpage

\section*{Введение}
\addcontentsline{toc}{section}{Введение}
Сортировка Хоара(Quicksort) - алгоритм сортировки, разработанный английским информатиком Тони Хоаром во время его работы в МГУ в 1960 году. Один из самых быстрых известных универсальных алгоритмов сортировки массивов: в среднем O(n*log(n)) обменов при упорядочении n элементов, из-за наличия ряда недостатков на практике обычно используется с некоторыми доработками.
\par QuickSort является существенно улучшенным вариантом алгоритма сортировки с помощью прямого обмена (его варианты известны как Пузырьковая сортировка и Шейкерная сортировка), известного в том числе своей низкой эффективностью. Принципиальное отличие состоит в том, что в первую очередь производятся перестановки на наибольшем возможном расстоянии и после каждого прохода элементы делятся на две независимые группы. Таким образом улучшение самого неэффективного прямого метода сортировки дало в результате один из наиболее эффективных улучшенных методов.
\par В данной лабораторной работе будет рассмотрен классический алгоритм сортировки Хоара с использованием технологий параллельных вычислений.
\newpage

\section*{Постановка задачи}
\addcontentsline{toc}{section}{Постановка задачи}
В текущей работе необходимо реализовать несколько функций с поддержкой различных технологий параллелизма, реализующих классический алгоритм сортировки Хоара.
\par В ходе данной работы должны быть разработаны:
\begin{itemize}
\item последовательный метод классической сортировки Хоара;
\item параллельный метод сортировки Хоара с помощью технологии OpenMP;
\item параллельный метод сортировки Хоара с помощью технологии TBB;
\item набор автоматических тестов с использованием Google C++ Testing Framework, подтверждающий корректность результатов для каждой из технологий;
\end{itemize}
\par После тестирования и замеров времени для каждой из технологий необходимо провести анализ результатов и составить соответствующие выводы.

\newpage

\section*{Описание алгоритма}
\addcontentsline{toc}{section}{Описание алгоритма}
\par Выбираем из массива элемент, называемый ведущим или опорным. Это может быть любой из элементов массива,однако в классической реализации данной сортировки выбирается центральный элемент. От выбора опорного элемента не зависит корректность алгоритма, но в отдельных случаях может сильно зависеть его эффективность.
\par Сравнить все остальные элементы с опорным и переставить их в массиве так, чтобы разбить массив на три непрерывных отрезка, следующих друг за другом: элементы меньшие опорного, равные и большие.
\par Для отрезков меньших и больших значений выполнить рекурсивно ту же последовательность операций, если длина отрезка больше единицы.

\newpage

\section*{Схема распараллеливания}
\addcontentsline{toc}{section}{Схема распараллеливания}
Распараллеливание сортировки происходит в два этапа:
\begin{itemize}
\item Параллельно сортируются несколько непересекающихся кусков массива (предварительно раздать каждому потоку соответствующий кусок);
\item Слияние полученных результатов в один массив, размер которого равен размеру исходного массива;
\end{itemize}
\newpage

\section*{Описание программной реализации}
\addcontentsline{toc}{section}{Описание программной реализации}
Функция последовательной сортировки принимает указатель на сортируемый вектор и границы, в рамках которых необходимо провести сортировку.
\begin{lstlisting}
void Hoarsort(std::vector<double>* vec, int first, int last);
\end{lstlisting}
\par В случае когда левая граница больше правой мы выходим из текущей функции.
\begin{lstlisting}
if (first >= last)return;
\end{lstlisting}
\par После того как центральный элемент исходного вектора стал опорным, с помощью цикла с предусловием находим элементы в левой и правой половинке массива больше  и меньше опорного соответственно. В случае если  индекс найденного элемента в левой половине меньше индекса найденного элемента в правой половине, то их необходимо поменять местами.
\par После прохождения всех итераций внешнего цикла, необходимо рекурсивно вызвать сортировку для каждой из получившихся частей массива.  

\subsection*{OpenMP}
\addcontentsline{toc}{subsection}{OpenMP}
Для распараллеливания алгоритма была создана параллельная секция с помощью следующей директивы:
\par \verb|#pragma omp parallel | 
\par Внутри данной параллельной секции каждый поток параллельно запускает сортировку для своей области массива. Далее с помощью директивы:  \par\verb|#pragma omp barrier | каждый поток приостанавливается в барьере, до тех пор пока все потоки не будут выполнять барьер. Это сделано для того, что бы на момент слияния всех отсортированных частей, каждый поток успел выполнить свою задачу.
\par Далее для корректной работы алгоритма необходимо выполнить слияние отсортированных частей, слияние будет выполнятся исключительно на основном потоке с помощью директивы:  \verb|#pragma omp master |.


\subsection*{TBB}
\addcontentsline{toc}{subsection}{TBB}
В версии TBB используется специальный инструмент:
\par\verb|tbb::parallel_for(...)|
\par Первым аргументом на вход подается итерационное пространство: \par\verb|tbb::blocked_range<int>(0, size, grainsize)| 
\par итерационное пространство с указанными границами [0; size), в данной работе size - число потоков, а размер порции вычислений (grainsize) = 1, так как необходимо, что бы каждый поток выполнял только одну итерацию цикла реализованную в классе функторе.
\par Вторым аргументом \verb|tbb::parallel_for(...)| является класс функтор в рамках которого указан участок кода, который должен выполняться параллельно.
\par Далее выполним слияние всех получившихся кусочков массива.
\newpage


\section*{Подтверждение корректности}
\addcontentsline{toc}{section}{Подтверждение корректности}
Для подтверждения корректности в программе представлен набор тестов, разработанных с помощью использования Google C++ Testing Framework.
\par Набор включает в себя тесты, проверяющие корректность работы различных параллельных версий соритировки Хоара в сравнении с последовательной версией. Массивы для сортировки генерируются случайным образом. 
\par Успешное прохождение всех тестов доказывает корректность работы системы.

\newpage

\section*{Результаты экспериментов}
\addcontentsline{toc}{section}{Результаты экспериментов}
Вычислительные эксперименты проводились на оборудовании со следующей аппаратной конфигурацией:

\begin{itemize}
\item Процессор: Intel Core i3-3220, 3.30 GHz, 2 ядра;
\item Оперативная память: 8 GB DDR4;
\item ОС: Microsoft Windows 10 Home.
\end{itemize}

\par Эксперименты проводились на массиве длиной в 3000000 элементов. 
\par Результаты экспериментов:
\begin{table}[!h]
\centering
\begin{tabular}{lllll}
Кол-во потоков & Последовательно,с & OpenMP,с & Ускорение  \\
2 & 0.22 & 0.1037 & 2.12  \\
4 & 0.22 & 0.1215 & 1.81  \\
8 & 0.22 & 0.1294 & 1.7 
\end{tabular}
\caption{Результаты вычислительных экспериментов OpenMP}
\end{table}

\begin{table}[!h]
\centering
\begin{tabular}{lllll}
Кол-во потоков & Последовательно,с & TBB,с & Ускорение  \\
2 & 0.22 & 0.1157 & 1.9  \\
4 & 0.22 & 0.1279 & 1.72  \\
8 & 0.22 & 0.1349 & 1.63 
\end{tabular}
\caption{Результаты вычислительных экспериментов TBB}
\end{table}


\par По результатам экспериментов видно, что параллельная реализация работает быстрее, чем последовательная для всех технологий. Однако, учитывая, что имеем машину с двух ядерным процессором, видно что при увелечении числа потоков время потраченное на слияние отсортированных кусков становится больше (4 и 8 кусков для слияния, в то время как реальных вычислителей только 2), отсюда и возникает увелечения времени работы параллельных реализаций. OpenMP работает несколько быстрее чем TBB, это может быть связано с накладными расходами используемых библеотек.


\section*{Заключение}
\addcontentsline{toc}{section}{Заключение}
В ходе данной лабораторной работы были разработаны последовательная и параллельные реализации алгоритма сортировки Хоара, для параллельной части использовалось простое слияние.
Из результатов видно, что параллельные методы действительно работают быстрее, чем последовательная.
После проведения анализа результатов, технология OpenMP оказалась более эффективной технологии TBB.
\par Также для оценки корректности выполнения каждой реализации, были разработаны и доведены до успешного выполнения тесты, созданные для данной программы с использованием Google C++ Testing Framework.

\newpage
\begin{thebibliography}{1}
\addcontentsline{toc}{section}{Список литературы}
\bibitem{Sysoev} Сысоев А.В., Мееров И.Б., Свистунов А.Н., Курылев А.Л., Сенин А.В., Шишков А.В., Корняков К.В., Сиднев А.А. «Параллельное программирование в системах с общей памятью. Инструментальная поддержка». Учебно-методические материалы по программе повышения квалификации «Технологии высокопроизводительных вычислений для обеспечения учебного процесса и научных исследований». Нижний Новгород, 2007, 110 с. 
\bibitem{Wiki1} Wikipedia: the free encyclopedia [Электронный ресурс] // URL: https://en.wikipedia.org/wiki/
\bibitem{Sidnev} А.А. Сиднев, А.В. Сысоев, И.Б. Мееров Библиотека Intel Threading Building Blocks – краткое описание, 2007, 29 с. 
\end{thebibliography}
\newpage


\section*{Приложение}
\addcontentsline{toc}{section}{Приложение}

\par 1. Последовательная реализация.
\begin{lstlisting}
// Copyright 2021 Loganov Andrei
#ifndef MODULES_TASK_1_LOGANOV_A_QSORT_QSORT_H_
#define MODULES_TASK_1_LOGANOV_A_QSORT_QSORT_H_
#include <vector>
void Hoarsort(std::vector<double>* vec, int first, int last);
std::vector<double> getRandomVector(int size);
std::vector<double> copyvec(std::vector<double> vec);
#endif  // MODULES_TASK_1_LOGANOV_A_QSORT_QSORT_H_

\end{lstlisting}
\begin{lstlisting}
// Copyright 2021 Loganov Andrei
#include <ctime>
#include <vector>
#include <random>
#include <algorithm>
#include "../../../../modules/task_1/loganov_a_qsort/qsort.h"
std::vector<double> copyvec(std::vector<double> vec) {
    std::vector<double> result(vec.size());
    for (int i=0; i<static_cast<int>(vec.size()); i++) {
        result[i] = vec[i];
    }
    return result;
}
std::vector<double> getRandomVector(int size) {
    std::mt19937 gen;
    std::vector<double> result(size);
    const int start = 0;
    const int end = 1000;
    gen.seed(static_cast<unsigned int>(time(0)));
    std::uniform_real_distribution<double> rand(start, end);
    for (int i = 0; i < static_cast<int>(result.size()); i++) {
        result[i] = rand(gen);
    }
    return result;
}
void Hoarsort(std::vector<double>* vec, int first, int last) {
    if (first >= last)return;
    int i = first, j = last;
    double  mid = (*vec)[(last + first) / 2];
    while (i <= j) {
        while ((*vec)[i] < mid) i++;
        while ((*vec)[j] > mid)j--;
        if (i < j) std::swap((*vec)[i], (*vec)[j]);
        if (i <= j) {
            i++;
            j--;
        }
    }
    Hoarsort(vec, first, j);
    Hoarsort(vec, i, last);
}


\end{lstlisting}
\begin{lstlisting}
// Copyright 2021 Loganov Andrei
#include <gtest/gtest.h>
#include <vector>
#include <numeric>
#include <algorithm>
#include "./qsort.h"
TEST(Q_SORT_TESTS, Test1) {
    std::vector<double> vec;
    std::vector<double> vec2;
    vec = {10.17, -0.45, 40, -1, 12, 34.27, 77, -0.4, 17};
    vec2 = {-1, -0.45, -0.4, 10.17, 12, 17, 34.27, 40, 77};
    Hoarsort(&vec, 0, static_cast<int>(vec.size())-1);
    ASSERT_EQ(vec2, vec);
}
TEST(Q_SORT_TESTS, Test2) {
    std::vector<double> vec;
    std::vector<double> vec2;
    vec2 = copyvec(vec);
    Hoarsort(&vec, 0, static_cast<int>(vec.size())-1);
    std::sort(vec2.begin(), vec2.end(),
    [](double a, double b) {return a < b; });
    ASSERT_EQ(vec2, vec);
}
TEST(Q_SORT_TESTS, Test3) {
     std::vector<double> vec;
    std::vector<double> vec2;
    vec = getRandomVector(5000);
    vec2 = copyvec(vec);
    Hoarsort(&vec, 0, static_cast<int>(vec.size())-1);
    std::sort(vec2.begin(), vec2.end(),
    [](double a, double b) {return a < b; });
    ASSERT_EQ(vec2, vec);
}
TEST(Q_SORT_TESTS, Test4) {
     std::vector<double> vec;
    std::vector<double> vec2;
    vec = getRandomVector(1);
    vec2 = copyvec(vec);
    Hoarsort(&vec, 0, static_cast<int>(vec.size())-1);
    std::sort(vec2.begin(), vec2.end(),
    [](double a, double b) {return a < b; });
    ASSERT_EQ(vec2, vec);
}
TEST(Q_SORT_TESTS, Test5) {
    std::vector<double> vec;
    std::vector<double> vec2;
    vec = getRandomVector(50000);
    vec2 = copyvec(vec);
    Hoarsort(&vec, 0, static_cast<int>(vec.size())-1);
    std::sort(vec2.begin(), vec2.end(),
    [](double a, double b) {return a < b; });
    ASSERT_EQ(vec2, vec);
}
int main(int argc, char **argv) {
    ::testing::InitGoogleTest(&argc, argv);
    return RUN_ALL_TESTS();
}

\end{lstlisting}

\par 2. Реализация на OpenMP.
\begin{lstlisting}
// Copyright 2021 Loganov Andrei
#ifndef MODULES_TASK_2_LOGANOV_A_QSORT_QSORT_H_
#define MODULES_TASK_2_LOGANOV_A_QSORT_QSORT_H_
#include <vector>
std::vector<double> getRandomVector(int size);
void Hoarsort(std::vector<double>* vec, int first, int last);
void MergePartsOfVector(double* vec1, int n1, int second, int n2);
void QsortOMP(std::vector<double>* vec);
std::vector<double> copyvec(std::vector<double> vec);
#endif  // MODULES_TASK_2_LOGANOV_A_QSORT_QSORT_H_

\end{lstlisting}
\begin{lstlisting}
// Copyright 2021 Loganov Andrei
#include <omp.h>
#include <ctime>
#include <vector>
#include <random>
#include <algorithm>
#include "../../../../modules/task_2/loganov_a_qsort/qsort.h"
std::vector<double> getRandomVector(int size) {
    std::mt19937 gen;
    std::vector<double> result(size);
    const int start = 0;
    const int end = 1000;
    gen.seed(static_cast<unsigned int>(time(0)));
    std::uniform_real_distribution<double> rand(start, end);
    for (int i = 0; i < static_cast<int>(result.size()); i++) {
    result[i] = rand(gen);
    }
    return result;
}
std::vector<double> copyvec(std::vector<double> vec) {
    std::vector<double> result(vec.size());
    for (int i = 0; i < static_cast<int>(vec.size()); i++) {
    result[i] = vec[i];
    }
    return result;
}
void Hoarsort(std::vector<double>* vec, int first, int last) {
    if (first >= last)return;
    int i = first, j = last;
    double  mid = (*vec)[(last + first) / 2];
    while (i <= j) {
    while ((*vec)[i] < mid) i++;
    while ((*vec)[j] > mid)j--;
    if (i < j) std::swap((*vec)[i], (*vec)[j]);
    if (i <= j) {
        i++;
        j--;
    }
    }
    Hoarsort(vec, first, j);
    Hoarsort(vec, i, last);
}
void MergePartsOfVector(double* vec1, int n1, int second, int n2) {
    int k = 0;
    int g = second;
    int f = 0;
    double* r = new double[static_cast<int64_t>(n1 + n2)];
     for (f = 0; k < n1 && g < static_cast<int64_t>(second + n2); f++) {
     if (vec1[k] < vec1[g]) {
         r[f] = vec1[k];
         k++;
     } else {
         r[f] = vec1[g];
         g++;
     }
     }
    if (g == second + n2) {
    for (int i = k; i < n1; i++) {
        r[f] = vec1[i];
        f++;
    }
    }
    if (k == n1) {
    for (int i = g; i < second + n2; i++) {
        r[f] = vec1[i];
        f++;
    }
    }
    int i = 0;
    while (i < n1 + n2) {
    vec1[i] = r[i];
    i++;
    }
}
void QsortOMP(std::vector<double>* vec) {
#pragma omp parallel
    {
    int threads_count = omp_get_num_threads();
    int sz = (*vec).size();
    int delta = sz / threads_count;
    int remainder = sz % threads_count;
    int id = omp_get_thread_num();
    if (id == 0) {
        Hoarsort(vec, 0, delta + remainder - 1);
    } else {
        Hoarsort(vec, remainder + delta + delta * (id - 1),
        remainder + delta + delta * id - 1);
    }
#pragma omp barrier
#pragma omp master
        {
        int size = delta + remainder;
        for (int i = 1; i < threads_count; i++) {
        MergePartsOfVector(vec->data(), size,
        remainder + delta + delta * (i - 1), delta);
        size += delta;
        }
    }
    }
}


\end{lstlisting}
\begin{lstlisting}
// Copyright 2021 Loganov Andrei
#include <time.h>
#include <omp.h>
#include <gtest/gtest.h>
#include <vector>
#include <numeric>
#include <algorithm>
#include <iostream>
#include "./qsort.h"
TEST(Q_SORT_OMP_TESTS, Test1) {
    std::vector<double> vec;
    std::vector<double> vec2;
    vec = {10.17, -0.45, 40, -1, 12, 34.27, 77, -0.4, 17};
    vec2 = {-1, -0.45, -0.4, 10.17, 12, 17, 34.27, 40, 77};
    QsortOMP(&vec);
    ASSERT_EQ(vec2, vec);
}
TEST(Q_SORT_OMP_TESTS, Test2) {
    std::vector<double> vec;
    std::vector<double> vec2;
    vec2 = copyvec(vec);
    Hoarsort(&vec, 0, static_cast<int>(vec.size())-1);
    QsortOMP(&vec2);
    ASSERT_EQ(vec2, vec);
}
TEST(Q_SORT_OMP_TESTS, Test3) {
    std::vector<double> vec;
    std::vector<double> vec2;
    vec = getRandomVector(213);
    vec2 = copyvec(vec);
    Hoarsort(&vec, 0, static_cast<int>(vec.size())-1);
    QsortOMP(&vec2);
    ASSERT_EQ(vec2, vec);
}
TEST(Q_SORT_OMP_TESTS, Test4) {
    std::vector<double> vec;
    std::vector<double> vec2;
    vec = getRandomVector(564);
    vec2 = copyvec(vec);
    Hoarsort(&vec, 0, static_cast<int>(vec.size())-1);
    QsortOMP(&vec2);
    ASSERT_EQ(vec2, vec);
}
TEST(Q_SORT_OMP_TESTS, Test5) {
    std::vector<double> vec;
    std::vector<double> vec2;
    // time_t t1, t2, t3, t4;
    vec = getRandomVector(600);
    vec2 = copyvec(vec);
    // t1 = clock();
    Hoarsort(&vec, 0, static_cast<int>(vec.size())-1);
    // t2 = clock();
    // double lin = t2 - t1;
    // std::cout << "lin" << lin / CLOCKS_PER_SEC << std::endl;
    // t3 = clock();
    QsortOMP(&vec2);
    // t4 = clock();
    // double par = t4 - t3;
    // std::cout << "par" << par / CLOCKS_PER_SEC << std::endl;
    // std::cout << "div" << lin / par << std::endl;
    ASSERT_EQ(vec2, vec);
}

int main(int argc, char **argv) {
    ::testing::InitGoogleTest(&argc, argv);
    return RUN_ALL_TESTS();
}

\end{lstlisting}
\par 3. Реализация на TBB.
\begin{lstlisting}
// Copyright 2021 Loganov Andrei
#ifndef MODULES_TASK_3_LOGANOV_A_QSORT_LOGANOV_A_QSORT_H_
#define MODULES_TASK_3_LOGANOV_A_QSORT_LOGANOV_A_QSORT_H_
#include <vector>
std::vector<double> getRandomVector(int size);
void Hoarsort(std::vector<double>* vec, int first, int last);
void MergePartsOfVector(double* vec1, int n1, int second, int n2);
void QsortTBB(std::vector<double>* vec, int threads_count = 4);
std::vector<double> copyvec(std::vector<double> vec);
#endif  // MODULES_TASK_3_LOGANOV_A_QSORT_LOGANOV_A_QSORT_H_

\end{lstlisting}
\begin{lstlisting}
// Copyright 2021 Loganov Andrei
#include <tbb/blocked_range.h>
#include <tbb/parallel_for.h>
#include <ctime>
#include <vector>
#include <random>
#include <algorithm>
#include "../../../../modules/task_3/loganov_a_qsort/loganov_a_qsort.h"
std::vector<double> getRandomVector(int size) {
    std::mt19937 gen;
    std::vector<double> result(size);
    const int start = 0;
    const int end = 1000;
    gen.seed(static_cast<unsigned int>(time(0)));
    std::uniform_real_distribution<double> rand(start, end);
    for (int i = 0; i < static_cast<int>(result.size()); i++) {
    result[i] = rand(gen);
    }
    return result;
}
std::vector<double> copyvec(std::vector<double> vec) {
    std::vector<double> result(vec.size());
    for (int i = 0; i < static_cast<int>(vec.size()); i++) {
    result[i] = vec[i];
    }
    return result;
}
void Hoarsort(std::vector<double>* vec, int first, int last) {
    if (first >= last)return;
    int i = first, j = last;
    double  mid = (*vec)[(last + first) / 2];
    while (i <= j) {
    while ((*vec)[i] < mid) i++;
    while ((*vec)[j] > mid)j--;
    if (i < j) std::swap((*vec)[i], (*vec)[j]);
    if (i <= j) {
        i++;
        j--;
    }
    }
    Hoarsort(vec, first, j);
    Hoarsort(vec, i, last);
}
void MergePartsOfVector(double* vec1, int n1, int second, int n2) {
    int k = 0;
    int g = second;
    int f = 0;
    double* r = new double[static_cast<int64_t>(n1 + n2)];
     for (f = 0; k < n1 && g < static_cast<int64_t>(second + n2); f++) {
     if (vec1[k] < vec1[g]) {
         r[f] = vec1[k];
         k++;
     } else {
         r[f] = vec1[g];
         g++;
     }
     }
    if (g == second + n2) {
    for (int i = k; i < n1; i++) {
        r[f] = vec1[i];
        f++;
    }
    }
    if (k == n1) {
    for (int i = g; i < second + n2; i++) {
        r[f] = vec1[i];
        f++;
    }
    }
    int i = 0;
    while (i < n1 + n2) {
    vec1[i] = r[i];
    i++;
    }
}
void QsortTBB(std::vector<double>* vec, int threads_count) {
    int sz = (*vec).size();
    int delta = sz / threads_count;
    int remainder = sz % threads_count;
    tbb::parallel_for(tbb::blocked_range<int>(0, threads_count, 1),
    [&](const tbb::blocked_range<int>& range){
        for (int i = range.begin(); i != range.end(); ++i) {
            if (i == 0) {
                Hoarsort(vec, 0, delta + remainder - 1);
            } else {
        Hoarsort(vec, remainder + delta + delta * (i - 1),
        remainder + delta + delta * i - 1);
            }
        }
    } , tbb::auto_partitioner());
        int size = delta + remainder;
        for (int i = 1; i < threads_count; i++) {
        MergePartsOfVector(vec->data(), size,
        remainder + delta + delta * (i - 1), delta);
        size += delta;
        }
}

\end{lstlisting}
\begin{lstlisting}
// Copyright 2021 Loganov Andrei
#include <time.h>
#include <tbb/tick_count.h>
#include <gtest/gtest.h>
#include <vector>
#include <numeric>
#include <algorithm>
#include <iostream>

#include "./loganov_a_qsort.h"
TEST(Q_SORT_OMP_TESTS, Test1) {
    std::vector<double> vec;
    std::vector<double> vec2;
    vec = {10.17, -0.45, 40, -1, 12, 34.27, 77, -0.4, 17};
    vec2 = {-1, -0.45, -0.4, 10.17, 12, 17, 34.27, 40, 77};
    QsortTBB(&vec);
    ASSERT_EQ(vec, vec2);
}
TEST(Q_SORT_OMP_TESTS, Test2) {
    std::vector<double> vec;
    std::vector<double> vec2;
    vec2 = copyvec(vec);
    Hoarsort(&vec, 0, static_cast<int>(vec.size())-1);
    QsortTBB(&vec2);
    ASSERT_EQ(vec, vec2);
}
TEST(Q_SORT_OMP_TESTS, Test3) {
    std::vector<double> vec;
    std::vector<double> vec2;
    vec = getRandomVector(213);
    vec2 = copyvec(vec);
    Hoarsort(&vec, 0, static_cast<int>(vec.size())-1);
    QsortTBB(&vec2);
    ASSERT_EQ(vec, vec2);
}
TEST(Q_SORT_OMP_TESTS, Test4) {
    std::vector<double> vec;
    std::vector<double> vec2;
    vec = getRandomVector(564);
    vec2 = copyvec(vec);
    Hoarsort(&vec, 0, static_cast<int>(vec.size())-1);
    QsortTBB(&vec2);
    ASSERT_EQ(vec, vec2);
}
TEST(Q_SORT_OMP_TESTS, Test5) {
    std::vector<double> vec;
    std::vector<double> vec2;
    // tbb::tick_count t1, t2, t3, t4;
    vec = getRandomVector(702);
    vec2 = copyvec(vec);
    // t1 = tbb::tick_count::now();
    Hoarsort(&vec, 0, static_cast<int>(vec.size())-1);
    // t2 = tbb::tick_count::now();
    // auto lin = (t2 - t1).seconds();
    // std::cout << "lin" << lin << std::endl;
    // t3 = tbb::tick_count::now();
    QsortTBB(&vec2);
    // t4 = tbb::tick_count::now();
    // auto par = (t4 - t3).seconds();
    // std::cout << "par" << par  << std::endl;
    // std::cout << "div" << lin / par << std::endl;
    ASSERT_EQ(vec, vec2);
}

int main(int argc, char **argv) {
    ::testing::InitGoogleTest(&argc, argv);
    return RUN_ALL_TESTS();
}

\end{lstlisting}


\end{document}