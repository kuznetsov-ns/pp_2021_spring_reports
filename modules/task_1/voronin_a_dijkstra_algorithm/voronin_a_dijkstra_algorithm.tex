\documentclass{report}

\usepackage[T2A]{fontenc}
\usepackage[utf8]{luainputenc}
\usepackage[english, russian]{babel}
\usepackage[pdftex]{hyperref}
\usepackage[14pt]{extsizes}
\usepackage{listings}
\usepackage{color}
\usepackage{geometry}
\usepackage{enumitem}
\usepackage{multirow}
\usepackage{graphicx}
\usepackage{indentfirst}
\usepackage[ruled,vlined,linesnumbered]{algorithm2e}

\geometry{a4paper,top=2cm,bottom=3cm,left=2cm,right=1.5cm}
\setlength{\parskip}{0.5cm}
\setlist{nolistsep, itemsep=0.3cm,parsep=0pt}

\newenvironment{pseudocode}[1][htb]
  {\renewcommand{\algorithmcfname}{Алгоритм}
   \begin{algorithm}[#1]%
  }{\end{algorithm}}


\lstset{language=C++,
		basicstyle=\footnotesize,
		keywordstyle=\color{blue}\ttfamily,
		stringstyle=\color{red}\ttfamily,
		commentstyle=\color{green}\ttfamily,
		morecomment=[l][\color{magenta}]{\#},
		tabsize=4,
		breaklines=true,
  		breakatwhitespace=true,
  		title=\lstname,
}

\makeatletter
\renewcommand\@biblabel[1]{#1.\hfil}
\makeatother

\begin{document}

\begin{titlepage}

\begin{center}
Министерство науки и высшего образования Российской Федерации
\end{center}

\begin{center}
Федеральное государственное автономное образовательное учреждение высшего образования \\
Национальный исследовательский Нижегородский государственный университет им. Н.И. Лобачевского
\end{center}

\begin{center}
Институт информационных технологий, математики и механики
\end{center}

\vspace{4em}

\begin{center}
\textbf{\LargeОтчет по лабораторной работе} \\
\end{center}
\begin{center}
\textbf{\LargeПоиск кратчайших путей из одной вершины (алгоритм Дейкстры)} \\
\end{center}

\vspace{4em}

\newbox{\lbox}
\savebox{\lbox}{\hbox{text}}
\newlength{\maxl}
\setlength{\maxl}{\wd\lbox}
\hfill\parbox{7cm}{
\hspace*{5cm}\hspace*{-5cm}\textbf{Выполнил:} \\ студент группы 381808-1 \\ Воронин А.А\\
\\
\hspace*{5cm}\hspace*{-5cm}\textbf{Проверил:}\\ доцент кафедры МОСТ, \\ кандидат технических наук \\ Сысоев А. В.
}

\vspace{\fill}

\begin{center} Нижний Новгород \\ 2021 \end{center}

\end{titlepage}

\setcounter{page}{2}

\tableofcontents
\newpage

\section*{Введение}
\addcontentsline{toc}{section}{Введение}
Граф — математическая абстракция реальной системы объектов любой природы, обладающих парными связями. Граф как математический объект есть совокупность двух множеств —
множества самих объектов, называемого множеством вершин и множеством их парных связей, называемой множеством рёбер. Элемент множества рёбер есть пара элементов множества вершин.
Графы находят широкое применение в современной науке и технике. Они используются и в естественных науках (физике и химии) и в социальных науках (например, социологии),
но наибольших масштабов применение графов получило в информатике и сетевых технологиях.
Задaча о кратчaйшем пути — задача поиска самого короткого пути между двумя точками (вершинами) на графе,
в которой минимизируется сумма весов рёбер, составляющих путь. Задача о кратчайшем пути является одной из важнейших классических задач теории графов. Сегодня известно множество алгоритмов для её решения.Один из них — алгоритм Дейкстры
Эдсгера Дейкстры. Алгоритм был создан в 1959 и позволяет найти все кратчайшие пути из одной изначально заданной вершины графа до всех остальных. Алгоритм работает только для графов без рёбер отрицательного веса.
Реализация алгоритма Дейкстры, а также вариенты его параллельных реализваций будут представлены в этой лабораторной работе.
\newpage

\section*{Постановка задачи}
\addcontentsline{toc}{section}{Постановка задачи}
Цель данной лабораторной работы - разработка программы на языке C++ реализующего алгоритм Дейкстры, а также реализация ускоренных версий с помощью фреймворков для параллелизации - OpenMP и TBB.
\par В работе представлены:
\begin{itemize}
\item Последовательный версия алгоритма Дейкстры;
\item Параллельный алгоритм Дейкстры реализванный на OpenMP;
\item Параллельный алгоритм Дейкстры реализванный на TBB;
\end{itemize}
\newpage

\section*{Метод решения}
\addcontentsline{toc}{section}{Метод решения}
Каждой вершине из V сопоставим метку — минимальное известное расстояние от этой вершины до a.
Алгоритм работает пошагово — на каждом шаге он «посещает» одну вершину и пытается уменьшать метки.Работа алгоритма завершается, когда все вершины посещены.
На начальном этапе - метка самой вершины a полагается равной 0, метки остальных вершин — бесконечности.
Это отражает то, что расстояния от a до других вершин пока неизвестны.Все вершины графа помечаются как непосещённые.
Затем, с помощью повторений, пока все вершины не будут посещены выбирается непосещённая вершина u, имеющая минимальную метку.
Мы рассматриваем всевозможные маршруты, в которых u является предпоследним пунктом. Вершины, в которые ведут рёбра из u, назовём соседями этой вершины.
Для каждого соседа вершины u, кроме отмеченных как посещённые, рассмотрим новую длину пути, равную сумме значений текущей метки u и длины ребра, соединяющего u с этим соседом.
\newpage

+\section*{Схема распараллеливания}
\addcontentsline{toc}{section}{Схема распараллеливания}
В алгоритме Дейкстры можно выделить два основных участка для эффективного распараллеливания:
\begin{itemize}
\item поиск вершины с минимальным расстоянием до исходной точки(редукция).;
\item поиск и запоминание расстояний от исходной точки до всех непройденных вершин,смежных с обрабатываемой вершиной;
\end{itemize}
\par Соответственно на каждой итерации цикла на первом этапе каждый поток находит локальный минимум проходя по свеому набору вершин, затем избирается глобальный минимум.
После чего каждый из потоков работает со своим наборов перезаписываемых расстояний от исходной вершины до других.
\newpage

\section*{Описание программной реализации}
\addcontentsline{toc}{section}{Описание программной реализации}
Ниже представлена послежовательная реализация алгоритма - вначале отбор специальных случаев, создание необходимых переменных.
Затем исполнение основного цикла и формирование результата.
В качестве входных параметров передаются: граф и номер исходной вершины. номер вершины до которой ищем кратчайший путь.
\begin{lstlisting}
std::vector<int> dijkstra(const std::vector<int>& graph, int start, int end) {
    if (graph.size() == 0) {
        throw "Empty graph";
    }

    if (start == end)
        return std::vector<int>(1, 0);

    if (start > end) {
        std::swap(start, end);
    }

    int points_count = sqrt(graph.size());

    if (sqrt(graph.size()) != points_count) {
        throw "Wrong size";
    }

    int max_weight = std::numeric_limits<int>::max();
    int min, min_point, tmp;
    std::vector<int> points_len(points_count, max_weight);
    std::vector<int> path;
    std::vector<bool> processed(points_count, false);

    // Align start and end with array indexes
    --start;
    --end;

    points_len[start] = 0;

    do {
        min_point = max_weight;
        min = max_weight;

        // Choose a point to work with
        for (int i = 0; i < points_count; i++) {
            if (!processed[i] && points_len[i] < min) {
                min_point = i;
                min = points_len[i];
            }
        }

        if (min_point != max_weight) {
            for (int i = 0; i < points_count; i++) {
                if (graph[min_point * points_count + i] > 0) {
                    tmp = min + graph[min_point * points_count + i];
                    if (points_len[i] > tmp) {
                        points_len[i] = tmp;
                    }
                }
            }
            processed[min_point] = true;
        }
    } while (min_point < max_weight);

    // Configuring a path
    path.push_back(end + 1);
    int weight = points_len[end];

    while (end != start) {
        for (int i = 0; i < points_count; i++) {
            if (graph[end * points_count + i] < 0) {
                throw "Graph weight can not be less then zero.";
            }
            if (graph[end * points_count + i] > 0) {
                tmp = weight - graph[end * points_count + i];
                if (points_len[i] == tmp) {
                    weight = tmp;
                    end = i;
                    path.push_back(i + 1);
                }
            }
        }
    }

    return path;
}
\end{lstlisting}
\par Параллельные реализации - копируют схему работы, за исключением вышеописанных секций.

\subsection*{OpenMP}
\addcontentsline{toc}{subsection}{OpenMP}
Первая часть распараллеливания была реализована с секциями
\par \verb| #pragma omp parallel ,  #pragma omp for, #pragma omp critical|
\par Работа осуществялется в параллельной секции (omp parallel) цикл поиска минимальных расстояний распараллелен с помощью omp for и редукция выполняется в секции omp critical
\par Для распараллеливания втрой секции используется \verb|  #pragma omp parallel for private(tmp)|.

\subsection*{TBB}
\addcontentsline{toc}{subsection}{TBB}
TBB реализация редукции проще в связи с наличием готовой функции \verb|tbb::parallel_reduce(...)|. Разделение на участки проиходит автоматически, обработка осуществляется внутри\verb| tbb:blocked_range|.
\par торой участок кода распараллеленс с помощью  \verb|tbb::parallel_for(...)|.


\newpage


\section*{Подтверждение корректности}
\addcontentsline{toc}{section}{Подтверждение корректности}
Для подтверждения корректности использован GTest framework.
\par Логически тесты можно разделить на проверку отрицательных и положительных сценариев
\begin{itemize}
\item Отрицательный - создание графа с отрицательным весом;
\item Отрицательный - создание неверного графа;
\item Отрицательный - создание пустого графа;
\item Положительный- поиск пути в начальную вершину;
\item Положительный- обработка графа из 3 вершин;
\item Положительный- обработка графа из 6 вершин;
\item Положительный- проверка создания графа;
\end{itemize}

\par Тесты для параллельных частей
\begin{itemize}
\item обработка графа из 6 вершин;
\item сравнение времени и результата с последовательной секцией;
\end{itemize}

\par Если все тесты пройдены, то работу алгоритма можно полагать корректной.

\newpage

\section*{Результаты экспериментов}
\addcontentsline{toc}{section}{Результаты экспериментов}
Оборудование используемое в ходе работ:

\begin{itemize}
\item Процессор: Intel(R) Core(TM) i5-8250U CPU @ 1.60GHz, 3015 MHz, ядер: 8;
\item Оперативная память: 8 GB (DDR4);
\item ОС: Ubuntu 20.10;
\end{itemize}

\par Для замеров времени использован единичный граф с 8000 вершинами.
\par Результаты замеров приведены ниже.


\begin{table}[!h]
\caption{Резултаты вычислительных экспериментов}
\centering
\begin{tabular}{|c|c|c|c|c|c|c|c|}
\hline
\multirow{2}{*}
	{\begin{tabular}[c]{@{}c@{}}Thread\\ count\end{tabular}} &
\multirow{2}{*}
	{\begin{tabular}[c]{@{}c@{}}Sequential\\ algorithm\end{tabular}} &
\multicolumn{2}{c|}
	{Prallel algorithm}	\\
	\cline{3-4} & &
	\multicolumn{1}{c|}{OpenMP} &
	\multicolumn{1}{c|}{TBB}
	\\ \cline{2-4}
	& t, с	    & t, с 		& t, с 			\\ \hline
2   & 1.43     & 0.89       & 0.78       	\\ \hline
4   &  1.43     & 0.67       & 0.62         \\ \hline
\end{tabular}
\end{table}

\par Эксперименты показывают,что параллельные вычисления в алгоритме Дейкстры позволяют получить выигрыш в скорости работы, однако он не так велик. Это связано с тем, что в любых алгоритмах на графах приходится делать много секций синхронизаций.
\newpage



\section*{Заключение}
\addcontentsline{toc}{section}{Заключение}
В процессе работы был реализован алгоритм Дейкстры для поиска кратчайшего пути. С помощью параллельных вычислений получилось ускорить работу алгоритма как с помощью OpenMP так и TBB. TBB реализация показала чуть лучшие результаты. Решение было протестировано c помощью GTest.
\newpage


\begin{thebibliography}{1}
\addcontentsline{toc}{section}{Список литературы}
\item Гергель В.П., Стронгин Р.Г. Основы параллельных вычислений для многопроцессорных вычислительных систем. Учебное пособие – Нижний Новгород: Изд-во ННГУ им. Н.И. Лобачевского, 2003. 184 с. ISBN 5-85746-602-4.(дата обращения: 1.03.2021)
\bibitem{Course} Сысоев А.В., Мееров И.Б., Свистунов А.Н., Курылев А.Л., Сенин А.В., Шишков А.В., Корняков К.В., Сиднев А.А. «Параллельное программирование в системах с общей памятью. Инструментальная поддержка». Учебно-методические материалы по программе повышения квалификации «Технологии высокопроизводительных вычислений для обеспечения учебного процесса и научных исследований». Нижний Новгород, 2007, 110 с.(дата обращения: 1.03.2021)
\bibitem{Wiki1} Wikipedia: the free encyclopedia [Электронный ресурс] // URL: https://en.wikipedia.org/wiki/ (дата обращения: 25.03.2021)
\bibitem{Parallel} Parallel: Что такое OpenMP [Электронный ресурс] // URL: \url { https://parallel.ru/tech/tech_dev/openmp.html} (дата обращения: 10.04.2021)
\bibitem{Intel} Intel: Документация по TBB [Электронный ресурс] // URL: \url {https://software.intel.com/content/www/ru/ru/develop/articles/tbb_async_io.html} (дата обращения: 10.05.2021)
\end{thebibliography}
\newpage

\end{document}