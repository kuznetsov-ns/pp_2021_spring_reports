\documentclass{report}

\usepackage[T2A]{fontenc}
\usepackage[utf8]{luainputenc}
\usepackage[english, russian]{babel}
\usepackage[pdftex]{hyperref}
\usepackage[14pt]{extsizes}
\usepackage{listings}
\usepackage{color}
\usepackage{geometry}
\usepackage{enumitem}
\usepackage{multirow}
\usepackage{graphicx}
\usepackage{indentfirst}
\usepackage[ruled,vlined,linesnumbered]{algorithm2e}

\geometry{a4paper,top=2cm,bottom=3cm,left=2cm,right=1.5cm}
\setlength{\parskip}{0.5cm}
\setlist{nolistsep, itemsep=0.3cm,parsep=0pt}

\newenvironment{pseudocode}[1][htb]
  {\renewcommand{\algorithmcfname}{Алгоритм}
   \begin{algorithm}[#1]%
  }{\end{algorithm}}


\lstset{language=C++,
		basicstyle=\footnotesize,
		keywordstyle=\color{blue}\ttfamily,
		stringstyle=\color{red}\ttfamily,
		commentstyle=\color{green}\ttfamily,
		morecomment=[l][\color{magenta}]{\#}, 
		tabsize=4,
		breaklines=true,
  		breakatwhitespace=true,
  		title=\lstname,       
}

\makeatletter
\renewcommand\@biblabel[1]{#1.\hfil}
\makeatother

\begin{document}

\begin{titlepage}

\begin{center}
Министерство науки и высшего образования Российской Федерации
\end{center}

\begin{center}
Федеральное государственное автономное образовательное учреждение высшего образования \\
Национальный исследовательский Нижегородский государственный университет им. Н.И. Лобачевского
\end{center}

\begin{center}
Институт информационных технологий, математики и механики
\end{center}

\vspace{4em}

\begin{center}
\textbf{\LargeОтчет по лабораторной работе} \\
\end{center}
\begin{center}
\textbf{\LargeВычисление многомерных интегралов с использованием многошаговой схемы (метод прямоугольников)} \\
\end{center}

\vspace{4em}

\newbox{\lbox}
\savebox{\lbox}{\hbox{text}}
\newlength{\maxl}
\setlength{\maxl}{\wd\lbox}
\hfill\parbox{7cm}{
\hspace*{5cm}\hspace*{-5cm}\textbf{Выполнил:} \\ студент группы 381808-1 \\ Новиков Д.М.\\
\\
\hspace*{5cm}\hspace*{-5cm}\textbf{Проверил:}\\ доцент кафедры МОСТ, \\ кандидат технических наук \\ Сысоев А. В.
}

\vspace{\fill}

\begin{center} Нижний Новгород \\ 2021 \end{center}

\end{titlepage}

\setcounter{page}{2}

\tableofcontents
\newpage

\section*{Введение}
\addcontentsline{toc}{section}{Введение}
Вычисление определенных интегралов по формуле Ньютона-Лейбница не всегда возможно. Многие подынтегральные функции не имеют первообразных в виде элементарных функций, поэтому мы во многих случаях не можем найти точное значение определенного интеграла по формуле Ньютона-Лейбница. С другой стороны, точное значение не всегда и нужно. На практике нам часто достаточно знать приближенное значение определенного интеграла с некоторой заданной степенью точности (например, с точностью до одной тысячной). В этих случаях нам на помощь приходят методы численного интегрирования, такие как метод прямоугольников, метод трапеций, метод Симпсона (парабол) и т.п.
\par Суть метода прямоугольников заключается в том, что в качестве приближенного значения определенного интеграла берётся интегральная сумма
\par В лабораторной работе рассматривается параллельная реализация метода прямоугольников для вычисления многомерных интегралов. В качестве технологий распараллеливания вычислений используются OpenMP и TBB.
\newpage

\section*{Постановка задачи}
\addcontentsline{toc}{section}{Постановка задачи}
В рамках лабораторной работы поставлена задача разработки несколько версий библиотек, производящих вычисление многомерных интегралов при помощи метода прямоугольников.
\par Версии библиотек, реализующиеся в данной лабораторной работе:
\begin{itemize}
\item Последовательная версия метода прямоугольников. Необходима для определения корректности результата работы параллельных версий и для сравнения времени работы различных реализаций
\item Параллельные реализации метода прямоугольников при помощи следующих технологий:
    \begin{itemize}
    \item OpenMP
    \item Threading Building Blocks (TBB)
    \end{itemize}
\end{itemize}
\par Каждая из версий представляет из себя отдельную библиотеку, состоящую из реализации метода и набора юнит-тестов, написанных с использованием фреймворка Google C++ Testing Framework. Они позволяют автоматизировать тестирование корректности реализации. Также, в виде отдельного теста представлено измерение времени работы конкретной реализации.
\newpage

\section*{Метод решения}
\addcontentsline{toc}{section}{Метод решения}
В методе прямоугольников подынтегральную функцию на частичном отрезке $\left[x_{j-1}, x_{j}\right]$ заменяют полиномом Лагранжа нулевого порядка, построенным в одной точке. В качестве этой точки можно выбрать серединц частичного отрезка $x_{j-1}$, $x_{j-0.5}$=$x_{j}$-0.5h. Тогда значение интеграла на частичном отрезке:
\begin{equation}
\int_{x_{j-1}}^{x_{j}}{f\left(x \right)dx \approx f\left(x_{j-0.5}\right)\cdot h}
\end{equation}
\par Подставив это выражение в квадратурную формулу вычисления интегралов, получим составную формулу средних прямоугольников:
\begin{equation}
\int_{x_{j-1}}^{x_{j}}{f\left(x \right)dx \approx \sum_{j=1}^{N}{f\left(x_{j-0.5} \right)\cdot h}}
\end{equation}
\par Площадь криволинейной трапеции приближенно заменяется площадью многоугольника, составленного из N прямоугольников. Таким образом, вычисление определенного интеграла сводится к нахождению суммы N элементарных прямоугольников.
\newpage

\section*{Схема распараллеливания}
\addcontentsline{toc}{section}{Схема распараллеливания}
В методе прямоугольников используется распараллеливание подсчета суммы N площадей прямоугольников.
\par В программной реализации подсчет значений суммы площадей производится при помощи цикла. На каждой итерации цикла подсчитывается значение функции и накапливается в соответствующую переменную.
\par Так как все итерации цикла являются одинаковыми по алгоритмической сложности, то подойдет простая схема распределения итераций на блоки для параллельного вычисления. Блоки в самом простом случае могут быть непрерывным отрезком итераций, которые каждое вычислительное устройство будет выполнять параллельно. По окончании вычислений результаты из каждого блока необходимо объединить друг с другом. В OpenMP и TBB для этого предусмотрены специальные циклы for с редукцией.
\newpage

\section*{Описание программной реализации}
\addcontentsline{toc}{section}{Описание программной реализации}
Метод прямоугольников для подсчета многомерных интегралов реализован в виде функции, принимающей на вход интегрируемую функцию; начальную и конечную точку; число шагов интегрирования.
\begin{lstlisting}
double rectangles_base(Function function, std::vector <double> begin_point,
    std::vector <double> end_point, const int number_of_partitions) {
    if (number_of_partitions <= 0) {
        throw 1;
    }

    if (begin_point.size() != end_point.size()) {
        throw 1;
    }

    int dimension = begin_point.size();
    std::vector<double> h(dimension);

    for (int i = 0; i < dimension; ++i) {
        h[i] = (end_point[i] - begin_point[i]) /
        static_cast<double>(number_of_partitions);
    }
    std::vector <double> half_point(dimension);
    double result = 0.0;
    int k;
    int sectors = std::pow(number_of_partitions, dimension);
    for (int i = 0; i < sectors; ++i) {
        for (k = 0; k < dimension; k++) {
            int a = std::pow(number_of_partitions, k+1);
            int b = (number_of_partitions * k == 0 ? 1 :
                     std::pow(number_of_partitions, k));
            int p = (i % a) / b;
            half_point[k] = begin_point[k] + h[k] * p + h[k] * 0.5;
        }
        result += function(half_point);
    }
    for (int i = 0; i < dimension; ++i) {
        result *= h[i];
    }
    return result;
}
\end{lstlisting}
\par Параллельные реализации реализованы также в виде отдельной функции. Сигнатуры всех версий различаются только названием, параметры и возвращаемое значение совпадают.
\begin{lstlisting}
double parallelRectanglesMethod(
   Function function, std::vector <double> begin_point,
    std::vector <double> end_point, const int number_of_partitions);
\end{lstlisting}

\subsection*{OpenMP}
\addcontentsline{toc}{subsection}{OpenMP}

\begin{lstlisting}
for (int i = 1; i < stepsCount; ++i) {
    for (int j = 0; j < static_cast<int>(segments.size()); ++j)
        currentPoint[j] = firstPoint[j] + i * segmentsSteps[j];

    if (i % 2 == 0)
        evenValues += func(currentPoint);
    else
        oddValues += func(currentPoint);
}
\end{lstlisting}
\par Данный цикл возможно распараллелить при помощи следующий директивы OpenMP:
\par \verb|#pragma omp parallel for reduction (+: result) |
\par Так как необходимо сумма всех значений функции, то для данной задачи подходит использование цикла с редукцией. Конструкция \verb|reduction(+: result)| означает, что OpenMP должен применить редукцию после окончания работы цикла, а именно просуммировать все значения по потокам в соответствующую переменную result. Таким образом, после параллельного блока в переменной окажется корректное значение.

\subsection*{TBB}
\addcontentsline{toc}{subsection}{TBB}
Технология TBB также как и OpenMP предоставляет возможность распараллеливания цикла с последующей редукцией. Такая возможность достигается при помощи функции \verb|tbb::parallel_reduce| В реализации данная функция используется следующим образом:
\begin{lstlisting}
 result = tbb::parallel_reduce(
                    tbb::blocked_range<int>(0, sectors), 0.0,
                    [&](tbb::blocked_range<int> r, double running_total)
\end{lstlisting}
\par\verb|tbb::blocked_range| представляет собой одномерное итерационное пространство. Это полуинтервал $[begin, end)$, который затем рекурсивно делится - расщепляется. В данном случае \verb|tbb::blocked_range<int>(0, sectors)| задает целочисленный полуинтервал $[0, sectors)$, где sectors - число секторов.



\newpage

\section*{Подтверждение корректности}
\addcontentsline{toc}{section}{Подтверждение корректности}
В библиотеке содержится набор юнит-тестов, разработанных при помощи фреймворка тестирования Google C++ Testing Framework. Данные тесты позволяют подтвердить корректность каждой из реализаций метода прямоугольников. Успешное прохождение тестов означает, что версия реализована правильно.
\par Набор включает в себя тестирование вычисления определенного интеграла от одномерных и многомерных функций с заранее подсчитанным результатом интегрирования. Функции, используемые при тестировании:
\begin{itemize}
\item $f(x)=5$ - константная функция
\item $f(x)=2x^2 - cos(x)$ - одномерная функция
\item $f(x,y)=x^2 + y^3$ - двумерная функция
\item $f(x,y,z)=x - y + sin(z)$ - трехмерная функция
\end{itemize}

\par Набор содержит в себе также нагрузочный тест. В нем содержатся измерения времени работы последовательной и параллельных версий. Данный тест позволяет измерить действительное ускорение времени работы параллельной версии по сравнению с последовательной.
\par В наборе также присутствуют тесты на некорректные данные. Такие тесты позволяют протестировать корректность поведения реализации, если поданы некорректные данные. Добавлен тесты, проверяющие невозможность работы системы если сегменты интегрирования пусты, а также случай, когда количество шагов интегрирования равно нулю.

\newpage

\section*{Результаты экспериментов}
\addcontentsline{toc}{section}{Результаты экспериментов}
Вычислительные эксперименты для оценки эффективности метода прямоугольников для вычисления многомерного интеграла проводились на оборудовании со следующей аппаратной конфигурацией:

\begin{itemize}
\item ЦП: AMD Ryzen 5 3500U with Radeon Vega, 2.40GHz. 4 ядра, 8 логических процессоров
\end{itemize}

\par Для определения времени работы каждой из реализации была выбрана функция $f(x,y,z)=x - y + sin(z)$, количество шагов интегрирования равно 10000000. 
\par Результаты проведенных экспериментов представлены в Таблице 1.

\begin{table}[!h]
\caption{Результаты нагрузочных экспериментов}
\centering
\begin{tabular}{|c|c|c|c|c|c|c|c|}
\hline
\multirow{3}{*}
	{\begin{tabular}[c]{@{}c@{}}Кол-во\\ потоков\end{tabular}} & 
\multirow{2}{*}
	{\begin{tabular}[c]{@{}c@{}}Последовательный\\ алгоритм\end{tabular}} & 
\multicolumn{6}{c|}
	{Параллельный алгоритм}	\\ 
	\cline{3-8} & & 
	\multicolumn{2}{c|}{OpenMP} & 
	\multicolumn{2}{c|}{TBB} &  
	\\ \cline{2-8}
	& t, с	    & t, с & speedup	& t, с & speedup		\\ \hline
2   & 14.21   & 10.25 & 1.38      & 6.37 & 2.23        	           \\ \hline
3   & 14.21     & 6.49 & 2.18       & 5.57 & 2.55        	           \\ \hline
4   & 14.21     & 6.21 & 2.28       &5.12 & 2.77         	           \\ \hline
\end{tabular}
\end{table}
\par Видно, что каждая из параллельных реализаций имеет ускорение относительно последовательной версии. OpenMP при любом количестве потоков показала худшие результаты, чем TBB версия. Вероятнее всего OpenMP требует больше ресурсов при работе с циклами с редукцией.
\parДля данной реализации метода прямоугольников очевидно наиболее предпочтительна технология TBB, так как она показывает лучшие результаты по сравнению с OpenMP, а также она удобна в работе.  . OpenMP обладает самым простым, понятным интерфейсом, но результаты экспериментов оставляют желать лучшего.
\newpage


\section*{Заключение}
\addcontentsline{toc}{section}{Заключение}
В результате выполнения данной лабораторной работы были разработаны библиотеки, производящие вычисления многомерных интегралов при помощи метода прямоугольников. Для параллельных реализаций использованы технологии OpenMP и Threading Building Blocks. Были проведены измерения производительности каждой из версий.
\par Для подтверждения корректности работы каждой из реализаций был разработан набор юнит-тестов, использующих фреймворк Google C++ Testing Framework. Успешное прохождение всех тестов подтверждает, что параллельные реализации реализованы корректно.
\newpage


\begin{thebibliography}{1}
\addcontentsline{toc}{section}{Список литературы}
\bibitem{Wiki1} Учебное пособие по курсу "Численные методы в оптике" - Университет ИТМО // URL: http://aco.ifmo.ru/el\_books/numerical\_methods/lectures/glava2\_2.htm
\bibitem{Sysoev} Сысоев А.В., Мееров И.Б., Свистунов А.Н., Курылев А.Л., Сенин А.В., Шишков А.В., Корняков К.В., Сиднев А.А. «Параллельное программирование в системах с общей памятью. Инструментальная поддержка». Учебно-методические материалы по программе повышения квалификации «Технологии высокопроизводительных вычислений для обеспечения учебного процесса и научных исследований». Нижний Новгород, 2007, 110 с.l
\bibitem{TBB} Intel® Threading Building Blocks Documentation [Электронный ресурс] // URL: https://software.intel.com/en-us/tbb-documentation
\end{thebibliography}
\newpage

\end{document}