\documentclass{report}

\usepackage[T2A]{fontenc}
\usepackage[utf8]{luainputenc}
\usepackage[english, russian]{babel}
\usepackage[pdftex]{hyperref}
\usepackage[14pt]{extsizes}
\usepackage{color}
\usepackage{geometry}
\usepackage{enumitem}
\usepackage{multirow}
\usepackage{graphicx}
\usepackage{indentfirst}

\usepackage{listings}
\usepackage{xcolor}

%New colors defined below
\definecolor{codegreen}{rgb}{0,0.6,0}
\definecolor{codegray}{rgb}{0.5,0.5,0.5}
\definecolor{codepurple}{rgb}{0.58,0,0.82}
\definecolor{backcolour}{rgb}{0.95,0.95,0.92}

%Code listing style named "mystyle"
\lstdefinestyle{mystyle}{
  backgroundcolor=\color{backcolour},   commentstyle=\color{codegreen},
  keywordstyle=\color{magenta},
  numberstyle=\tiny\color{codegray},
  stringstyle=\color{codepurple},
  basicstyle=\ttfamily\footnotesize,
  breakatwhitespace=false,         
  breaklines=true,                 
  captionpos=b,                    
  keepspaces=true,                 
  numbers=left,                    
  numbersep=5pt,                  
  showspaces=false,                
  showstringspaces=false,
  showtabs=false,                  
  tabsize=2
}

%"mystyle" code listing set
\lstset{style=mystyle}

\geometry{a4paper,top=2cm,bottom=3cm,left=2cm,right=1.5cm}
\setlength{\parskip}{0.5cm}
\setlist{nolistsep, itemsep=0.3cm,parsep=0pt}

\lstset{language=C++,
		basicstyle=\footnotesize,
		keywordstyle=\color{blue}\ttfamily,
		stringstyle=\color{red}\ttfamily,
		commentstyle=\color{green}\ttfamily,
		morecomment=[l][\color{magenta}]{\#}, 
		tabsize=4,
		breaklines=true,
  		breakatwhitespace=true,
  		title=\lstname,       
}

\makeatletter
\renewcommand\@biblabel[1]{#1.\hfil}
\makeatother

\begin{document}

\begin{titlepage}

\begin{center}
Министерство науки и высшего образования Российской Федерации
\end{center}

\begin{center}
Федеральное государственное автономное образовательное учреждение высшего образования \\
Национальный исследовательский Нижегородский государственный университет им. Н.И. Лобачевского
\end{center}

\begin{center}
Институт информационных технологий, математики и механики
\end{center}

\vspace{4em}

\begin{center}
\textbf{\LargeОтчет по лабораторной работе} \\
\end{center}
\begin{center}
\textbf{\Large«Вычисление многомерных интегралов с использованием многошаговой схемы. Метод Симпсона»} \\
\end{center}

\vspace{4em}

\newbox{\lbox}
\savebox{\lbox}{\hbox{text}}
\newlength{\maxl}
\setlength{\maxl}{\wd\lbox}
\hfill\parbox{7cm}{
\hspace*{5cm}\hspace*{-5cm}\textbf{Выполнил:} \\ студент группы 381806-1 \\ Власов М. С.\\
\\
\hspace*{5cm}\hspace*{-5cm}\textbf{Проверил:}\\ доцент кафедры МОСТ, \\ кандидат технических наук \\ Сысоев А. В.\\
}
\vspace{\fill}

\begin{center} Нижний Новгород \\ 2021 \end{center}

\end{titlepage}

\setcounter{page}{2}

% Содержание
\tableofcontents
\newpage

% Введение
\section*{Введение}
\addcontentsline{toc}{section}{Введение}
Интеграл — одно из важнейших понятий математического анализа, которое возникает при решении задач о нахождении площади под кривой, пройденного пути при неравномерном движении, массы неоднородного тела и других свойств математических и физических объектов. Но это недостаточно для полного описания и изучения некоторых объектов и их состояний, таких как объем тела, площадь произвольной поверхности, масса физического тела, координаты центра масс однородного тела, количество вещества. В этом случае на помощь приходят многомерные интегралы от функций с более чем одной переменной.
\par Для вычисления интегралов с помощью компьютера широко применяются методы численного интегрирования. Численное интегрирование — приближенное вычисление значения определенного интеграла.
\par Итак, в данной лабораторной работе будет реализован один из алгоритмов численного многомерного интегрирования с использованием многошаговой схемы - метод Симпсона.
\newpage

% Постановка задачи
\section*{Постановка задачи}
\addcontentsline{toc}{section}{Постановка задачи}
В рамках лабораторной работы необходимо реализовать последовательное и параллельное вычисление многомерного интеграла с использованием многошаговой схемы по формуле Симпсона, проверить корректность работы алгоритмов, провести эксперименты для оценки эффективности параллелизации. По полученным результатам сделать выводы.
\par Для реализации параллельных версий необходимо использовать OpenMP и oneTBB. Для проверки корректности работы алгоритмов используется Google Testing Framework.
\newpage

% Описание алгоритмов
\section*{Описание алгоритмов}
\addcontentsline{toc}{section}{Описание алгоритмов}
Формула Симпсона относится к приёмам численного интегрирования. Суть классического метода заключается в приближении подынтегральной функции $f(x)$ на отрезке $[a, b]$ интерполяционным многочленом второй степени $p_{2}(x)$, то есть приближение графика функции на отрезке параболой. Метод Симпсона имеет порядок погрешности 4 и алгебраический порядок точности 3.
\par Формулой Симпсона называется интеграл от интерполяционного многочлена второй степени на отрезке $[a, b]$:
$$
    \int\limits_a^b {f(x) dx} \approx \int\limits_a^b {p_2(x) dx} = \frac{b - a}{6} \left( f(a) + 4f \left( \frac{a - b}{2} \right) + f(b) \right)
$$
\par Для более точного вычисления интеграла, интервал $[a, b]$ разбивают на $N = 2n$ элементарных отрезков одинаковой длины и применяют формулу Симпсона на составных отрезках. Каждый составной отрезок состоит из соседней пары элементарных отрезков. Значение исходного интеграла является суммой результатов интегрирования на составных отрезках:
$$
    \int\limits_a^b {f(x) dx} \approx \frac{h}{3} \left( f(x_0) + 2 \sum_{j = 1}^{\frac{N}{2} - 1} f(x_{2j}) + 4 \sum_{j = 1}^{\frac{N}{2}} f(x_{2j - 1})+ f(x_N) \right) ,
$$
где $h = \frac{b - a}{N}$ - величина шага, а $x_j = a + jh$ - чередующиеся границы и середины составных отрезков, на которых применяется формула Симпсона.
\par Эта более точная формула также называется формулой Котеса.
\par Для многомерных интегралов формула имеет следующее очевидное обобщение:
$$
    \int\limits_{a_1}^{b_1} \int\limits_{a_2}^{b_2} ... \int\limits_{a_n}^{b_n} {f(X) dx_1 dx_2 ... dx_n} \approx
$$$$
    \approx \frac{h_1 h_2 ... h_n}{3N} \left( f(X_0) + 2 \sum_{j = 1}^{\frac{N}{2} - 1} f(X_{2j}) + 4 \sum_{j = 1}^{\frac{N}{2}} f(X_{2j - 1})+ f(X_N) \right) ,
$$
где $N$ - число шагов, $X = (x_1, x_2, ..., x_n)$ - вектор переменных-аргументов функции, $h_1 = |a_1 - b_1|, h_2 = |a_2 - b_2|, ... , h_n = |a_n - b_n|$ - величины шага.
\newpage

% Схема распараллеливания
\section*{Схема распараллеливания}
\addcontentsline{toc}{section}{Схема распараллеливания}
Для распараллеливания алгоритма вычисления многомерного интеграла с помощью формулы Симпсона необходимо распределить итерации цикла, выполняющего пошаговое вычисление, между доступными потоками, таким образом, чтобы на каждый получилось одинаковое число итераций. При этом результаты вычислений от каждого потока необходимо сложить.
\par Если случится так, что число шагов (общее число итераций) не делится нацело на число доступных потоков, необходимо досчитать неучтенные итерации после параллельного выполнения одним из потоков и так же прибавить к общему результату.
\newpage

% Описание программной реализации
\section*{Описание программной реализации}
\addcontentsline{toc}{section}{Описание программной реализации}
Алгоритм последовательного вычисления многомерного интеграла с использованием многошагового метода Симпсона вызывается с помощью функции:
\begin{lstlisting}
double SimpsonMethod::sequential(
    const std::function<double(const std::vector<double>&)>& func,
    const std::vector<double>& seg_begin,
    const std::vector<double>& seg_end,
    int steps_count);
\end{lstlisting}
\par Входными параметрами являются: функция, которую необходимо проинтегрировать (в виде функтора, указателя на функцию или другого функционального объекта, приводимого к типу std::function), векторы значений - начал и концов отрезков для каждого из параметров интегрируемой функции, а также число шагов. Выходными данными является непосредственно значение интеграла.
\par Реализация распараллеливания метода Симпсона представлена в функции:
\begin{lstlisting}
double SimpsonMethod::parallel(
    const std::function<double(const std::vector<double>&)>& func,
    const std::vector<double>& seg_begin,
    const std::vector<double>& seg_end,
    int steps_count);
\end{lstlisting}
\par Она имеет те же входные параметры и выходные данные, что и предыдущая.
\par Эта функция, в зависимости от используемой библиотеки, использует для параллелизации средства либо OpenMP, либо oneTBB.
\newpage

% Подтверждение корректности
\section*{Подтверждение корректности}
\addcontentsline{toc}{section}{Подтверждение корректности}
Для подтверждения корректности в программе представлен набор тестов, разработанных с Google Testing Framework.
\par Набор представляет из себя тесты, которые проверяют корректность вычислений (сравнивается значение интеграла, полученное благодаря параллельному вычслению, со значением, полученным с помощью последовательной реализации), а также эффективность (вычисление занимаемого последовательной и параллельной реализациями времени и сравнение полученных данных).
\par Успешное прохождение всех тестов подтверждает корректность работы написанной программы.
\newpage

% Результаты экспериментов
\section*{Результаты экспериментов}
\addcontentsline{toc}{section}{Результаты экспериментов}
Эксперименты для оценки эффективности проводились на ПК со следующими характеристиками:

\begin{itemize}
\item Процессор: AMD A6, 2.60 GHz, 2 ядра;
\item Оперативная память: 8 ГБ;
\item ОС: Microsoft Windows 10 Pro.
\end{itemize}

\par Для проведения экспериментов производилось вычисление многомерного интеграла функции $x^2 + y^2 = z$ с помощью метода Симпсона - последовательно и параллельно - с количеством шагов, равным $10^7$.

\begin{table}[!h]
\caption{Результаты экспериментов. Сравнение с OpenMP}
\centering
\begin{tabular}{lllll}
Число потоков & Последовательно & Параллельно & Ускорение \\
1 & 2,935 & 3,298 & 0,899 \\
2 & 3,312 & 1,826 & 1,814 \\
3 & 2,930 & 1,742 & 1,682 \\
4 & 2,784 & 1,701 & 1,637
\end{tabular}
\end{table}

\begin{table}[!h]
\caption{Результаты экспериментов. Сравнение с oneTBB}
\centering
\begin{tabular}{lllll}
Число потоков & Последовательно & Параллельно & Ускорение \\
1 & 3,075 & 3,079 & 0,999 \\
2 & 2,903 & 1,573 & 1,846 \\
3 & 2,847 & 1,646 & 1,729 \\
4 & 2,834 & 1,658 & 1,709
\end{tabular}
\end{table}

\par По данным, полученным в результате экспериментов, можно сделать вывод о том, что параллельный случай работает действительно быстрее, чем последовательный.
\par Если увеличить количество выделяемых потоков до 4, возможно получить деградацию ускорения. Так происходит из-за увеличения накладных расходов на менеджмент ресурсов (переключение между потоками, расчет частичных данных итераций). При дальнейшем увеличении количества потоков ускорения при данном количестве процессорных ядер компьютера, очевидно, не будет.
\newpage

% Заключение
\section*{Заключение}
\addcontentsline{toc}{section}{Заключение}
В ходе выполнения лабораторной работы были разработаны последовательная и параллельная реализации алгоритма вычисления многомерных интегралов с помощью многошаговой схемы по формуле Симпсона.
\par Задача работы была успешно достигнута, поскольку результаты проведенных экспериментов по оценке эффективности показывают, что параллельная реализация работает быстрее, чем последовательная.
\par Кроме того, были написаны тесты с использованием Google Testing Framework, необходимые для подтверждения корректности работы программы.
\newpage

% Список литературы
\begin{thebibliography}{1}
\addcontentsline{toc}{section}{Список литературы}
\bibitem{wikipedia} Википедия: Формула Симпсона [Электронный ресурс] // URL: \url {https://en.wikipedia.org/wiki/Simpson's_rule} (дата обращения: 08.05.2021)
\bibitem{studwood} Studwood: Формула Симпсона для вычисления двойных интегралов [Электронный ресурс] // URL: \url {https://studwood.ru/1710068/matematika_himiya_fizika/formula_simpsona_vychisleniya_dvoynyh_integralov} (дата обращения: 08.05.2021)
\end{thebibliography}
\newpage

% Приложение
\section*{Приложение}
\addcontentsline{toc}{section}{Приложение}
В этом разделе находится листинг всего кода, написанного в рамках лабораторной работы.
\par 1. Библиотека с последовательной реализацией. Файл: simpson\_method.h
\begin{lstlisting}[language=C++]
// Copyright 2021 Vlasov Maksim

#ifndef MODULES_TASK_1_VLASOV_M_SIMPSON_METHOD_SIMPSON_METHOD_H_
#define MODULES_TASK_1_VLASOV_M_SIMPSON_METHOD_SIMPSON_METHOD_H_

#include <functional>
#include <vector>

namespace SimpsonMethod {

double integrate(const std::function<double(const std::vector<double>&)>& func,
                 const std::vector<double>& seg_begin,
                 const std::vector<double>& seg_end, int steps_count);

}  // namespace SimpsonMethod

#endif  // MODULES_TASK_1_VLASOV_M_SIMPSON_METHOD_SIMPSON_METHOD_H_
\end{lstlisting}

\par 2. Библиотека с последовательной реализацией. Файл: simpson\_method.cpp

\begin{lstlisting}[language=C++]
// Copyright 2021 Vlasov Maksim

#include "../../../modules/task_1/vlasov_m_simpson_method/simpson_method.h"

#include <algorithm>
#include <cassert>
#include <cmath>
#include <numeric>
#include <stdexcept>
#include <utility>

static void sumUp(std::vector<double>* accum, const std::vector<double>& add) {
    assert(accum->size() == add.size());
    for (size_t i = 0; i < accum->size(); i++)
        accum->at(i) += add[i];
}

double SimpsonMethod::integrate(
    const std::function<double(const std::vector<double>&)>& func,
    const std::vector<double>& seg_begin, const std::vector<double>& seg_end,
    int steps_count) {
    if (steps_count <= 0)
        throw std::runtime_error("Steps count must be positive");
    if (seg_begin.empty() || seg_end.empty())
        throw std::runtime_error("No segments");
    if (seg_begin.size() != seg_end.size())
        throw std::runtime_error("Invalid segments");
    size_t dim = seg_begin.size();
    std::vector<double> steps(dim), segments(dim);
    for (size_t i = 0; i < dim; i++) {
        steps[i] = (seg_end[i] - seg_begin[i]) / steps_count;
        segments[i] = seg_end[i] - seg_begin[i];
    }
    std::pair<double, double> sum = std::make_pair(0.0, 0.0);
    std::vector<double> args = seg_begin;
    for (int i = 0; i < steps_count; i += 2) {
        sumUp(&args, steps);
        sum.first += func(args);
        sumUp(&args, steps);
        sum.second += func(args);
    }
    double seg_prod = std::accumulate(segments.begin(), segments.end(), 1.0,
                                      [](double p, double s) { return p * s; });
    return (func(seg_begin) + 4 * sum.first + 2 * sum.second - func(seg_end)) *
           seg_prod / (3.0 * steps_count);
}
\end{lstlisting}

\par 3. Библиотека с последовательной реализацией. Файл: main.cpp

\begin{lstlisting}[language=C++]
// Copyright 2021 Vlasov Maksim

#include <gtest/gtest.h>

#include <cassert>
#include <cmath>
#include <iostream>
#include <vector>

#include "./simpson_method.h"

#define MULTIDIM_FUNC(FNAME, FVARCOUNT, FCOMP)                                 \
    double FNAME(const std::vector<double>& x) {                               \
        assert(x.size() == (FVARCOUNT));                                       \
        return (FCOMP);                                                        \
    }

MULTIDIM_FUNC(generic, 1, 0);
MULTIDIM_FUNC(parabola, 1, -x[0] * x[0] + 4);
MULTIDIM_FUNC(body, 2, x[0] * x[0] + x[1] * x[1]);
MULTIDIM_FUNC(super, 3, std::sin(x[0] + 3) - std::log(x[1]) + x[2] * x[2]);

TEST(Sequential_SimpsonMethodTest, can_integrate_2d_function) {
    std::vector<double> seg_begin = {0};
    std::vector<double> seg_end = {2};
    double square = SimpsonMethod::integrate(parabola, seg_begin, seg_end, 100);
    ASSERT_NEAR(16.0 / 3.0, square, 1e-6);
}

TEST(Sequential_SimpsonMethodTest, can_integrate_3d_function) {
    std::vector<double> seg_begin = {0, 0};
    std::vector<double> seg_end = {1, 1};
    double volume = SimpsonMethod::integrate(body, seg_begin, seg_end, 100);
    ASSERT_NEAR(2.0 / 3.0, volume, 1e-6);
}

// Calculated by WolframAlpha with the following query:
// integrate (sin(x + 3) - ln(y) + z^2), x=[-2, 1], y=[1, 3], z=[0, 2]
TEST(Sequential_SimpsonMethodTest, can_integrate_super_function) {
    std::vector<double> seg_begin = {-2, 1, 0};
    std::vector<double> seg_end = {1, 3, 2};
    double integral = SimpsonMethod::integrate(super, seg_begin, seg_end, 100);
    ASSERT_NEAR(13.0007625, integral, 1e-6);
}

TEST(Sequential_SimpsonMethodTest, cannot_accept_empty_segment_vectors) {
    ASSERT_ANY_THROW(SimpsonMethod::integrate(generic, {}, {}, 100));
    ASSERT_ANY_THROW(SimpsonMethod::integrate(generic, {0}, {}, 100));
    ASSERT_ANY_THROW(SimpsonMethod::integrate(generic, {}, {0}, 100));
}

TEST(Sequential_SimpsonMethodTest, cannot_accept_invalid_segment_vectors) {
    ASSERT_ANY_THROW(SimpsonMethod::integrate(generic, {1, 2}, {1, 2, 3}, 100));
    ASSERT_ANY_THROW(SimpsonMethod::integrate(generic, {1, 2, 3}, {1, 2}, 100));
}

TEST(Sequential_SimpsonMethodTest, cannot_accept_invalid_steps_count) {
    ASSERT_ANY_THROW(SimpsonMethod::integrate(generic, {0}, {0}, 0));
    ASSERT_ANY_THROW(SimpsonMethod::integrate(generic, {0}, {0}, -1));
}

int main(int argc, char** argv) {
    ::testing::InitGoogleTest(&argc, argv);
    return RUN_ALL_TESTS();
}
\end{lstlisting}

\par 4. Библиотека с реализацией на OpenMP. Файл: simpson\_method.h

\begin{lstlisting}[language=C++]
// Copyright 2021 Vlasov Maksim

#ifndef MODULES_TASK_2_VLASOV_M_SIMPSON_METHOD_SIMPSON_METHOD_H_
#define MODULES_TASK_2_VLASOV_M_SIMPSON_METHOD_SIMPSON_METHOD_H_

#include <functional>
#include <vector>

namespace SimpsonMethod {

double sequential(const std::function<double(const std::vector<double>&)>& func,
                  const std::vector<double>& seg_begin,
                  const std::vector<double>& seg_end, int steps_count);

double parallel(const std::function<double(const std::vector<double>&)>& func,
                std::vector<double> seg_begin, std::vector<double> seg_end,
                int steps_count);

}  // namespace SimpsonMethod

#endif  // MODULES_TASK_2_VLASOV_M_SIMPSON_METHOD_SIMPSON_METHOD_H_
\end{lstlisting}

\par 5. Библиотека с реализацией на OpenMP. Файл: simpson\_method.cpp

\begin{lstlisting}[language=C++]
// Copyright 2021 Vlasov Maksim

#include "../../../modules/task_2/vlasov_m_simpson_method/simpson_method.h"

#include <omp.h>

#include <algorithm>
#include <cassert>
#include <cmath>
#include <iostream>
#include <numeric>
#include <stdexcept>
#include <utility>

static void sumUp(std::vector<double>* accum, const std::vector<double>& add) {
    assert(accum->size() == add.size());
    for (size_t i = 0; i < accum->size(); i++)
        accum->at(i) += add[i];
}

double SimpsonMethod::sequential(
    const std::function<double(const std::vector<double>&)>& func,
    const std::vector<double>& seg_begin, const std::vector<double>& seg_end,
    int steps_count) {
    if (steps_count <= 0)
        throw std::runtime_error("Steps count must be positive");
    if (seg_begin.empty() || seg_end.empty())
        throw std::runtime_error("No segments");
    if (seg_begin.size() != seg_end.size())
        throw std::runtime_error("Invalid segments");
    size_t dim = seg_begin.size();
    std::vector<double> steps(dim), segments(dim);
    for (size_t i = 0; i < dim; i++) {
        steps[i] = (seg_end[i] - seg_begin[i]) / steps_count;
        segments[i] = seg_end[i] - seg_begin[i];
    }
    std::pair<double, double> sum = std::make_pair(0.0, 0.0);
    std::vector<double> args = seg_begin;
    for (int i = 0; i < steps_count; i++) {
        sumUp(&args, steps);
        if (i % 2 == 0)
            sum.first += func(args);
        else
            sum.second += func(args);
    }
    double seg_prod = std::accumulate(segments.begin(), segments.end(), 1.0,
                                      [](double p, double s) { return p * s; });
    return (func(seg_begin) + 4 * sum.first + 2 * sum.second - func(seg_end)) *
           seg_prod / (3.0 * steps_count);
}

double SimpsonMethod::parallel(
    const std::function<double(const std::vector<double>&)>& func,
    std::vector<double> seg_begin, std::vector<double> seg_end,
    int steps_count) {
    if (steps_count <= 0)
        throw std::runtime_error("Steps count must be positive");
    if (seg_begin.empty() || seg_end.empty())
        throw std::runtime_error("No segments");
    if (seg_begin.size() != seg_end.size())
        throw std::runtime_error("Invalid segments");
    size_t dim = seg_begin.size();
    std::vector<double> steps(dim), segments(dim);
    for (size_t i = 0; i < dim; i++) {
        steps[i] = (seg_end[i] - seg_begin[i]) / steps_count;
        segments[i] = seg_end[i] - seg_begin[i];
    }
    double sum_first = 0, sum_second = 0;
    std::vector<double> args(dim);
    int t_count = 0, t_steps = 0;
#pragma omp parallel firstprivate(args) reduction(+ : sum_first, sum_second)
    {
        int t_id = omp_get_thread_num();
        t_count = omp_get_num_threads();
        t_steps = steps_count / t_count;
        for (size_t i = 0; i < dim; i++)
            args[i] = seg_begin[i] + steps[i] * t_id * t_steps;
        int t_start = t_id * t_steps;
        int t_end = t_start + t_steps;
        for (int i = t_start; i < t_end; i++) {
            sumUp(&args, steps);
            if (i % 2 == 0)
                sum_first += func(args);
            else
                sum_second += func(args);
        }
    }
    if (steps_count % t_count != 0) {
        int passed_steps_count = t_count * t_steps;
        for (size_t i = 0; i < dim; i++)
            args[i] = seg_begin[i] + steps[i] * passed_steps_count;
        for (int i = passed_steps_count; i < steps_count; i++) {
            sumUp(&args, steps);
            if (i % 2 == 0)
                sum_first += func(args);
            else
                sum_second += func(args);
        }
    }
    double seg_prod = std::accumulate(segments.begin(), segments.end(), 1.0,
                                      [](double p, double s) { return p * s; });
    return (func(seg_begin) + 4 * sum_first + 2 * sum_second - func(seg_end)) *
           seg_prod / (3.0 * steps_count);
}
\end{lstlisting}

\par 6. Библиотека с реализацией на OpenMP. Файл: main.cpp

\begin{lstlisting}[language=C++]
// Copyright 2021 Vlasov Maksim

#include <gtest/gtest.h>
#include <omp.h>

#include <cassert>
#include <cmath>
#include <iostream>
#include <vector>

#include "./simpson_method.h"

#define MULTIDIM_FUNC(FNAME, FVARCOUNT, FCOMP)                                 \
    double FNAME(const std::vector<double>& x) {                               \
        assert(x.size() == (FVARCOUNT));                                       \
        return (FCOMP);                                                        \
    }

MULTIDIM_FUNC(generic, 1, 0);
MULTIDIM_FUNC(parabola, 1, -x[0] * x[0] + 4);
MULTIDIM_FUNC(body, 2, x[0] * x[0] + x[1] * x[1]);
MULTIDIM_FUNC(super, 3, std::sin(x[0] + 3) - std::log(x[1]) + x[2] * x[2]);

// Performance test - for demo purposes, not for CI
/*TEST(Parallel_SimpsonMethodTest, same_result_as_sequential) {
    std::vector<double> seg_begin = {0, 0};
    std::vector<double> seg_end = {1, 1};
    std::pair<double, double> time = {0, 0};
    int num_threads;
    std::cout << "num_threads: ";
    std::cin >> num_threads;
    omp_set_num_threads(num_threads);
    time.first = omp_get_wtime();
    double seq = SimpsonMethod::sequential(body, seg_begin, seg_end, 10000000);
    time.second = omp_get_wtime();
    std::cout << "Sequential " << (time.second - time.first) << ' ' << seq
              << std::endl;
    time.first = omp_get_wtime();
    double par = SimpsonMethod::parallel(body, seg_begin, seg_end, 10000000);
    time.second = omp_get_wtime();
    std::cout << "Parallel " << (time.second - time.first) << ' ' << par
              << std::endl;
    ASSERT_NEAR(seq, par, 1e-6);
}*/

TEST(Parallel_SimpsonMethodTest, can_integrate_2d_function) {
    std::vector<double> seg_begin = {0};
    std::vector<double> seg_end = {2};
    double square = SimpsonMethod::parallel(parabola, seg_begin, seg_end, 100);
    ASSERT_NEAR(16.0 / 3.0, square, 1e-6);
}

TEST(Parallel_SimpsonMethodTest, can_integrate_3d_function) {
    std::vector<double> seg_begin = {0, 0};
    std::vector<double> seg_end = {1, 1};
    double volume = SimpsonMethod::parallel(body, seg_begin, seg_end, 100);
    ASSERT_NEAR(2.0 / 3.0, volume, 1e-6);
}

// Calculated by WolframAlpha with the following query:
// integrate (sin(x + 3) - ln(y) + z^2), x=[-2, 1], y=[1, 3], z=[0, 2]
TEST(Parallel_SimpsonMethodTest, can_integrate_super_function) {
    std::vector<double> seg_begin = {-2, 1, 0};
    std::vector<double> seg_end = {1, 3, 2};
    double integral = SimpsonMethod::parallel(super, seg_begin, seg_end, 100);
    ASSERT_NEAR(13.0007625, integral, 1e-6);
}

TEST(Parallel_SimpsonMethodTest, cannot_accept_empty_segment_vectors) {
    ASSERT_ANY_THROW(SimpsonMethod::parallel(generic, {}, {}, 100));
    ASSERT_ANY_THROW(SimpsonMethod::parallel(generic, {0}, {}, 100));
    ASSERT_ANY_THROW(SimpsonMethod::parallel(generic, {}, {0}, 100));
}

TEST(Parallel_SimpsonMethodTest, cannot_accept_invalid_segment_vectors) {
    ASSERT_ANY_THROW(SimpsonMethod::parallel(generic, {1, 2}, {1, 2, 3}, 100));
    ASSERT_ANY_THROW(SimpsonMethod::parallel(generic, {1, 2, 3}, {1, 2}, 100));
}

TEST(Parallel_SimpsonMethodTest, cannot_accept_invalid_steps_count) {
    ASSERT_ANY_THROW(SimpsonMethod::parallel(generic, {0}, {0}, 0));
    ASSERT_ANY_THROW(SimpsonMethod::parallel(generic, {0}, {0}, -1));
}

int main(int argc, char** argv) {
    ::testing::InitGoogleTest(&argc, argv);
    return RUN_ALL_TESTS();
}
\end{lstlisting}

\par 7. Библиотека с реализацией на oneTBB. Файл: simpson\_method.h

\begin{lstlisting}[language=C++]
// Copyright 2021 Vlasov Maksim

#ifndef MODULES_TASK_3_VLASOV_M_SIMPSON_METHOD_SIMPSON_METHOD_H_
#define MODULES_TASK_3_VLASOV_M_SIMPSON_METHOD_SIMPSON_METHOD_H_

#include <functional>
#include <vector>

namespace SimpsonMethod {

double sequential(const std::function<double(const std::vector<double>&)>& func,
                  const std::vector<double>& seg_begin,
                  const std::vector<double>& seg_end, int steps_count);

double parallel(const std::function<double(const std::vector<double>&)>& func,
                std::vector<double> seg_begin, std::vector<double> seg_end,
                int steps_count);

}  // namespace SimpsonMethod

#endif  // MODULES_TASK_3_VLASOV_M_SIMPSON_METHOD_SIMPSON_METHOD_H_
\end{lstlisting}

\par 8. Библиотека с реализацией на oneTBB. Файл: simpson\_method.cpp

\begin{lstlisting}[language=C++]
// Copyright 2021 Vlasov Maksim

#include "../../../modules/task_3/vlasov_m_simpson_method/simpson_method.h"

#include <tbb/blocked_range.h>
#include <tbb/parallel_reduce.h>

#include <algorithm>
#include <cassert>
#include <cmath>
#include <numeric>
#include <stdexcept>
#include <utility>

static void sumUp(std::vector<double>* accum, const std::vector<double>& add) {
    assert(accum->size() == add.size());
    for (size_t i = 0; i < accum->size(); i++)
        accum->at(i) += add[i];
}

double SimpsonMethod::sequential(
    const std::function<double(const std::vector<double>&)>& func,
    const std::vector<double>& seg_begin, const std::vector<double>& seg_end,
    int steps_count) {
    if (steps_count <= 0)
        throw std::runtime_error("Steps count must be positive");
    if (seg_begin.empty() || seg_end.empty())
        throw std::runtime_error("No segments");
    if (seg_begin.size() != seg_end.size())
        throw std::runtime_error("Invalid segments");
    size_t dim = seg_begin.size();
    std::vector<double> steps(dim), segments(dim);
    for (size_t i = 0; i < dim; i++) {
        steps[i] = (seg_end[i] - seg_begin[i]) / steps_count;
        segments[i] = seg_end[i] - seg_begin[i];
    }
    std::pair<double, double> sum = std::make_pair(0.0, 0.0);
    std::vector<double> args = seg_begin;
    for (int i = 0; i < steps_count; i++) {
        sumUp(&args, steps);
        if (i % 2 == 0)
            sum.first += func(args);
        else
            sum.second += func(args);
    }
    double seg_prod = std::accumulate(segments.begin(), segments.end(), 1.0,
                                      [](double p, double s) { return p * s; });
    return (func(seg_begin) + 4 * sum.first + 2 * sum.second - func(seg_end)) *
           seg_prod / (3.0 * steps_count);
}

double SimpsonMethod::parallel(
    const std::function<double(const std::vector<double>&)>& func,
    std::vector<double> seg_begin, std::vector<double> seg_end,
    int steps_count) {
    if (steps_count <= 0)
        throw std::runtime_error("Steps count must be positive");
    if (seg_begin.empty() || seg_end.empty())
        throw std::runtime_error("No segments");
    if (seg_begin.size() != seg_end.size())
        throw std::runtime_error("Invalid segments");
    size_t dim = seg_begin.size();
    std::vector<double> steps(dim), segments(dim);
    for (size_t i = 0; i < dim; i++) {
        steps[i] = (seg_end[i] - seg_begin[i]) / steps_count;
        segments[i] = seg_end[i] - seg_begin[i];
    }
    std::pair<double, double> sum = std::make_pair(0.0, 0.0);
    sum = tbb::parallel_reduce(
        tbb::blocked_range<int>(0, steps_count), std::make_pair(0.0, 0.0),
        [&func, &steps, &seg_begin, &dim](const tbb::blocked_range<int>& range,
                                          std::pair<double, double> sum) {
            int t_begin = range.begin();
            int t_end = range.end();
            std::vector<double> args(dim);
            for (size_t i = 0; i < dim; i++)
                args[i] = seg_begin[i] + steps[i] * t_begin;
            for (int i = t_begin; i < t_end; i++) {
                sumUp(&args, steps);
                if (i % 2 == 0)
                    sum.first += func(args);
                else
                    sum.second += func(args);
            }
            return sum;
        },
        [](const std::pair<double, double>& lhs,
           const std::pair<double, double>& rhs) {
            return std::make_pair(lhs.first + rhs.first,
                                  lhs.second + rhs.second);
        });
    double seg_prod = std::accumulate(segments.begin(), segments.end(), 1.0,
                                      [](double p, double s) { return p * s; });
    return (func(seg_begin) + 4 * sum.first + 2 * sum.second - func(seg_end)) *
           seg_prod / (3.0 * steps_count);
}
\end{lstlisting}

\par 9. Библиотека с реализацией на oneTBB. Файл: main.cpp

\begin{lstlisting}[language=C++]
// Copyright 2021 Vlasov Maksim

#include <gtest/gtest.h>
#include <tbb/tick_count.h>

#include <cassert>
#include <cmath>
#include <iostream>
#include <vector>

#include "./simpson_method.h"

#define MULTIDIM_FUNC(FNAME, FVARCOUNT, FCOMP)                                 \
    double FNAME(const std::vector<double>& x) {                               \
        assert(x.size() == (FVARCOUNT));                                       \
        return (FCOMP);                                                        \
    }

MULTIDIM_FUNC(generic, 1, 0);
MULTIDIM_FUNC(parabola, 1, -x[0] * x[0] + 4);
MULTIDIM_FUNC(body, 2, x[0] * x[0] + x[1] * x[1]);
MULTIDIM_FUNC(super, 3, std::sin(x[0] + 3) - std::log(x[1]) + x[2] * x[2]);

// Performance test - for demo purposes, not for CI
/*TEST(TBB_SimpsonMethodTest, same_result_as_sequential) {
    std::vector<double> seg_begin = {0, 0};
    std::vector<double> seg_end = {1, 1};
    std::pair<tbb::tick_count, tbb::tick_count> time;
    time.first = tbb::tick_count::now();
    double seq = SimpsonMethod::sequential(body, seg_begin, seg_end, 10000000);
    time.second = tbb::tick_count::now();
    std::cout << "Sequential " << (time.second - time.first).seconds() << ' '
              << seq << std::endl;
    time.first = tbb::tick_count::now();
    double par = SimpsonMethod::parallel(body, seg_begin, seg_end, 10000000);
    time.second = tbb::tick_count::now();
    std::cout << "Parallel " << (time.second - time.first).seconds() << ' '
              << par << std::endl;
    ASSERT_NEAR(seq, par, 1e-6);
}*/

TEST(TBB_SimpsonMethodTest, can_integrate_2d_function) {
    std::vector<double> seg_begin = {0};
    std::vector<double> seg_end = {2};
    double square = SimpsonMethod::parallel(parabola, seg_begin, seg_end, 100);
    ASSERT_NEAR(16.0 / 3.0, square, 1e-6);
}

TEST(TBB_SimpsonMethodTest, can_integrate_3d_function) {
    std::vector<double> seg_begin = {0, 0};
    std::vector<double> seg_end = {1, 1};
    double volume = SimpsonMethod::parallel(body, seg_begin, seg_end, 100);
    ASSERT_NEAR(2.0 / 3.0, volume, 1e-6);
}

// Calculated by WolframAlpha with the following query:
// integrate (sin(x + 3) - ln(y) + z^2), x=[-2, 1], y=[1, 3], z=[0, 2]
TEST(TBB_SimpsonMethodTest, can_integrate_super_function) {
    std::vector<double> seg_begin = {-2, 1, 0};
    std::vector<double> seg_end = {1, 3, 2};
    double integral = SimpsonMethod::parallel(super, seg_begin, seg_end, 100);
    ASSERT_NEAR(13.0007625, integral, 1e-6);
}

TEST(TBB_SimpsonMethodTest, cannot_accept_empty_segment_vectors) {
    ASSERT_ANY_THROW(SimpsonMethod::parallel(generic, {}, {}, 100));
    ASSERT_ANY_THROW(SimpsonMethod::parallel(generic, {0}, {}, 100));
    ASSERT_ANY_THROW(SimpsonMethod::parallel(generic, {}, {0}, 100));
}

TEST(TBB_SimpsonMethodTest, cannot_accept_invalid_segment_vectors) {
    ASSERT_ANY_THROW(SimpsonMethod::parallel(generic, {1, 2}, {1, 2, 3}, 100));
    ASSERT_ANY_THROW(SimpsonMethod::parallel(generic, {1, 2, 3}, {1, 2}, 100));
}

TEST(TBB_SimpsonMethodTest, cannot_accept_invalid_steps_count) {
    ASSERT_ANY_THROW(SimpsonMethod::parallel(generic, {0}, {0}, 0));
    ASSERT_ANY_THROW(SimpsonMethod::parallel(generic, {0}, {0}, -1));
}

int main(int argc, char** argv) {
    ::testing::InitGoogleTest(&argc, argv);
    return RUN_ALL_TESTS();
}
\end{lstlisting}

\end{document}
