\documentclass{report}

\usepackage[T2A]{fontenc}
\usepackage[utf8]{luainputenc}
\usepackage[english, russian]{babel}
\usepackage[pdftex]{hyperref}
\usepackage[14pt]{extsizes}
\usepackage{listings}
\usepackage{color}
\usepackage{geometry}
\usepackage{enumitem}
\usepackage{multirow}
\usepackage{graphicx}
\usepackage{indentfirst}

\geometry{a4paper,top=2cm,bottom=3cm,left=2cm,right=1.5cm}
\setlength{\parskip}{0.5cm}
\setlist{nolistsep, itemsep=0.3cm,parsep=0pt}

\lstset{language=C++,
		basicstyle=\footnotesize,
		keywordstyle=\color{blue}\ttfamily,
		stringstyle=\color{red}\ttfamily,
		commentstyle=\color{green}\ttfamily,
		morecomment=[l][\color{magenta}]{\#}, 
		tabsize=4,
		breaklines=true,
  		breakatwhitespace=true,
  		title=\lstname,       
}

\makeatletter
\renewcommand\@biblabel[1]{#1.\hfil}
\makeatother

\begin{document}

\begin{titlepage}

\begin{center}
Министерство науки и высшего образования Российской Федерации
\end{center}

\begin{center}
Федеральное государственное автономное образовательное учреждение высшего образования \\
Национальный исследовательский Нижегородский государственный университет им. Н.И. Лобачевского
\end{center}

\begin{center}
Институт информационных технологий, математики и механики
\end{center}

\vspace{4em}

\begin{center}
\textbf{\LargeОтчет по лабораторной работе} \\
\end{center}
\begin{center}
\textbf{\Large«Поразрядная сортировка целых чисел с простым слиянием»} \\
\end{center}

\vspace{4em}

\newbox{\lbox}
\savebox{\lbox}{\hbox{text}}
\newlength{\maxl}
\setlength{\maxl}{\wd\lbox}
\hfill\parbox{7cm}{
\hspace*{5cm}\hspace*{-5cm}\textbf{Выполнила:} \\ студентка группы 381808-2 \\ Щелянскова Анастасия\\
\\
\hspace*{5cm}\hspace*{-5cm}\textbf{Проверил:}\\ доцент кафедры МОСТ, \\ кандидат технических наук \\ Сысоев А.В.\\
}
\vspace{\fill}

\begin{center} Нижний Новгород \\ 2021 \end{center}

\end{titlepage}

\setcounter{page}{2}

% Введение
\section*{Введение}
\addcontentsline{toc}{section}{Введение}
Поразрядная сортировка имеет две модификации: Most Significant Digit
(MSD) и Least Significant Digit (LSD). В данной работе будет рассмотрен
алгоритм LSD как наиболее эффективный.Идея поразрядной восходящей сортировки (Least Significant Digit (LSD) Radix Sort) заключается в том, что выполняется последовательная сортировка чисел по разрядам (от младшего разряда к старшему).
\newpage

% Постановка задачи
\section*{Постановка задачи}
\addcontentsline{toc}{section}{Постановка задачи}
В данной работе необходимо реализовать поразрядную сортировку целых чисел с простым слиянием. Данная работа должна содержать последовательную и несколько параллельных версий, с использованием следующих технологий: OpenMP, TBB . 
\par
На основе разработанной программы необходимо провести вычислительные эксперименты, сравнив время работы последовательного и параллельных алгоритмов на разных входных данных, затем сделать вывод об эффективности параллельных версий.
\par
Также следует экспериментально проверить корректность разработанных версий алгоритма с помощью Google тестов.
\newpage

% Описание алгоритма
\section*{Описание алгоритма}
\addcontentsline{toc}{section}{Описание алгоритма}
Сравнение производится поразрядно: сначала сравниваются значения одного крайнего разряда, и элементы группируются по результатам этого сравнения, затем сравниваются значения следующего разряда, соседнего, и элементы либо упорядочиваются по результатам сравнения значений этого разряда внутри образованных на предыдущем проходе групп, либо переупорядочиваются в целом, но сохраняя относительный порядок, достигнутый при предыдущей сортировке. Затем аналогично делается для следующего разряда, и так до конца.
\newpage

% Схема распараллеливания
\section*{Схема распараллеливания}
\addcontentsline{toc}{section}{Схема распараллеливания}
Для распараллеливания алгоритма поразрядной сортировки необходимо разделить массив на равные подмассивы, количество которых равно количеству потоков. Если массив не делится на цело на число потоков, необходимо записать остаток от деления в один из подмассивов (в моем случае в последний). Каждый поток берет соответствующую часть массива и запускает локальную поразрядную сортировку для своего подмассива.
\par Затем отсортированные подмассивы сливаются алгоритмом простого слияния. На каждом шаге мы берём меньший из двух первых элементов подмассивов и записываем его в результирующий массив. Счётчики номеров элементов результирующего массива и подмассива, из которого был взят элемент, увеличиваем на 1. Когда один из подмассивов закончился, мы добавляем все оставшиеся элементы второго подмассива в результирующий массив.
\newpage

% Описание программной реализации
\section*{Описание программной реализации}
\addcontentsline{toc}{section}{Описание программной реализации}
В программе имеются следующие функции:

\begin{itemize}
\item Алгоритм последовательной поразрядной сортировки целых чисел вызывается с помощью функции:
\end{itemize}
\begin{lstlisting}
	std::vector<int> RadixSort(std::vector<int> vect, int size);
\end{lstlisting}
\begin{itemize}
\item Алгоритм слияния отсортированных массивов вызывается с помощью функции:
\end{itemize}
\begin{lstlisting}
	std::vector<int> Merge(const std::vector<int>& vect_left, const std::vector<int>& vect_right);
\end{lstlisting}
\begin{itemize}
\item Функция, реализующая OpenMP версию паралельной поразрядной сортировки целых чисел : 
\end{itemize}
\begin{lstlisting}
	std::vector<int> RadixSortParallel(std::vector<int> vect, int size, int threads);
\end{lstlisting}
\begin{itemize}
\item Функция, реализующая TBB версию паралельной поразрядной сортировки целых чисел : 
\end{itemize}
\begin{lstlisting}
	std::vector<int> RadixSort_tbb(const std::vector<int>& v);
\end{lstlisting}
\newpage

% Подтверждение корректности
\section*{Подтверждение корректности}
\addcontentsline{toc}{section}{Подтверждение корректности}
Для подтверждения корректности в программе представлен набор тестов, разработанных с помощью Google Testing Framework.
\par Набор представляет из себя тесты, которые проверяют корректность вычислений: сравниваются отсортированный с помощью функции стандартной библиотеки вектор и вектор, отсортированный с помощью написанной нами функции.

\par Успешное прохождение всех тестов подтверждает корректность работы написанной программы.
\newpage

% Описание экспериментов
\section*{Описание экспериментов}
\addcontentsline{toc}{section}{Описание экспериментов}
Эксперименты проводились на 4-х ядерном компьютере с операционной системой – Windows 10, который имеет 8 логических процессоров. Результаты экспериментов приведены в таблице:
\begin{table}[!h]
\caption{Результаты экспериментов. Сравнение с OpenMP.}
\centering
\begin{tabular}{lllll}
Размер вектора & Последовательно & Параллельно & Ускорение  \\
1500000        & 0.427         & 0.132     & 3.3       \\
15000000        & 4.2         & 1.4     & 2.9       \\
150000000       & 41.7179         & 12.198     & 3.42       
\end{tabular}
\end{table}

\begin{table}[!h]
\caption{Результаты экспериментов. Сравнение с TBB.}
\centering
\begin{tabular}{lllll}
Размер вектора & Последовательно & Параллельно & Ускорение  \\
1500000        & 	0.4231         & 0.199     & 	2.122       \\
15000000        & 4.235         & 1.5459     & 2.74       \\
150000000       & 41.8891         & 15.5868     & 2.68748       
\end{tabular}
\end{table}
\par Из полученных результатов можно сделать несколько выводов:
\begin{enumerate} 
\item Благодаря параллельному исполнению время работы сортировки сократилось.
\item Сложность и затратность в выполнении такого этапа сортировки, как слияние отсортированных массивов, сильно увеличивает время выполнения сортировки.
\end{enumerate} 
\newpage

% Заключение
\section*{Заключение}
\addcontentsline{toc}{section}{Заключение}
В ходе выполнения работы была успешно реализован алгорим поразрядной сортировки целых чисел с простым слиянием.
\par 
Основной задачей лабораторной работы была реализация эффективной параллельной версии. Эта цель была успешно достигнута, что подтверждается результатами экспериментов, проведенных в ходе работы. 
\par 
Работоспособность программы была проверена с помощью библиотеки для модульного тестирования.
\newpage

% Список литературы
\begin{thebibliography}{1}
\addcontentsline{toc}{section}{Список литературы}
\bibitem{Gergel}
Гергель В. П., Стронгин Р. Г. Основы параллельных вычислений для многопроцессорных вычислительных систем. – 2003.
\bibitem{algolist} Поразрядная сортировка. [Электронный ресурс] // URL: \url {https://http://algolist.ru/sort/radix_sort.php} (дата обращения: 25.03.2021)
\end{thebibliography}
\newpage

% Приложение
\section*{Приложение}
\addcontentsline{toc}{section}{Приложение}
\begin{lstlisting}
							// main.cpp
// Copyright 2021 Schelyanskova Anastasiia

#include <gtest/gtest.h>

#include <vector>

#include "./radix_sort.h"

TEST(getRandomVector, wrong_size) {
 ASSERT_ANY_THROW(getRandomVector(-100)); 
}

TEST(getRandomVector, works) {
  std::vector<int> vect = getRandomVector(100);
  ASSERT_EQ(static_cast<int>(vect.size()), 100);
}

TEST(getRandomVector, no_exceptions) {
 ASSERT_NO_THROW(getRandomVector(100)); 
}

TEST(get_max_power, no_exceptions) {
  std::vector<int> vect = {1, 75, 87, 9, 400, 5};
  int power = 3;
  int max_power = get_max_power(vect);
  ASSERT_EQ(power, max_power);
}

TEST(radix_sort, correct_works) {
  std::vector<int> vect = {1, 7, 8, 9, 4, 5};
  std::vector<int> correctAnswer = {1, 4, 5, 7, 8, 9};
  std::vector<int> myAnswer = RadixSort(vect, 6);
  ASSERT_EQ(correctAnswer, myAnswer);
}

TEST(radix_sort, works_bubble) {
  std::vector<int> vect = {1, 17, 8, 95, 4, 51};
  std::vector<int> bubble = bubble_sort_vector(vect);
  std::vector<int> radix = RadixSort(vect, 6);
  ASSERT_EQ(bubble, radix);
}

int main(int argc, char **argv) {
  ::testing::InitGoogleTest(&argc, argv);
  return RUN_ALL_TESTS();
}

\end{lstlisting}
\begin{lstlisting}
							//radix_sort.cpp
// Copyright 2021 Schelyanskova Anastasiia

#include "../../../modules/task_1/schelyanskova_a_radix_sort/radix_sort.h"

#include <math.h>

#include <algorithm>
#include <ctime>
#include <iostream>
#include <random>
#include <string>
#include <vector>

std::vector<int> getRandomVector(int sz) {
  std::random_device dev;
  std::mt19937 gen(dev());
  std::vector<int> vect(sz);
  for (int i = 0; i < sz; i++) {
    vect[i] = gen() % 100;
  }
  return vect;
}
int get_max_power(std::vector<int> vect) {
  int max_power = 0;
  int tmp = vect[0];
  int size = vect.size();
  for (int i = 0; i < size; i++) {
    if (tmp < vect[i]) {
      tmp = vect[i];
    }
  }
  if (tmp == 0) {
    return 1;
  }
  while (tmp != 0) {
    tmp = tmp / 10;
    max_power++;
  }
  return max_power;
}


bool vector_sort(std::vector<int> vect) {
  int size = vect.size();
  for (int i = 0; i < size - 1; i++) {
    if (vect[i] > vect[i + 1]) {
      return false;
    }
  }
  return true;
}

std::vector<int> RadixSort(std::vector<int> vect, int size) {
  std::vector<std::vector<int>> vect_start(10);
  std::vector<int> vect1 = vect;
  int tmp = 0;
  int power = 1;
  int max_power;
  if (vector_sort(vect)) {
    return vect;
  }
  max_power = get_max_power(vect);
  while (power <= max_power) {
    for (int i = 0; i < size; i++) {
      tmp = (vect1[i] % static_cast<unsigned int>(pow(10, power)) / static_cast<unsigned int>(pow(10, power - 1)));
      vect_start[tmp].push_back(vect1[i]);
    }
    vect1.clear();
    for (int i = 0; i < 10; i++) {
      if (!vect_start[i].empty()) {
        int size = vect_start[i].size();
        for (int j = 0; j < size; j++) {
          vect1.push_back(vect_start[i][j]);
        }
        vect_start[i].clear();
      }
    }
    power++;
  }
  return vect1;
}
std::vector<int> Merge(const std::vector<int>& vect_left,
                       const std::vector<int>& vect_right) {
  std::vector<int> result((vect_left.size() + vect_right.size()));

  auto const leftSize = static_cast<int>(vect_left.size());
  auto const rightSize = static_cast<int>(vect_right.size());

  auto i = 0, j = 0, k = 0;
  for (k = 0; k < leftSize + rightSize - 1; k++) {
    if (vect_left[i] < vect_right[j])
      result[k] = vect_left[i++];
    else
      result[k] = vect_right[j++];
  }

  while (i < leftSize) {
    result[k++] = vect_left[i++];
  }
  while (j < rightSize) {
    result[k++] = vect_right[j++];
  }

  return result;
}

\end{lstlisting}
\begin{lstlisting}
							// radix_sort.h
// Copyright 2021 Schelyanskova Anastasiia

#ifndef MODULES_TASK_1_SCHELYANSKOVA_A_RADIX_SORT_RADIX_SORT_H_
#define MODULES_TASK_1_SCHELYANSKOVA_A_RADIX_SORT_RADIX_SORT_H_

#include <string>
#include <vector>

std::vector<int> getRandomVector(int size);
int get_max_power(std::vector<int> vect);
bool vector_sort(std::vector<int> vect);
std::vector<int> RadixSort(std::vector<int> vect, int size);
std::vector<int> Merge(const std::vector<int>& vect_left,
                       const std::vector<int>& vect_right);

#endif  // MODULES_TASK_1_SCHELYANSKOVA_A_RADIX_SORT_RADIX_SORT_H_

\end{lstlisting}
\begin{lstlisting}
							// radix_sort_omp.cpp
// Copyright 2021 Schelyanskova Anastasiia

#include "../../../modules/task_2/schelyanskova_a_radix_sort_omp/radix_sort_omp.h"

#include <math.h>
#include <omp.h>

#include <algorithm>
#include <ctime>
#include <iostream>
#include <random>
#include <string>
#include <vector>

std::vector<int> getRandomVector(int sz) {
  std::random_device dev;
  std::mt19937 gen(dev());
  std::vector<int> vect(sz);
  for (int i = 0; i < sz; i++) {
    vect[i] = gen() % 100;
  }
  return vect;
}

int get_max_power(std::vector<int> vect) {
  int max_power = 0;
  int tmp = vect[0];
  int size = vect.size();
  for (int i = 0; i < size; i++) {
    if (tmp < vect[i]) {
      tmp = vect[i];
    }
  }
  if (tmp == 0) {
    return 1;
  }
  while (tmp != 0) {
    tmp = tmp / 10;
    max_power++;
  }
  return max_power;
}

bool vector_sort(std::vector<int> vect) {
  int size = vect.size();
  for (int i = 0; i < size - 1; i++) {
    if (vect[i] > vect[i + 1]) {
      return false;
    }
  }
  return true;
}

std::vector<int> RadixSort(std::vector<int> vect, int size) {
  std::vector<std::vector<int>> vect_start(10);
  std::vector<int> vect1 = vect;
  int tmp = 0;
  int power = 1;
  int max_power;
  if (vector_sort(vect)) {
    return vect;
  }
  max_power = get_max_power(vect);
  while (power <= max_power) {
    for (int i = 0; i < size; i++) {
      tmp = (vect1[i] % static_cast<unsigned int>(pow(10, power)) /
             static_cast<unsigned int>(pow(10, power - 1)));
      vect_start[tmp].push_back(vect1[i]);
    }
    vect1.clear();
    for (int i = 0; i < 10; i++) {
      if (!vect_start[i].empty()) {
        int size = vect_start[i].size();
        for (int j = 0; j < size; j++) {
          vect1.push_back(vect_start[i][j]);
        }
        vect_start[i].clear();
      }
    }
    power++;
  }
  return vect1;
}
 std::vector<int> Merge(std::vector<int> &vect_left,
                       std::vector<int> &vect_right) {
  std::vector<int> result((vect_left.size() + vect_right.size()));

  int leftSize = static_cast<int>(vect_left.size());
  int rightSize = static_cast<int>(vect_right.size());

  auto i = 0, j = 0, k = 0;
  for (k = 0; k < leftSize + rightSize - 1; k++) {
    if (i >= vect_left.size() || j >= vect_right.size()) {
      break;
    }
    if (vect_left[i] < vect_right[j])
      result[k] = vect_left[i++];
    else
      result[k] = vect_right[j++];
  }

  while (i < leftSize) {
    result[k++] = vect_left[i++];
  }
  while (j < rightSize) {
    result[k++] = vect_right[j++];
  }

 return result;
 }

std::vector<int> RadixSortParallel(std::vector<int> vect, int size,
                                   int threads) {
  std::vector<std::vector<int>> localSortedVectors;


#pragma omp parallel num_threads(threads)
  {
    int threadID = omp_get_thread_num();
    localSortedVectors.resize(threads);
    int chunkSize = size / threads;
    int firstIndex = chunkSize * threadID;
    int secondIndex = chunkSize * (threadID + 1);
    std::vector<int> local;

    if (threadID != threads - 1) {
      local = std::vector<int>(vect.begin() + firstIndex,
                               vect.begin() + secondIndex);
    } else {
      local = std::vector<int>(vect.begin() + firstIndex, vect.end());
    }

    localSortedVectors[threadID] = RadixSort(local, local.size());
  }

  std::vector<int> result;
   for (int i = 0; i < localSortedVectors.size(); i++) {
    result = Merge(result, localSortedVectors[i]);
   }
  return result;
}

\end{lstlisting}
\begin{lstlisting}
							// radix_sort_omp.h

// Copyright 2021 Schelyanskova Anastasiia
#ifndef MODULES_TASK_2_SCHELYANSKOVA_A_RADIX_SORT_OMP_RADIX_SORT_OMP_H_
#define MODULES_TASK_2_SCHELYANSKOVA_A_RADIX_SORT_OMP_RADIX_SORT_OMP_H_

#include <omp.h>

#include <string>
#include <vector>

std::vector<int> getRandomVector(int size);
int get_max_power(std::vector<int> vect);
bool vector_sort(std::vector<int> vect);
std::vector<int> RadixSort(std::vector<int> vect, int size);
std::vector<int> RadixSortParallel(std::vector<int> vect, int size,
                                   int threads);

 std::vector<int> Merge(const std::vector<int> &vect_left,
                      const std::vector<int> &vect_right);


#endif  // MODULES_TASK_2_SCHELYANSKOVA_A_RADIX_SORT_OMP_RADIX_SORT_OMP_H_

\end{lstlisting}
\begin{lstlisting}
							// bitwise_sort_simple_tbb.cpp

// Copyright 2021 Schelyanskova Anastasiia
#include <tbb/tbb.h>
#include <random>
#include <vector>
#include <algorithm>
#include <iostream>
#include "../../../modules/task_3/schelyanskova_a_radix_sort_tbb/bitwise_sort_simple_tbb.h"


std::vector<int> getRandomVector(int sz) {
    std::random_device dev;
    std::mt19937 gen(dev());
    std::vector<int> vect(sz);
    for (int i = 0; i < sz; i++) {
        vect[i] = gen() % 100;
    }
    return vect;
}

bool vector_sort(std::vector<int> vect) {
    int size = vect.size();
    for (int i = 0; i < size - 1; i++) {
        if (vect[i] > vect[i + 1]) {
            return false;
        }
    }
    return true;
}

int get_max_power(std::vector<int> vect) {
    int max_power = 0;
    int tmp = vect[0];
    int size = vect.size();
    for (int i = 0; i < size; i++) {
        if (tmp < vect[i]) {
            tmp = vect[i];
        }
    }
    if (tmp == 0) {
        return 1;
    }
    while (tmp != 0) {
        tmp = tmp / 10;
        max_power++;
    }
    return max_power;
}

std::vector<int> RadixSort(std::vector<int> vect, int size) {
    std::vector<std::vector<int>> vect_start(10);
    std::vector<int> vect1 = vect;
    int tmp = 0;
    int power = 1;
    int max_power;
    if (vector_sort(vect)) {
        return vect;
    }
    max_power = get_max_power(vect);
    while (power <= max_power) {
        for (int i = 0; i < size; i++) {
            tmp = (vect1[i] % static_cast<unsigned int>(pow(10, power))
                    / static_cast<unsigned int>(pow(10, power - 1)));
            vect_start[tmp].push_back(vect1[i]);
        }
        vect1.clear();
        for (int i = 0; i < 10; i++) {
            if (!vect_start[i].empty()) {
                int size_tmp = vect_start[i].size();
                for (int j = 0; j < size_tmp; j++) {
                    vect1.push_back(vect_start[i][j]);
                }
                vect_start[i].clear();
            }
        }
        power++;
    }
    return vect1;
}

std::vector<int> RadixSort_tbb(const std::vector<int>& v) {
    int num_threads = 4;
    if (static_cast<int>(v.size()) < num_threads) {
        return RadixSort(v, v.size());
    }
    std::vector<int> copy(v);
    if (v.size() == 1) {
        return copy;
    }

    int delta = v.size() / num_threads;

    std::vector<std::vector<int>> split_vec(0);
    std::vector<int> tmp;

    for (int thread_num = 0; thread_num < num_threads; thread_num++) {
        if (thread_num == num_threads - 1) {
            tmp.insert(tmp.end(), copy.begin() + delta * thread_num, copy.end());
        } else {
            tmp.insert(tmp.end(), copy.begin() + delta * thread_num,
                       copy.begin() + delta * (thread_num + 1));
        }
        split_vec.push_back(tmp);
        tmp.clear();
    }

    tbb::task_scheduler_init init(num_threads);
    tbb::parallel_for(tbb::blocked_range<size_t>(0, split_vec.size(), 1),
                      [&split_vec](const tbb::blocked_range<size_t>& range){
                          for (size_t i = range.begin(); i != range.end(); ++i) {
                              split_vec[i] = RadixSort(split_vec[i], split_vec[i].size());
                          }
                      }, tbb::simple_partitioner());

    std::vector<int> merged = split_vec[0];
    for (size_t i = 1; i < split_vec.size(); ++i) {
        merged = Merge(merged, split_vec[i]);
    }

    return merged;
}

std::vector<int> Merge(const std::vector<int>& vect_left,
                       const std::vector<int>& vect_right) {
    std::vector<int> result((vect_left.size() + vect_right.size()));

    int i = 0, j = 0, k = 0;
    while (i < static_cast<int>(vect_left.size()) && j < static_cast<int>(vect_right.size())) {
        if (vect_left[i] < vect_right[j])
            result[k] = vect_left[i++];
        else
            result[k] = vect_right[j++];
        k++;
    }
    while (i < static_cast<int>(vect_left.size())) {
        result[k++] = vect_left[i++];
    }
    while (j < static_cast<int>(vect_right.size())) {
        result[k++] = vect_right[j++];
    }

    return result;
}

\end{lstlisting}
\begin{lstlisting}
							// bitwise_sort_simple_tbb.h

// Copyright 2021 Schelyanskova Anastasiia
#ifndef MODULES_TASK_3_SCHELYANSKOVA_A_RADIX_SORT_TBB_BITWISE_SORT_SIMPLE_TBB_H_
#define MODULES_TASK_3_SCHELYANSKOVA_A_RADIX_SORT_TBB_BITWISE_SORT_SIMPLE_TBB_H_

#include <vector>


std::vector<int> getRandomVector(int size);
std::vector<int> RadixSort(std::vector<int> vect, int size);
std::vector<int> RadixSort_tbb(const std::vector<int>& v);
std::vector<int> Merge(const std::vector<int>& vect_left,
                          const std::vector<int>& vect_right);


#endif  // MODULES_TASK_3_SCHELYANSKOVA_A_RADIX_SORT_TBB_BITWISE_SORT_SIMPLE_TBB_H_
\end{lstlisting}
\end{document}
