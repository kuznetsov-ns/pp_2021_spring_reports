\documentclass{report}

\usepackage[T2A]{fontenc}
\usepackage[utf8]{luainputenc}
\usepackage[english, russian]{babel}
\usepackage[pdftex]{hyperref}
\usepackage[14pt]{extsizes}
\usepackage{listings}
\usepackage{color}
\usepackage{geometry}
\usepackage{enumitem}
\usepackage{multirow}
\usepackage{graphicx}
\usepackage{indentfirst}

\geometry{a4paper,top=2cm,bottom=3cm,left=2cm,right=1.5cm}
\setlength{\parskip}{0.5cm}
\setlist{nolistsep, itemsep=0.3cm,parsep=0pt}

\lstset{language=C++,
		basicstyle=\footnotesize,
		keywordstyle=\color{blue}\ttfamily,
		stringstyle=\color{red}\ttfamily,
		commentstyle=\color{green}\ttfamily,
		morecomment=[l][\color{magenta}]{\#}, 
		tabsize=4,
		breaklines=true,
  		breakatwhitespace=true,
  		title=\lstname,       
}

\makeatletter
\renewcommand\@biblabel[1]{#1.\hfil}
\makeatother

\begin{document}

\begin{titlepage}

\begin{center}
Министерство науки и высшего образования Российской Федерации
\end{center}

\begin{center}
Федеральное государственное автономное образовательное учреждение высшего образования \\
Национальный исследовательский Нижегородский государственный университет им. Н.И. Лобачевского
\end{center}

\begin{center}
Институт информационных технологий, математики и механики
\end{center}

\vspace{4em}

\begin{center}
\textbf{\LargeОтчет по лабораторной работе} \\
\end{center}
\begin{center}
\textbf{\Large«Поразрядная сортировка для вещественных чисел (тип double) с четно-нечетным слиянием Бэтчера.»} \\
\end{center}

\vspace{4em}

\newbox{\lbox}
\savebox{\lbox}{\hbox{text}}
\newlength{\maxl}
\setlength{\maxl}{\wd\lbox}
\hfill\parbox{7cm}{
\hspace*{5cm}\hspace*{-5cm}\textbf{Выполнил:} \\ студент группы 381806-1 \\ Белик Ю. А.\\
\\
\hspace*{5cm}\hspace*{-5cm}\textbf{Проверил:}\\ доцент кафедры МОСТ, \\ кандидат технических наук \\ Сысоев А. В.\\
}
\vspace{\fill}

\begin{center} Нижний Новгород \\ 2021 \end{center}

\end{titlepage}

\setcounter{page}{2}

% Содержание
\tableofcontents
\newpage

% Введение
\section*{Введение}
\addcontentsline{toc}{section}{Введение}
\addcontentsline{toc}{section}{Введение}
В настоящее время наука и технологии шагают вперед семимильными шагами. Чтобы идти в ногу с прогрессом, приходится постоянно совершенствовать свои знания, умения и навыки, а на это требуется достаточно много времени. В целях освобождения времени для таких фундаментальных задач нужно минимализировать его затраты на небольшие задачи, которые можно решить один раз вместо того, чтобы каждый раз делать все заново.
\par Именно с этой целью пишутся программы, совершающие элементарные операции по обработке данных. Примером служит данная работа, в которой реализован один из методов сортировок данных.
\newpage

% Постановка задачи
\section*{Постановка задачи}
\addcontentsline{toc}{section}{Постановка задачи}
Необходимо реализовать последовательную и параллельную реализации поразрядной сортировки вещественных чисел с четно-нечетным слиянием Бэтчера, провести тесты для оценки времени и подтверждения корректности алгоритма. Сделать выводы по полученным результатам.
\newpage

% Описание алгоритма
\section*{Описание алгоритма}
\addcontentsline{toc}{section}{Описание алгоритма}
При поразрядной сортировке на вход приходит массив вещественных чисел.
\par В новом цикле по всем байтам:
\begin{itemize}
\item Выбирается позиция байта, начиная с младшего
\item Запоминается значение для каждого элемента массива по выбранной позиции байта
\item Запоминается количество таких значений во всем массиве
\item Новый массив формируется из элементов старого, выставляя элементы по позициям, указанных в массиве.
\item Старый вектор приранивается к новому для дальнейшей сортировки
\end{itemize}
\par При прохождении по старшему байту учитывается знак числа.
\par Таким образом, пройдя по всем разрядам, на выходе мы получим массив из упорядоченных элементов исходного вектора.
\par В алгоритме слияния Бэтчера исходный вектор разбивается на куски, которые локально упорядочиваются. Затем происходит последовательное слияние этих фрагментов, сначала попарно четных с нечетными, затем попарно четные кратные четырем с четными не кратными четырем, после попарно сливаются кратные восьми с фрагментами, индекс которых кратен четырем, но не кратен восьми, и так далее до образования единого целого упорядоченного вектора.
\newpage

% Схема распараллеливания
\section*{Схема распараллеливания}
\addcontentsline{toc}{section}{Схема распараллеливания}
Первым шагом алгоритма является разбиение вектора с дальнейшей локальной поразрядной сортировкой и перетасовкой элементов (элементы, стоящие на четных позициях перемещаются в первую половину вектора, а на нечетных во вторую) на каждом.
\par Далее в цикле по числу слияний, которое равно округленному в большую сторону логарифму общего числа потоков по основанию 2, происходят последовательные слияния. Сначала сливаются локальные вектора четных (верхний) и нечетных (нижний) потоков, потом потоков, у которых остаток от деления ранга на 4 равен 0 (верхний) и 2 (нижний), затем потоков с рангами, при делении на 8 дают остатки 0 (верхний) и 4 (нижний), и так далее пока не закончатся потоки (где верхний поток будет содержать итоговый вектор каждого слияния).
\newpage

% Описание программной реализации
\section*{Описание программной реализации}
\addcontentsline{toc}{section}{Описание программной реализации}
Алгоритм поразрядной сортировки реализован в функции
\begin{lstlisting}
	void RadixSort(double* vec, size_t len, double* vec2);
\end{lstlisting}
\par Входным параметром явлется массив, элементы которого надо упорядочить, число элементов этого массива и итоговый массив, который нужно заполнить.
\par Алгоритм слияния реализован в функции
\begin{lstlisting}
	std::vector<double> MergeBatcherPar(std::vector<double> vec, int size, int thr);
\end{lstlisting}
\par Входным параметром явлется вектор, элементы которого надо упорядочить, количество элементов этого вектора и число потоков. На выходе получаем вектор из упорядоченных элементов.
\par Внутри функции вектор делится по числу потоков, число элементов, которое будет сортировать каждый поток, сохраняется в массив lens, номер элемента, с которого будет начинаться работа каждого потока, сохраняется в массив offsets. 
\par Для работы механизма слияния реализованы некоторые дополнительные функции.
\par Функция перетасовки элеметов
\begin{lstlisting}
	void Shuffle(double* vec, size_t len, double* vec2);
\end{lstlisting}
\par Входным параметром явлется массив, элементы которого надо перетасовать, число элементов массива и итоговый массив для заполнения. На выходе итоговый массив содержит элементы исходного массива, но элеметы, стоящие в исходном векторе на нечетных позициях, перенесены во вторую половину, а элементы, стоящие на четных, перенесены в первую половину.
\par Алгоритм частичного слияния реализован в функции
\begin{lstlisting}
	void PMerge(double* start1, double* start2, double* start3, size_t len1, size_t len2);
\end{lstlisting}
\par Входным параметром являются указатели на 2 массива, которые должны слиться, число элементов в них и указатель на итоговый массив. На выходе получаем итоговый массив из упорядоченных элементов.
\par Заметим, что принцип распараллеливания везде один и тот же, меняются лишь инструменты параллелизации. Рассмотрим их далее для каждой из 3 технологий:OpenMP, TBB и std::threads.
\newpage

% OpenMP
\section*{OpenMP}
\addcontentsline{toc}{section}{OpenMP}
Определения именованных констант, прототипов функций и типов данных OpenMP содержится в файле "omp.h".
\par Для параллелизации циклов сортировки, перетасовки и слияния в алгориме используется директива "pragma omp parallel for".
\par Псевдокод параллельного фрагмента алгоритма с использованием OpenMP:
\begin{lstlisting}
	#pragma omp parallel for num_threads(thr)
	for (int i = 0; i < thr; i++) {
		RadixSort(vectmp + offsets[i], lens[i], reztmp + offsets[i]);
		Shuffle(reztmp + offsets[i], lens[i], vectmp + offsets[i]);
	}
	for (int h = 0; h < nummerge; h++) {
		#pragma omp parallel for num_threads(thr)
		for (int i = 0; i < thr; i++) {
			if ((i % mergecount == 0) && (i + offset < thr))
			PMerge(vectmp + offsets[i], vectmp + offsets[i + offset], reztmp + offsets[i], len1, len2);
			if ((i - offset >= 0) && ((i - offset) % mergecount == 0)) 
			PMerge(vectmp + start1, vectmp + start2, reztmp + start3, lens[i - offset] / 2, lens[i] / 2);
			
		}
		#pragma omp parallel for num_threads(thr)
		for (int i = 0; i < thr; i++)
		if ((i % mergecount == 0) && (i + offset < thr)) ;
		PMerge(reztmp + offsets[i], reztmp + offsets[i] + evencount, vectmp + offsets[i], evencount, oddcount);
	}
\end{lstlisting}
\newpage

% TBB
\section*{TBB}
\addcontentsline{toc}{section}{TBB}
Определения именованных констант, прототипов функций и типов данных TBB содержится в файле "tbb.h".
\par Для параллелизации циклов сортировки, перетасовки и слияния в алгориме используется функция "tbb::parallel for". В качестве параметров эта функция принимает:
\begin{enumerate}
	\item Одномерное итерационное пространство tbb::blocked range - рекурсивно делимый полуинтервал [0; N), где N - число потоков. Значение параметра размера порции вычислений grainsize устанавливается явно и равно 1, поскольку разбиение массива на куски происходит раньше.
	\item Лямбда, реализующая вычисления тела цикла. Она совершает сортировку, перетасовку и слияние кусков массива.
\end{enumerate}
\par Псевдокод параллельного фрагмента алгоритма с использованием TBB:
\begin{lstlisting}
	tbb_radixsort tbbrs(vectmp, reztmp, offsets, lens);
	//void operator() (const tbb::blocked_range<int> &range) const {
		//	RadixSort(in + offsets[range.begin()], lens[range.begin()], out + offsets[range.begin()]);
		//}
	tbb::parallel_for(tbb::blocked_range<int>(0, thr, 1), tbbrs);
	tbb_shuffle tbbsh(reztmp, vectmp, offsets, lens);
	//void operator() (const tbb::blocked_range<int> &range) const {
		//	if (lens[range.begin()] != 0) {
			//		Shuffle(in + offsets[range.begin()], lens[range.begin()], out + offsets[range.begin()]);
			//	}
		//}
	tbb::parallel_for(tbb::blocked_range<int>(0, thr, 1), tbbsh);
	for (int h = 0; h < nummerge; h++) {
		tbb_pmerge tbbpm(vectmp, reztmp, offsets, lens, mergecount, offset, thr);
		//	void operator() (const tbb::blocked_range<int> &range) const {
			//		if ((range.begin() % mergecount == 0) && (range.begin() + offset < thr))
			//			PMerge(in + offsets[range.begin()], in + offsets[range.begin() + offset], out + offsets[range.begin()], len1, len2);
			//	if ((range.begin() - offset >= 0) && ((range.begin() - offset) % mergecount == 0)) 
			//		PMerge(in + start1, in + start2, out + start3, lens[range.begin() - offset] / 2, lens[range.begin()] / 2);
			//}	
		tbb::parallel_for(tbb::blocked_range<int>(0, thr, 1), tbbpm);
		tbb_ppmerge tbbppm(reztmp, vectmp, offsets, lens, mergecount, offset, thr);
		//void operator() (const tbb::blocked_range<int> &range) const {
			//	if ((range.begin() % mergecount == 0) && (range.begin() + offset < thr)) 
			//		PMerge(in + offsets[range.begin()], in + offsets[range.begin()] + evencount, out + offsets[range.begin()], evencount, oddcount);
			//}
		tbb::parallel_for(tbb::blocked_range<int>(0, thr, 1), tbbppm);
	}
\end{lstlisting}
\newpage

% std::threads
\section*{std::threads}
\addcontentsline{toc}{section}{std::threads}
Праллелизация программы с использованием API std::thread осуществляется путем создания объектов std::thread. Под каждый поток нужен свой объект класса. В данной работе реализованы вектора объектов std::thread, их размеры равны числу потоков.
\par Для паралллелизации циклов сортировки, перетасовки и слияния, нужно в цикле по числу потоков, проинициализировать каждый поток ламбдой. Сразу после инициализации поток переходит в состояние исполнения. 
\par В лямбду передается номер потока, на котором она будет исполняться и все необходимые данные для работы.
\par Псевдокод параллельного фрагмента алгоритма с использованием std::threads:
\begin{lstlisting}
	for (int i = 0; i < thr; i++) {
		threadssort.push_back(std::thread([i, &vectmp, &offsets, &lens, &reztmp]() {
			RadixSort(vectmp + offsets[i], lens[i], reztmp + offsets[i]);
			Shuffle(reztmp + offsets[i], lens[i], vectmp + offsets[i]);
		}));
	}
	for (auto& t : threadssort)
	t.join();
	for (int h = 0; h < nummerge; h++) {
		for (int i = 0; i < thr; i++) {
			threadsmp.push_back(std::thread([i, &vectmp, &offsets, &lens, &reztmp, mergecount, offset, thr]() {
				if ((i % mergecount == 0) && (i + offset < thr)) 
				PMerge(vectmp + offsets[i], vectmp + offsets[i + offset], reztmp + offsets[i], len1, len2);
				if ((i - offset >= 0) && ((i - offset) % mergecount == 0)) 
				PMerge(vectmp + start1, vectmp + start2, reztmp + start3, lens[i - offset] / 2, lens[i] / 2);
			}));
		}
		for (auto& t : threadsmp)
		t.join();
		for (int i = 0; i < thr; i++) {
			threadsm.push_back(std::thread([i, &vectmp, &offsets, &lens, &reztmp, mergecount, offset, nummerge, thr, h]() {
				if ((i % mergecount == 0) && (i + offset < thr)) 
				PMerge(reztmp + offsets[i], reztmp + offsets[i] + evencount, vectmp + offsets[i], evencount, oddcount);
			}));
		}
		for (auto& t : threadsm)
		t.join();
	}
\end{lstlisting}
\newpage

% Описание экспериментов
\section*{Описание экспериментов}
\addcontentsline{toc}{section}{Описание экспериментов}
Для подтверждения корректности в программе представлен набор тестов, разработанных с помощью использования Google C++ Testing Framework.
\par Набор представляет из себя тесты, которые проверяют работу функций на разных количествах элементов вектора.
\par Успешное прохождение всех тестов доказывает корректность работы программного комплекса.
\newpage

% Результаты экспериментов
\section*{Результаты экспериментов}
\addcontentsline{toc}{section}{Результаты экспериментов}
Для проведения тестов производилась сортировка векторов размером 50'000'000 значений каждый с разбиением на 4 части. Эксперимент проведен 3 раза, за время работы каждого алгоритма принято среднее время. 
\par Результаты тестов представлены в Таблице 1.
\begin{table}[!h]
\caption{Результаты экспериментов}
\centering
\begin{tabular}{lllll}
		Механизм        & Время работы    & Разница по времени & Ускорение  \\
		Последовательно & 6.077           & -                  & -          \\
		OpenMP          & 2.972           & 3.105              & 2.045      \\
		TBB             & 2.952           & 3.125              & 2.059      \\
		std::thread     & 2.946           & 3.131              & 2.063      
\end{tabular}
\end{table}
\par По данным, полученным в результате тестов, можно сделать вывод о том, что параллельная функция работает действительно быстрее, чем последовательная.
\par Можно заметить, что все параллельные реализации работают примерно с одинаковым ускорением.
\newpage
% Заключение
\section*{Заключение}
\addcontentsline{toc}{section}{Заключение}
Все поставленные задачи выполнены. Реализованы параллельная и последовательная функции выполняющие поразрядную сортировку вещественных чисел с четно-нечетным слиянием Бэтчера. На тестах продемонстрирована эффективность работы параллельной функции. Также реализованы и пройдены тесты, проверящие корректность рабыты функций.
\newpage

% Список литературы
\begin{thebibliography}{1}
	\addcontentsline{toc}{section}{Список литературы}
	\bibitem{1} Вирт Н. - "Алгоритмы+структуры данных=программы"
	\bibitem{2} Кнут Д. - "Искусство программирования. Том 3. Сортировка и поиск"
	\bibitem{3} Дябилкин Д.А. - "Сортирующие сети"
\end{thebibliography}
\newpage

% Приложение
\section*{Приложение}
\addcontentsline{toc}{section}{Приложение}
В данном разделе находится код, написанный в рамках лабораторной работы.
\begin{lstlisting}
	// RadixSortB.h
	
	// Copyright 2021 Belik Julia
	#ifndef MODULES_TASK_1_BELIK_J_RADIX_SORT_RADIXSORTB_H_
	#define MODULES_TASK_1_BELIK_J_RADIX_SORT_RADIXSORTB_H_
	
	#include <vector>
	
	void RadixSort(double* vec, size_t len, double* vec2);
	std::vector<double> Vector(size_t n, double a, double b);
	std::vector<double> MergeBatcherOMP(std::vector<double> vec, size_t size, int thr);
	std::vector<double> MergeBatcherTBB(std::vector<double> vec, size_t size, int thr);
	std::vector<double> MergeBatcherSTD(std::vector<double> vec, size_t size, int thr);
	void Shuffle(double* vec, size_t len, double* vec2);
	void PMerge(double* start1, double* start2, double* start3, size_t len1, size_t len2);
	
	class tbb_radixsort {
		private:
		double *in;
		double *out;
		std::vector<size_t> offsets;
		std::vector<size_t> lens;
		public:
		tbb_radixsort(double *in_, double *out_, const std::vector<size_t>& offsets_,
		const std::vector<size_t>& lens_) : in(in_), out(out_), offsets(offsets_), lens(lens_) {}
		void operator() (const tbb::blocked_range<int> &range) const {
			RadixSort(in + offsets[range.begin()], lens[range.begin()], out + offsets[range.begin()]);
		}
	};
	
	class tbb_pmerge {
		private:
		double *in;
		double *out;
		std::vector<size_t> offsets;
		std::vector<size_t> lens;
		int mergecount;
		int offset;
		int thr;
		
		public:
		tbb_pmerge(double *in_, double *out_, const std::vector<size_t>& offsets_, const std::vector<size_t>& lens_,
		int mergecount_, int offset_, int thr_) : in(in_), out(out_), offsets(offsets_), lens(lens_),
		mergecount(mergecount_), offset(offset_), thr(thr_) {}
		void operator() (const tbb::blocked_range<int> &range) const {
			if ((range.begin() % mergecount == 0) && (range.begin() + offset < thr)) {
			size_t len1 = lens[range.begin()] / 2 + lens[range.begin()] % 2;
			size_t len2 = lens[range.begin() + offset] / 2 + lens[range.begin() + offset] % 2;
			PMerge(in + offsets[range.begin()], in + offsets[range.begin() + offset],
			out + offsets[range.begin()], len1, len2);
		}
		if ((range.begin() - offset >= 0) && ((range.begin() - offset) % mergecount == 0)) {
		size_t start3 = offsets[range.begin() - offset] + lens[range.begin() - offset] / 2
		+ lens[range.begin() - offset] % 2 + lens[range.begin()] / 2 + lens[range.begin()] % 2;
		size_t start1 = offsets[range.begin() - offset] + lens[range.begin() - offset] / 2
		+ lens[range.begin() - offset] % 2;
		size_t start2 = offsets[range.begin()] + lens[range.begin()] / 2 + lens[range.begin()] % 2;
		PMerge(in + start1, in + start2, out + start3, lens[range.begin() - offset] / 2, lens[range.begin()] / 2);
	}
}
};

class tbb_ppmerge {
private:
double *in;
double *out;
std::vector<size_t> offsets;
std::vector<size_t> lens;
int mergecount;
int offset;
int thr;
public:
tbb_ppmerge(double *in_, double *out_, const std::vector<size_t>& offsets_, const std::vector<size_t>& lens_,
int mergecount_, int offset_, int thr_) : in(in_), out(out_), offsets(offsets_), lens(lens_),
mergecount(mergecount_), offset(offset_), thr(thr_) {}
void operator() (const tbb::blocked_range<int> &range) const {
	if ((range.begin() % mergecount == 0) && (range.begin() + offset < thr)) {
	size_t evencount = lens[range.begin()] / 2 + lens[range.begin() + offset] / 2 + lens[range.begin()] % 2
	+ lens[range.begin() + offset] % 2;
	size_t oddcount = lens[range.begin()] / 2 + lens[range.begin() + offset] / 2;
	PMerge(in + offsets[range.begin()], in + offsets[range.begin()] + evencount, out + offsets[range.begin()],
	evencount, oddcount);
}
}
};

class tbb_shuffle {
private:
double *in;
double *out;
std::vector<size_t> offsets;
std::vector<size_t> lens;
public:
tbb_shuffle(double *in_, double *out_, const std::vector<size_t>& offsets_, const std::vector<size_t>& lens_)
: in(in_), out(out_), offsets(offsets_), lens(lens_) {}
void operator() (const tbb::blocked_range<int> &range) const {
if (lens[range.begin()] != 0) {
	Shuffle(in + offsets[range.begin()], lens[range.begin()], out + offsets[range.begin()]);
}
}
};

class tbb_sshuffle {
private:
double *in;
double *out;
std::vector<size_t> offsets;
std::vector<size_t> lens;
int mergecount;
int offset;
int thr;
public:
tbb_sshuffle(double *in_, double *out_, const std::vector<size_t>& offsets_, const std::vector<size_t>& lens_,
int mergecount_, int offset_, int thr_) : in(in_), out(out_), offsets(offsets_), lens(lens_),
mergecount(mergecount_), offset(offset_), thr(thr_) {}
void operator() (const tbb::blocked_range<int> &range) const {
if ((range.begin() % mergecount == 0) && (range.begin() + offset < thr)) {
Shuffle(in + offsets[range.begin()], lens[range.begin()], out + offsets[range.begin()]);
}
}
};

#endif  // MODULES_TASK_1_BELIK_J_RADIX_SORT_RADIXSORTB_H_
\end{lstlisting}
\begin{lstlisting}
// RadixSortB.cpp
// Copyright 2021 Belik Julia
#include <tbb/tbb.h>
#include <omp.h>
#include <vector>
#include <string>
#include <random>
#include <ctime>
#include <algorithm>
#include <iterator>
#include <utility>
#include <iostream>
#include "../../../modules/task_1/belik_j_radix_sort/RadixSortB.h"
#include "../../../3rdparty/unapproved/unapproved.h"
void RadixSort(double* vec, size_t len, double* vec2) {
double* vec1 = new double[len];
for (size_t i = 0; i < len; i++)
vec1[i] = vec[i];
for (size_t i = 0; i < 7; i++) {
unsigned char* bvec = (unsigned char*)vec1;
int countb[256] = { 0 };
int offset[256] = { 0 };
for (size_t j = 0; j < len; j++)
countb[bvec[8 * j + i]]++;
offset[0] = 0;
for (int j = 1; j < 256; j++)
offset[j] = offset[j - 1] + countb[j - 1];
for (size_t j = 0; j < len; j++)
vec2[offset[bvec[8 * j + i]]++] = vec1[j];
for (size_t i = 0; i < len; i++)
vec1[i] = vec2[i];
}
unsigned char* bvec = (unsigned char*)vec1;
int countb[256] = { 0 };
int offset[256] = { 0 };
for (size_t i = 0; i < len; i++)
countb[bvec[8 * i + 7]]++;
offset[255] = offset[255] - 1;
for (int i = 254; i >= 128; i--)
offset[i] = offset[i + 1] + countb[i];
offset[0] = offset[128] + 1;
for (int i = 1; i < 128; i++)
offset[i] = offset[i - 1] + countb[i - 1];
for (size_t i = 0; i < len; i++) {
if (bvec[8 * i + 7] >= 128) {
vec2[offset[bvec[8 * i + 7]]--] = vec1[i];
} else {
vec2[offset[bvec[8 * i + 7]]++] = vec1[i];
}
}
delete[] vec1;
}
std::vector<double> Vector(size_t n, double a, double b) {
std::mt19937 gen;
gen.seed(static_cast<unsigned int>(time(0)));
std::uniform_real_distribution<double> rand(a, b);
std::vector<double> vec(n);
for (size_t i = 0; i < n; i++)
vec[i] = rand(gen);
return vec;
}

std::vector<double> MergeBatcherOMP(std::vector<double> vec, int size, int thr) {
const int len = size / thr;
size_t* lens = new size_t[thr];
double* vectmp = new double[size];
#pragma omp parallel for num_threads(thr)
for (int j = 0; j < size; j++)
vectmp[j] = vec[j];
#pragma omp parallel for num_threads(thr)
for (int j = 0; j < thr; j++)
lens[j] = len;
lens[thr - 1] += size % thr;
size_t* offsets = new size_t[thr];
offsets[0] = 0;
for (int j = 1; j < thr; j++)
offsets[j] = offsets[j - 1] + lens[j - 1];
std::vector<double> rez(size);
double* reztmp = new double[size];
#pragma omp parallel for num_threads(thr)
for (int i = 0; i < thr; i++) {
RadixSort(vectmp + offsets[i], lens[i], reztmp + offsets[i]);
if (thr > 1) {
Shuffle(reztmp + offsets[i], lens[i], vectmp + offsets[i]);
} else {
for (size_t j = 0; j < lens[i]; j++)
vectmp[offsets[i] + j] = reztmp[offsets[i] + j];
}
}
int nt = thr;
int mergecount = 2, offset = 1;
int nummerge = 1, pow = 2;
while (pow < thr) {
pow *= 2;
nummerge++;
}
for (int h = 0; h < nummerge; h++) {
#pragma omp parallel for num_threads(thr)
for (int i = 0; i < thr; i++) {
if ((i % mergecount == 0) && (i + offset < thr)) {
size_t len1 = lens[i] / 2 + lens[i] % 2;
size_t len2 = lens[i + offset] / 2 + lens[i + offset] % 2;
PMerge(vectmp + offsets[i], vectmp + offsets[i + offset], reztmp + offsets[i], len1, len2);
}
if ((i - offset >= 0) && ((i - offset) % mergecount == 0)) {
size_t start1 = offsets[i - offset] + lens[i - offset] / 2 + lens[i - offset] % 2;
size_t start2 = offsets[i] + lens[i] / 2 + lens[i] % 2;
size_t start3 = start1 + lens[i] / 2 + lens[i] % 2;
PMerge(vectmp + start1, vectmp + start2, reztmp + start3, lens[i - offset] / 2, lens[i] / 2);
}
}
#pragma omp parallel for num_threads(thr)
for (int i = 0; i < thr; i++) {
if ((i % mergecount == 0) && (i + offset < thr)) {
size_t evencount = lens[i] / 2 + lens[i + offset] / 2 + lens[i] % 2 + lens[i + offset] % 2;
size_t oddcount = lens[i] / 2 + lens[i + offset] / 2;
PMerge(reztmp + offsets[i], reztmp + offsets[i] + evencount, vectmp + offsets[i], evencount, oddcount);
if (h != nummerge - 1) {
Shuffle(vectmp + offsets[i], lens[i] + lens[i + offset], reztmp + offsets[i]);
}
lens[i] += lens[i + offset];
lens[i + offset] = 0;
}
}
if (h != nummerge - 1) {
#pragma omp parallel for num_threads(thr)
for (int j = 0; j < size; j++)
vectmp[j] = reztmp[j];
}
nt /= 2;
mergecount *= 2;
offset *= 2;
for (int j = 1; j < thr; j++)
offsets[j] = offsets[j - 1] + lens[j - 1];
}		
delete[] lens;
delete[] offsets;
#pragma omp parallel for num_threads(thr)
for (int j = 0; j < size; j++)
vec[j] = vectmp[j];
delete[] vectmp;
delete[] reztmp;
return vec;
}

std::vector<double> MergeBatcherTBB(std::vector<double> vec, size_t size, int thr) {
tbb::task_scheduler_init init(thr);
const size_t len = size / thr;
std::vector<size_t> lens(thr);
double* vectmp = new double[size];
for (size_t j = 0; j < size; j++)
vectmp[j] = vec[j];
for (int j = 0; j < thr; j++)
lens[j] = len;
lens[thr - 1] += size % thr;
std::vector<size_t> offsets(thr);
offsets[0] = 0;
for (int j = 1; j < thr; j++)
offsets[j] = offsets[j - 1] + lens[j - 1];
std::vector<double> rez(size);
double* reztmp = new double[size];
if (thr == 1) {
RadixSort(vectmp, size, reztmp);
for (size_t j = 0; j < size; j++)
rez[j] = reztmp[j];
return rez;
}
tbb_radixsort tbbrs(vectmp, reztmp, offsets, lens);
tbb::parallel_for(tbb::blocked_range<int>(0, thr, 1), tbbrs);
tbb_shuffle tbbsh(reztmp, vectmp, offsets, lens);
tbb::parallel_for(tbb::blocked_range<int>(0, thr, 1), tbbsh);
int nt = thr;
int mergecount = 2, offset = 1;
int nummerge = 1, pow = 2;
while (pow < thr) {
pow *= 2;
nummerge++;
}
for (int h = 0; h < nummerge; h++) {
tbb_pmerge tbbpm(vectmp, reztmp, offsets, lens, mergecount, offset, thr);
tbb::parallel_for(tbb::blocked_range<int>(0, thr, 1), tbbpm);
tbb_ppmerge tbbppm(reztmp, vectmp, offsets, lens, mergecount, offset, thr);
tbb::parallel_for(tbb::blocked_range<int>(0, thr, 1), tbbppm);
for (int j = 0; j < thr; j++)
if ((j % mergecount == 0) && (j + offset < thr)) {
lens[j] += lens[j + offset];
lens[j + offset] = 0;
}
for (int j = 1; j < thr; j++)
offsets[j] = offsets[j - 1] + lens[j - 1];
if (h != nummerge - 1) {
tbb_sshuffle tbbssh(vectmp, reztmp, offsets, lens, mergecount, offset, thr);
tbb::parallel_for(tbb::blocked_range<int>(0, thr, 1), tbbssh);
for (size_t j = 0; j < size; j++)
vectmp[j] = reztmp[j];
}
nt /= 2;
mergecount *= 2;
offset *= 2;
}
for (size_t j = 0; j < size; j++)
rez[j] = vectmp[j];
delete[] vectmp;
delete[] reztmp;
return rez;
}

std::vector<double> MergeBatcherSTD(std::vector<double> vec, size_t size, int thr) {
const size_t len = size / thr;
size_t* lens = new size_t[thr];
double* vectmp = new double[size];
for (size_t j = 0; j < size; j++)
vectmp[j] = vec[j];
for (int j = 0; j < thr; j++)
lens[j] = len;
lens[thr - 1] += size % thr;
size_t* offsets = new size_t[thr];
offsets[0] = 0;
for (int j = 1; j < thr; j++)
offsets[j] = offsets[j - 1] + lens[j - 1];
std::vector<double> rez(size);
double* reztmp = new double[size];
if (thr == 1) {
RadixSort(vectmp, size, reztmp);
for (size_t j = 0; j < size; j++)
rez[j] = reztmp[j];
delete[] lens;
delete[] offsets;
delete[] vectmp;
delete[] reztmp;
return rez;
}
std::vector<std::thread> threadssort;
for (int i = 0; i < thr; i++) {
threadssort.push_back(std::thread([i, &vectmp, &offsets, &lens, &reztmp]() {
RadixSort(vectmp + offsets[i], lens[i], reztmp + offsets[i]);
Shuffle(reztmp + offsets[i], lens[i], vectmp + offsets[i]);
}));
}
for (auto& t : threadssort)
t.join();
int nt = thr;
int mergecount = 2, offset = 1;
int nummerge = 1, pow = 2;
while (pow < thr) {
pow *= 2;
nummerge++;
}
for (int h = 0; h < nummerge; h++) {
std::vector<std::thread> threadsmp;
for (int i = 0; i < thr; i++) {
threadsmp.push_back(std::thread([i, &vectmp, &offsets, &lens, &reztmp, mergecount, offset, thr]() {
if ((i % mergecount == 0) && (i + offset < thr)) {
size_t len1 = lens[i] / 2 + lens[i] % 2;
size_t len2 = lens[i + offset] / 2 + lens[i + offset] % 2;
PMerge(vectmp + offsets[i], vectmp + offsets[i + offset], reztmp + offsets[i], len1, len2);
}
if ((i - offset >= 0) && ((i - offset) % mergecount == 0)) {
size_t start3 = offsets[i - offset] + lens[i - offset] / 2 + lens[i - offset] % 2
+ lens[i] / 2 + lens[i] % 2;
size_t start1 = offsets[i - offset] + lens[i - offset] / 2 + lens[i - offset] % 2;
size_t start2 = offsets[i] + lens[i] / 2 + lens[i] % 2;
PMerge(vectmp + start1, vectmp + start2, reztmp + start3, lens[i - offset] / 2, lens[i] / 2);
}
}));
}
for (auto& t : threadsmp)
t.join();
std::vector<std::thread> threadsm;
for (int i = 0; i < thr; i++) {
threadsm.push_back(std::thread([i, &vectmp, &offsets, &lens,
&reztmp, mergecount, offset, nummerge, thr, h]() {
if ((i % mergecount == 0) && (i + offset < thr)) {
size_t evencount = lens[i] / 2 + lens[i + offset] / 2 + lens[i] % 2 + lens[i + offset] % 2;
size_t oddcount = lens[i] / 2 + lens[i + offset] / 2;
PMerge(reztmp + offsets[i], reztmp + offsets[i] + evencount,
vectmp + offsets[i], evencount, oddcount);
if (h != nummerge - 1) {
Shuffle(vectmp + offsets[i], lens[i] + lens[i + offset], reztmp + offsets[i]);
}
lens[i] += lens[i + offset];
lens[i + offset] = 0;
}
}));
}
for (auto& t : threadsm)
t.join();
if (h != nummerge - 1) {
for (size_t j = 0; j < size; j++)
vectmp[j] = reztmp[j];
}
nt /= 2;
mergecount *= 2;
offset *= 2;
for (int j = 1; j < thr; j++)
offsets[j] = offsets[j - 1] + lens[j - 1];
}
delete[] lens;
delete[] offsets;
for (size_t j = 0; j < size; j++)
vec[j] = vectmp[j];
delete[] vectmp;
delete[] reztmp;
return vec;
}

void Shuffle(double* vec, size_t len, double* vec2) {
for (size_t i = 0; i < len; i++)
vec2[i / 2 + (i % 2) * (len / 2 + len % 2)] = vec[i];
}
void PMerge(double* start1, double* start2, double* start3, size_t len1, size_t len2) {
size_t i = 0, j = 0, k = 0;
while ((i < len1) && (j < len2)) {
if (start1[i] < start2[j])
start3[k++] = start1[i++];
else
start3[k++] = start2[j++];
}
while (i < len1)
start3[k++] = start1[i++];
while (j < len2)
start3[k++] = start2[j++];
}
\end{lstlisting}
\begin{lstlisting}
// main.cpp
// Copyright 2021 Belik Julia
#include <gtest/gtest.h>
#include <vector>
#include <numeric>
#include <algorithm>
#include <iostream>
#include "./RadixSortB.h"

TEST(Radix_Sort_Merge_Batcher, Test_Sort) {
std::vector<double> vec;
const int n = 10;
vec = Vector(n, -10000.0, 10000.0);
double* tmp1 = new double[n];
double* tmp2 = new double[n];
for (int i = 0; i < n; i++)
tmp1[i] = vec[i];
RadixSort(tmp1, n, tmp2);
std::vector<double> v1(n);
for (int i = 0; i < n; i++)
v1[i] = tmp2[i];
std::sort(vec.begin(), vec.end());
delete[] tmp1;
delete[] tmp2;
ASSERT_EQ(v1, vec);
}
TEST(Radix_Sort_Merge_Batcher, Test_Sort_OMP_4thr) {
std::vector<double> vec;
const int n = 10;
vec = Vector(n, -10.0, 10.0);
std::vector<double> v1 = MergeBatcherOMP(vec, n, 4);
std::sort(vec.begin(), vec.end());
ASSERT_EQ(v1, vec);
}
TEST(Radix_Sort_Merge_Batcher, Test_Sort_TBB_4thr) {
std::vector<double> vec;
const int n = 10;
vec = Vector(n, -10.0, 10.0);
std::vector<double> v1 = MergeBatcherTBB(vec, n, 4);
std::sort(vec.begin(), vec.end());
ASSERT_EQ(v1, vec);
}
TEST(Radix_Sort_Merge_Batcher, Test_Sort_STD_4thr) {
std::vector<double> vec;
const int n = 10;
vec = Vector(n, -10.0, 10.0);
std::vector<double> v1 = MergeBatcherSTD(vec, n, 4);
std::sort(vec.begin(), vec.end());
ASSERT_EQ(v1, vec);
}
TEST(Radix_Sort_Merge_Batcher, Test_Sort_Par) {
std::vector<double> vec;
const int n = 20;
int thr = 4;
vec = Vector(n, -10000.0, 10000.0);
std::vector<double> v1 = MergeBatcherOMP(vec, n, thr);
std::vector<double> v2 = MergeBatcherTBB(vec, n, thr);
std::vector<double> v3 = MergeBatcherSTD(vec, n, thr);
ASSERT_EQ(v1, v2);
}
int main(int argc, char **argv) {
::testing::InitGoogleTest(&argc, argv);
return RUN_ALL_TESTS();
}
\end{lstlisting}

\end{document}